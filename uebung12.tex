\documentclass{algblatt}
\usepackage{multicol}
\loesungenfalse

\geometry{tmargin=2.0cm,bmargin=2.0cm,lmargin=2.7cm,rmargin=2.7cm}

%\setlength{\titleskip}{0.7em}
\setlength{\aufgabenskip}{1.0em}

\begin{document}

\vspace*{-1.5cm}
\maketitle{12}{Abgabe bis 8. Juli 2013, 17:00 Uhr}

\begin{aufgabe}{Allgemeines zu Gruppen}
\begin{enumerate}
\item Gibt es in der Permutationsgruppe~$S_5$ eine Untergruppe mit~$70$
Elementen?

\item Sei~$G$ eine Gruppe. Sei~$H$ eine Untergruppe von~$G$ und~$K$ eine
Untergruppe von~$H$. Wieso ist~$K$ dann auch eine Untergruppe von~$G$?

\item Sei~$G$ eine Gruppe und~$\sigma \in G$. Zeige, dass~$\sigma^i \circ
\sigma^j = \sigma^{i + j}$ für beliebige \emph{ganze} Zahlen~$i,j$.
\end{enumerate}
\end{aufgabe}

\begin{aufgabe}{Elementordnungen}
\begin{enumerate}
\item Sei~$G$ eine Gruppe und $\sigma \in G$ ein Element der Ordnung~$n$.
Zeige, dass die Ordnung einer be\-lie\-bi\-gen Potenz~$\sigma^m$ durch~$n \mathrel{/}
\operatorname{ggT}(n,m)$ gegeben ist.

\item Bestimme die Ordnungen aller Elemente der zyklischen Gruppe~$\CCC_n$.

\item Bestimme alle Erzeuger der zyklischen Gruppe~$\CCC_n$.
\end{enumerate}
\end{aufgabe}

\begin{aufgabe}{Kreisteilungspolynome}
\begin{enumerate}
\item Berechne die Kreisteilungspolynome~$\Phi_3(X)$, $\Phi_6(X)$
und~$\Phi_9(X)$.

\item Zerlege das Polynom~$X^3 + X^2 + X + 1$ über den rationalen Zahlen
in irreduzible Faktoren.
\end{enumerate}
\end{aufgabe}

\begin{aufgabe}{Etwas Zahlentheorie}
Sei~$p$ eine Primzahl.
\begin{enumerate}
\item Gib eine Primfaktorzerlegung von~$X^{p-1}-1$
modulo~$p$ an.
\item Zeige, dass der
Binomialkoeffizient~$\binom{p^2}{p}$ durch~$p$, aber nicht durch~$p^2$ teilbar
ist.
\end{enumerate}
\end{aufgabe}

\begin{aufgabe}{Primitive Wurzeln}
\begin{enumerate}
\item Gib alle primitiven Wurzeln modulo~$5$ an.

\item Sei~$X$ die Menge der~$n$-ten komplexen Einheitswurzeln. Zeige, dass die
Abbildung
\[ \sigma_d : X \longrightarrow X,\ \zeta \longmapsto \zeta^d \]
genau dann eine Bijektion ist, wenn die feste natürliche Zahl~$d$ teilerfremd
zu~$n$ ist.
\end{enumerate}
\end{aufgabe}

Zur Erinnerung: \textbf{Algebra-Treffen} am 10. Juli um 18:30 Uhr in Raum
2004/L1.

\end{document}
