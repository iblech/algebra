\documentclass{algblatt}
\usepackage{multicol}
\usepackage{xstring}
\IfSubStr{\jobname}{\detokenize{loesung}}{\loesungentrue}{\loesungenfalse}

\geometry{tmargin=2.0cm,bmargin=2.0cm,lmargin=2.75cm,rmargin=2.75cm}

%\setlength{\titleskip}{0.7em}
\setlength{\aufgabenskip}{1.0em}

\begin{document}

\vspace*{-1.5cm}
\maketitle{14}{Abgabetermin entscheidet ihr!}

\begin{aufgabe}{Illustrationen des Hauptsatzes}
\begin{enumerate}
\item Zeige, dass die einzigen Zwischenerweiterungen von~$\QQ(\sqrt{2})$
über~$\QQ$ die beiden trivialen (ganz $\QQ(\sqrt{2})$ und nur~$\QQ$) sind.
\item Finde ein normiertes separables Polynom~$f(X)$ mit rationalen Koeffizienten,
sodass der Index der Untergruppe~$\Gal_{\QQ(\sqrt[3]{2})}(x_1,\ldots,x_n)$
in~$\Gal_\QQ(x_1,\ldots,x_n)$ gleich~$3$ ist. Dabei seien~$x_1,\ldots,x_n$ die
Nullstellen von~$f(X)$. Ist diese Untergruppe ein Normalteiler?
\item Sei~$f(X)$ ein normiertes separables Polynom mit rationalen
Koeffizienten, welches mindestens eine echt komplexe Nullstelle besitzt.
Zeige, dass die Galoisgruppe der Nullstellen von~$f(X)$ mindestens ein Element
der Ordnung~$2$ besitzt.
\end{enumerate}
\end{aufgabe}

\begin{aufgabe}{Wurzelausdrücke}
\begin{enumerate}
\item Sei~$x$ eine durch Wurzeln ausdrückbare Zahl und~$x'$ ein galoissch
Konjugiertes von~$x$. Zeige, dass~$x'$
ebenfalls durch Wurzeln ausdrückbar ist, und zwar durch
denselben Wurzelausdruck wie~$x$.
\item Zeige, dass jede primitive~$n$-te Einheitswurzel durch Wurzeln, deren
Exponenten höchstens $\max\{ 2, \frac{n-1}{2}
\}$ sind, ausgedrückt werden kann.
\end{enumerate}
\end{aufgabe}

\begin{aufgabe}{Normalteiler}
\begin{enumerate}
\item Sei~$G$ eine Gruppe mit~$G \neq \{ \id \}$. Finde
zwei verschiedene Normalteiler in~$G$.
\item Sei~$G$ eine beliebige Gruppe. Zeige, dass das Zentrum von~$G$ ein
Normalteiler in~$G$ ist.
\item Ist die symmetrische Gruppe~$\SSS_5$ einfach?
\end{enumerate}
\end{aufgabe}

\begin{aufgabe}{Diedergruppen}
\begin{enumerate}
\item Bestimme explizit die Symmetriegruppe eines ebenen regelmäßigen~$n$-Ecks
in der Ebene, die sog. \emph{Diedergruppe~$\DDD_n \subseteq \SSS_n$}. Zeige, dass
diese von zwei Elementen erzeugt werden kann und insgesamt~$2n$ Elemente
enthält.
\item Zeige, dass der Index von~$\DDD_4$ in~$\SSS_4$ gleich~$3$ ist.
\item Zeige, dass~$\DDD_4$ kein Normalteiler in~$\SSS_4$ ist.
\end{enumerate}
\end{aufgabe}

\begin{aufgabe}{Auflösbarkeit von Gleichungen}
\begin{enumerate}
\item Finde ein normiertes irreduzibles Polynom~$f(X)$ fünften Grads mit
rationalen Koeffizienten, sodass die Gleichung~$f(X) = 0$ auflösbar ist.
\item Zeige, dass die Gleichung~$X^5 - 23\,X + 1 = 0$ nicht auflösbar ist.
\end{enumerate}
\end{aufgabe}

\begin{aufgabe}{Kriterium für Konstruierbarkeit}
Sei~$x$ eine algebraische Zahl und~$t$ ein primitives Element zu allen galoissch
Konjugierten von~$x$. Zeige, dass~$x$ genau dann konstruierbar ist, wenn der
Grad von~$t$ eine Zweierpotenz ist.
\end{aufgabe}

\end{document}
