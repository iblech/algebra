\documentclass{algblatt}
\usepackage{multicol}
\loesungenfalse

\geometry{tmargin=2.0cm,bmargin=2.0cm,lmargin=2.75cm,rmargin=2.75cm}

%\setlength{\titleskip}{0.7em}
\setlength{\aufgabenskip}{1.0em}

\begin{document}

\vspace*{-1.5cm}
\maketitle{14}{Abgabetermin entscheidet ihr!}

\begin{aufgabe}{Illustrationen des Hauptsatzes}
\begin{enumerate}
\item Zeige, dass die einzigen Zwischenerweiterungen von~$\QQ(\sqrt{2})$
über~$\QQ$ die beiden trivialen (ganz $\QQ(\sqrt{2})$ und nur~$\QQ$) sind.
\item Finde ein normiertes separables Polynom~$f(X)$ mit rationalen Koeffizienten,
sodass der Index der Untergruppe~$\Gal_{\QQ(\sqrt[3]{2})}(x_1,\ldots,x_n)$
in~$\Gal_\QQ(x_1,\ldots,x_n)$ gleich~$3$ ist. Dabei seien~$x_1,\ldots,x_n$ die
Nullstellen von~$f(X)$. Ist diese Untergruppe ein Normalteiler?
\item Sei~$f(X)$ ein normiertes separables Polynom mit rationalen
Koeffizienten, welches mindestens eine echt komplexe Nullstelle besitzt.
Zeige, dass die Galoisgruppe der Nullstellen von~$f(X)$ mindestens ein Element
der Ordnung~$2$ besitzt.
\end{enumerate}

\begin{loesungE}
\item
\emph{Variante über den Hauptsatz:} Das Polynom~$X^2 - 2$ hat die
Nullstellen~$\pm\sqrt{2}$; ein primitives Element der Nullstellen
ist~$\sqrt{2}$, und daher können wir den Hauptsatz verwenden, um Auskunft über
die Zwischenerweiterungen von~$\QQ(\sqrt{2})$ zu erhalten: Diese stehen in
1:1--Korrespondenz zu den Untergruppen der Galoisgruppe der beiden Nullstellen.
Diese ist~$\{ \id, \sigma \}$, wobei~$\sigma = (1, 2)$; es gibt also genau zwei
Untergruppen, entsprechend den Zwischenerweiterungen~$\QQ$ und~$\QQ(\sqrt{2})$.

\emph{Direkte Variante:} Sei~$\QQ(\sqrt{2}) \supseteq L \supseteq \QQ$ eine
Zwischenerweiterung. Nach der Gradformel muss~$[L:\QQ]$ gleich~$2$ oder~$1$
sein. Im ersten Fall gilt~$L = \QQ(\sqrt{2})$, im zweiten~$L = \QQ$.

\item Wir setzen~$f(X) = X^3 - 2$. Die Nullstellen sind
\[ x_1 = \sqrt[3]{2},\quad
  x_2 = \omega \sqrt[3]{2},\quad
  x_3 = \omega^2 \sqrt[3]{2}, \]
wobei~$\omega = \exp(2\pi\i/3)$. Da~$x_1$ kein primitives Element
für~$\QQ(x_1,x_2,x_3)$ ist, folgt mit Aufgabe~2b) von Blatt~11,
dass~$G := \Gal_\QQ(x_1,x_2,x_3) = \SSS_3$.

Nun gibt es die Zwischenerweiterung~$\QQ(x_1,x_2,x_3) \supseteq
\QQ(\sqrt[3]{2}) \supseteq \QQ$. Ihr Grad über~$\QQ$ ist~$3$, daher ist der
Index der zugehörigen Untergruppe~$H$ der Galoisgruppe ebenfalls~$3$. Explizit ist
sie nach Aufgabe~5b) von Blatt~13 durch
\[ H = \{ \id, (2, 3) \} \]
gegeben. Damit kann man nachrechnen, dass~$H$ kein Normalteiler
in~$G$ ist: Denn die Konjugation von~$(2,3) \in H$
durch~$(1,2) \in G$ ist
\[ (1,2) \circ (2,3) \circ (1,2)^{-1} =
  (1,3) \]
und liegt also nicht in~$H$.

\emph{Bemerkung:} Man kann sich auch die Motivation über die
Zwischenerweiterung sparen und direkt die Untergruppe~$H$ angeben.

\item Seien~$x_1,\ldots,x_n$ die Nullstellen von~$f(X)$. Dann definieren wir
eine Permutation~$\sigma \in \SSS_n$ durch die Forderung
\[ x_{\sigma(i)} = \overline{x_i} \]
für~$i = 1,\ldots,n$. Da mit~$x_i$ auch~$\overline{x_i}$ eine Nullstelle ist,
treten auf der rechten Seite alle Nullstellen genau einmal auf, sodass durch
diese Forderung wirklich eine Permutation definiert wird.

Ferner liegt diese Permutation tatsächlich in der Galoisgruppe: Denn
gilt~$H(x_1,\ldots,x_n) = 0$ für ein Polynom~$H \in \QQ[X_1,\ldots,X_n]$, so
gilt auch
\[ H(x_{\sigma(1)},\ldots,x_{\sigma(n)}) =
  H(\overline{x_1},\ldots,\overline{x_n}) =
  \overline{H}(\overline{x_1},\ldots,\overline{x_n}) =
  \overline{H(x_1,\ldots,x_n)} =
  \overline{0} = 0. \]
Nun gilt~$\sigma^2 = \id$, denn es gilt
\[ x_{\sigma^2(i)} = x_{\sigma(\sigma(i))} = \overline{x_{\sigma(i)}} =
\overline{\overline{x_i}} = x_i \]
für alle~$i = 1,\ldots,n$. Damit ist also die Ordnung von~$\sigma$ gleich~$1$
oder~$2$. Die Voraussetzung, dass mindestens eine Nullstelle echt komplex ist,
garantiert nun, dass~$\sigma \neq \id$ ist; also hat~$\sigma$ Ordnung~$2$.

\emph{Bemerkung:} Unter der der Korrespondenz des Hauptsatzes entspricht die
Untergruppe~$\{\id,\sigma\}$ der Zwischenerweiterung~$\QQ(x_1,\ldots,x_n) \cap
\RR$. Wenn man~$\QQ(t) = \QQ(x_1,\ldots,x_n)$ schreibt, kann man Erzeuger
dieser Zwischenerweiterung explizit bestimmen:
\[ \QQ(t)^{\{\id,\sigma\}} = \QQ(t + \overline{t}, t \cdot \overline{t}) =
  \QQ(2 \mathrm{Re}(t), \mathrm{Re}(t)^2 + \mathrm{Im}(t)^2) =
  \QQ(\mathrm{Re}(t), \mathrm{Im}(t)^2). \]
Man kann auch explizit das Minimalpolynom von~$t$
über~$\QQ(t)^{\{\id,\sigma\}}$ angeben: Es lautet
\[ X^2 - (t + \overline{t}) X + t \cdot \overline{t}. \]

%\item Sei~$t$ ein primitives Element der Nullstellen~$x_1,\ldots,x_n$; dann
%sind bekanntlich die Elemente der Galoisgruppe durch ihre Wirkung auf~$t$
%bereits eindeutig bestimmt, und zwar gibt es für jedes galoissch
%Konjugiertes~$t'$ von~$t$ genau ein Element~$\sigma$ mit~$\sigma \cdot t = t'$.
%
%Da mindestens eine Nullstelle echt komplex ist, muss auch~$t$ echt komplex sein
%(sonst würde~$x_i \in \QQ(x_i) \subseteq \QQ(x_1,\ldots,x_n) = \QQ(t) \subseteq
%\RR$ folgen). Folglich ist das komplex Konjugierte~$\overline{t}$, das
%auch ein galoissch Konjugiertes ist (wieso?), von~$t$ verschieden.
%
%Wir betrachten nun das eindeutig bestimmte Element~$\sigma$ der Galoisgruppe
%mit~$\sigma \cdot t = \overline{t}$. Da~$t \neq \overline{t}$, ist~$\sigma \neq
%\id$. Ferner gilt~$\sigma^2 \cdot t = \sigma \cdot \overline{t} =
\end{loesungE}
\end{aufgabe}

\begin{aufgabe}{Wurzelausdrücke}
\begin{enumerate}
\item Sei~$x$ eine durch Wurzeln ausdrückbare Zahl und~$x'$ ein galoissch
Konjugiertes von~$x$. Zeige, dass~$x'$
ebenfalls durch Wurzeln ausdrückbar ist, und zwar durch
denselben Wurzelausdruck wie~$x$.
\item Zeige, dass jede primitive~$n$-te Einheitswurzel durch Wurzeln, deren
Exponenten höchstens $\max\{ 2, \frac{n-1}{2}
\}$ sind, ausgedrückt werden kann.
\end{enumerate}

\begin{loesungE}
\item Da~$x$ durch Wurzeln ausdrückbar ist, gibt es
\begin{itemize}
\item eine natürliche Zahl~$n \geq 0$,
\item komplexe Zahlen~$z_1,\ldots,z_n$,
\item Primzahlen~$p_1,\ldots,p_n$,
\item Polynome~$f_i \in \QQ[Z_1,\ldots,Z_{i-1}]$, $i = 1,\ldots,n$ mit
\[ z_i^{p_i} = f_i(z_1,\ldots,z_{i-1}) \quad\text{und}\quad
  z_i \not\in \QQ(z_1,\ldots,z_{i-1}), \]
denn das ist gleichbedeutend damit,
dass~$z_i$ eine über~$\QQ(z_1,\ldots,z_{i-1})$ algebraisch eindeutige~$p_i$-te
Wurzel ist), sowie
\item ein Polynom~$g \in \QQ[Z_1,\ldots,Z_n]$ mit~$x = g(z_1,\ldots,z_n)$.
\end{itemize}
Wir können dann schreiben:
\[ x = g\Bigl(\sqrt[p_1]{f_1}, \sqrt[p_2]{f_2(\sqrt[p_1]{f_1})}, \ldots,
  \sqrt[p_n]{f_n(\ldots)}\Bigr). \]

Wir können nun ein Polynom~$h(X) \in \QQ[X]$ finden, das separabel ist und die
Zahlen~$x, z_1, \ldots, z_n$ als Nullstellen besitzt: Etwa dadurch, indem wir
annihilierende Polynome für diese Zahlen aufmultiplizieren und dann den größten
gemeinsamen Teiler abdividieren. Seien~$u_1,\ldots,u_m$ die Nullstellen
von~$h(X)$.

Die Zahlen~$x$ und~$x'$ sowie~$z_1,\ldots,z_n$ liegen dann alle
in~$\QQ(u_1,\ldots,u_m)$. Da~$x$ zu~$x'$ galoissch konjugiert ist, gibt es nach
Aufgabe~1c) von Blatt~11 eine Permutation~$\sigma \in \Gal_\QQ(u_1,\ldots,u_m)$
mit~$\sigma \cdot x = x'$. Nun lassen wir diese Permutation auf die~$z_i$
wirken -- dann sehen wir
\[ (\sigma \cdot z_i)^{p_i} = \sigma \cdot z_i^{p_i} =
  \sigma \cdot f_i(z_1,\ldots,z_{i-1}) =
  f_i(\sigma \cdot z_1,\ldots,\sigma \cdot z_{i-1}) \]
und
\[ \sigma \cdot z_i \not\in \QQ(\sigma \cdot z_1,\ldots,\sigma \cdot z_{i-1}).
\]
Ferner gilt
\[ x' = \sigma \cdot x = \sigma \cdot g(z_1,\ldots,z_n) =
  g(\sigma \cdot z_1,\ldots,\sigma \cdot z_n). \]
Also bezeugen dieselben Polynome~$f_1,\ldots,f_n$ und~$g$, dass~$x'$ durch
Wurzeln ausdrückbar ist.

\item Dazu schauen wir uns den Beweis von Hilfssatz~5.26 genauer an: In ihm
werden an insgesamt drei Stellen Wurzeln gezogen, und wir müssen zeigen, dass
wir die Situation jeweils so arrangieren können, dass die Wurzelexponenten
höchstens~$\max\{ 2, \frac{n-1}{2} \}$ sind.

Eine Vorbemerkung: Es gilt~$\max\{ 2, \frac{n-1}{2} \} = 2$ genau dann, wenn~$2
\geq \frac{n-1}{2}$; das ist genau dann der Fall, wenn~$n \leq 5$.

Der Fall~$n = 1$ ist klar.

\emph{Erster Fall im Beweis:} Die Zahl~$n$ ist eine zusammengesetzte Zahl. Dann
können wir~$n = p q$ schreiben, wobei~$p$ der \emph{kleinste} Primfaktor von~$n$ sein
soll und~$q$ die restlichen Faktoren aufsammelt. (Im Original durfte~$p$
auch ein größerer Primfaktor von~$n$ sein.) Im Beweis wird dann eine~$p$-te
Wurzel gezogen, also müssen wir zeigen:~$p \leq \max\{ 2, \frac{n-1}{2} \}$.
Falls~$n \leq 5$ -- also~$n = 4$ --, gilt~$p = 2$ und die Behauptung stimmt.
Falls~$n > 5$, gilt
\[ \frac{n-1}{2} = \frac{pq-1}{2} \geq \frac{p\cdot3-1}{2} = \frac{2p + p-1}{2}
= p + \frac{p-1}{2} \geq p \]
und die Behauptung stimmt ebenfalls. Dabei haben wir~$q \geq 3$ verwendet: $q =
1$ kann nicht sein (sonst wäre~$n$ prim) und~$q = 2$ kann auch nicht sein
(da~$p$ der kleinste Primfaktor von~$n$ ist, wäre sonst~$n = p q = 2 \cdot 2 =
4$ im Widerspruch zu~$n > 5$).

\emph{Zweiter Fall im Beweis:} Die Zahl~$n$ ist eine Primzahl. Im Beweis wird
dann eine~$(n-1)$-te Wurzel gezogen. Falls~$n = 2$, ist die Behauptung klar.
Sonst ist~$n-1$ eine gerade Zahl, also können wir~$n-1 = 2a$ mit~$a \geq 1$
schreiben. Nach Hilfssatz~5.19 können wir statt der~$(n-1)$-ten Wurzel auch
eine zweite Wurzel (passt, ist sicher höchstens~$\max\{ 2, \ldots \}$) gefolgt
von einer~$a$-ten Wurzel (passt ebenso, da~$a = \frac{n-1}{2} \leq \max\{
\ldots, \frac{n-1}{2} \}$) ziehen.
\end{loesungE}
\end{aufgabe}

\ifloesungen\newpage\fi
\begin{aufgabe}{Normalteiler}
\begin{enumerate}
\item Sei~$G$ eine Gruppe mit~$G \neq \{ \id \}$. Finde
zwei verschiedene Normalteiler in~$G$.
\item Sei~$G$ eine beliebige Gruppe. Zeige, dass das Zentrum von~$G$ ein
Normalteiler in~$G$ ist.
\item Ist die symmetrische Gruppe~$\SSS_5$ einfach?
\end{enumerate}

\begin{loesungE}
\item Stets sind die Untergruppen~$\{\id\}$ und~$G$ Normalteiler (wieso?). Nach
Voraussetzung sind das zwei verschiedene.

\item Das \emph{Zentrum} enthält diejenigen Elemente~$\tau \in G$, für die für
alle~$\sigma \in G$ die Identität~$\sigma \circ \tau \circ \sigma^{-1} = \tau$
gilt (äquivalent:~$\sigma \circ \tau = \tau \circ \sigma$).

Zum Nachweis der Normalteilereigenschaft sei~$\tau \in \ZZZ(G)$ und~$\sigma \in
G$ beliebig gegeben. Dann müssen wir
zeigen, dass~$\sigma \circ \tau \circ \sigma^{-1}$ ebenfalls in~$\ZZZ(G)$
liegt. Das ist klar, denn wie bemerkt ist dieses Element gerade gleich~$\tau
\in \ZZZ(G)$.

\item Nein, denn die Untergruppe~$\AAA_5 \subseteq \SSS_5$ ist ein
Normalteiler: Sei~$\tau \in \AAA_5$ und~$\sigma \in \SSS_5$. Dann gilt
\[ \sgn(\sigma \circ \tau \circ \sigma^{-1}) =
  \sgn\sigma \cdot \sgn\tau \cdot (\sgn\sigma)^{-1} =
  \sgn\tau = 1, \]
also liegt das konjugierte Element~$\sigma \circ \tau \circ \sigma^{-1}$ wieder
in~$\AAA_5$.

\emph{Bemerkung:} Völlig analog zeigt man, dass auch die Gruppen~$\SSS_n$,~$n
\geq 3$ jeweils nicht einfach sind.
\end{loesungE}
\end{aufgabe}

\begin{aufgabe}{Diedergruppen}
\begin{enumerate}
\item Bestimme explizit die Symmetriegruppe eines ebenen regelmäßigen~$n$-Ecks,
die sog. \emph{Diedergruppe~$\DDD_n \subseteq \SSS_n$}. Zeige, dass
diese von zwei Elementen erzeugt werden kann und insgesamt~$2n$ Elemente
enthält.
\item Zeige, dass der Index von~$\DDD_4$ in~$\SSS_4$ gleich~$3$ ist.
\item Zeige, dass~$\DDD_4$ kein Normalteiler in~$\SSS_4$ ist.
\end{enumerate}

\begin{loesungE}
\item Genau die folgenden Bewegungen der Ebene bilden das
regelmäßige~$n$-Eck auf sich selbst ab (wieso?):
\begin{itemize}
\item die Drehungen~$R_i$, $i = 0,\ldots,{n-1}$ um $\frac{i}{n} \cdot
360^\circ$ im Gegenuhrzeigersinn um den Mittelpunkt sowie
\item die Spiegelungen~$S_1,\ldots,S_n$ an~$n$ Achsen: Für gerades~$n$ gehen
diese jeweils entweder durch zwei gegenüberliegende Ecken oder durch die Mittelpunkte
zweier gegenüberliegender Kanten. Für ungerades~$n$ gehen diese jeweils durch
eine Ecke und den Mittelpunkt der gegenüberliegenden Seite.
\end{itemize}
Es gilt also~$\DDD_n = \{ R_0,\ldots,R_{n-1}, S_1,\ldots,S_n \}$.

Erzeugt werden kann die Diedergruppe durch~$(R_1,S_1)$, d.\,h. es
gilt~$\DDD_n = \langle R_1, S_1 \rangle$: Denn die Drehungen~$R_i$ lassen sich
als Potenz der Basisdrehung~$R_1$ darstellen ($R_i = R_0^i$) und die
Spiegelungen erhält man als Verkettung der Basisspiegelung~$S_1$ mit einer
geeigneten Drehung.

\emph{Bemerkung:} Für~$n \geq 3$ kann kein Element der Diedergruppe alleine die
volle Diedergruppe erzeugen, d.\,h. für~$n \geq 3$ ist die Diedergruppe nicht
zyklisch.

\item Es gilt~$\gra{\SSS_4}{\DDD_4} = (4!) \mathrel{/} (2\cdot4) = 3$.

\item Die Konjugation von~$R_1 \in \DDD_4$ durch~$(1,2) \in \SSS_4$ ist das
Element
\[ (1,2) \circ R_1 \circ (1,2)^{-1} =
  \begin{pmatrix}
    1 & 2 & 3 & 4 \\
    3 & 1 & 4 & 2
  \end{pmatrix} =
  (1,3,4,2), \]
welches nicht in der Diedergruppe liegt.

\emph{Bemerkung:} Diese Permutation ist durchaus eine Symmetrie -- aber einer
anderen Figur, nämlich der, die durch Vertauschung zweier Ecken eines Quadrats
entsteht: "`$\triangledown\atop\vartriangle$"'
\end{loesungE}
\end{aufgabe}

\begin{aufgabe}{Auflösbarkeit von Gleichungen}
\begin{enumerate}
\item Finde ein normiertes irreduzibles Polynom~$f(X)$ fünften Grads mit
rationalen Koeffizienten, sodass die Gleichung~$f(X) = 0$ auflösbar ist.
\item Zeige, dass die Gleichung~$X^5 - 23\,X + 1 = 0$ nicht auflösbar ist.
\end{enumerate}

\begin{loesungE}
\item Ein Beispiel ist das Polynom~$f(X) = X^5 - 2$. Dessen Nullstellen sind
nämlich~$\zeta^i \sqrt[5]{2}$, $i = 0,\ldots,4$, wobei~$\zeta$ eine primitive
fünfte Einheitswurzel ist. Da primitive Einheitswurzeln durch Wurzeln
ausdrückbar sind (Satz~5.25) und die Zahl~$\sqrt[5]{2}$ sogar ganz sicher durch
Wurzeln ausdrückbar ist, sind die Lösungen der Gleichung~$f(X) = 0$ also durch
Wurzeln ausdrückbar.

\item Wir zeigen, dass das Polynom~$f(X) = X^5 - 23\,X + 1$ irreduzibel ist und
genau zwei nicht reelle Nullstellen besitzt. Dann folgt nämlich aus
Hilfssatz~5.39, dass die Galoisgruppe der Nullstellen die volle~$\SSS_5$ ist,
und diese ist nicht auflösbar (siehe Seite~196 oben).

\emph{Nachweis der Irreduzibilität:} Rationale Nullstellen besitzt~$f(X)$
keine, denn diese könnten nur Teiler von~$1$ sein, aber~$\pm 1$ sind keine
Nullstellen. Bleibt zu zeigen, dass~$f(X)$ nicht in Faktoren der Grade~2
und~3 zerfällt. Nach dem Satz von Gauß genügt es, Faktoren mit ganzzahligen
Koeffizienten auszuschließen. Aus dem Ansatz
\[ f(X) = (a + bX + cX^2) \cdot (d + eX + fX^2 + gX^3) \]
mit ganzzahligen Koeffizienten~$a,b,c,d,e,f,g$ folgen die Gleichungen
\begin{align*}
  1 &= ad, \\
  -23 &= ae + bd, \\
  0 &= be + af + cd, \\
  0 &= ag + bf + ce, \\
  0 &= cf + bg, \\
  1 &= cg.
\end{align*}
Mit einigem Rechnen sieht man: $a = d = \pm 1$, $c = g = {\tilde\pm} 1$, $f =
-b$, $e = cb^2 - a$, $cb^2 - a + b = \mp 23$. Daraus erhält man die Beziehung
\[ b \cdot (cb + 1) = \mp 22. \]
Daraus folgen nur acht Fälle für~$b$: $b = 1$, $b = 2$, $b = 11$, $b = 22$ und
jeweils mit negativem Vorzeichen. Alle Fälle führen zu einem Widerspruch.

\emph{Nachweis der Nullstelleneigenschaft:} Am einfachsten zeigt man das
numerisch: Die Nullstellen sind
\begin{align*}
  x_1 &\approx -2{,}20, \\
  x_2 &\approx 0{,}04, \\
  x_3 &\approx 2{,}18, \\
  x_4 &\approx -0{,}01 - 2{,}19\,\i, \\
  x_5 &\approx -0{,}01 + 2{,}19\,\i.
\end{align*}
Alternativ führt man eine Kurvendiskussion, kann sich so den groben Verlauf des
reellen Graphen erschließen und daraus auch ablesen, dass es genau drei reelle
Nullstellen gibt.
\end{loesungE}
\end{aufgabe}

\begin{aufgabe}{Kriterium für Konstruierbarkeit}
Sei~$x$ eine algebraische Zahl und~$t$ ein primitives Element zu allen galoissch
Konjugierten von~$x$. Zeige, dass~$x$ genau dann konstruierbar ist, wenn der
Grad von~$t$ eine Zweierpotenz ist.

\begin{loesung}
Seien~$x_1,\ldots,x_n$ alle galoissch Konjugierten von~$x$ und
seien~$t_1,\ldots,t_m$ alle galoissch Konjugierten von~$t$. Nach
Proposition~4.4 sind diese ebenfalls primitive Elemente für~$x_1,\ldots,x_n$,
d.\,h. es gilt jeweils~$\QQ(t_i) = \QQ(t)$. Folglich gilt
insbesondere~$t_1,\ldots,t_m \in \QQ(t)$; also sind alle galoissch Konjugierten
von~$t$ in~$t$ rational und es greift Proposition~4.34: Die Zahl~$t$ ist genau
dann konstruierbar, wenn der Grad von~$t$ eine Zweierpotenz ist.

Ferner halten wir fest, dass jedes galoissch Konjugierte einer konstruierbaren
Zahl selbst konstruierbar ist. Das folgt aus Aufgabe~2a), denn konstruierbare
Zahlen sind ja nichts anderes als Zahlen, die durch Wurzeln ausdrückbar sind,
wobei alle vorkommenden Wurzelexponenten gleich~$2$ sein müssen.

Mit diesem Vorwissen zeigen wir nun die beiden Richtungen:

"`$\Longleftarrow$"': Da~$t$ konstruierbar ist, ist auch~$x$
konstruierbar, da~$x$ ja eine in~$t$ rationale Zahl ist.

"`$\Longrightarrow$"': Da~$x$ konstruierbar ist, ist auch jedes galoissch
Konjugierte von~$x$ konstruierbar. Da~$t$ in diesen galoissch Konjugierten
rational ist, ist~$t$ daher ebenfalls konstruierbar.
%Wir erinnern an die Charakterisierung von
%Konstruierbarkeit (Satz~1.13): Eine Zahl~$z$ ist genau dann konstruierbar, wenn
%es eine Folge komplexer Zahlen~$z_1,\ldots,z_{n-1},z_n=z$ gibt, sodass für
%alle~$i = 1,\ldots,n$ jeweils gilt:
%\begin{itemize}
%\item $z_i \in \QQ(z_1,\ldots,z_{i-1})$ oder
%\item $z_i$ ist eine Quadratwurzel aus einer Zahl
%aus~$\QQ(z_1,\ldots,z_{i-1})$.
%\end{itemize}
\end{loesung}
\end{aufgabe}

\end{document}
