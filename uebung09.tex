\documentclass{algblatt}
\usepackage{tikz}
\loesungenfalse

\geometry{tmargin=2cm,bmargin=2cm,lmargin=2.4cm,rmargin=2.4cm}

%\setlength{\titleskip}{0.7em}
\setlength{\aufgabenskip}{1.4em}

\begin{document}

\vspace*{-1.5cm}
\maketitle{9}{Abgabe bis 17. Juni 2013, 17:00 Uhr}

\begin{aufgabe}{Linearkombinationen}
\begin{enumerate}
\item Sei~$x$ eine Lösung der Gleichung~$X^4-3\,X^3+10\,X-10=0$. Drücke~$x^6$
als Linearkombination der Zahlen~$1,x,x^2,x^3$ mit rationalen Koeffizienten
aus.
\item Sei~$z := \sqrt{5} + \sqrt[3]{5}$ gegeben. Gib eine
natürliche Zahl~$n$ und eine verschwindende nichttriviale Linearkombination
von~$1,z,z^2,\ldots,z^n$ mit rationalen Koeffizienten an.
\item Finde zwei komplexe Zahlen, die über~$\RR$ linear unabhängig und
über~$\CC$ linear abhängig sind.
\end{enumerate}

\begin{loesungE}
\item \emph{Variante 1:} Wir rechnen unter Verwendung der Beziehung~$x^4 = 3x^3
- 10x + 10$:
\begin{align*}
  x^6 &= x^4 x^2 = (3x^3 - 10x + 10) x^2 \\
  &= 3x^4 x - 10x^3 + 10x^2 \\
  &= 3 \cdot (3x^3 - 10x + 10)x - 10x^3 + 10x^2 \\
  &= 9x^4 - 30x^2 + 30x - 10x^3 + 10x^2 \\
  &= 9 \cdot (3x^3 - 10x + 10) - 30x^2 + 30x - 10x^3 + 10x^2 \\
  &= 27x^3 - 90x + 90 - 30x^2 + 30x - 10x^3 + 10x^2 \\
  &= 17x^3 - 20x^2 - 60x + 90
\end{align*}

\emph{Variante 2:} Wir führen in einem Schritt eine Polynomdivision durch:
\[ X^6 = (X^4-3\,X^3+10\,X-10) \cdot (X^2 + 3\,X + 9) + (17\,X^3 - 20\,X^2 -
60\,X + 90). \]
Setzt man nun~$x$ für~$X$ ein, erhält man dasselbe Ergebnis, da die erste
Klammer verschwindet.

\item Wir rechnen:
\begin{align*}
  && z &= \sqrt{5} + \sqrt[3]{5} \\
  \Longleftrightarrow && \sqrt[3]{5} &= z - \sqrt{5} \\
  \Longrightarrow && 5 &= (z - \sqrt{5})^3 \\
  \Longleftrightarrow && 5 &= z^3 - 3\sqrt{5}z^2 + 15z - 5\sqrt{5} \\
  \Longleftrightarrow && 5-15z-z^3 &= -\sqrt{5} \cdot (5 + 3z^2) \\
  \Longrightarrow && (5-15z-z^3)^2 &= 5 \cdot (25 + 30z^2 + 9z^4) \\
  \Longleftrightarrow && 0 &= z^6 - 15z^4 - 10z^3 + 75z^2 - 150z - 100 
\end{align*}
Also können wir etwa~$n = 6$ setzen, die gesuchte nichttriviale und trotzdem
verschwindende Linearkombination steht schon da.

\emph{Bemerkung:} Tatsächlich ist der Ausdruck in der letzten Zeile (wenn
man~"`$X$"' statt~"`$z$"' schreibt) schon das Minimalpolynom von~$z$
über~$\QQ$. Schmerzlos kann man durch eine schnelle Gradüberlegung erkennen: Da
die Grade von~$\sqrt{5}$ und~$\sqrt[3]{5}$ teilerfremd sind (sie sind~$2$
bzw.~$3$), ist der Grad von~$\QQ(\sqrt{5},\sqrt[3]{5})$ gerade durch das
Produkt der Grade, also durch~$2 \cdot 3 = 6$ gegeben; und~$z$ ist gerade ein
primitives Element für diese Erweiterung.

\item Zum Beispiel~$1$ und~$\i$: Diese sind über~$\RR$ sicherlich linear
unabhängig, denn für reelle Zahlen~$a,b \in \RR$ folgt aus
\[ a \cdot 1 + b \cdot \i = 0 \]
sofort~$a = b = 0$, da eine komplexe Zahl genau dann null ist, wenn ihr Real-
und Imaginärteil null sind. Dagegen bezeugt die verschwindende und trotzdem
nichttriviale Linearkombination
\[ \i \cdot 1 + (-1) \cdot \i = 0, \]
dass die beiden Zahlen über~$\CC$ linear abhängig sind.

\emph{Bemerkung:} Letzteres muss auch so sein, denn~$\CC$ ist
als~$\CC$-Vektorraum nur eindimensional.
\end{loesungE}
\end{aufgabe}

\begin{aufgabe}{Grade algebraischer Zahlen}
\begin{enumerate}
\item Berechne den Grad von~$\sqrt{3} + \i$ über~$\QQ$, über~$\QQ(\sqrt{3})$ und
über~$\QQ(\i)$.
\item Finde ein Polynom mit rationalen Koeffizienten, das über~$\QQ$
irreduzibel ist, über~$\QQ(\sqrt{3})$ in genau zwei und über~$\QQ(\sqrt{3}+\i)$
in genau vier irreduzible Polynome zerfällt.
\item Seien~$a$ und~$d$ ganze Zahlen. Zeige, dass~$a + \sqrt{d}$ eine ganz
algebraische Zahl ist und berechne ihren Grad in Abhängigkeit von~$a$ und~$d$.
\item Sei~$\zeta$ eine Lösung der Polynomgleichung~$X^4 + X^3 + X^2 + X + 1 =
0$. Zeige, dass~$\zeta$ eine in~$\alpha := \exp(\pi\i/5)$ rationale Zahl ist,
und gib eine Basis von~$\QQ(\alpha)$ über~$\QQ(\zeta)$ an.
\end{enumerate}

\begin{loesungE}
\item \emph{Variante 1 (direkt, etwas länglich):} Sei~$z := \sqrt{3} + \i$. Wir
wollen zunächst den Grad von~$z$ über~$\QQ(\sqrt{3})$ bestimmen. Dazu suchen
wir das Minimalpolynom:
\begin{align*}
  && z &= \sqrt{3} + \i \\
  \Longleftrightarrow&& z - \sqrt{3} &= \i \\
  \Longrightarrow&& (z - \sqrt{3})^2 &= -1 \\
  \Longleftrightarrow&& z^2 - 2\sqrt{3}z + 4 &= 0
\end{align*}
Das Polynom~$X^2 - 2\sqrt{3}\,X + 4$ ist tatsächlich über~$\QQ(\sqrt{3})$
irreduzibel: Es hat Grad~2 und seine Nullstellen~$\sqrt{3} \pm \i$ sind echt
komplex und liegen daher nicht in~$\QQ(\sqrt{3}) \subseteq \RR$. Also ist es
das Minimalpolynom von~$z$ über~$\QQ(\sqrt{3})$; der Grad von~$z$
über~$\QQ(\sqrt{3})$ ist also~$2$.

Nun wollen wir das Minimalpolynom über~$\QQ(\i)$ bestimmen. Dazu rechnen wir:
\begin{align*}
  && z &= \sqrt{3} + \i \\
  \Longleftrightarrow&& z - \i &= \sqrt{3} \\
  \Longrightarrow&& (z - \i)^2 &= 3 \\
  \Longleftrightarrow&& z^2 - 2\i z - 4 &= 0
\end{align*}
Das Polynom~$X^2 - 2\i\,X - 4$ ist tatsächlich über~$\QQ(\i)$
irreduzibel: Es hat Grad~2 und seine Nullstellen~$\pm\sqrt{3} + \i$ liegen
nicht in~$\QQ(\i)$: Wenn doch, läge auch~$\sqrt{3}$ in~$\QQ(\i)$ (wieso?), also
gäbe es rationale Zahlen~$a, b \in \QQ$ mit~$\sqrt{3} = a + b \i$.
Realteilvergleich würde dann~$\sqrt{3} = a \in \QQ$ liefern, ein Widerspruch.
Also ist das Polynom tatsächlich das Minimalpolynom von~$z$ über~$\QQ(\i)$; der
Grad von~$z$ über~$\QQ(\i)$ ist also~$2$.

Nun bleibt es, den Grad von~$z$ über~$\QQ$ zu bestimmen. Dazu müssen wir unsere
Rechnungen fortsetzen:
\begin{align*}
  && z &= \sqrt{3} + \i \\
  \Longrightarrow&& z^2 - 2\i z - 4 &= 0 \\
  \Longleftrightarrow&& z^2 - 4 &= 2 \i z \\
  \Longrightarrow&& (z^2 - 4)^2 &= -4z^2 \\
  \Longleftrightarrow&& z^4 - 4z^2 + 16 &= 0
\end{align*}
Nun kann man nachrechnen, dass das Polynom~$X^4 - 4\,X^2 + 16$ tatsächlich über
den rationalen Zahlen irreduzibel ist. Das gelingt etwa über unseren
numerischen Irreduzibilitätstest. Ist das getan, folgt, dass der Grad von~$z$
über~$\QQ$ genau~4 ist.

\emph{Variante 2 (schneller mit der Gradformel):} Sei~$z := \sqrt{3} + \i$. Die
Zahl~$\sqrt{3}$ liegt in~$\QQ(z)$. Das kann man durch kurzes Knobeln erkennen
(es gilt~$\sqrt{3} = z - z^3/8$) oder auch daran, dass nach dem Verfahren der
Vorlesung~$z$ ein primitives Element von~$\sqrt{3}$ und~$\i$ ist (die
Ausnahmemenge~$S$ enthält nur~$0$ und~$\sqrt{3} \cdot \i$) und daher
sogar~$\QQ(z) = \QQ(\sqrt{3}, \i)$ gilt.

Auf jeden Fall liegt daher der Rechenbereich~$\QQ(\sqrt{3})$ in~$\QQ(z)$ und
wir können das Diagramm
\begin{center}\begin{tikzpicture}[node distance=1.5cm]
  \node (Q)                  {$\mathbb{Q}$};
  \node (Q3)  [above of=Q]   {$\mathbb{Q}(\sqrt{3})$};
  \node (Q3i) [above of=Q3]  {$\mathbb{Q}(\sqrt{3} + \i)$};
  \draw (Q) -- (Q3);
  \draw (Q3) -- (Q3i);
\end{tikzpicture}\end{center}
zeichnen. Nach der Gradformel gilt also
\[ \gra{\QQ(\sqrt{3}+\i)}{\QQ} = \gra{\QQ(\sqrt{3}+\i)}{\QQ(\sqrt{3})} \cdot
  \gra{\QQ(\sqrt{3})}{\QQ}. \]
Wie die erste Rechnung in Variante~1 gezeigt hat, ist der erste Faktor auf der
rechten Seite gleich~2, und vom zweiten Faktor weiß man sowieso, dass er
gleich~2 ist (Minimalpolynom ist~$X^2 - 3$). Folglich ist der Grad von~$z$
über~$\QQ$ gleich~$4$.

Bleibt, den Grad von~$z$ über~$\QQ(\i)$ zu bestimmen. Da~$\QQ(\i) \subseteq
\QQ(z)$ (da~$\i = z^3/8$), können wir dafür das Diagramm
\begin{center}\begin{tikzpicture}[node distance=1.5cm]
  \node (Q)                  {$\mathbb{Q}$};
  \node (Qi)  [above of=Q]   {$\mathbb{Q}(\sqrt{i})$};
  \node (Q3i) [above of=Qi]  {$\mathbb{Q}(\sqrt{3} + \i)$};
  \draw (Q) -- (Qi);
  \draw (Qi) -- (Q3i);
\end{tikzpicture}\end{center}
zeichnen und daher die Gradformel verwenden:
\[ \gra{\QQ(\sqrt{3}+\i)}{\QQ} = \gra{\QQ(\sqrt{3}+\i)}{\QQ(\sqrt{i})} \cdot
  \gra{\QQ(\sqrt{i})}{\QQ}. \]
Gesucht ist der erste Faktor auf der rechten Seite, die anderen Terme kennen
wir. Aufgelöst ergibt sich~$\gra{\QQ(\sqrt{3}+\i)}{\QQ(\sqrt{i})} =
\deg_{\QQ(\i)} z = 2$.

\emph{Bemerkung:} Die beiden Varianten kann man auf mehrere Arten und Weisen
miteinander kombinieren.

\item Von dem Polynom~$f(X) = X^4 - 4\,X^2 + 16 \in \QQ[X]$ haben wir schon
gesehen, dass es über~$\QQ$ irreduzibel ist. Seine vier Nullstellen sind die
Zahlen
\[ x_1 = \sqrt{3} + \i,\quad x_2 = \sqrt{3} - \i,\quad x_3 = -\sqrt{3} +
\i,\quad x_4 = -\sqrt{3} - \i, \]
welche alle in~$\QQ(\sqrt{3} + \i)$ liegen, da~$\sqrt{3} + \i$ ein primitives
Element für~$\QQ(\sqrt{3},\i)$ ist. Also zerfällt~$f(X)$
über~$\QQ(\sqrt{3}+\i)$ in die vier Linearfaktoren
\[ f(X) = (X-x_1) \cdot (X-x_2) \cdot (X-x_3) \cdot (X-x_4). \]
Über~$\QQ(\sqrt{3})$ erhalten wir die Zerlegung
\[ f(X) = (X-x_1) (X-x_2) \cdot (X-x_3) (X-x_4) =
  (X^2 - 2\sqrt{3}\,X + 4) \cdot (X^2 + 2\sqrt{3}\,X + 4). \]
Dabei sind die beiden auftretenden Faktoren sicherlich über~$\QQ(\sqrt{3})$
irreduzibel, da sie vom Grad~2 sind und ihre Nullstellen nicht
in~$\QQ(\sqrt{3})$ liegen, da sie echt komplex sind,
aber~$\QQ(\sqrt{3})$ nur reelle Zahlen enthält.

\item Wir setzen~$z := a + \sqrt{d}$ und rechnen:
\begin{align*}
  && z &= a + \sqrt{d} \\
  \Longleftrightarrow&& z - a &= \sqrt{d} \\
  \Longrightarrow&& (z-a)^2 &= d \\
  \Longleftrightarrow && z^2 - 2az + a^2-d &= 0
\end{align*}
Die Zahl~$z$ ist also als Lösung der Polynomgleichung~$X^2 - 2aX + a^2-d = 0$
mit ganzzahligen Koeffizienten ganz algebraisch. Außerdem ist damit klar, dass
der Grad von~$z$ höchstens~$2$ ist. Er ist genau dann~$1$, wenn dieses Polynom
reduzibel ist. Das ist genau dann der Fall, wenn eine seiner beiden
Nullstellen, etwa~$z$, schon ganzzahlig ist. Das wiederum ist genau dann der
Fall, wenn~$\sqrt{d}$ in~$\ZZ$ liegt; das ist äquivalent dazu, dass~$d$ eine
Quadratzahl ist.

\item Nach Aufgabe~4c) von Blatt~3 ist~$\zeta$ eine fünfte Einheitswurzel (aber
nicht die~$1$). Insbesondere ist~$\zeta$ damit auch eine zehnte Einheitswurzel
(denn~$\zeta^{10} = (\zeta^5)^2 = 1^2 = 1$). Die Zahl~$\alpha = \exp(\pi\i/5) =
\exp(2\pi\i/10)$ ist eine primitive zehnte Einheitswurzel, daher muss es einen
Exponent~$k \in \ZZ$ mit~$\alpha^k = \zeta$ geben. Also ist~$\zeta$ in~$\alpha$
rational, d.\,h. es gilt~$\QQ(\zeta) \subseteq \QQ(\alpha)$.

Umgekehrt gilt auch~$\QQ(\alpha) \subseteq \QQ(\zeta)$: Die Zahl~$(-\zeta)$ ist
nämlich eine primitive zehnte Einheitswurzel (siehe unten). Da~$\alpha$
(irgend-)eine zehnte Einheitswurzel ist, gibt es daher einen Exponent~$\ell
\in \ZZ$ mit~$(-\zeta)^\ell = \alpha$. Also ist~$\alpha$ in~$\zeta$ rational,
d.\,h. es gilt~$\QQ(\alpha) \subseteq \QQ(\zeta)$.

Zusammengenommen gilt somit~$\QQ(\alpha) = \QQ(\zeta)$. Daher ist der Grad der
Erweiterung~$1$ und eine mögliche Basis ist durch die Familie~$(1)$ der
Länge~$1$ gegeben.

Nun müssen wir noch zu begründen, wieso~$(-\zeta)$ eine primitive zehnte Einheitswurzel
ist. Klar ist zumindest, dass~$(-\zeta)$ überhaupt eine zehnte Einheitswurzel
ist, denn es gilt~$(-\zeta)^{10} = \zeta^{10} = (\zeta^5)^2 = 1$. Um die
Primitivität nachzuweisen, zeigen wir, dass~$(-\zeta)^j$ erst für~$j = 10$ (und
nicht schon für~$j = 1,2,\ldots,9$) wieder~$1$ ist:

Gelte~$(-\zeta)^j = (-1)^j \zeta^j = 1$, also~$\zeta^j = (-1)^j$. Dann kann~$j$
nicht ungerade sein, denn dann gälte~$\zeta^j = -1$, aber~$(-1)$ ist keine
fünfte Einheitswurzel. Also ist~$j$ gerade und es gilt~$\zeta^j = 1$. Da~$\zeta$ eine
\emph{primitive} fünfte Einheitswurzel ist, muss~$j$ ein Vielfaches von~$5$
sein. Da es außerdem gerade ist, muss~$j$ sogar ein Vielfaches von~$10$ sein.

\emph{Variante für den zweiten Teil, wenn man Kreisteilungspolynome kennt:} Wir wollen
zeigen, dass~$\QQ(\alpha) = \QQ(\zeta)$. Da sicher die
Inklusion~"`$\supseteq$"' gilt, genügt es zu zeigen, dass beide Vektorräume
dieselbe Dimension über~$\QQ$ haben, dass also~$\deg_\QQ \alpha = \deg_\QQ
\zeta$ gilt. Das Minimalpolynom von~$\zeta$ ist das fünfte
Kreisteilungspolynom,~$\Phi_5 = X^4 + X^3 + X^2 + X + 1$, das von~$\alpha$ ist
das zehnte Kreisteilungspolynom,~$\Phi_{10} = X^4 - X^3 + X^2 - X + 1$. Also
haben beide Zahlen in der Tat denselben Grad, nämlich~$4$.

\emph{Explizite Variante für beide Teile:} Wir gehen alle vier Möglichkeiten
für~$\zeta$ durch und drücken jeweils~$\zeta$ durch~$\alpha$ und
umgekehrt~$\alpha$ durch~$\zeta$ aus:
\begin{align*}
  \zeta &= \exp(2\pi\i/5), \text{ dann:} & \zeta &= \alpha^2, & \alpha &= -\zeta^3. \\
  \zeta &= \exp(4\pi\i/5), \text{ dann:} & \zeta &= \alpha^4, & \alpha &= -\zeta^4. \\
  \zeta &= \exp(6\pi\i/5), \text{ dann:} & \zeta &= \alpha^6, & \alpha &= -\zeta. \\
  \zeta &= \exp(8\pi\i/5), \text{ dann:} & \zeta &= \alpha^8, & \alpha &= -\zeta^2.
\end{align*}
In jedem Fall gilt also~$\QQ(\zeta) = \QQ(\alpha)$, eine Basis ist also durch~$(1)$
gegeben.
\end{loesungE}
\end{aufgabe}

\begin{aufgabe}{Spiel und Spaß mit der Gradformel}
\begin{enumerate}
\item Seien~$x \in \overline{\QQ}$, $y \in \QQ(x)$ und~$z \in \QQ(y)$. Wie lässt sich~$\deg_{\QQ(z)} x$
aus~$\deg_{\QQ(y)} x$ und~$\deg_{\QQ(z)} y$ berechnen?

\item Seien~$x, y, z$ wie in~a). Zeige, dass~$\deg_{\QQ(z)} y$ ein Teiler
von~$\deg_{\QQ(z)} x$ ist.

\item Sei~$f$ ein normiertes und irreduzibles Polynom vom Grad~$\geq 2$ mit rationalen
Koeffizienten. Sei~$a$ eine algebraische Zahl, deren Grad teilerfremd zum Grad
von~$f$ ist. Zeige, dass keine Zahl aus~$\QQ(a)$ Nullstelle von~$f$ sein kann.
\end{enumerate}

\begin{loesungE}
\item Da~$\QQ(z) \subseteq \QQ(y) \subseteq \QQ(x)$, ist die Gradformel
anwendbar:
\[ \deg_{\QQ(z)} x = \gra{\QQ(x)}{\QQ(z)} = \gra{\QQ(x)}{\QQ(y)} \cdot
\gra{\QQ(y)}{\QQ(z)} = \deg_{\QQ(y)} x \cdot \deg_{\QQ(z)} y. \]
Auf diese Weise lässt sich also der gesuchte Grad berechnen.

\item Folgt sofort aus der in~a) hergeleiteten Beziehung.

\item Sei~$w \in \QQ(a)$ eine hypothetische Zahl mit~$f(w) = 0$. Da~$f$
normiert ist, rationale Koeffizienten hat und über den rationalen Zahlen
irreduzibel ist, ist~$f$ daher Minimalpolynom von~$z$, es gilt also~$\deg_\QQ w =
\deg f$. Somit folgt
\[ \deg_\QQ a = \gra{\QQ(a)}{\QQ} = \gra{\QQ(a)}{\QQ(w)} \cdot \gra{\QQ(w)}{\QQ} =
  \gra{\QQ(a)}{\QQ(w)} \cdot \deg f, \]
also ist der Grad von~$f$ einer Teiler vom Grad von~$a$. Wegen~$\deg f \geq 2$
ist das ein Widerspruch zur Teilerfremdheitsvoraussetzung.
\end{loesungE}
\end{aufgabe}

\begin{aufgabe}{Primitive Elemente}
\begin{enumerate}
\item Finde ein primitives Element zu~$\i$ und~$\sqrt[3]{2}$.
\item Drücke~$\sqrt{2}$ und~$\sqrt{3}$ als Polynome in~$\sqrt{2} + \sqrt{3}$
mit rationalen Koeffizienten aus.
\item Seien~$z_1,\ldots,z_n$ algebraische Zahlen. Zeige, dass es eine
algebraische Zahl~$z$ mit~$\QQ(z) = \QQ(z_1,\ldots,z_n)$ gibt.
\item Sei~$f(X)$ ein Polynom mit rationalen Koeffizienten. Zeige, dass eine
algebraische Zahl~$a$ existiert, sodass~$f(X)$ über~$\QQ(a)$ vollständig in
Linearfaktoren zerfällt.\ifloesungen\enlargethispage{2em}\fi
\end{enumerate}

\begin{loesungE}
\item
\emph{Variante 1 (Verfahren aus der Vorlesung):}
Die Minimalpolynome von~$x := \i$ und~$y := \sqrt[3]{2}$ sind~$f(X) = X^2 + 1$
bzw.~$g(X) = X^3 - 2$ mit den Nullstellen~$\pm \i$ bzw.~$\omega^k \cdot \sqrt[3]{2}$,
$k=0,1,2$. Die Ausnahmemenge~$S$ ist daher gleich
\begin{align*}
  S &= \left\{ \frac{x' - x}{y - y'} \,\middle|\, f(x') = 0, g(y') = 0, y \neq
  y' \right\} \\
  &= \left\{ 0, \frac{-2\i}{\sqrt[3]{2} - \omega \sqrt[3]{2}},
    \frac{-2\i}{\sqrt[3]{2} - \omega^2 \sqrt[3]{2}} \right\}.
\end{align*}
Näherungsweise ergibt sich
\vspace{-1em}
\begin{align*}
  \frac{-2\i}{\sqrt[3]{2} - \omega \sqrt[3]{2}} \approx \phantom{+}0{,}46 - 0{,}79 \,\i, \\
  \frac{-2\i}{\sqrt[3]{2} - \omega^2 \sqrt[3]{2}} \approx -0{,}46 - 0{,}79 \,\i,
\end{align*}
also ist zum Beispiel die Wahl~$\lambda := 1 \not\in S$ erlaubt und~$\i +
\sqrt[3]{2}$ daher ein primitives Element.

\emph{Variante 2 (durch stundenlanges Knobeln):}
Wir vermuten, dass~$z := \i + \sqrt[3]{2}$ ein primitives Element ist und
wollen diese Vermutung nur noch bestätigen. Klar ist zumindest, dass~$z$
in~$\i$ und~$\sqrt[3]{2}$ rational ist, dass also~$\QQ(z) \subseteq
\QQ(\i,\sqrt[3]{2})$ gilt. Umgekehrt kann man durch Vergleich
verschiedener~$z$-Potenzen auf die Beziehungen
\begin{align*}
  \i &= -\frac{91}{22} - \frac{39}{11}z + \frac{39}{11}z^2 - \frac{20}{11}z^3 +
  \frac{9}{22}z^4 - \frac{6}{11}z^5, \\[0.7em]
  \sqrt[3]{2} &= \phantom{+}\frac{91}{22} + \frac{50}{11}z - \frac{39}{11}z^2 +
  \frac{20}{11}z^3 - \frac{9}{22}z^4 + \frac{6}{11}z^5
\end{align*}
kommen; diese bezeugen, dass~$\QQ(\i)$ und~$\QQ(\sqrt[3]{2})$ jeweils
Teilmengen von~$\QQ(z)$ sind.

\emph{Variante 3 (ein anderes primitives Element):}
Wir vermuten, dass~$z := \i \cdot \sqrt[3]{2}$ ein primitives Element ist. Klar
ist zumindest, dass~$z$ in~$\i$ und~$\sqrt[3]{2}$ rational ist, dass
also~$\QQ(z) \subseteq \QQ(\i,\sqrt[3]{2})$ gilt. Die umgekehrte Inklusion
zeigen die Beziehungen
\begin{align*}
  \i &= -z^3/2, \\
  \sqrt[3]{2} &= z^4/2.
\end{align*}

\item Sei~$z := \sqrt{2} + \sqrt{3}$. Dann rechnen wir ein paar~$z$-Potenzen
aus:
\begin{align*}
  z &= \sqrt{2} + \sqrt{3} \\
  z^2 &= 5 + 2 \sqrt{6} \\
  z^3 &= 11\sqrt{2} + 9\sqrt{3}
\end{align*}
Die Darstellung der Potenz~$z^2$ hilft uns nicht weiter, da in ihr nicht nur
die Zahlen~$\sqrt{2}$ und~$\sqrt{3}$ vorkommen, sondern auch die für uns nicht
weiter relevante Zahl~$\sqrt{6}$. Aber~$z^1$ und~$z^3$ können wir geeignet
gegeneinander ausspielen:
\begin{align*}
  \sqrt{2} &= (z^3 - 9z) / 2 \\
  \sqrt{3} &= (z^3 - 11z) / (-2)
\end{align*}

\emph{Bemerkung:} Die explizite Rechnung zeigt, dass~$\sqrt{2}+\sqrt{3}$ ein
primitives Element für~$\QQ(\sqrt{2},\sqrt{3})$ ist:
Denn~$\QQ(\sqrt{2}+\sqrt{3}) \subseteq \QQ(\sqrt{2},\sqrt{3})$ gilt sowieso,
und die umgekehrte Inklusion gilt gerade deswegen, weil~$\sqrt{2}$
und~$\sqrt{3}$ in~$\sqrt{2}+\sqrt{3}$ rational sind.

\item Wir führen einen Induktionsbeweis. Der Induktionsanfang~$n = 1$ ist klar.
Für den Induktionsschritt~$n \to n + 1$ seien algebraische
Zahlen~$z_1,\ldots,z_{n+1}$ gegeben. Nach Induktionsvoraussetzung gibt es dann
ein primitives Element der Zahlen~$z_1,\ldots,z_n$, d.\,h. eine
eine algebraische Zahl~$t$ mit~$\QQ(t) = \QQ(z_1,\ldots,z_n)$. Ferner können
wir ein primitives Element~$t'$ zu~$t$ und~$z_{n+1}$ finden, also eine Zahl mit
\[ \QQ(t') = \QQ(t,z_{n+1}) = \QQ(t)(z_{n+1}) = \QQ(z_1,\ldots,z_n)(z_{n+1}) =
\QQ(z_1,\ldots,z_{n+1}). \]
Das beschließt den Beweis des Induktionsschritts.

\item Über den algebraischen Zahlen zerfällt~$f$ vollständig in
Linearfaktoren:~$f = (X-x_1) \cdots (X-x_n)$. Etwas genauer zerfällt~$f$ aber
auch schon in dem kleineren Rechenbereich~$\QQ(x_1,\ldots,x_n)$ vollständig in
Linearfaktoren. Nach Teilaufgabe~c) (anwendbar, da alle~$x_i$ algebraisch) ist
dieser von der geforderten Form~$\QQ(a)$ für eine geeignete algebraische
Zahl~$a$.
\end{loesungE}
\end{aufgabe}

\begin{aufgabe}{Irrationale Zahlen für Fortgeschrittene}
Zeige mit elementaren Methoden direkt über den Ansatz~$\sqrt{2} = a +
b\sqrt{3}$ mit rationalen Zahlen~$a$ und~$b$, dass~$\sqrt{2}$ kein Element
von~$\QQ(\sqrt{3})$ ist, also keine in~$\sqrt{3}$ rationale Zahl ist. Welchen
Grad hat~$\sqrt{2}$ daher über~$\QQ(\sqrt{3})$?

\begin{loesung}
\emph{Variante 1 (über Primfaktoren):} Angenommen, $\sqrt{2} = a + b \sqrt{3}$
für gewisse rationale Zahlen~$a = x/y$, $b = u/v \in \QQ$ mit~$x,y,u,v \in
\ZZ$. Dann rechnen wir:
\begin{align*}
  && \sqrt{2} &= a + b \sqrt{3} \\
  \Longleftrightarrow && \sqrt{2} y v &= xv + uy \sqrt{3} \\
  \Longrightarrow && 2 y^2 v^2 &= x^2 v^2 + 2 xyuv \sqrt{3} + 3 u^2 y^2 \\
  \Longleftrightarrow && 2 y^2 v^2 - x^2 v^2 - 3 u^2 y^2 &= 2 xyuv \sqrt{3} \\
  \Longrightarrow && (2 y^2 v^2 - x^2 v^2 - 3 u^2 y^2)^2 &= 12 \cdot (xyuv)^2
\end{align*}
Auf der linken Seite kommt der Primfaktor~$3$ eine gerade Anzahl von Malen vor,
auf der rechten Seite dagegen eine ungerade Anzahl von Malen (da er
in~$(xyuv)^2$ gerade oft und dann noch einmal im Vorfaktor~$12$ vorkommt), das
ist ein Widerspruch.

\emph{Variante 2 (mit Fallunterscheidungen):} Angenommen, $\sqrt{2} = a + b
\sqrt{3}$ für gewisse rationale Zahlen~$a, b \in \QQ$. Dann folgt
\[ 2 = a^2 + 3b^2 + 2ab \sqrt{3}. \]
Falls~$ab \neq 0$, ist das ein Widerspruch, denn dann können wir
nach~$\sqrt{3}$ auflösen und so als rationale Zahl ausdrücken. Nach Aufgabe~1a)
von Blatt~0 ist~$\sqrt{3}$ aber irrational.

Falls~$ab = 0$, gibt es zwei Unterfälle: Falls~$b = 0$, gilt~$\sqrt{2} = a \in
\QQ$ im Widerspruch zur Irrationalität von~$\sqrt{2}$. Falls~$b \neq 0$,
folgt~$a = 0$ und daher~$\sqrt{2/3} = b \in \QQ$. Das kann aber nicht sein:
Ist~$b = x/y$ mit~$x,y \in \ZZ$, folgt~$2/3 = x^2/y^2$, also~$2y^2 = 3x^2$. In
der linken Seite tritt der Primfaktor~$2$ ungerade oft auf (wieso?), rechts
aber gerade oft.

\emph{Variante 3 (mit anderen Fallunterscheidungen):} Angenommen, $\sqrt{2} = a
+ b \sqrt{3}$ für gewisse rationale Zahlen~$a, b \in \QQ$. Dann kann man
nach~$a$ auflösen, quadrieren und umstellen, sodass
\[ 2 b \sqrt{6} = 2 - 3 b^2 - a^2 \]
folgt. Falls~$b \neq 0$, kann man weiter nach~$\sqrt{6}$ auflösen und
damit~$\sqrt{6}$ als rational erkennen -- ein Widerspruch. Falls~$b = 0$, folgt
direkt aus der Ursprungsgleichung, dass~$\sqrt{2}$ rational ist -- ebenfalls
ein Widerspruch.

\emph{Folgerung über den Grad:} Wegen~$\sqrt{2} \not\in \QQ(\sqrt{3})$ ist der
Grad von~$\sqrt{2}$ über~$\QQ(\sqrt{3})$ mehr als~$1$. (Tatsächlich ist er
genau~$2$, denn das Polynom~$X^2 - 2 \in \QQ(\sqrt{3})[X]$ ist ein
annulierendes Polynom für~$\sqrt{2}$.)
\end{loesung}
\end{aufgabe}

\end{document}

Herausgeflogen:
* Zwei komplexe Zahlen, die über R lin. unabh., aber über C lin. abh. sind.
* Darstellungsmatrix und Minimalpolynom von Q(x) --> Q(x), y |--> x y.
