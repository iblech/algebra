\documentclass{algblatt}
\loesungenfalse

\geometry{tmargin=2cm,bmargin=2cm,lmargin=3.0cm,rmargin=3.0cm}

%\setlength{\titleskip}{0.7em}
\setlength{\aufgabenskip}{1.6em}

\begin{document}

\vspace*{-1.5cm}
\maketitle{9}{Abgabe bis 17. Juni 2013, 17:00 Uhr}

\begin{aufgabe}{\ldots}
\begin{enumerate}
\item Sei~$x \in \QQ(\sqrt{3})$. Liegt dann immer auch eine Quadratwurzel
von~$x$ in~$\QQ(\sqrt{3})$?
\item Zeige, dass der \emph{goldene Schnitt} $\Phi = \frac{1+\sqrt{5}}{2}$ eine
ganz algebraische Zahl ist, obwohl in diesem Ausdruck Nenner vorkommen, die
sich nicht offensichtlich wegkürzen lassen.
\end{enumerate}
\end{aufgabe}

\begin{aufgabe}{Linearkombinationen}
\begin{enumerate}
\item Sei~$x$ eine Lösung der Gleichung~$X^4-2\,X^3+12\,X-10=0$. Drücke~$x^6$
als Linearkombination der Zahlen~$1,x,x^2,x^3$ mit rationalen Koeffizienten
aus.
\item Sei~$z := \sqrt{2} + \sqrt[3]{2}$ gegeben. Gib eine
natürliche Zahl~$n$ und eine verschwindende nichttriviale Linearkombination
von~$1,z,z^2,\ldots,z^n$ mit rationalen Koeffizienten an.
\end{enumerate}
\end{aufgabe}

\begin{aufgabe}{Grade algebraischer Zahlen}
\begin{enumerate}
\item Berechne den Grad von~$\sqrt{2} + \i$ über~$\QQ$, über~$\QQ(\sqrt{2})$ und
über~$\QQ(\i)$.
\item Finde ein Polynom mit rationalen Koeffizienten, dass über~$\QQ$
irreduzibel ist, über~$\QQ(\sqrt{2})$ in genau zwei und über~$\QQ(\sqrt{2}+\i)$
in genau vier irreduzible Polynome zerfällt.
\item Seien~$a$ und~$d$ ganze Zahlen. Zeige, dass~$a + \sqrt{d}$ eine ganz
algebraische Zahl ist und berechne ihren Grad in Abhängigkeit von~$a$ und~$d$.
\item Sei~$\zeta$ eine Lösung der Polynomgleichung~$X^4 + X^3 + X^2 + X + 1 =
0$. Zeige, dass~$\zeta$ eine in~$\alpha := \exp(\pi\i/5)$ rationale Zahl ist,
und gib eine Basis von~$\QQ(\alpha)$ über~$\QQ(\zeta)$ an.
\end{enumerate}
\end{aufgabe}

\begin{aufgabe}{Primitive Elemente}
\begin{enumerate}
\item Finde ein primitives Element zu~$\i$ und~$\sqrt[3]{2}$.
\item Drücke~$\sqrt{2}$ und~$\sqrt{3}$ als Polynome in~$\sqrt{2} + \sqrt{3}$
mit rationalen Koeffizienten aus.
\item Seien~$z_1,\ldots,z_n$ algebraische Zahlen. Zeige, dass es eine
algebraische Zahl~$z$ mit~$\QQ(z) = \QQ(z_1,\ldots,z_n)$ gibt.
\item Sei~$f(X)$ ein Polynom mit rationalen Koeffizienten. Zeige, dass eine
algebraische Zahl~$y$ existiert, sodass~$f(X)$ über~$\QQ(y)$ vollständig in
Linearfaktoren zerfällt.
\end{enumerate}
\end{aufgabe}

\begin{aufgabe}{Irrationale Zahlen für Fortgeschrittene}
Zeige mit elementaren Methoden direkt über den Ansatz~$\sqrt{2} = a +
b\sqrt{3}$ mit rationalen Zahlen~$a$ und~$b$, dass~$\sqrt{2}$ kein Element
von~$\QQ(\sqrt{3})$ ist, also keine in~$\sqrt{3}$ rationale Zahl ist.
\end{aufgabe}

\end{document}

Herausgeflogen:
* Zwei komplexe Zahlen, die über R lin. unabh., aber über C lin. abh. sind.
* Darstellungsmatrix und Minimalpolynom von Q(x) --> Q(x), y |--> x y.
