\documentclass{algblatt}
\usepackage{multicol}
\loesungenfalse

\geometry{tmargin=2.0cm,bmargin=2.0cm,lmargin=2.0cm,rmargin=2.0cm}

%\setlength{\titleskip}{0.7em}
\setlength{\aufgabenskip}{1.3em}

\begin{document}

\vspace*{-1.5cm}
\maketitle{11}{Abgabe bis 1. Juli 2013, 17:00 Uhr}

\begin{aufgabe}{Wirkung der galoisschen Gruppe}
Seien~$x_1,\ldots,x_n$ die Nullstellen eines normierten separablen
Polynoms~$f(X)$ mit rationalen Koeffizienten.
\begin{enumerate}
\item Seien~$\sigma$ und~$\tau$ Symmetrien der Nullstellen. Zeige, dass~$\sigma \cdot
(\tau \cdot x_i) = (\sigma \circ \tau) \cdot x_i$ für alle~$i = 1,\ldots,n$.

\item Sei~$\sigma$ eine Symmetrie der Nullstellen und seien~$z, w \in
\QQ(x_1,\ldots,x_n)$. Zeige: $\sigma \cdot (z + w) = \sigma \cdot z + \sigma
\cdot w$ und~$\sigma \cdot (zw) = (\sigma \cdot z) \, (\sigma \cdot w)$.

\item Zeige, dass genau dann eine Symmetrie~$\sigma$ der Nullstellen
mit~$x_2 = \sigma \cdot x_1$ existiert, wenn~$x_1$ und~$x_2$ zueinander
galoissch konjugiert sind.
\end{enumerate}
\end{aufgabe}

\begin{aufgabe}{Abstrakte Beispiele für Galoisgruppen}
\begin{enumerate}
\item Sei~$f(X)$ ein normiertes quadratisches Polynom mit rationalen
Koeffizienten, dessen Nullstellen~$x_1$ und~$x_2$ verschieden seien.
Berechne die Galoisgruppe~$\Gal_\QQ(x_1,x_2)$ in Abhängigkeit der Diskriminante
von~$f(X)$.

\item Sei~$f(X)$ ein normiertes irreduzibles Polynom vom Grad~$3$ mit
rationalen Koeffizienten und Nullstellen~$x_1,x_2,x_3$. Sei~$x_1$ \emph{kein}
primitives Element zu~$\QQ(x_1,x_2,x_3)$. Zeige, dass die Galoisgruppe der
Nullstellen genau sechs Element enthält.
\end{enumerate}
\end{aufgabe}

\begin{aufgabe}{Kriterium für gerade Permutationen ???}
\begin{enumerate}
\item Zeige, dass die Menge~$A_n$ der geraden Permutationen in~$n$ Ziffern eine
Untergruppe der~$S_n$ ist.

\item Zeige, dass die Galoisgruppe der Nullstellen eines normierten separables
Polynoms~$f(X)$ mit rationalen Koeffizienten genau dann vollständig in der
alternierenden Gruppe~$A_n$ enthalten ist, wenn die Diskriminante von~$f(X)$
eine Quadratwurzel in den rationalen Zahlen besitzt.
\end{enumerate}
\end{aufgabe}

\begin{aufgabe}{???}
Sei~$f(X)$ ein normiertes separables Polynom vom Grad~$n$ und~$t$ ein
primitives Element seiner Nullstellen. Zeige, dass der Grad von~$t$
höchstens~$n!$ ist.
\end{aufgabe}

\begin{aufgabe}{Galoissche Resolventen}
\begin{enumerate}
\item Wieso ist das Konzept der galoisschen Resolvente nur für separable
Polynome definiert worden?

\item Finde eine galoissche Resolvente für das Polynom~$f(X) = X^2 + X + 1$.

\item Seien~$x_1,\ldots,x_n$ die Nullstellen eines normierten separablen
Polynoms~$f(X)$ mit rationalen Koeffizienten. Sei~$C$ eine natürliche Zahl mit
\[ n \cdot \left|\frac{x_i - x_j}{x_k - x_\ell}\right| \leq C \]
für alle~$i,j,k,\ell \in \{ 1,\ldots,n \}$ mit~$k \neq \ell$. Zeige, dass
\[ V(X_1,\ldots,X_n) := X_1 + C\,X_2 + C^2\,X_3 + \cdots + C^{n-1}\,X_n \]
eine galoissche Resolvente für~$f(X)$ ist.
\end{enumerate}
\end{aufgabe}

\end{document}

Nicht enthalten:
* Schreibe das Polynom ... als Polynom in X_1 und den elementarsymm. Fkt.
  von X_1, X_2 und X_3.
* Dasselbe allgemein.
