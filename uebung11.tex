\documentclass{algblatt}
\usepackage{multicol}
\loesungenfalse

\geometry{tmargin=2.0cm,bmargin=2.0cm,lmargin=2.0cm,rmargin=2.0cm}

%\setlength{\titleskip}{0.7em}
\setlength{\aufgabenskip}{1.3em}

\begin{document}

\vspace*{-1.5cm}
\maketitle{11}{Abgabe bis 1. Juli 2013, 17:00 Uhr}

\begin{aufgabe}{Wirkung der galoisschen Gruppe}
Seien~$x_1,\ldots,x_n$ die Nullstellen eines normierten separablen
Polynoms~$f(X)$ mit rationalen Koeffizienten.
\begin{enumerate}
\item Seien~$\sigma$ und~$\tau$ Symmetrien der Nullstellen. Zeige, dass~$\sigma \cdot
(\tau \cdot x_i) = (\sigma \circ \tau) \cdot x_i$ für alle~$i = 1,\ldots,n$.

\item Sei~$\sigma$ eine Symmetrie der Nullstellen und seien~$z, w \in
\QQ(x_1,\ldots,x_n)$. Zeige: $\sigma \cdot (z + w) = \sigma \cdot z + \sigma
\cdot w$ und~$\sigma \cdot (zw) = (\sigma \cdot z) \cdot (\sigma \cdot w)$.

\item Zeige, dass genau dann eine Symmetrie~$\sigma$ der Nullstellen
mit~$x_2 = \sigma \cdot x_1$ existiert, wenn~$x_1$ und~$x_2$ zueinander
galoissch konjugiert sind.
\end{enumerate}

\begin{loesungE}
\item Es gilt~$\sigma \cdot (\tau \cdot x_i) = \sigma \cdot x_{\tau(i)} =
x_{\sigma(\tau(i))} = x_{(\sigma \circ \tau)(i)} = (\sigma \circ \tau) \cdot
x_i$.

\item Da~$z$ und~$w$ in~$\QQ(x_1,\ldots,x_n)$ liegen, gibt es Polynome~$g, h
\in \QQ[X_1,\ldots,X_n]$ mit
\begin{align*}
  z &= g(x_1,\ldots,x_n), \\
  w &= h(x_1,\ldots,x_n). \\
\intertext{Daher folgt}
  \sigma \cdot (z+w) &= \sigma \cdot (g(x_1,\ldots,x_n) + h(x_1,\ldots,x_n)) =
  g(x_{\sigma(1)},\ldots,x_{\sigma(n)}) + h(x_{\sigma(1)},\ldots,x_{\sigma(n)})
  \\
  &=
  \sigma \cdot g(x_1,\ldots,x_n) + \sigma \cdot h(x_1,\ldots,x_n) =
  \sigma \cdot z + \sigma \cdot w,
\end{align*}
der Multiplikationsfall geht völlig analog.

\item "`$\Longrightarrow$"': Sei~$m$ das Minimalpolynom von~$x_1$. Dann gilt
\[ m(x_2) = m(\sigma \cdot x_1) = \sigma \cdot m(x_1) = \sigma \cdot 0 = 0, \]
also ist~$x_2$ in der Tat galoissch konjugiert zu~$x_1$.
Die Multiplikation mit~$\sigma$ darf man deswegen an~$m$ vorbeiziehen, weil~$m$
nur rationale Koeffizienten hat. Wenn man~$m = \sum_i a_i X^i$ schreibt, kann
man eine formale Begründung wie folgt führen:
\[ m(\sigma \cdot x_2) =
  \sum_i a_i (\sigma \cdot x_2)^i =
  \sum_i (\sigma \cdot a_i) \cdot (\sigma \cdot x_2)^i =
  \sum_i \sigma \cdot (a_i x_2^i) =
  \sigma \cdot \sum_i a_i x_2^i =
  \sigma \cdot m(x_2). \]

"`$\Longleftarrow$"': Wir kürzen die Galoisgruppe der Nullstellen kurz
mit~"`$G$"' ab. Dann betrachten wir das Polynom
\[ g(X) := \prod_{\sigma \in G} (X - \sigma \cdot x_1). \]
Dieses ist sicher normiert und hat~$x_1$ als Nullstelle. Außerdem sind seine
Koeffizienten rational: Denn sie sind invariant unter der Wirkung der
Galoisgruppe. Das folgt aus der für jedes~$\tau \in G$ gültigen Beziehung
\[ \tau \cdot g(X) = \tau \cdot \prod_{\sigma \in G} (X
- \sigma \cdot x_1) = \prod_{\sigma \in G} (X - \tau \cdot
  \sigma \cdot x_1) = \prod_{\sigma \in G} (X
  - \sigma \cdot x_1) = g(X) \]
(wieso gilt diese?) und Koeffizientenvergleich.

Da nach Voraussetzung~$x_1$ zu~$x_2$ galoissch konjugiert ist, muss nach
Aufgabe~4b) von Blatt 10 auch~$x_2$ eine Nullstelle von~$g(X)$ sein, einer der
Faktoren
\[ x_2 - \sigma \cdot x_1, \quad \sigma \in G, \]
muss also verschwinden. Das zeigt die Behauptung.

\emph{Bemerkung:} Das betrachtete Polynom~$g(X)$ wird im Allgemeinen nicht das
Minimalpolynom von~$x_1$ sein; das stört aber auch nicht weiter.
\end{loesungE}
\end{aufgabe}

\begin{aufgabe}{Abstrakte Beispiele für Galoisgruppen}
\begin{enumerate}
\item Sei~$f(X)$ ein normiertes \emph{separables} quadratisches Polynom mit rationalen
Koeffizienten.
Berechne die Galoisgruppe der Nullstellen von~$f(X)$ in Abhängigkeit der Diskriminante
von~$f(X)$.

\item Sei~$f(X)$ ein normiertes irreduzibles Polynom vom Grad~$3$ mit
rationalen Koeffizienten und Nullstellen~$x_1,x_2,x_3$. Sei~$x_1$ \emph{kein}
primitives Element zu~$\QQ(x_1,x_2,x_3)$. Zeige, dass die Galoisgruppe der
Nullstellen genau sechs Elemente enthält.
\end{enumerate}

\begin{loesungE}
\item Die gesuchte Galoisgruppe~$\Gal_\QQ(x_1,x_2)$ ist eine Teilmenge der
symmetrischen Gruppe~$S_2$, die nur zwei Elemente enthält: die
Identitätspermutation und die, die die beiden Ziffern vertauscht. Bei dieser
Aufgabe geht also nur um die Frage, ob diese zweite Permutation~$\sigma$ in der
Galoisgruppe enthalten ist oder nicht; mit Aufgabe~1c) kann man diese Frage
schnell klären.

\emph{Erster Fall:} Die Diskriminante ist ein Quadrat in~$\QQ$. Dann sind~$x_1$
und~$x_2$ rationale Zahlen. Da sie verschieden sind, sind sie nicht zueinander
galoissch konjugiert. Nach Aufgabe~1c) kann~$\sigma$ daher nicht in der
Galoisgruppe liegen.

\emph{Zweiter Fall:} Die Diskriminante ist kein Quadrat in~$\QQ$. Dann
ist~$f(X)$ irreduzibel und somit das gemeinsame Minimalpolynom von~$x_1$
und~$x_2$, die beiden Nullstellen sind also galoissch Konjugierte. Nach
Aufgabe~1c) muss die Galoisgruppe daher eine Permutation enthalten, die~$x_1$
auf~$x_2$ abbildet. Da die Identitätspermutation das nicht macht, muss noch
die zweite Permutation~$\sigma$ enthalten sein.

\emph{Variante:} Mit der Formel~$|\Gal_\QQ(x_1,x_2)| =
\gra{\QQ(x_1,x_2)}{\QQ}$ kann man die Fallunterscheidungen ein wenig kürzer
fassen und muss auch nicht mehr Aufgabe~1c) kennen. Falls die Diskriminante ein
Quadrat in~$\QQ$ ist, sind die beiden Lösungen rational und es gilt
einfach~$\QQ(x_1,x_2) = \QQ$, also wird der Grad in diesem Fall~$1$ sein.
Ansonsten wird der Grad mehr als~$1$ sein [tatsächlich genau~$2$], weswegen die
Galoisgruppe dann noch das zweite Element enthalten muss.

\item Die Galoisgruppe kann höchstens sechs Elemente enthalten, denn es gibt
nur sechs Permutationen in drei Ziffern. Umgekehrt muss die Galoisgruppe aber
auch mindestens sechs Elemente enthalten, denn
\[ |\Gal_\QQ(x_1,x_2,x_3)| = \gra{\QQ(x_1,x_2,x_3)}{\QQ} =
  \underbrace{\gra{\QQ(x_1,x_2,x_3)}{\QQ(x_1)}}_{\geq\,2} \cdot
  \underbrace{\gra{\QQ(x_1)}{\QQ}}_{=\,3} \geq
  2 \cdot 3 = 6. \]
Die Abschätzung gilt deswegen, weil die einzig andere
Option~$\gra{\QQ(x_1,x_2,x_3)}{\QQ(x_1)} = 1$ gleichbedeutend
mit~$\QQ(x_1,x_2,x_3) = \QQ(x_1)$ wäre, einem Widerspruch zur Voraussetzung.
\end{loesungE}
\end{aufgabe}

\ifloesungen\newpage\fi
\begin{aufgabe}{Manchmal sind alle Symmetrien gerade}
\begin{enumerate}
\item Zeige, dass die Menge~$\AAA_n$ der geraden Permutationen in~$n$ Ziffern eine
Untergruppe der~$\SSS_n$ ist.

\item Zeige, dass die Galoisgruppe der Nullstellen eines normierten separables
Polynoms~$f(X)$ mit rationalen Koeffizienten genau dann vollständig in der
alternierenden Gruppe~$\AAA_n$ enthalten ist, wenn die Diskriminante von~$f(X)$
eine Quadratwurzel in den rationalen Zahlen besitzt.
\end{enumerate}

\begin{loesungE}
\item Die Identitätspermutation liegt in~$\AAA_n$, da sie gerade ist.

Seien~$\sigma$ und~$\tau$ zwei Permutationen aus~$\AAA_n$, also zwei gerade
Permutationen. Dann ist auch die Komposition~$\sigma \circ \tau$ eine gerade
Permutation (wieso?) und daher in~$\AAA_n$ enthalten.

Sei schließlich~$\sigma$ eine Permutation aus~$\AAA_n$. Dann ist
auch~$\sigma^{-1}$ eine gerade Permutation (wieso?), also in~$\AAA_n$ enthalten.

\emph{Bemerkung:} Wenn man das nötige abstrakte Vorwissen mitbringt, kann man
auch wie folgt argumentieren: Die Signumsabbildung ist ein
Gruppenhomomorphismus~$\SSS_n \to \{ \pm1 \}$. Die Menge~$\AAA_n$ ist daher als Kern
dieses Homomorphismus eine Untergruppe.

\item Seien~$x_1, \ldots, x_n$ die Nullstellen von~$f(X)$. Wir betrachten eine
der beiden Quadratwurzeln der Diskriminante, und zwar
\[ \delta := \prod_{i < j} (x_i - x_j). \]
Diese Zahl liegt offensichtlich in~$\QQ(x_1,\ldots,x_n)$. Außerdem sieht man,
dass wegen der vorausgesetzten Separabilität eine Permutation~$\sigma$ der
Galoisgruppe genau dann~$\delta$ invariant lässt (d.\,h.~$\sigma \cdot \delta =
\delta$ erfüllt), wenn~$\sigma$ gerade ist (sonst entsteht ein Minuszeichen).

Nun besitzt die Diskriminante genau dann eine Quadratwurzel in den rationalen
Zahlen, wenn~$\delta$ in~$\QQ$ liegt [automatisch liegt dann auch~$-\delta$
in~$\QQ$]. Das ist genau dann der Fall, wenn~$\delta$ von allen Elementen der
Galoisgruppe invariant gelassen wird. Nach obiger Überlegung ist das genau dann
der Fall, wenn alle Elemente der Galoisgruppe gerade sind, wenn also die
Galoisgruppe eine Teilmenge der alternierenden Gruppe~$\AAA_n$ ist.
\end{loesungE}
\end{aufgabe}

\begin{aufgabe}{Grad primitiver Elemente}
Sei~$f(X)$ ein normiertes separables Polynom vom Grad~$n$ und~$t$ ein
primitives Element seiner Nullstellen.
\begin{enumerate}
\item Zeige, dass jedes weitere primitive Element~$t'$ denselben Grad wie~$t$
hat.
\item Zeige, dass der Grad von~$t$ höchstens~$n!$ ist.
\item Zeige, dass der Grad von~$t$ sogar ein Teiler von~$n!$ ist.
\end{enumerate}

\begin{loesungE}
\item Nach Voraussetzung gilt~$\QQ(t) = \QQ(x_1,\ldots,x_n) = \QQ(t')$. Daher
folgt $\deg_\QQ t = \gra{\QQ(t)}{\QQ} = \gra{\QQ(x_1,\ldots,x_n)}{\QQ} =
\gra{\QQ(t')}{\QQ} = \deg_\QQ t'$.

\item Wir wissen um die fundamentale Beobachtung, dass die Elemente der
Galoisgruppe in Bijektion mit den galoissch Konjugierten von~$t$ stehen.
Insbesondere enthält die Galoisgruppe also genau so viele Elemente, wie es
galoissch Konjugierte von~$t$ gibt. Daher ist der Grad von~$t$ gerade durch
die Anzahl der Elemente der Galoisgruppe gegeben. Diese ist höchstens~$n!$, da
es nur~$n!$ Permutationen in~$n$ Ziffern gibt. Als Formel:
\[ \deg_\QQ t = |\Gal_\QQ(x_1,\ldots,x_n)| \leq |\SSS_n| = n!. \]

\emph{Variante:} Es gilt für jedes~$k = 0,\ldots,n-1$ die Abschätzung
\[ \gra{\QQ(x_1,\ldots,x_k,x_{k+1})}{\QQ(x_1,\ldots,x_k)} \leq n - k, \]
denn mit Polynomdivision über dem Rechenbereich~$\QQ(x_1,\ldots,x_k)$ sieht
man, dass sich das Polynom~$f(X)$ über~$\QQ(x_1,\ldots,x_k)$ aufspaltet als
\[ f(X) = (X-x_1) \cdots (X-x_k) \cdot g_k(X) \]
für einen verbleidenden Faktor~$g_k(X) \in \QQ(x_1,\ldots,x_k)[X]$ vom
Grad~$n-k$. Da~$f(x_{k+1}) = 0$ und die anderen Nullstellen nicht gleich~$x_{k+1}$ sind,
muss~$g_k(x_{k+1})$ gelten, also ist~$g_k(X)$ ein annulierendes Polynom
für~$x_{k+1}$ über~$\QQ(x_1,\ldots,x_k)$.

Damit gilt die Abschätzungskette
\begin{align*}
  \deg_\QQ t &= \gra{\QQ(t)}{\QQ} = \gra{\QQ(x_1,\ldots,x_n)}{\QQ} \\
  &= \gra{\QQ(x_1,\ldots,x_n)}{\QQ(x_1,\ldots,x_{n-1})} \cdots
    \gra{\QQ(x_1,x_2)}{\QQ(x_1)} \cdot \gra{\QQ(x_1)}{\QQ} \\
  &\leq 1 \cdot 2 \cdots (n-1) \cdot n = n!.
\end{align*}

\item Da die Galoisgruppe der Nullstellen eine Untergruppe der~$\SSS_n$ ist, ist
nach dem Satz von Lagrange $|\Gal_\QQ(x_1,\ldots,x_n)|$ ein Teiler von~$|\SSS_n| =
n!$. Mit der Formel aus~b) zeigt das schon die Behauptung.

\emph{Bemerkung:} Die Variante für Teilaufgabe~b) lässt sich nicht auf die
stärkere Behauptung von Teilaufgabe~c) übertragen.
\end{loesungE}
\end{aufgabe}

\begin{aufgabe}{Galoissche Resolventen}
\begin{enumerate}
\item Wieso ist das Konzept der galoisschen Resolvente nur für separable
Polynome definiert worden?

\item Finde eine galoissche Resolvente für das Polynom~$f(X) = X^2 + X + 1$.

\item Seien~$x_1,\ldots,x_n$ die Nullstellen eines normierten separablen
Polynoms~$f(X)$ mit rationalen Koeffizienten. Sei~$C$ eine natürliche Zahl mit
\[ n \cdot \left|\frac{x_i - x_j}{x_k - x_\ell}\right| \leq C \]
für alle~$i,j,k,\ell \in \{ 1,\ldots,n \}$ mit~$k \neq \ell$. Zeige, dass
\[ V(X_1,\ldots,X_n) := X_1 + C\,X_2 + C^2\,X_3 + \cdots + C^{n-1}\,X_n \]
eine galoissche Resolvente für~$f(X)$ ist.
\end{enumerate}

\begin{loesungE}
\item Eine galoissche Resolvente~$V(X_1,\ldots,X_n)$ für ein Polynom~$f(X)$ mit
den Nullstellen~$x_1,\ldots,x_n$ ist ein Polynom, sodass für je zwei
verschiedene Permutationen~$\sigma,\tau \in \SSS_n$ jeweils die
Zahlen~$V(x_{\sigma(1)},\ldots,x_{\sigma(n)})$
und~$V(x_{\tau(1)},\ldots,x_{\tau(n)})$ verschieden sind. Das ist aber
unmöglich, wenn manche der Nullstellen übereinstimmen, d.\,h. wenn~$x_i = x_j$
für~$i \neq j$ gilt.

\item \emph{Variante 1 (durch Probieren):} Seien~$x_1$ und~$x_2$ die beiden
(verschiedenen) Nullstellen von~$X^2 + X + 1$.
(Sie sind~$\omega$ und~$\omega^2$, wobei~$\omega =
\exp(2\pi\i/3)$, aber das müssen wir für diese Aufgabe gar nicht wissen.) Dann
ist etwa~$V(X_1,X_2) := X_1$ eine galoissche Resolvente,
denn in der Liste
\begin{center}
  \begin{tabular}{c|c}
    $\sigma$ & $V(x_{\sigma(1)}, x_{\sigma(2)})$ \\\hline
    $\id$ & $V(x_1,x_2) = x_1$ \\
    $(1,2)$ & $V(x_2,x_1) = x_2$
  \end{tabular}
\end{center}
kommt keine Zahl doppelt vor.

\emph{Bemerkung:} Es stimmt also ganz allgemein, dass für quadratische
Polynome~$f(X) = X^2 + b\,X + c$ jede Nullstelle~$x_i$ schon ein primitives
Element für~$\QQ(x_1,x_2)$ ist. Das kann man auch direkt sehen, denn es
gilt~$x_2 = -b - x_1$ und~$x_1 = -b - x_1$ und daher~$\QQ(x_2) = \QQ(x_1)$.

\emph{Variante 2 (mit der Technik aus der nächsten Teilaufgabe):} Wir listen
für alle~$i,j,k,\ell \in \{ 1, 2 \}$ mit~$k \neq \ell$ die für Teilaufgabe~c)
relevanten Zahlen auf:
\begin{center}
  \begin{tabular}{c|c|c|c|l}
    $i$ & $j$ & $k$ & $\ell$ & $n \cdot |x_i - x_j| / |x_k - x_\ell|$ \\\hline
    $1$ & $1$ & $1$ & $2$ & $0$ \\
    $1$ & $2$ & $1$ & $2$ & $2$ \\
    $2$ & $1$ & $1$ & $2$ & $2$ \\
    $2$ & $2$ & $1$ & $2$ & $0$ \\
    $1$ & $1$ & $2$ & $1$ & $0$ \\
    $1$ & $2$ & $2$ & $1$ & $2$ \\
    $2$ & $1$ & $2$ & $1$ & $2$ \\
    $2$ & $2$ & $2$ & $1$ & $0$
  \end{tabular}
\end{center}
Wenn man nur kurz nachdenkt, kann man sich die meisten Kombinationen sogar
sparen. Auf jeden Fall folgt, dass die Wahl~$C := 2$ möglich ist und daher
\[ x_1 + 2 x_2 \]
ein primitives Element ist.

\item Seien~$\sigma$ und~$\tau$ zwei verschiedene Permutationen. Wir müssen
zeigen, dass~$V(x_{\sigma(1)},\ldots,x_{\sigma(n)}) \neq
V(x_{\tau(1)},\ldots,x_{\tau(n)})$. Dafür wird es hilfreich sein, den größten
Index~$k \in \{ 1,\ldots,n \}$ mit~$\sigma(k) \neq \tau(k)$ zu betrachten. Nach
Voraussetzung gilt Abschätzung
\[ |x_{\sigma(i)} - x_{\tau(i)}| \leq
  |x_{\sigma(k)} - x_{\tau(k)}| \cdot
  \frac{1}{n} \cdot C \]
für alle~$i = 1,\ldots,n$. Damit ergibt sich für den Betrag~$|\delta|$ der
Differenz:
\begin{align*}
  |\delta| &= |V(x_{\sigma(1)},\ldots,x_{\sigma(n)}) -
  V(x_{\tau(1)},\ldots,x_{\tau(n)})| \\
  &=
  \Biggl|\sum_{i=1}^n (x_{\sigma(i)} - x_{\tau(i)}) \cdot C^{i-1}\Biggr| \\
  &= \Biggl|\sum_{i=1}^k (x_{\sigma(i)} - x_{\tau(i)}) \cdot C^{i-1}\Biggr| \\
  &\geq |x_{\sigma(k)} - x_{\tau(k)}| \cdot C^{k-1} - \sum_{i=1}^{k-1}
  |x_{\sigma(i)} - x_{\tau(i)}| \cdot C^{i-1}\\
  &\geq |x_{\sigma(k)} - x_{\tau(k)}| \cdot C^{k-1} - \sum_{i=1}^{k-1}
  |x_{\sigma(k)} - x_{\tau(k)}| \cdot \frac{1}{n} \cdot C \cdot C^{i-1} \\
  &\geq |x_{\sigma(k)} - x_{\tau(k)}| \cdot C^{k-1} - |x_{\sigma(k)} -
  x_{\tau(k)}| \cdot \sum_{i=1}^{k-1}
  \frac{1}{n} \cdot C^{k-1} \\
  &= |x_{\sigma(k)} - x_{\tau(k)}| \cdot C^{k-1} \cdot \left(1 -
  \frac{k-1}{n}\right) \\
  &> 0.
\end{align*}
\end{loesungE}
\end{aufgabe}

\end{document}

Nicht enthalten:
* Schreibe das Polynom ... als Polynom in X_1 und den elementarsymm. Fkt.
  von X_1, X_2 und X_3.
* Dasselbe allgemein.
