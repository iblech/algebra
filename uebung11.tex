\documentclass{algblatt}
\usepackage{multicol}
\loesungenfalse

\geometry{tmargin=2.0cm,bmargin=2.0cm,lmargin=2.0cm,rmargin=2.0cm}

%\setlength{\titleskip}{0.7em}
\setlength{\aufgabenskip}{1.3em}

\begin{document}

\vspace*{-1.5cm}
\maketitle{11}{Abgabe bis 1. Juli 2013, 17:00 Uhr}

\begin{aufgabe}{Wirkung der galoisschen Gruppe}
Seien~$x_1,\ldots,x_n$ die Nullstellen eines normierten separablen
Polynoms~$f(X)$ mit rationalen Koeffizienten.
\begin{enumerate}
\item Seien~$\sigma$ und~$\tau$ Symmetrien der Nullstellen. Zeige, dass~$\sigma \cdot
(\tau \cdot x_i) = (\sigma \circ \tau) \cdot x_i$ für alle~$i = 1,\ldots,n$.

\item Sei~$\sigma$ eine Symmetrie der Nullstellen und seien~$z, w \in
\QQ(x_1,\ldots,x_n)$. Zeige: $\sigma \cdot (z + w) = \sigma \cdot z + \sigma
\cdot w$ und~$\sigma \cdot (zw) = (\sigma \cdot z) \, (\sigma \cdot w)$.

\item Zeige, dass genau dann eine Symmetrie~$\sigma$ der Nullstellen
mit~$x_2 = \sigma \cdot x_1$ existiert, wenn~$x_1$ und~$x_2$ zueinander
galoissch konjugiert sind.
\end{enumerate}

\begin{loesungE}
\item Es gilt~$\sigma \cdot (\tau \cdot x_i) = \sigma \cdot x_{\tau(i)} =
x_{\sigma(\tau(i))} = x_{(\sigma \circ \tau)(i)} = (\sigma \circ \tau) \cdot
x_i$.

\item Da~$z$ und~$w$ in~$\QQ(x_1,\ldots,x_n)$ liegen, gibt es Polynome~$g, h
\in \QQ[X_1,\ldots,X_n]$ mit
\begin{align*}
  z &= g(x_1,\ldots,x_n), \\
  w &= h(x_1,\ldots,x_n). \\
\intertext{Daher folgt}
  \sigma \cdot (z+w) &= \sigma \cdot (g(x_1,\ldots,x_n) + h(x_1,\ldots,x_n)) =
  g(x_{\sigma(1)},\ldots,x_{\sigma(n)}) + h(x_{\sigma(1)},\ldots,x_{\sigma(n)})
  \\
  &=
  \sigma \cdot g(x_1,\ldots,x_n) + \sigma \cdot h(x_1,\ldots,x_n) =
  \sigma \cdot z + \sigma \cdot w,
\end{align*}
der Multiplikationsfall geht völlig analog.
\end{loesungE}
\end{aufgabe}

\begin{aufgabe}{Abstrakte Beispiele für Galoisgruppen}
\begin{enumerate}
\item Sei~$f(X)$ ein normiertes quadratisches Polynom mit rationalen
Koeffizienten, dessen Nullstellen~$x_1$ und~$x_2$ verschieden seien.
Berechne die Galoisgruppe~$\Gal_\QQ(x_1,x_2)$ in Abhängigkeit der Diskriminante
von~$f(X)$.

\item Sei~$f(X)$ ein normiertes irreduzibles Polynom vom Grad~$3$ mit
rationalen Koeffizienten und Nullstellen~$x_1,x_2,x_3$. Sei~$x_1$ \emph{kein}
primitives Element zu~$\QQ(x_1,x_2,x_3)$. Zeige, dass die Galoisgruppe der
Nullstellen genau sechs Element enthält.
\end{enumerate}

\begin{loesungE}
\item Die gesuchte Galoisgruppe~$\Gal_\QQ(x_1,x_2)$ ist eine Teilmenge der
symmetrischen Gruppe~$S_2$, die nur zwei Elemente enthält: die
Identitätspermutation und die, die die beiden Ziffern vertauscht. Bei dieser
Aufgabe geht also nur um die Frage, ob diese zweite Permutation~$\sigma$ in der
Galoisgruppe enthalten ist oder nicht; mit Aufgabe~1c) kann man diese Frage
schnell klären.

\emph{Erster Fall:} Die Diskriminante ist ein Quadrat in~$\QQ$. Dann sind~$x_1$
und~$x_2$ rationale Zahlen. Da sie verschieden sind, sind sie nicht zueinander
galoissch konjugiert. Nach Aufgabe~1c) kann~$\sigma$ daher nicht in der
Galoisgruppe liegen.

\emph{Zweiter Fall:} Die Diskriminante ist kein Quadrat in~$\QQ$. Dann
ist~$f(X)$ irreduzibel und somit das gemeinsame Minimalpolynom von~$x_1$
und~$x_2$, die beiden Nullstellen sind also galoissch Konjugierte. Nach
Aufgabe~1c) muss die Galoisgruppe daher eine Permutation enthalten, die~$x_1$
auf~$x_2$ abbildet. Da die Identitätspermutation das nicht macht, muss noch
die zweite Permutation~$\sigma$ enthalten sein.

\item Die Galoisgruppe kann höchstens sechs Elemente enthalten, denn es gibt
nur sechs Permutationen in drei Ziffern. Umgekehrt muss die Galoisgruppe aber
auch mindestens sechs Elemente enthalten, denn
\[ |\Gal_\QQ(x_1,x_2,x_3)| = \gra{\QQ(x_1,x_2,x_3)}{\QQ} =
  \underbrace{\gra{\QQ(x_1,x_2,x_3)}{\QQ(x_1)}}_{\geq\,2} \cdot
  \underbrace{\gra{\QQ(x_1)}{\QQ}}_{=\,3} \geq
  2 \cdot 3 = 6. \]
Die Abschätzung gilt deswegen, weil die einzig andere
Option~$\gra{\QQ(x_1,x_2,x_3)}{\QQ(x_1)} = 1$ gleichbedeutend
mit~$\QQ(x_1,x_2,x_3) = \QQ(x_1)$ wäre, einem Widerspruch zur Voraussetzung.
\end{loesungE}
\end{aufgabe}

\begin{aufgabe}{Kriterium für gerade Permutationen ???}
\begin{enumerate}
\item Zeige, dass die Menge~$A_n$ der geraden Permutationen in~$n$ Ziffern eine
Untergruppe der~$S_n$ ist.

\item Zeige, dass die Galoisgruppe der Nullstellen eines normierten separables
Polynoms~$f(X)$ mit rationalen Koeffizienten genau dann vollständig in der
alternierenden Gruppe~$A_n$ enthalten ist, wenn die Diskriminante von~$f(X)$
eine Quadratwurzel in den rationalen Zahlen besitzt.
\end{enumerate}

\begin{loesungE}
\item \ldots

\item Seien~$x_1, \ldots, x_n$ die Nullstellen von~$f(X)$. Wir betrachten eine
der beiden Quadratwurzeln der Diskriminante,
\[ \delta := \prod_{i < j} (x_i - x_j). \]
Diese Zahl liegt offensichtlich in~$\QQ(x_1,\ldots,x_n)$. Außerdem sieht man,
dass eine Permutation~$\sigma$ der Galoisgruppe genau dann~$\delta$ invariant
lässt (d.\,h.~$\sigma \cdot \delta = \delta$ erfüllt), wenn~$\sigma$ gerade
ist (sonst entsteht ein Minuszeichen).

Nun besitzt die Diskriminante genau dann eine Quadratwurzel in den rationalen
Zahlen, wenn~$\delta$ in~$\QQ$ liegt [automatisch liegt dann auch~$-\delta$
in~$\QQ$]. Das ist genau dann der Fall, wenn~$\delta$ von allen Elementen der
Galoisgruppe invariant gelassen wird. Nach obiger Überlegung ist das genau dann
der Fall, wenn alle Elemente der Galoisgruppe gerade sind, wenn also die
Galoisgruppe eine Teilmenge der alternierenden Gruppe~$A_n$ ist.
\end{loesungE}
\end{aufgabe}

\begin{aufgabe}{???}
Sei~$f(X)$ ein normiertes separables Polynom vom Grad~$n$ und~$t$ ein
primitives Element seiner Nullstellen. Zeige, dass der Grad von~$t$
höchstens~$n!$ ist.

\begin{loesung}
Wir wissen um die fundamentale Beobachtung, dass die Elemente der
Galoisgruppe in Bijektion mit den galoissch Konjugierten von~$t$ stehen.
Insbesondere enthält die Galoisgruppe also genau so viele Elemente, wie es
galoissch Konjugierte von~$t$ gibt. Daher ist der Grad von~$t$ gerade durch
die Anzahl der Elemente der Galoisgruppe gegeben. Diese ist höchstens~$n!$, da
es nur~$n!$ Permutationen in~$n$ Ziffern gibt.
\end{loesung}
\end{aufgabe}

\begin{aufgabe}{Galoissche Resolventen}
\begin{enumerate}
\item Wieso ist das Konzept der galoisschen Resolvente nur für separable
Polynome definiert worden?

\item Finde eine galoissche Resolvente für das Polynom~$f(X) = X^2 + X + 1$.

\item Seien~$x_1,\ldots,x_n$ die Nullstellen eines normierten separablen
Polynoms~$f(X)$ mit rationalen Koeffizienten. Sei~$C$ eine natürliche Zahl mit
\[ n \cdot \left|\frac{x_i - x_j}{x_k - x_\ell}\right| \leq C \]
für alle~$i,j,k,\ell \in \{ 1,\ldots,n \}$ mit~$k \neq \ell$. Zeige, dass
\[ V(X_1,\ldots,X_n) := X_1 + C\,X_2 + C^2\,X_3 + \cdots + C^{n-1}\,X_n \]
eine galoissche Resolvente für~$f(X)$ ist.
\end{enumerate}

\begin{loesungE}
\item \ldots

\item \ldots

\item Seien~$\sigma$ und~$\tau$ zwei verschiedene Permutationen. Wir müssen
zeigen, dass~$V(x_{\sigma(1)},\ldots,x_{\sigma(n)}) \neq
V(x_{\tau(1)},\ldots,x_{\tau(n)})$. Dafür wird es hilfreich sein, den größten
Index~$k \in \{ 1,\ldots,n \}$ mit~$\sigma(k) \neq \tau(k)$ zu betrachten. Nach
Voraussetzung gilt Abschätzung
\[ |x_{\sigma(i)} - x_{\tau(i)}| \leq
  |x_{\sigma(k)} - x_{\tau(k)}| \cdot
  \frac{1}{n} \cdot C \]
für alle~$i = 1,\ldots,n$. Damit ergibt sich:
\begin{align*}
  |V(x_{\sigma(1)},\ldots,x_{\sigma(n)}) - V(x_{\tau(1)},\ldots,x_{\tau(n)})| &=
  \Biggl|\sum_{i=1}^n (x_{\sigma(i)} - x_{\tau(i)}) \cdot C^{i-1}\Biggr| \\
  &= \Biggl|\sum_{i=1}^k (x_{\sigma(i)} - x_{\tau(i)}) \cdot C^{i-1}\Biggr| \\
  &\geq |x_{\sigma(k)} - x_{\tau(k)}| \cdot C^{k-1} - \sum_{i=1}^{k-1}
  |x_{\sigma(i)} - x_{\tau(i)}| \cdot C^{i-1}\\
  &\geq |x_{\sigma(k)} - x_{\tau(k)}| \cdot C^{k-1} - \sum_{i=1}^{k-1}
  |x_{\sigma(k)} - x_{\tau(k)}| \cdot \frac{1}{n} \cdot C \cdot C^{i-1} \\
  &\geq |x_{\sigma(k)} - x_{\tau(k)}| \cdot C^{k-1} - |x_{\sigma(k)} -
  x_{\tau(k)}| \cdot \sum_{i=1}^{k-1}
  \frac{1}{n} \cdot C^{k-1} \\
  &= |x_{\sigma(k)} - x_{\tau(k)}| \cdot C^{k-1} \cdot \left(1 -
  \frac{k-1}{n}\right) \\
  &> 0.
\end{align*}
\end{loesungE}
\end{aufgabe}

\end{document}

Nicht enthalten:
* Schreibe das Polynom ... als Polynom in X_1 und den elementarsymm. Fkt.
  von X_1, X_2 und X_3.
* Dasselbe allgemein.
