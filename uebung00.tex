\documentclass{algblatt}
\usepackage{xstring}
\IfSubStr{\jobname}{\detokenize{loesung}}{\loesungentrue}{\loesungenfalse}


\begin{document}

\maketitle{0}{
  \vbox{Präsenzblatt, Besprechung in der ersten Übung, keine Abgabe \\[1.6em]
  
  \begin{minipage}{0.93\textwidth}
  \setlength\parskip{\medskipamount}
  Willkommen zur Veranstaltung \emph{Algebra I} im Sommersemester 2013!

  Die Übungen beginnen am 18.~April, die Anmeldung ist bis zum 17.~April um 16:00~Uhr auf
  \url{https://digicampus.uni-augsburg.de/} möglich.
  In der ersten Übung besprechen wir dieses
  Präsenzblatt, dessen Lösung nicht abgegeben werden muss.
  
  Übungsblatt~1 ist
  dann schriftlich zu bearbeiten und mit Namen und Übungs\-grup\-pen\-num\-mer versehen
  bis zum 23.~April im Briefkasten im Erdgeschoss des Mathe\-gebäu\-des abzugeben.
  Tipps dazu, wie man ein Übungsblatt bearbeitet, gibt es unter
  \url{http://xrl.us/uebungsblatt}.
  \end{minipage}
}}

\begin{aufgabe}{Irrationale Zahlen}
Zeige, dass folgende Zahlen jeweils nicht rational sind:
\begin{enumerate}
\item $\sqrt{3}$
\item $\sqrt{12}$
\item $\sqrt[3]{25}$
\end{enumerate}
\begin{loesungE}
\item Angenommen, $\sqrt{3}$ ist doch rational, also~$\sqrt{3} = p/q$ für
gewisse ganze Zahlen~$p,q$. Dann folgt
\[ 3 q^2 = p^2. \]
Auf der rechten Seite kommt der Primfaktor~3 eine gerade Anzahl von Malen vor
(wieso?), links aber eine ungerade Anzahl von Malen. Das ist ein Widerspruch.
\item Aus~$\sqrt{12} = p/q$ folgt~$12 q^2 = 2^2 \cdot 3 \cdot q^2 = p^2$, also
liefert dasselbe Argument den Widerspruch.

\emph{Bemerkung:} Alternativ beobachtet man wegen~$\sqrt{12} = 2 \sqrt{3}$,
dass aus der Rationalität von~$\sqrt{12}$ die Rationalität von~$\sqrt{3}$
folgen würde, die wir in Teilaufgabe~a) aber widerlegt haben. Hierbei nutzt man
aus, dass das Produkt rationaler Zahlen rational ist (wieso?).

\item Aus~$\sqrt[3]{25} = p/q$ folgt~$25 q^3 = p^3$. Die Anzahl der Male, wie
oft der Primfaktor~5 vorkommt, ist rechts ein Vielfaches von~3 und links nicht
(es bleibt der Rest~2, wieso?).
\end{loesungE}

\emph{Bemerkung:} Die Entdeckung der Irrationalität ist mathematisches
Kulturgut!
\end{aufgabe}

\ifloesungen\newpage\fi
\begin{aufgabe}{Beispiele für Polynomgleichungen}
Seien ganze Zahlen~$x_1, \ldots, x_n$ gegeben. Finde eine Polynomgleichung mit
ganzzahligen Koeffizienten, die diese Zahlen als Lösungen besitzt.
\begin{loesung}$(X-x_1) \cdots (X-x_n) = 0$ (wieso hat diese Gleichung
ganzzahlige Koeffizienten?).

\emph{Bemerkung:} Natürlich gibt es noch viele weitere passende Gleichungen.
\end{loesung}
\end{aufgabe}

\begin{aufgabe}{Rationale Lösungen sind schon ganzzahlige Lösungen}
\begin{enumerate}
  \item Zeige, dass eine ganze Zahl~$a$ genau dann eine~$n$-te Wurzel in den
  rationalen Zahlen besitzt, wenn sie eine~$n$-te Wurzel in den ganzen Zahlen
  besitzt.
  \item Zeige, dass jede rationale Lösung einer normierten Polynomgleichung mit
  ganzzahligen Koeffizienten schon eine ganze Zahl ist.
  \item Was haben die Teilaufgaben~a) und~b) miteinander zu tun?
\end{enumerate}
\begin{loesungE}
\item Zu zeigen ist also: $\sqrt[n]{a} \in \QQ \Leftrightarrow \sqrt[n]{a} \in
\ZZ$.

Dabei ist die Richtung "`$\Leftarrow$"' klar. Für die
Richtung~"`$\Rightarrow$"' können wir eine vollständig gekürzte
Bruchdarstellung von~$\sqrt[n]{a}$ ansetzen: $\sqrt[n]{a} = p/q$.

Dann folgt~$a = p^n / q^n \in \ZZ$. Da diese Division aufgeht, kommt jeder
Primfaktor von~$q$ auch in~$p^n$ und damit in~$p$ vor. Da~$p$ und~$q$ keine
gemeinsamen Primfaktoren haben, besteht~$q$ also aus überhaupt keinem
Primfaktor und ist daher (wieso?) gleich~$\pm 1$.
\item Sei~$x = p/q$ eine vollständig gekürzte Bruchdarstellung einer Lösung der
Gleichung
\[ X^n + a_{n-1} X^{n-1} + \cdots + a_1 X + a_0 = 0. \]
Dann folgt
\[ p^n = -q \cdot (p^{n-1} a_{n-1} + q p^{n-2} a_{n-2} + \cdots + q^{n-2} p
a_1 + q^{n-1} a_0), \]
also ist~$q$ ein Teiler von~$p^n$. Daher kommt jeder Primfaktor von~$q$ auch
in~$p^n$ und daher in~$p$ vor (wieso?). Da nach Voraussetzung~$p$ und~$q$ keine
gemeinsamen Primfaktoren haben, besteht~$q$ aus überhaupt keinem Primfaktor und
ist daher (wieso?) gleich~$\pm1$. Also ist~$x = \pm p$ eine ganze Zahl.
\item Eine~$n$-te Wurzel von~$a$ ist eine Lösung der Polynomgleichung~$X^n - a
= 0$. Da das eine normierte Polynomgleichung mit ganzzahligen Koeffizienten
ist, folgt die Behauptung von Teilaufgabe~a) mit Teilaufgabe~b).
\end{loesungE}
\end{aufgabe}

\begin{aufgabe}{Rechnen mit komplexen Zahlen}
Schreibe folgende komplexe Zahlen in der Form~$x + y \i$, wobei~$x$ und~$y$
reelle Zahlen sind:
\begin{enumerate}
\item $(1 + 2 \i) \cdot (3 - 4 \i)$
\item $\dfrac{1 + \i}{1 - \i}$
\item alle vier Lösungen der Polynomgleichung $X^4 - 1 = 0$
\end{enumerate}

\begin{loesungE}
\item $11 + 2 \, \i$.
\item $0 + 1 \, \i$.
\item $1 + 0 \, \i$, $0 + 1 \, \i$, $-1 + 0 \, \i$ und~$0 + (-1) \, \i$.
\end{loesungE}
\end{aufgabe}

\end{document}
