\documentclass{algblatt}

\begin{document}

\maketitle{0}{
  \vbox{Präsenzblatt, Besprechung in der ersten Übung, keine Abgabe \\[1.6em]
  
  \begin{minipage}{0.93\textwidth}
  \setlength\parskip{\medskipamount}
  Willkommen zur Veranstaltung \emph{Algebra I} im Sommersemester 2013!

  Die Übungen beginnen am 18.~April, die Anmeldung ist bis zum 17.~April um 16:00~Uhr auf
  \url{https://digicampus.uni-augsburg.de/} möglich.
  In der ersten Übung besprechen wir dieses
  Präsenzblatt, dessen Lösung nicht abgegeben werden muss.
  
  Übungsblatt~1 ist
  dann schriftlich zu bearbeiten und mit Namen und Übungs\-grup\-pen\-num\-mer versehen
  bis zum ??.~April im Briefkasten im Erdgeschoss des Mathe\-gebäu\-des abzugeben.
  Tipps dazu, wie man ein Übungsblatt bearbeitet, gibt es unter
  \url{http://xrl.us/uebungsblatt}.
  \end{minipage}
}}

\begin{aufgabe}{Irrationale Zahlen}
Zeige, dass folgende Zahlen jeweils nicht rational sind:
\begin{enumerate}
\item $\sqrt{3}$
\item $\sqrt{12}$
\item $\sqrt[3]{25}$
\end{enumerate}
\end{aufgabe}

\begin{aufgabe}{Beispiele für Polynomgleichungen}
Seien ganze Zahlen~$x_1, \ldots, x_n$ gegeben. Finde eine Polynomgleichung mit
ganzzahligen Koeffizienten, die diese Zahlen als Lösungen besitzt.
\end{aufgabe}

\begin{aufgabe}{Rationale Lösungen sind schon ganzzahlige Lösungen}
\begin{enumerate}
  \item Zeige, dass eine ganze Zahl~$a$ genau dann eine~$n$-te Wurzel in den
  rationalen Zahlen besitzt, wenn sie eine~$n$-te Wurzel in den ganzen Zahlen
  besitzt.
  \item Zeige, dass jede rationale Lösung einer normierten Polynomgleichung mit
  ganzzahligen Koeffizienten schon eine ganze Zahl ist.
  \item Was haben die Teilaufgaben~a) und~b) miteinander zu tun?
\end{enumerate}
\end{aufgabe}

\begin{aufgabe}{Rechnen mit komplexen Zahlen}
Schreibe folgende komplexe Zahlen in der Form~$x + y \i$, wobei~$x$ und~$y$
reelle Zahlen sind:
\begin{enumerate}
\item $(1 + 2 \i) \cdot (3 - 4 \i)$
\item $\dfrac{1 + \i}{1 - \i}$
\item alle vier Lösungen der Polynomgleichung $X^4 - 1 = 0$
\end{enumerate}
\end{aufgabe}

\end{document}
