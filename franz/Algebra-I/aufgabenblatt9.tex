\documentclass{algsheet}
    \usepackage{amssymb} 
\usepackage[protrusion=true,expansion=false]{microtype}

%%%%%%%%%%%%%%%%%%%%%%%%%%%%%%%%%%%%%%%%%%%%%%%%%%%%%%%%%%%%%%%%%%%%%%%%%%%%%%%%%%%%%%%%%%%%%%%%%%%%%%%%%%%%%%%%%%%%%%%%%%%%%%%%%%%%%%%%%%%%%%%%%%%%%%%%%%%%

   \lecture{Algebra I\\ \small{Einführung in die Algebra}}
        
        \semester{Wintersemester 2010/2011}
        \sheet{9.\ Aufgabenblatt}
        \author{Dipl.-Math.~Franz Vogler}
        \date{14.~Dezember 2010}
        
        %/usr/local/share/texmf/tex/algmacros.sty
    \usepackage[arrow,curve,matrix]{xy}    
        \begin{document}
                \maketitle





%1
\begin{exercise}\textbf{Quickies} (1+1+2+2 Punkte): Die schnellen Aufgaben für zwischendurch.\newline

\begin{itemize}
 \item [\textbf{(Q1)}]
     
    Sei \(x\) eine in \(\sqrt 3\) rationale Zahl. Ist im allgemeinen dann auch
    eine Quadratwurzel aus \(x\) eine in \(\sqrt 3\) rationale Zahl?

\item [\textbf{(Q2)}]
         Gib zwei komplexe Zahlen an, welche über den reellen Zahlen linear unabhängig,
    über den komplexen Zahlen allerdings linear abhängig sind.
 
\item [\textbf{(Q3)}]
     Zeige, daß der \emph{goldene Schnitt}
    \begin{equation}
        \upphi = \frac{1 + \sqrt 5} 2
        \label{eq:golden}
    \end{equation}
    eine ganze algebraische Zahl ist, obwohl in der
    Darstellung~\eqref{eq:golden} Nenner vorkommen, welche sich nicht offensichtlich
    wegkürzen lassen.

\item [\textbf{(Q4)}]
       Sei \(x\) eine Lösung der Gleichung
    \[
        X^4 - 2 X^3 + 12 X - 10 = 0.
    \]
    Drücke \(x^6\) durch eine Linearkombination von
    \(1\), \(x\), \(x^2\) und \(x^3\) mit rationalen Koeffizienten aus.

\end{itemize}
\end{exercise}


%2
\begin{exercise}(3 Punkte)\newline
    Seien \(a\) und \(d\) zwei ganze Zahlen. Sei \(d\) positiv. Zeige, daß
    \(a + \sqrt d\) eine ganze algebraische Zahl ist und berechne ihren
    Grad in Abhängigkeit von \(a\) und \(d\).
\end{exercise}



\begin{exercise}(2 Punkte)\newline
    Sei die algebraische Zahl \(x = \sqrt{2} + \sqrt[3] 2\) gegeben. Gib eine
    natürliche Zahl \(n\) und eine verschwindende 
    nicht triviale Linearkombination 
    von \(1\), \(x\), \(x^2\), \dots, \(x^n\) mit rationalen Koeffizienten an.
\end{exercise}



\begin{exercise}(5 Punkte)\newline
    Sei \(f(X) = X^n + a_{n - 1} X^{n - 1} + \dots a_1 X + a_0\) das Minimalpolynom
    der algebraischen Zahl \(x\) über den rationalen Zahlen. Multiplikation
    mit \(x\) induziert eine lineare Abbildung
    \[
        \phi\colon \set Q(x) \to \set Q(x),\quad y \mapsto x \cdot y.
    \]
    Gib die Darstellungsmatrix \(A\) dieses Endomorphismus' bezüglich der Basis
    \(1\), \(x\), \(x^2\), \dots, \(x^{n - 1}\) von \(\set Q(x)\) über
    \(\set Q\) an. Zeige, daß \(f(X)\) das Minimalpolynom von \(A\) ist.
\end{exercise}




\begin{exercise}(3 Punkte)\newline
    Finde ein primitives Element zu \(\mathrm i\) und \(\sqrt[3] 2\).
\end{exercise}

\begin{exercise}(4 Punkte)\newline
    Drücke \(\sqrt 2\) und \(\sqrt 3\) als Polynome in \(\sqrt 2 + \sqrt 3\) 
    mit rationalen Koeffizienten aus.
\end{exercise}

\begin{exercise}(3 Punkte)\newline
    Zeige mit elementaren Methoden direkt über den Ansatz
    \(\sqrt 2 = a + b \sqrt 3\) mit rationalen Zahlen \(a\) und \(b\), daß
    \(\sqrt 2\) keine in \(\sqrt 3\) rationale Zahl ist.
\end{exercise}

\begin{exercise}(3 Punkte)\newline
    Seien \(x_1\), \dots, \(x_n\) algebraische Zahlen. Zeige, daß eine algebraische
    Zahl \(z\) existiert, welche in \(x_1\), \dots,
    \(x_n\) rational ist und so daß jeweils \(x_1\), \dots, \(x_n\)
    in \(z\) rational ist, daß also
    \[
        \set Q(z) = \set Q(x_1, \dotsc, x_n).
    \]
\end{exercise}

\begin{exercise}(2 Punkte)\newline
    Sei \(f(X)\) ein Polynom rationalen Koeffizienten. Zeige, daß eine algebraische
    Zahl \(y\) existiert, so daß \(f(X)\) über \(y\) vollständig in Linearfaktoren
    zerfällt.
\end{exercise}

\begin{exercise}(6 Punkte)\newline
    Berechne den jeweils Grad von \(\sqrt 2 + \mathrm i\) über den rationalen
    Zahlen, über \(\sqrt 2\) und über \(\mathrm i\).
\end{exercise}

\begin{exercise}(3 Punkte)\newline
    Gib ein Polynom mit rationalen Koeffizienten an, welches über den rationalen
    Zahlen irreduzibel ist, über \(\sqrt 2\) in zwei irreduzible Polynome zerfällt und
    über \(\sqrt 2 + \mathrm i\) in vier irreduzible Polynome zerfällt.
\end{exercise}  

\begin{exercise}(3 Punkte)\newline
    Sei \(\zeta\) eine Lösung der Polynomgleichung \(X^4 + X^3 + X^2 + X + 1 = 0\). 
    Zeige, daß \(\zeta\) eine in \(\alpha \coloneqq \mathrm e^{\frac{\uppi \mathrm i}{5}}\)
    rationale Zahl ist, und gib eine Basis von \(\set Q(\alpha)\) über
    \(\set Q(\zeta)\) an.
\end{exercise}


\end{document} 