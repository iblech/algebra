\documentclass{algsheet}
    \usepackage{amssymb} 
\usepackage[protrusion=true,expansion=false]{microtype}

%%%%%%%%%%%%%%%%%%%%%%%%%%%%%%%%%%%%%%%%%%%%%%%%%%%%%%%%%%%%%%%%%%%%%%%%%%%%%%%%%%%%%%%%%%%%%%%%%%%%%%%%%%%%%%%%%%%%%%%%%%%%%%%%%%%%%%%%%%%%%%%%%%%%%%%%%%%%

   \lecture{Algebra I\\ \small{Einführung in die Algebra}}
        
        \semester{Wintersemester 2010/2011}
        \sheet{Globalübung}
        \author{Dipl.-Math.~Arturo Mancino}
        \date{08.~Dezember 2010}
        
        %/usr/local/share/texmf/tex/algmacros.sty
    \usepackage[arrow,curve,matrix]{xy}    
        \begin{document}
                \maketitle


\begin{exercise}(x Punkte)\newline
    Sei \(f(X) = X^6 + X^5 + X^4 + X^3 + X^2 + X + 1 = 0\). Zeige, daß
    \(f(X)\) irreduzibel ist.
\end{exercise}

\begin{exercise}(x Punkte)\newline
    Sei \(f(X) = X^5 + X^4 + X^3 + X^2 + X + 1\). Zeige, daß \(f(X)\) nicht
    irreduzibel ist.
\end{exercise}

\begin{exercise}(x Punkte)\newline
	Gib ein Beispiel dafür an, daß die Bedingung im
	Eisensteinschen Irreduzibilitätskriterium,
	daß \(p^2\) kein Teiler von \(a_0\) ist, notwendig ist.
\end{exercise}

\begin{exercise}(x Punkte)\newline
	Gib ein Beispiel dafür an, daß die Bedingung im
	Eisensteinschen Irreduzibilitätskriterium,
	daß \(p\) ein Teiler von allen \(a_0\), \dots, \(a_{n - 1}\) ist,
	notwendig ist. 
\end{exercise}

\begin{exercise}(x Punkte)\newline
    Seien \(f(X)\) und \(g(X)\) zwei normierte Polynome positiven Grades
    und mit rationalen Koeffizienten, so daß \(f(g(X))\) irreduzibel ist.
    Ist dann auch \(g(X)\) irreduzibel?
\end{exercise}



%5
\begin{exercise}
 \[\begin{split} f(X+1)&=(X+1)^6+(X+1)^5+(X+1)^4+(X+1)3+(X+1)2+(X+1)+1\\
     &=(X^6+6X^5+15X^4+20X^3+15X^2+6X+1)+(X^5+5X^4+10X^3+10X^2+5X+1)\\&\quad+(X^4+4X^3+6X^2+4X+1)
        +(X^3 + 3 X^2 + 3 X + 1) + (X^2 + 2 X
        + 1) + X + 1 + 1\\
 &=(X^6+6X^5+15X^4+20X^3+15X^2+6X+1)+(X^5+5X^4+10X^3+10X^2+5X+1)\\&\quad+X^4 + 5 X^3 + 10 X^2 + 10 X + 5\\
 &=X^6+7X^5+21X^4+35X^3+35X^2+21X+7.  \end{split}\]                                                       
Jetzt können wir das Eisensteinkriterium (Proposition 3.18) mit $p=7$ anwenden und erhalten, dass $f(X+1)$ irreduzibel ist.
Gäbe es nun eine Zerlegung $f(X)=g(X)h(X)$, so wäre auch $f(X+1)=g(X+1)h(X+1)$ eine Zerlegung, was nicht sein kann. 
\end{exercise}



%6
\begin{exercise}
 Durch Hinschauen sehen wir, dass $-1$ eine Nullstelle des gegebenen Polynoms ist, also ist
 \[X^5+X^4+X^3+X^2+X+1=(X+1)(X^4+X^2+1)\]
eine Faktorisierung.
\end{exercise}



%7
\begin{exercise}
 Beispielsweise $X^2+4X+4=(X+2)^2$ mit $p=2$.
\end{exercise}


%8
\begin{exercise}
 Zum Beispiel $X^2+2X+1=(X+1)^2$.
\end{exercise}



%9
\begin{exercise}
 Nein: Definiere $f(X):=X+1$ und $g(X):=X^2+2X+1$. Dann ist $f(g(X))=X^2+2X+2$ irreduzibel nach dem Eisensteinkriterium.
\end{exercise}









\end{document} 