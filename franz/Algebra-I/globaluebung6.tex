\documentclass{algsheet}
    \usepackage{amssymb} 
\usepackage[protrusion=true,expansion=false]{microtype}

%%%%%%%%%%%%%%%%%%%%%%%%%%%%%%%%%%%%%%%%%%%%%%%%%%%%%%%%%%%%%%%%%%%%%%%%%%%%%%%%%%%%%%%%%%%%%%%%%%%%%%%%%%%%%%%%%%%%%%%%%%%%%%%%%%%%%%%%%%%%%%%%%%%%%%%%%%%%

   \lecture{Algebra I\\ \small{Einführung in die Algebra}}
        
        \semester{Wintersemester 2010/2011}
        \sheet{Globalübung}
        \author{Dipl.-Math.~Arturo Mancino}
        \date{08.~Dezember 2010}
        
        %/usr/local/share/texmf/tex/algmacros.sty
    \usepackage[arrow,curve,matrix]{xy}    
        \begin{document}
                \maketitle


\begin{exercise}(x Punkte)\newline
    Sei \(f(X) = X^6 + X^5 + X^4 + X^3 + X^2 + X + 1 = 0\). Zeige, daß
    \(f(X)\) irreduzibel ist.
\end{exercise}

\begin{exercise}(x Punkte)\newline
    Sei \(f(X) = X^5 + X^4 + X^3 + X^2 + X + 1\). Zeige, daß \(f(X)\) nicht
    irreduzibel ist.
\end{exercise}

\begin{exercise}(x Punkte)\newline
	Gib ein Beispiel dafür an, daß die Bedingung im
	Eisensteinschen Irreduzibilitätskriterium,
	daß \(p^2\) kein Teiler von \(a_0\) ist, notwendig ist.
\end{exercise}

\begin{exercise}(x Punkte)\newline
	Gib ein Beispiel dafür an, daß die Bedingung im
	Eisensteinschen Irreduzibilitätskriterium,
	daß \(p\) ein Teiler von allen \(a_0\), \dots, \(a_{n - 1}\) ist,
	notwendig ist. 
\end{exercise}

\begin{exercise}(x Punkte)\newline
    Seien \(f(X)\) und \(g(X)\) zwei normierte Polynome positiven Grades
    und mit rationalen Koeffizienten, so daß \(f(g(X))\) irreduzibel ist.
    Ist dann auch \(g(X)\) irreduzibel?
\end{exercise}









\end{document} 