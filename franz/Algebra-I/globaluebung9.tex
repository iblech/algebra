\documentclass{algsheet}
    \usepackage{amssymb} 
\usepackage[protrusion=true,expansion=false]{microtype}

%%%%%%%%%%%%%%%%%%%%%%%%%%%%%%%%%%%%%%%%%%%%%%%%%%%%%%%%%%%%%%%%%%%%%%%%%%%%%%%%%%%%%%%%%%%%%%%%%%%%%%%%%%%%%%%%%%%%%%%%%%%%%%%%%%%%%%%%%%%%%%%%%%%%%%%%%%%%

   \lecture{Algebra I\\ \small{Einführung in die Algebra}}
        
        \semester{Wintersemester 2010/2011}
        \sheet{Globalübung}
        \author{Dipl.-Math.~Arturo Mancino}
        \date{15.~Dezember 2010}
        
        %/usr/local/share/texmf/tex/algmacros.sty
    \usepackage[arrow,curve,matrix]{xy}    
        \begin{document}
                \maketitle






\begin{exercise}(x Punkte)\newline
    Sei \(x\) eine algebraische Zahl. Zeige, daß \(x\) genau dann Grad \(1\)
    über den rationalen Zahlen besitzt, wenn \(x\) rational ist. 
\end{exercise}

\begin{exercise}(x Punkte)\newline
    Gib eine algebraische Zahl vom Grad \(7\) über den rationalen Zahlen
    an.
\end{exercise}

\begin{exercise}(x Punkte)\newline
    Zeige, daß für jede positive natürliche Zahl \(n\) eine algebraische Zahl
    vom Grad \(n\) über den rationalen Zahlen existiert.
\end{exercise}

\begin{exercise}(x Punkte)\newline
    Seien \(x\) und \(y\) algebraische Zahlen, deren Grade über
    den rationalen Zahlen \(n\) beziehungsweise \(m\) seien. Zeige, daß
    die Grade von \(x + y\) und \(x y\) höchstens \(n \cdot m\) sind.
\end{exercise}


\begin{exercise}(x Punkte)\newline
    Sei \(z\) Lösung der Gleichung
    \begin{equation}
        X^n + a_{n - 1} X^{n - 1} + \dots + a_1 X + a_0 = 0,
    \end{equation}
    wobei die \(a_0\), \dots, \(a_{n - 1}\) algebraische Zahlen sind.
    Gib eine obere Schranke für den Grad von \(z\) in Termen von
    \(n\) und den Graden von \(a_0\), \dots, \(a_{n - 1}\) über den rationalen
    Zahlen an.
\end{exercise}



\begin{exercise}(x Punkte)\newline
    Gib eine algebraische Zahl an, welche keine ganze algebraische Zahl ist.
\end{exercise}

AAAAAAAAAAAAAAAAAAAAAAAAAAAAAAAAAAAAAAAAAAAAAAAAAAAAAAAAAAAAAAAAAAAAAAAAAAAAAAAAAAAA



%1
\begin{exercise}
 Ist $x$ rational, so ist $X-x$ ein Polynom in $\mathbb Q[X]$, der Grad von $x$ also gleich eins.
\newline
Ist umgekehrt $[\mathbb Q(x):\mathbb Q]=1$, so hat das Minimalpolynom von $x$ den Grad eins und ist also von der Form $X-x$,
d.h. $x$ ist eine rationale Zahl. 
\end{exercise}



%2
\begin{exercise}
  Zum Beispiel $\xi_7=\exp(\mathrm i\frac{2\pi}{7})$.
\end{exercise}




%3
\begin{exercise}
 Es ist $\sqrt[n]{n}$ eien Lösung von $X^n-n$  und damit auch algebraisch - und zwar vom Grad $n$.
 
\end{exercise}



%4
\begin{exercise}
 Das ist gerade Hilfssatz 1.4 in Verbindung mit Aufgabenblatt 3 Aufgabe 3.
\end{exercise}


%9
\begin{exercise}
Es ist $z\in\mathbb Q(a_0,a_1,\dots,a_{n-1})$ und
\[[\mathbb Q(z)\colon\mathbb Q]=[\mathbb Q(z)\colon\mathbb Q(a_0,a_1,\dots,a_{n-1})]\cdot
    [\mathbb Q(a_0,a_1,\dots,a_{n-1})\colon\mathbb Q].\]
Damit ist eine obere Schranke für den Grad von $z$ duch $n$ mal das Produkt über die Grade der $a_i$ gegeben.
\end{exercise}


%6
\begin{exercise}
 Zum Beispiel $\frac12$, die Lösung von $X-\frac12$ ist.
\end{exercise}


\end{document} 