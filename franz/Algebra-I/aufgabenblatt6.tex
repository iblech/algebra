\documentclass{algsheet}
    \usepackage{amssymb} 
\usepackage[protrusion=true,expansion=false]{microtype}

%%%%%%%%%%%%%%%%%%%%%%%%%%%%%%%%%%%%%%%%%%%%%%%%%%%%%%%%%%%%%%%%%%%%%%%%%%%%%%%%%%%%%%%%%%%%%%%%%%%%%%%%%%%%%%%%%%%%%%%%%%%%%%%%%%%%%%%%%%%%%%%%%%%%%%%%%%%%

   \lecture{Algebra I\\ \small{Einführung in die Algebra}}
        
        \semester{Wintersemester 2010/2011}
        \sheet{6.\ Aufgabenblatt}
        \author{Dipl.-Math.~Franz Vogler}
        \date{23.~November 2010}
        
        %/usr/local/share/texmf/tex/algmacros.sty
    \usepackage[arrow,curve,matrix]{xy}    
        \begin{document}
                \maketitle






\begin{exercise}(3 Punkte)\newline
     Sei \(X^3 + p X + q = 0\) eine reduzierte kubische Gleichung mit ganzzahligen
     Koeffizienten \(p\) und \(q\). Zeige, daß die Gleichung drei verschiedene
     Lösungen besitzt, wenn \(q\) ungerade ist.
\end{exercise}

\begin{exercise}(3 Punkte)\newline
    Berechne die Diskriminante einer allgemeinen kubischen Gleichung
    \[X^3 + a X^2 + b X + c = 0.\]
\end{exercise}

\begin{exercise}(3 Punkte)\newline
    Zeige, daß \(X^3 - 5 X^2 + 3 X  + 9=0\) höchstens zwei verschiedene Lösungen
    hat.
\end{exercise}

\begin{exercise}(4 Punkte)\newline
    Zeige, daß die Diskriminante einer Polynomgleichung
    \(X^n + a_{n - 1} X^{n - 1} + \dotsb + a_1 X + a_0 = 0\) mit Lösungen
    \(x_1\), \dots, \(x_n\) durch das Quadrat der \emph{Vandermondschen Determinante}
    \[
      \Delta_n(x_1,\dots,x_n)=\det \begin{pmatrix} 1 & \dots & 1 \\
                x_1 & \dots & x_n \\
                \vdots & & \vdots \\
                x_1^{n - 1} & \dots & x_n^{n - 1}
             \end{pmatrix}
    \]
    gegeben ist.
\end{exercise}

\begin{exercise}(2 Punkte)\newline
    Sei \(X^n + a_{n - 1} X^{n - 1} + \dotsb + a_1 X + a_0=0\) eine normierte
    Polynomgleichung mit rationalen Koeffizienten. Zeige, daß sie mindestens
    eine nicht reelle Nullstelle besitzt, wenn ihre Diskriminante negativ ist.
\end{exercise}

\begin{exercise}(4 Punkte)\newline
    Seien \(f(X)\) und \(g(X)\) zwei normierte Polynome mit Nullstellen (mit Vielfachheiten)
    \(x_1\), \dots, \(x_n\) beziehungsweise \(y_1\), \dots, \(y_m\). Zeige, daß
    der Ausdruck \(R = \prod\limits_{i, j} (x_i - y_j)\) ein Polynom in den
    elementarsymmetrischen Funktionen der Koeffizienten von \(f(X)\)
    und den elementarsymmetrischen Funktionen der Koeffizienten von \(g(X)\) ist.
\end{exercise}

\begin{exercise}(4 Punkte)\newline
    Seien \(X^2 + a X + b = 0\) und \(X^2 + c X + d = 0\) zwei quadratische
    Gleichungen. Gib einen in \(a\), \(b\), \(c\) und \(d\) polynomiellen
    Ausdruck an, der genau dann verschwindet, wenn die beiden Gleichungen eine
    gemeinsame Lösung besitzen.
\end{exercise}




\begin{exercise}(2 Punkte)\newline
    Sei \((z_n)\) eine konvergente komplexe Zahlenfolge mit Grenzwert \(z\).
    Seien die \(z_n\) algebraisch. Ist dann auch \(z\) algebraisch?
\end{exercise}



\begin{exercise}(2 Punkte)\newline
     Ist \(\sqrt[3] \uppi\) eine algebraische Zahl? Ist \(\uppi^3\) algebraisch?
\end{exercise}




\begin{exercise}(4 Punkte)\newline
    Ist folgendes Problem lösbar: Gegeben ein Kreis, konstruiere nur mit Zirkel
    und Lineal ein gleichseitiges Dreieck mit demselben Flächeninhalt?
\end{exercise}




\begin{exercise}(2 Punkte)\newline
     Konstruiere eine Folge paarweise verschiedener transzendenter Zahlen.
\end{exercise}



\begin{exercise}(4 Punkte)\newline
    Sei \(n\) eine natürliche Zahl. Gib eine Konstruktionsvorschrift für eine
    Primzahl \(p > n\).
\end{exercise}



\end{document} 