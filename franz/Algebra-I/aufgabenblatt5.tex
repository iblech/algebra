\documentclass{algsheet}
    \usepackage{amssymb} 
\usepackage[protrusion=true,expansion=false]{microtype}

%%%%%%%%%%%%%%%%%%%%%%%%%%%%%%%%%%%%%%%%%%%%%%%%%%%%%%%%%%%%%%%%%%%%%%%%%%%%%%%%%%%%%%%%%%%%%%%%%%%%%%%%%%%%%%%%%%%%%%%%%%%%%%%%%%%%%%%%%%%%%%%%%%%%%%%%%%%%

   \lecture{Algebra I\\ \small{Einführung in die Algebra}}
        
        \semester{Wintersemester 2010/2011}
        \sheet{5.\ Aufgabenblatt}
        \author{Dipl.-Math.~Franz Vogler}
        \date{16.~November 2010}
        
        %/usr/local/share/texmf/tex/algmacros.sty
    \usepackage[arrow,curve,matrix]{xy}    
        \begin{document}
                \maketitle



\begin{exercise}(2 Punkte)\newline
    Gib eine normierte Polynomgleichung dritten Grades an, welche \(1\)
    als zweifache Lösung, \(2\) als einfache Lösung und keine weiteren Lösungen
    besitzt.
\end{exercise}

\begin{exercise}(3 Punkte)\newline
    Warum sind beide Seiten von 
\begin{equation}
 (gh)'(X)=g'(X)h(X)+g(X)h'(X)
\end{equation}
 sowohl in
    \(g(X)\) als auch in \(h(X)\) linear und warum reicht es daher, die
    Gleichung (1) nur für \(g(X) = X^m\) und \(h(X) = X^n\)
    nachzurechnen?
\end{exercise}





\begin{exercise}(4 Punkte)\newline
    Seien \(g(X)\) und \(h(X)\) zwei Polynome. Begründe, warum
    mittels vollständiger Induktion
    \[
        (g h)^{(k)}(X)
        = \sum_{i + j = k} \binom k i \, g^{(i)}(X) \, h^{(j)}(X)
    \]
    aus Gleichung (1) in Aufgabe 2 folgt.
\end{exercise}

\begin{exercise}(3 Punkte)\newline
    Sei \(f(X)\) ein Polynom. Zeige, daß \(f^{(n + 1)} = 0 \iff \deg f \leq n\)
    für jede natürliche Zahl \(n\).
\end{exercise}
\newpage

\begin{exercise}(4 Punkte)\newline
    Sei \(f(X)\) ein Polynom und \(x\) eine komplexe Zahl. Zeige, daß die
    Entwicklung von \(f(X)\) nach \(X - x\) durch die \emph{Taylorsche%
    \footnote{Brooke Taylor, 1685--1731, britischer Mathematiker} Formel}
    \[
        f(X) = \sum_{k = 0}^\infty \frac{f^{(k)}(x)}{k!} \, (X - x)^k
    \]
    gegeben ist.
\end{exercise}

\begin{exercise}(2 Punkte)\newline
    Gib die zweite elementarsymmetrische Funktion in den fünf Unbestimmten
    \(X\), \(Y\), \(Z\), \(U\) und \(V\) explizit an.
\end{exercise}

\begin{exercise}(4 Punkte)\newline
    Sei \(X^4 + a_3 X^3 + a_2 X^2 + a_1 X + a_0 = 0\) eine normierte Polynomgleichung
    vierten Grades, deren Lösungen mit Vielfachheiten \(x_1\), \(x_2\), \(x_3\)
    und \(x_4\) seien. Drücke die Koeffizienten \(a_0\), \(a_1\), \(a_2\) und
    \(a_3\) explizit als Polynome in den \(x_i\) aus.
\end{exercise}

\begin{exercise}(4 Punkte)\newline
    Verwende den Vietaschen Satz für \(n = 2\) um die bekannte Lösungsformel
    für normierte quadratische Gleichungen herzuleiten.
\end{exercise}




\begin{exercise}(8 Punkte)\newline
    Sei \(f(X, Y, Z, W) \coloneqq X Y + Z W + X Y Z W\).
    Wieviele \(4\)-stellige Permutationen \(\sigma\) gibt es, so daß
    \(\sigma \cdot f = f\)?
\end{exercise}

\begin{exercise}(3 Punkte)\newline
     Zeige, daß \(e_k(\underbrace{1, \dots, 1}_n) = \binom n k\).
\end{exercise}
 
\begin{exercise}(1 Punkt)\newline
    Schreibe \(X^2 + Y^2 + Z^2\) als Polynom in den elementarsymmetrischen
    Funktionen in \(X\), \(Y\) und \(Z\).
\end{exercise}

\begin{exercise}(1 Punkt)\newline
    Schreibe \(X_1^2 + X_2^2 + \dotsb + X_n^2\) als Polynom in den
    elementarsymmetrischen Funktionen in \(X_1\), \dots, \(X_n\).
\end{exercise}

\begin{exercise}(2 Punkte)\newline
    Sei \(X^3 + p X + q = 0\) eine reduzierte kubische Gleichung. Zeige, daß
    ihre Diskrimante durch \(-4 p^3 - 27 q^2\) gegeben ist.
\end{exercise}



\end{document} 