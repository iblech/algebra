\documentclass{algsheet}
    \usepackage{amssymb} 
\usepackage[protrusion=true,expansion=false]{microtype}

%%%%%%%%%%%%%%%%%%%%%%%%%%%%%%%%%%%%%%%%%%%%%%%%%%%%%%%%%%%%%%%%%%%%%%%%%%%%%%%%%%%%%%%%%%%%%%%%%%%%%%%%%%%%%%%%%%%%%%%%%%%%%%%%%%%%%%%%%%%%%%%%%%%%%%%%%%%%

   \lecture{Algebra I\\ \small{Einführung in die Algebra}}
        
        \semester{Wintersemester 2010/2011}
        \sheet{8.\ Aufgabenblatt}
        \author{Dipl.-Math.~Franz Vogler}
        \date{07.~Dezember 2010}
        
        %/usr/local/share/texmf/tex/algmacros.sty
    \usepackage[arrow,curve,matrix]{xy}    
        \begin{document}
                \maketitle





\begin{exercise}(4 Punkte)\newline
    Bestimme numerisch die Nullstellen von \(f(X) = X^4 - 10 X^2 + 1\)
    bis auf wenige Stellen nach dem Komma, und nutze diese Information um zu
    zeigen, daß \(f(X)\) über den rationalen Zahlen irreduzibel ist.
\end{exercise}

\begin{exercise}(3 Punkte)\newline
    Sei \(f(X)\) ein nicht verschwindendes Polynom mit rationalen Koeffizienten.
    Zeige, daß der Inhalt von \(f(X)\) genau dann ganzzahlig ist, wenn 
    \(f(X)\) ganzzahlige Koeffizienten besitzt.
\end{exercise}

\begin{exercise}(3 Punkte)\newline
    Sei \(f(X)\) ein normiertes Polynom mit rationalen Koeffizienten. Zeige,
    daß der Inhalt von \(f(X)\) das Inverse des führenden Koeffizienten von
    \(\tilde f(X)\) ist.
\end{exercise}

\begin{exercise}(4 Punkte)\newline
    Sei \(f(X)\) ein Polynom mit rationalen Koeffizienten. Zeige, daß
    der Inhalt von \(f(X)\) das Inverse einer ganzen Zahl ist, wenn
    \(f(X)\) normiert ist. Gilt auch die Umkehrung?
\end{exercise}





\begin{exercise}(2 Punkte)\newline
    Seien \(a\), \(a'\), \(b\) und \(b'\) vier ganze Zahlen, so daß die
    Kongruenzen \(a \equiv b\) und \(a' \equiv b'\) modulo einer weiteren
    ganzen Zahl \(n\) gelten.
    Rechne explizit nach, daß \(a + a' \equiv b + b'\) modulo \(n\).
\end{exercise}

\begin{exercise}(3 Punkte)\newline
    Sei \(a\) eine ganze Zahl mit \(a \equiv 1\) modulo \(3\). Für welche
    Exponenten \(n\) ist dann \(a^n \equiv 2\) modulo \(3\)?
\end{exercise}

\begin{exercise}(3 Punkte)\newline
    Berechne zwei Inverse von \(6\) modulo \(35\).
\end{exercise}

\begin{exercise}(4 Punkte)\newline
    Seien \(n\) eine positive ganze Zahl
    und \(a\) eine ganze Zahl, welche teilerfremd zu
    \(n\) ist. Seien \(b\) und \(b'\) zwei ganze Zahlen mit
    \(a b \equiv a b' \equiv 1 \pmod n\). Zeige, daß
    \(b \equiv b'  \pmod n\).
\end{exercise}

\begin{exercise}(3 Punkte)\newline
    Sei \(f(X)\) ein normiertes Polynom mit ganzzahligen Koeffizienten.
    Sei \(p\) eine Primzahl, so daß \(f(X)\) modulo \(p\) nicht irreduzibel
    ist. Ist dann auch \(f(X)\) nicht irreduzibel?
\end{exercise}

\begin{exercise}(5 Punkte)\newline
    Seien \(b_0\), \dots, \(b_m\) ganze, von Null verschiedene Zahlen. Zeige, daß es nur endlich viele Polynome
    \(g(X)\) mit ganzzahligen Koeffizienten und \(\deg g(X) \leq m\) gibt, so daß für alle
    \(i \in \{0, \dots, m\}\) die ganze Zahl \(g(i)\) ein Teiler von \(b_i\) ist.
\end{exercise}

\begin{exercise}(6 Punkte)\newline
    Sei \(f(X)\) ein primitives Polynom vom Grade \(n\) mit ganzzahligen Koeffizienten. Sei
    \(f(i) \neq 0\) für alle ganzen Zahlen \(i\) mit \(0 \leq i \leq \frac n 2\).
    
    Sei
    \begin{equation}
        f(X) = g(X) \cdot h(X)
        \label{eq:kronecker}
    \end{equation}
    eine Faktorisierung, wobei \(g(X)\) und \(h(X)\) Polynome mit ganzzahligen Koeffizienten sind.
    Ohne Einschränkung nehmen
    wir an, daß \(\deg g(X) \leq \frac n 2\) (ansonsten vertauschen wir beide Faktoren).
    
    Überlege, daß für alle ganzen Zahlen \(i\) mit \(0 \leq i \leq \frac n 2\) die ganze Zahl \(g(i)\)
    ein Teiler von \(f(i)\) ist. Folgere daraus, daß nur endlich viele ganzzahlige Polynome \(g(X)\)
    mit~\eqref{eq:kronecker} existieren können. Leite daraus ein Verfahren festzustellen ab, ob \(f(X)\) über den
    ganzen Zahlen irreduzibel ist oder nicht.
    
    Wie läßt sich dieses Verfahren auf alle primitiven Polynome vom Grade \(n\) mit ganzzahligen Koeffizienten
    ausdehnen?
    
    (Dieses Verfahren ist zuerst von Leopold Kronecker\footnote{Leopold Kronecker, 1823--1891, deutscher
    Mathematiker} angegeben worden, der zuerst auf die Notwendigkeit eines
    Verfahrens hinwies, die Irreduzibilität eines Polynoms mit ganzzahligen Koeffizienten festzustellen.)
\end{exercise}






\end{document} 