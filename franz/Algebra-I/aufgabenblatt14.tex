\documentclass[a4paper,10pt]{algsheet}
    \usepackage{amssymb} 
\usepackage[protrusion=true,expansion=false]{microtype}

%%%%%%%%%%%%%%%%%%%%%%%%%%%%%%%%%%%%%%%%%%%%%%%%%%%%%%%%%%%%%%%%%%%%%%%%%%%%%%%%%%%%%%%%%%%%%%%%%%%%%%%%%%%%%%%%%%%%%%%%%%%%%%%%%%%%%%%%%%%%%%%%%%%%%%%%%%%%

   \lecture{Algebra I\\ \small{Einführung in die Algebra}}
        
        \semester{Wintersemester 2010/2011}
        \sheet{14.\ Aufgabenblatt}
        \author{Dipl.-Math.~Franz Vogler}
        \date{01.~Februar 2011}
        
        %/usr/local/share/texmf/tex/algmacros.sty
    \usepackage[arrow,curve,matrix]{xy}    
        \begin{document}
                \maketitle



\begin{exercise}(5 Punkte)\newline
    Bestimme explizit die Symmetriegruppe eines ebenen regelmäßigen \(n\)-Ecks
    im Raume, die sogenannte \emph{Dieder-Gruppe \(\mathrm D_n\)}. Zeige, daß diese
    von zwei Elementen erzeugt werden kann und insgesamt \(2n\) Elemente besitzt.
\end{exercise}


\begin{exercise}(10 Punkte)\newline
    Bestimme explizit alle Untergruppen der Diedergruppe \(\mathrm D_4\)
    sowie der \(\mathrm D_6\) und veranschauliche
    sie nach Möglichkeit am Quadrat bzw. regelmäßigen 6--Eck im Raume.
\end{exercise}


\begin{exercise}(10 Punkte)\newline
    Bestimme die Galoissche Gruppe der Nullstellen
    \(x_1\), \dots, \(x_4\)
    des Polynoms \(X^4 + 1\) über den rationalen Zahlen, ihre Untergruppen
    und die diesen Untergruppen entsprechenden
    Zwischenerweiterungen von \(\mathbb Q(x_1, \dotsc, x_4)\) über \(\mathbb Q\).
\end{exercise}


%%%%%%%%%%%%%%%%%%%%%%%%%%%%%%%%%%%%%%%%%%%%%%%%%%%%%%%%%%%%%%%%%%%%%%%%%%%


\begin{exercise}(5 Punkte + 20 Bonuspunkte)\newline
    Im folgenden kürzen wir einen Variablensatz der Form \(X_1, \dotsc, X_n\)
    mit \(\underline X\) ab. Entsprechend steht zum Beispiel
    \(\underline Y\) für \(Y_1, \dotsc, Y_n\) oder \(\underline X_i\) für
    \(X_{1i}, \dotsc, X_{ni}\). 
    Wir nennen ein Polynom
    \begin{align*}
        f(\underline X_1, \dotsc, \underline X_m)
    \end{align*}
    mit ganzzahligen Koeffizienten
    \emph{symmetrisch} in \(\underline X_1\), \dots, \(\underline X_m\), falls für
    jede
    \(m\)-stellige Permutation \(\sigma\) gilt, daß
    \begin{align*}
        \sigma \cdot f(\underline X_1, \dotsc, \underline X_m)
        \coloneqq f(\underline X_{\sigma(1)}, \dotsc, \underline X_{\sigma(m)})
        = f(\underline X_1, \dotsc, \underline X_m).
    \end{align*}
    Seien die Polynome
    \begin{align*}
        e_\lambda(\underline X_1, \dotsc, \underline X_m) \in \mathbb Z[\underline X_1,
        \dotsc, \underline X_m]
    \end{align*}
    diejenigen Polynome, so daß die Gleichheit
    \begin{equation}
        \sum_{k_1 + \dotsb + k_n \leq m} 
        e_{(k_1, \dotsc, k_n)}(\underline X_1, \dotsc, \underline X_m) \cdot T_1^{k_1}
        \dotsm T_n^{k_n} = \prod_{j = 1}^m (1 + X_{1j} T_1 + \dotsb + X_{nj} T_n)
    \end{equation}
    in \(\mathbb     Z[\underline X_1, \dotsc, \underline X_m, \underline T]\) gilt,
    wobei die Summe über alle Tupel \(\lambda = (k_1, \dotsc, k_n)\)
    natürlicher Zahlen mit \(k_1 + \dotsc + k_n \leq m\) geht.  Diese Polynome
    heißen die \emph{Mac~Mahonschen\footnote{Percy Alexander Mac~Mahon, 1854--1929,
    britischer Mathematiker} verallgemeinerten symmetrischen Funktionen}.

\begin{itemize}
 \item [\textbf{(1)}] Bestimme 
    $e_{(l,0 \dotsc, 0)}(\underline X_1, \dotsc, \underline X_m)$. 
\item [\textbf{(2)}]  Zeige, daß
    die \(e_\lambda(\underline X_1, \dotsc, \underline X_m)\)
    jeweils symmetrisch in \(\underline X_1\), \dots, \(\underline X_m\) sind.
\item [\textbf{(3)}]
   Sei \(g(\underline X_1, \dotsc, \underline X_m)\) ein in \(\underline X_1,
    \dotsc, \underline X_m\) symmetrisches Polynom mit ganzzahligen (oder rationalen
    oder algebraischen) Koeffizienten. Zeige analog zum Hauptsatz über die
    elementarsymmetrischen Funktionen, daß \(g(\underline X_1, \dotsc, \underline
    X_m)\) dann als Polynom in den Mac~Mahonschen verallgemeinerten symmetrischen
    Funktionen \newline 
    \(e_\lambda(\underline X_1, \dotsc, \underline X_m)\) mit ganzzahligen
    (oder rationalen oder algebraischen) Koeffizienten geschrieben werden kann.
\item [\textbf{(4)}] 
    Sei \(K\) ein Koeffizientenbereich.
    Seien \(x_1\), \dots, \(x_n\) die Nullstellen eines separablen Polynoms \(f(X)\)
    über \(K\). Sei \(H = \{\sigma_1, \dots, \sigma_m\}\) eine
    Untergruppe der Galoisschen Gruppe \(G\) von \(x_1\), \dots, \(x_n\) über \(K\).
    Zeige, daß dann
    \begin{align*}
        K(x_1, \dotsc, x_n)^H =
        K(e_\lambda(\sigma_1 \cdot \underline x, \dotsc, \sigma_m \cdot \underline x)),
    \end{align*}
    wobei \(\lambda\) über alle Tupel \((k_1, \dotsc, k_n)\) natürlicher Zahlen
    mit \(k_1 + \dotsb + k_n \leq m\) läuft.
\end{itemize}
\end{exercise}





\begin{exercise}(10 Punkte + 20 Bonuspunkte)\newline
Es sei das Polynom \(f=X^6 - 2 X^3 - 1\) über den rationalen Zahlen gegeben.
\begin{itemize}
 \item[$(\alpha)$] Zeige, dass $f$ irreduzibel über den rationalen Zahlen ist
                   und die Nullstellen durch 
         $y_1=x,\;y_2=\omega x,\;y_3=\omega^2 x,\;
             y_4=-\frac{1}{x},\;y_5=-\frac{\omega}{x},
             \;y_6=-\frac{\omega^2}{x}$ 
mit geeignetem $x\in\overline{\mathbb Q}$
 und $\omega^3=1$ gegeben sind. (Hinweis: Aufgabenblatt 10.)

\item[$(\beta)$] Begründe die Gleichheit $[\mathbb{Q}(x,\omega)\colon\mathbb Q]=12$  
 und folgere daraus, dass\newline 
  $\mathrm{Gal}_\mathbb Q(\underline y):=\mathrm{Gal}_\mathbb Q(y_1,y_2,y_3,y_4,y_5,y_6)$ von Ordnung 12 ist.

\item[$(\gamma)$] Wie wirken die zwölf Elemente aus $\mathrm{Gal}_\mathbb Q(\underline y)$ 
   auf die Nullstellen von $f$? Finde dazu zwei Elememte $d,s$ mit $d\colon\omega\mapsto\omega^2$, 
  $d\colon x\mapsto\frac{-\omega^2}{x}$ sowie $s\colon x\mapsto x$, $s\colon\omega\mapsto\omega^2$.

\item[$(\delta)$] Warum gilt $\mathbb{Q}(x,\omega)=\mathbb{Q}(x+\omega)$?

\item[$(\epsilon)$] Kannst Du damit die Struktur einer
             Dir bekannten Gruppe an $\mathrm{Gal}_\mathbb Q(\underline y)$ entdecken? Was sind
             die Untergruppen von $\mathrm{Gal}_\mathbb Q(\underline y)$?
             (Hinweis: Aufgabe 2.)
\item[$(\eta)$] Bestimme alle Zwischenerweiterungen $L_1,\dots,L_{16}$ von 
                $\mathbb Q(x,\omega)$ über $\mathbb Q$, welche nach dem Hauptsatz der
             Galoischen Theorie in Bijektion zu den von Dir in Teil $(\epsilon)$
             gefundenen Untergruppen von $\mathrm{Gal}_\mathbb Q(\underline y)$
             stehen. (Hinweis: Aufgabe 4 Teil \textbf{(4)}.)
\end{itemize}



\end{exercise}
    













\end{document} 