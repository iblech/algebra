\documentclass{algsheet}
    \usepackage{amssymb} 
\usepackage[protrusion=true,expansion=false]{microtype}

%%%%%%%%%%%%%%%%%%%%%%%%%%%%%%%%%%%%%%%%%%%%%%%%%%%%%%%%%%%%%%%%%%%%%%%%%%%%%%%%%%%%%%%%%%%%%%%%%%%%%%%%%%%%%%%%%%%%%%%%%%%%%%%%%%%%%%%%%%%%%%%%%%%%%%%%%%%%

   \lecture{Algebra I\\ \small{Einführung in die Algebra}}
        
        \semester{Wintersemester 2010/2011}
        \sheet{7.\ Aufgabenblatt}
        \author{Dipl.-Math.~Franz Vogler}
        \date{30.~November 2010}
        
        %/usr/local/share/texmf/tex/algmacros.sty
    \usepackage[arrow,curve,matrix]{xy}    
        \begin{document}
                \maketitle




\begin{exercise}(4 Punkte)\newline
    Seien
    \[
        f(X)  = X^3 - 2 X^2 + 2 X - 4\quad
        \text{und}\quad
     	g(X)  = X^2 - 3 X + 2 
    \]
    gegeben. Gib Polynome \(p(X)\) und \(q(X)\) mit
    \((X - 2) = p(X) \cdot f(X) + q(X) \cdot g(X)\) an.
\end{exercise}

\begin{exercise}(3 Punkte)\newline
    Gib einen Beweis dafür an, dass der euklidische Algorithmus für ganze Zahlen genauso funktioniert wie für Polynome (S. 71, Proposition 3.3 im Skript).
\end{exercise}

\begin{exercise}(2 Punkte)\newline
    Seien \(f(X)\) und \(g(X)\) zwei Polynome mit algebraischen Koeffizienten.
    Zeige, daß genau ein normiertes Polynom \(d(X)\) (der Fall \(d(X) = 0\)
    ist ausdrücklich zugelassen) existiert,
    welches ein größter gemeinsamer Teiler von
    \(f(X)\) und \(g(X)\) ist. 
\end{exercise}

\begin{exercise}(4 Punkte)\newline
    Seien \(f(X)\) und \(g(X)\) zwei Polynome mit algebraischen Koeffizienten.
    Definiere und konstruiere danach das \emph{kleinste gemeinsame Vielfache
    von \(f(X)\) und \(g(X)\)}.
\end{exercise}

\begin{exercise}(4 Punkte)\newline
    Sei \(f(X)\) ein normiertes Polynom über den rationalen Zahlen. Zeige, daß \(f\)
    genau dann separabel ist, wenn der größte
    gemeinsame Teiler von \(f(X)\) und \(f'(X)\) das konstante Polynom \(1\) ist.
\end{exercise}

\begin{exercise}(4 Punkte)\newline
    Gib eine normierte Polynomgleichung minimalen Grades über den rationalen
    Zahlen an, welche dieselben Lösungen
    (ohne Vielfachheiten) wie die Gleichung
   \[
        X^7 + X^6 - 4 X^4 - 4 X^3 + 4 X + 4 = 0
    \]
    besitzt.
\end{exercise}

\begin{exercise}(4 Punkte)\newline
    Sei \(a\) eine algebraische Zahl. Sei
    \(f(X) = X^3 + 2 a^2 X - a + 5\). Konstruiere eine Polynomgleichung mit
    rationalen Koeffizienten, welche von \(a\) genau dann erfüllt wird, wenn
    \(f(X)\) kein separables Polynom ist.
\end{exercise}



\begin{exercise}(2 Punkte)\newline
    Zeige, daß das Polynom
    \[
        f(X) = X^3 - \frac 3 2 X^2 + X - \frac 6 5
    \]
    keine rationale Nullstelle besitzt.
\end{exercise}

\begin{exercise}(4 Punkte)\newline
    Seien \(f(X)\) und \(g(X)\) zwei normierte Polynome mit rationalen 
    Koeffizienten.
    Gib ein Verfahren für die Berechnung des größten gemeinsamen Teilers von
    \(f(X)\) und \(g(X)\) über die Zerlegung von \(f(X)\) und \(g(X)\) in
    irreduzible Polynome an.
\end{exercise}

\begin{exercise}(2 Punkte)\newline
    Sei \(f(X) \neq 1\) ein normiertes Polynom über den rationalen Zahlen, welches prim ist, das heißt:
    Teilt \(f(X)\) ein Produkt \(g(X) \cdot h(X)\) von Polynomen \(g(X)\) und \(h(X)\) mit rationalen
    Koeffizienten, so teilt \(f(X)\) mindestens einen der Faktoren \(g(X)\) und \(h(X)\). Zeige, daß
    \(f(X)\) ein irreduzibles Polynom ist.
\end{exercise}

\begin{exercise}(3 Punkte)\newline
    Sei \(f(X)\) ein normiertes irreduzibles Polynom mit rationalen
    Koeffizienten. Seien \(g_1(X)\), \dots, \(g_n(X)\) weitere Polynome mit rationalen Koeffizienten.
    Sei \(f(X)\) ein Teiler des Produktes \(g_1(X) \dotsm g_n(X)\). Zeige, daß dann ein
    \(i \in \{1, \dotsc, n\}\) existiert, so daß \(f(X)\) ein Teiler von \(g_i(X)\) ist.
\end{exercise}




\begin{exercise}\textbf{Quickies:} (1+1+1+2 Punkte)\newline Die folgenden Aufgaben können ganz kurz in ein bis zwei Zeilen gelöst werden.  
\begin{itemize}
 \item[\textbf{(Q1)}]%(1 Punkt)\newline
        Ist ein normiertes Polynom vom Grad \(1\) und mit rationalen
    Koeffizienten immer irreduzibel?
  \item[\textbf{(Q2)}]%(1 Punkt)\newline
                      Seien \(x\) und \(y\) zwei algebraische Zahlen mit \(x \cdot y = 0\).
    Zeige, daß dann \(x = 0\) oder \(y = 0\) (oder beides).

\item[\textbf{(Q3)}]%(1 Punkt)\newline
           Zeige, daß normierte Polyome vom Grad \(2\) und \(3\) über den rationalen
    Zahlen genau dann irreduzibel sind, wenn sie keine rationale Nullstelle
    besitzen.    

\item[\textbf{(Q4)}]%(2 Punkte)\newline
       Zeige, daß sich die Bewertung wie ein Logarithmus verhält. Damit ist folgendes
    gemeint: Sei \(p(X)\) ein irreduzibles normiertes Polynom über den rationalen
    Zahlen. Dann gelten
    \begin{align}
        \mathrm{ord}_{p(X)} 1 & = 0\\
        \shortintertext{und}
        \mathrm{ord}_{p(X)} (f(X) \cdot g(X)) & = \mathrm{ord}_{p(X)} f(X) + \mathrm{ord}_{p(X)} g(X)
    \end{align}
    für beliebige normierte Polynome \(f(X)\) und \(g(X)\).
\end{itemize}


   
\end{exercise}





\end{document} 
