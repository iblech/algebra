\documentclass{algsheet}
    \usepackage{amssymb} 
\usepackage[protrusion=true,expansion=false]{microtype}

%%%%%%%%%%%%%%%%%%%%%%%%%%%%%%%%%%%%%%%%%%%%%%%%%%%%%%%%%%%%%%%%%%%%%%%%%%%%%%%%%%%%%%%%%%%%%%%%%%%%%%%%%%%%%%%%%%%%%%%%%%%%%%%%%%%%%%%%%%%%%%%%%%%%%%%%%%%%

   \lecture{Algebra I\\ \small{Einführung in die Algebra}}
        
        \semester{Wintersemester 2010/2011}
        \sheet{Globalübung}
        \author{Dipl.-Math.~Arturo Mancino}
        \date{12.~Januar 2011}
        
        %/usr/local/share/texmf/tex/algmacros.sty
    \usepackage[arrow,curve,matrix]{xy}    
        \begin{document}
                \maketitle



\begin{exercise}
    Sei \(n\) eine natürliche Zahl. Gib ein Verfahren an, alle \(n\)-stelligen Permutationen
    aufzulisten und zeige, daß es davon insgesamt \(n!\) Stück gibt. 
\end{exercise}

\begin{exercise}
    Sei \(n\) eine natürliche Zahl. Sei \(\sigma\) diejenige \(n\)-stellige Permutation, welche
    \(1\) auf \(2\), \(2\) auf \(3\), \dots und schließlich \(n - 1\) auf \(n\) und
    \(n\) auf \(1\) abbildet. Berechne ihr Signum in Abhängigkeit von \(n\).
\end{exercise}

\begin{exercise}
    Berechne die Symmetriegruppe eines regelmäßigen Fünfecks in der Ebene.
\end{exercise}

\begin{exercise}
    Berechne die Symmetriegruppe eines regelmäßigen \(n\)-Ecks in der Ebene.
\end{exercise}

\begin{exercise}
    Berechne die Symmetriegruppe des Oktaeders.
\end{exercise}

\begin{exercise}
    Gib alle möglichen Untergruppen der symmetrischen Gruppe \(\SG_3\) an.
\end{exercise}


\begin{exercise}
    Sei \(f(X) = g_1(X) \dotsm g_m(X)\), wobei \(f(X)\) und \(g_1(X)\), \dots, \(g_m(X)\) normierte 
    Polynome
    mit rationalen Koeffizienten sind. Sei \(f(X)\) separabel. Zeige, daß jede
    Symmetrie der Nullstellen von \(f(X)\) die Nullstellen der \(g_i(X)\) jeweils
    nur untereinander permutiert.
\end{exercise}


\begin{exercise}
    Sei \(f(X)\) ein irreduzibles normiertes Polynom dritten Grades über den rationalen Zahlen mit
    Nullstellen \(x_1\), \(x_2\) und \(x_3\). Sei \(x_1\) ein primitives Element zu \(x_1\), \(x_2\)
    und \(x_3\). Zeige, daß die Galoissche Gruppe zu den Nullstellen \(x_1\), \(x_2\) und \(x_3\) von
    \(f(X)\) genau drei Elemente hat. Gib diese Elemente an.
\end{exercise}



\begin{exercise}
    Seien \(x_1\), \dots, \(x_n\) die Nullstellen eines normierten separablen Polynoms \(f(X)\) mit
    rationalen Koeffizienten. Seien
    \(y_1\), \dots, \(y_n\) die Nullstellen desselben Polynoms in (möglicherweise) anderer Anordnung.
    Wie läßt sich \(\mathrm{Gal}_{\set Q}(y_1, \dotsc, y_n)\) durch \(\mathrm{Gal}_{\set Q}(x_1, \dotsc, x_n)\) beschreiben?
\end{exercise}


\begin{exercise}
    Sei \(f(X)\) ein normiertes separables Polynom mit \(n\) Nullstellen. Zeige, daß es immer gelingt, eine
    Galoissche
    Resolvente \(V(X_1, \dotsc, X_n)\) von \(f(X)\) zu finden, die von \(X_1\) nicht abhängt, also schon ein
    Polynom in \(X_2\), \dots, \(X_n\) ist.
\end{exercise}


\begin{exercise}
    Berechne eine Galoissche Gruppe des Polynoms \(f(X) = X^3 - 3 X - 4\) über den rationalen
    Zahlen mit Hilfe der Methode der Galoisschen Resolvente.
\end{exercise}


\end{document}
 