\documentclass{algsheet}
    \usepackage{amssymb} 
\usepackage[protrusion=true,expansion=false]{microtype}

%%%%%%%%%%%%%%%%%%%%%%%%%%%%%%%%%%%%%%%%%%%%%%%%%%%%%%%%%%%%%%%%%%%%%%%%%%%%%%%%%%%%%%%%%%%%%%%%%%%%%%%%%%%%%%%%%%%%%%%%%%%%%%%%%%%%%%%%%%%%%%%%%%%%%%%%%%%%

   \lecture{Algebra I\\ \small{Einführung in die Algebra}}
        
        \semester{Wintersemester 2010/2011}
        \sheet{Globalübung}
        \author{Dipl.-Math.~Arturo Mancino}
        \date{26.~Januar 2011}
        
        %/usr/local/share/texmf/tex/algmacros.sty
    \usepackage[arrow,curve,matrix]{xy}    
        \begin{document}
                \maketitle



\begin{exercise}
    Sei \(x\) eine algebraische Zahl, deren galoissch Konjugierte
    durch \(x_1 = x\), \(x_2\), \dots, \(x_n\) gegeben sind. Zeige, daß
    \(x\) genau dann konstruierbar ist, wenn der Grad eines zu
    \(x_1\), \dots, \(x_n\) primitiven Elementes über den rationalen Zahlen
    durch eine Zweierpotenz gegeben ist.
\end{exercise}


\end{document}
 