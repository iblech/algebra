\documentclass{algsheet}
    \usepackage{amssymb} 
\usepackage[protrusion=true,expansion=false]{microtype}

%%%%%%%%%%%%%%%%%%%%%%%%%%%%%%%%%%%%%%%%%%%%%%%%%%%%%%%%%%%%%%%%%%%%%%%%%%%%%%%%%%%%%%%%%%%%%%%%%%%%%%%%%%%%%%%%%%%%%%%%%%%%%%%%%%%%%%%%%%%%%%%%%%%%%%%%%%%%

   \lecture{Algebra I\\ \small{Einführung in die Algebra}}
        
        \semester{Wintersemester 2010/2011}
        \sheet{Globalübung}
        \author{Dipl.-Math.~Arturo Mancino}
        \date{17.~November 2010}
        
        %/usr/local/share/texmf/tex/algmacros.sty
    \usepackage[arrow,curve,matrix]{xy}    
        \begin{document}
                \maketitle



\begin{exercise}(x Punkte)\newline
    Sei \(f(X)\) ein Polynom mit algebraischen Koeffizienten mit \(\deg f \leq n\)
    für eine natürliche Zahl \(n\). Seien \(x_0\), \dots, \(x_n\) paarweise verschiedene algebraische Zahlen
    mit \(f(x_0) = \dotsb = f(x_n) = 0\). Zeige, daß \(f(X)\) das Nullpolynom ist.
\end{exercise}

\begin{exercise}(x Punkte)\newline
    Seien \(f(X)\) und \(g(X)\) zwei Polynome mit algebraischen Koeffizienten und
    \(\deg f\), \(\deg g \leq n\) für eine natürliche Zahl \(n\). Seien \(x_0\), \dots,
    \(x_n\) paarweise verschiedene algebraische Zahlen mit \(f(x_0) = g(x_0)\), \dots, \(f(x_n) = g(x_n)\). Zeige, 
    daß dann \(f(X) = g(X)\) gilt.
\end{exercise}

\begin{exercise}(x Punkte)\newline
    Seien \(x_0\), \dots, \(x_n\) paarweise verschiedene algebraische Zahlen. Sei
    \(i \in \{0, \dots, n\}\). Zeige, daß genau ein Polynom \(f(X)\) mit algebraischen
    Koeffizienten und
    \[
        f(x_j) = \begin{cases} 1 & \text{für \(j = i\) und}\\ 0
                                 & \text{für \(j \in \{0, \dotsc, i - 1, i + 1, \dotsc, n\}\)} \end{cases}
    \]
    existiert.
\end{exercise}

\begin{exercise}(x Punkte)\newline
    Seien \(x_0\), \dots, \(x_n\) algebraische Zahlen. Seien \(y_0\),
    \dots, \(y_n\) weitere algebraische Zahlen. Zeige, daß genau ein Polynom \(f(X)\) mit algebraischen
    Koeffizienten und \(f(x_i) = y_i\) für \(i \in \{0, \dots, n\}\) existiert.
\end{exercise}






\end{document} 