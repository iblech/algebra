\documentclass{algsheet}
    \usepackage{amssymb} 
\usepackage[protrusion=true,expansion=false]{microtype}

%%%%%%%%%%%%%%%%%%%%%%%%%%%%%%%%%%%%%%%%%%%%%%%%%%%%%%%%%%%%%%%%%%%%%%%%%%%%%%%%%%%%%%%%%%%%%%%%%%%%%%%%%%%%%%%%%%%%%%%%%%%%%%%%%%%%%%%%%%%%%%%%%%%%%%%%%%%%

   \lecture{Algebra I\\ \small{Einführung in die Algebra}}
        
        \semester{Wintersemester 2010/2011}
        \sheet{10.\ Aufgabenblatt}
        \author{Dipl.-Math.~Franz Vogler}
        \date{21.~Dezember 2010}
        
        %/usr/local/share/texmf/tex/algmacros.sty
    \usepackage[arrow,curve,matrix]{xy}    
        \begin{document}
                \maketitle



Euer Algebra-I--Team wünscht Euch allen ein gesegnetes Weihnachtsfest und ein gesundes und erfolgreiches 
neues Jahr. 

\begin{exercise}(1 Punkt)\newline
    Sei \(x\) eine algebraische Zahl, sei \(y\) eine algebraische Zahl, welche in
    \(x\) rational ist, und sei \(z\) eine algebraische Zahl, welche in \(y\)
    rational ist. Wie läßt sich der Grad von \(x\) über \(z\) aus dem Grad von
    \(x\) über \(y\) und dem Grad von \(y\) über \(z\) berechnen?
\end{exercise}

\begin{exercise}(2 Punkte)\newline
    Sei \(x\) eine algebraische Zahl, sei \(y\) eine algebraische Zahl, welche in
    \(x\) rational ist, und sei \(z\) eine algebraische Zahl, welche in \(y\)
    rational ist. Zeige, daß der Grad von \(y\) über \(z\) ein Teiler des Grades
    von \(x\) über \(z\) ist.
\end{exercise}

\begin{exercise}(2 Punkte)\newline
    Seien \(x\), \(y\) und \(z\) drei algebraische Zahlen, so daß
    sowohl \(x\) als auch \(y\) in \(z\) rational sind. Zeige, daß
    \begin{equation}
        [\set Q(z) : \set Q(x)] \cdot [\set Q(x) : \set Q]
        = [\set Q(z) : \set Q(y)] \cdot [\set Q(y) : \set Q],
    \end{equation}
    und gib ein Diagramm zur Veranschaulichung an.
\end{exercise}

\begin{exercise}(2 Punkte)\newline
    Sei \(x\) eine algebraische Zahl. Sei \(f(X)\) ein nichtlineares Polynom mit rationalen
    Koeffizienten, welches irreduzibel über den rationalen Zahlen ist. Zeige unter
    der Annahme, daß der Grad von \(f(X)\) teilerfremd zum Grad von \(x\) über
    den rationalen Zahlen ist, daß keine in \(x\) rationale Zahl Nullstelle von
    \(f(X)\) ist.
\end{exercise}

\begin{exercise}(3 Punkte)\newline
    Sei \(x\) eine algebraische Zahl. Sei \(y\) eine algebraische Zahl, welche
    in \(x\) rational ist. Sei \(f(X)\) ein Polynom über \(y\) vom Grad größer eins, welches über
    \(y\) irreduzibel ist. Zeige unter der Annahme, daß der Grad von \(f(X)\)
    teilerfremd zum Grad von \(x\) über \(y\) ist, daß keine in \(x\) rationale
    Zahl Nullstelle von \(f(X)\) ist.
\end{exercise}

\begin{exercise}(4 Punkte)\newline
    Ist die folgende Aussage richtig oder falsch?
    
    Seien \(x\) und \(y\) algebraische Zahlen, und sei \(z\) ein
    primitives Element zu \(x\) und \(y\). Dann ist der Grad von \(z\) über den
    rationalen Zahlen ein Teiler des Produktes der Grade von \(x\) und \(y\) über den
    rationalen Zahlen.
\end{exercise}





\begin{exercise}(1 Punkt)\newline
    Gib zwei algebraische Zahlen an, die nicht zueinander galoissch konjugiert sind.
\end{exercise}

\begin{exercise}(2 Punkte)\newline
    Sei \(t\) eine algebraische Zahl. Begründe warum das Produkt von \(t\) mit allen seinen
    galoissch Konjugierten eine rationale Zahl ist. Wie ist es mit der Summe?
\end{exercise}



\begin{exercise}(3 Punkte)\newline
    Seien \(x\), \(y\) und \(z\) drei algebraische Zahlen. Seien \(x\) galoissch konjugiert zu
    \(y\) und \(y\) galoissch konjugiert zu \(z\). Zeige, daß \(x\) galoissch konjugiert zu
    \(z\) ist.
\end{exercise}

\begin{exercise}(6 Punkte)\newline
    Wieviele galoissch Konjugierte hat \(x_1 = \sqrt[3]{1 + \sqrt 2}\)?
\end{exercise}

\begin{exercise}(4 Punkte)\newline
    Seien \(p\) und \(q\) zwei verschiedene Primzahlen. Berechne die galoissch Konjugierten von
    \(\sqrt p + \sqrt q\).
\end{exercise}

\begin{exercise}(3 Punkte)\newline
    Zeige, daß Proposition 4.2 auf Seite 116 die galoissch Konjugierten einer algebraischen
    Zahl \(t\) charakterisiert, das heißt, eine algebraische Zahl \(t'\) ist genau ein
    galoissch Konjugiertes zu \(t\) (der Fall \(t' = t\) ist hier eingeschlossen), wenn
    jedes Polynom \(f(X)\) mit rationalen Koeffizienten, welches \(t\) als Nullstelle hat,
    auch \(t'\) als Nullstelle hat.
\end{exercise}  

\begin{exercise}(3 Punkte)\newline
    Sei \(f(X)\) ein Polynom mit rationalen Koeffizienten. Sei \(x \coloneqq f(t)\), wobei
    \(t\) eine algebraische Zahl ist. Seien \(t'\) eine weitere algebraische Zahl und
    \(x' \coloneqq f(t')\). Zeigen Sie, daß \(x\) und \(x'\) galoissch konjugiert sind,
    wenn \(t\) und \(t'\) galoissch konjugiert sind.
\end{exercise}

\begin{exercise}(2 Punkte)\newline
    Zeige an einem Beispiel, daß Hilfssatz 4.3 auf Seite 116 falsch wird, wenn \(x_1\), \dots,
    \(x_n\) nicht als Lösungen (mit Vielfachheiten) einer Polynomgleichung mit rationalen
    Koeffizienten vorausgesetzt werden, sondern beliebige algebraische Zahlen sein können.
\end{exercise}


\begin{exercise}(2 Punkte)\newline
    Zeige an einem Beispiel, daß Proposition 4.4 auf Seite 117 falsch wird, wenn \(x_1\), \dots,
    \(x_n\) nicht als Lösungen (mit Vielfachheiten) einer Polynomgleichung mit rationalen
    Koeffizienten vorausgesetzt werden, sondern beliebige algebraische Zahlen sein können.
\end{exercise}







\end{document} 