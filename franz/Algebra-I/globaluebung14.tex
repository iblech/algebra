\documentclass{algsheet}
    \usepackage{amssymb} 
\usepackage[protrusion=true,expansion=false]{microtype}

%%%%%%%%%%%%%%%%%%%%%%%%%%%%%%%%%%%%%%%%%%%%%%%%%%%%%%%%%%%%%%%%%%%%%%%%%%%%%%%%%%%%%%%%%%%%%%%%%%%%%%%%%%%%%%%%%%%%%%%%%%%%%%%%%%%%%%%%%%%%%%%%%%%%%%%%%%%%

   \lecture{Algebra I\\ \small{Einführung in die Algebra}}
        
        \semester{Wintersemester 2010/2011}
        \sheet{Globalübung}
        \author{Dipl.-Math.~Arturo Mancino}
        \date{02.~Februar 2011}
        
        %/usr/local/share/texmf/tex/algmacros.sty
    \usepackage[arrow,curve,matrix]{xy}    
        \begin{document}
                \maketitle


\begin{exercise}(5 Punkte)\newline
    Seien \(x_1\), \dots, \(x_n\) die Nullstellen eines separablen Polynoms
    mit rationalen Koeffizienten und \(G\) ihre Galoissche Gruppe. Seien
    \(\sigma\) und \(\tau\) zwei Symmetrien in \(G\). Sei weiter \(t\) eine
    in \(x_1\), \dots, \(x_n\) rationale Zahl, welche unter \(\sigma\) und \(\tau\)
    invariant ist. Zeige, daß \(t\) auch unter \(\id\), \(\tau \circ \sigma\)
    und \(\sigma^{-1}\) invariant ist. 
\end{exercise}


\begin{exercise}
    Gib einen Koeffizientenbereich an, in dem \(7\) keine algebraisch eindeutige \(5\)-te Wurzel besitzt
    und in dem \(X^5 - 7\) nicht in Linearfaktoren zerfällt.
\end{exercise}

\begin{exercise}
    Gib einen Koeffizientenbereich an, in dem \(7\) keine algebraisch eindeutige \(5\)-te Wurzel besitzt
    und in dem \(X^5 - 7\) in Linearfaktoren zerfällt.
\end{exercise}



\begin{exercise}(x Punkte)\newline
    Gib ein Beispiel dafür an, daß Hilfssatz 5.17 im Skript falsch wird, wenn wir nicht annehmen,
    daß \(p\) eine Primzahl ist.
\end{exercise}




\end{document}
 