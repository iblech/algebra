\documentclass{algsheet}
    \usepackage{amssymb} 
\usepackage[protrusion=true,expansion=false]{microtype}

%%%%%%%%%%%%%%%%%%%%%%%%%%%%%%%%%%%%%%%%%%%%%%%%%%%%%%%%%%%%%%%%%%%%%%%%%%%%%%%%%%%%%%%%%%%%%%%%%%%%%%%%%%%%%%%%%%%%%%%%%%%%%%%%%%%%%%%%%%%%%%%%%%%%%%%%%%%%

   \lecture{Algebra I\\ \small{Einführung in die Algebra}}
        
        \semester{Wintersemester 2010/2011}
        \sheet{15.\ Aufgabenblatt}
        \author{Dipl.-Math.~Franz Vogler}
        \date{08.~Februar 2011}
        
        %/usr/local/share/texmf/tex/algmacros.sty
    \usepackage[arrow,curve,matrix]{xy}    
        \begin{document}
                \maketitle


Wichtige Termine in der vorlesungsfreien Zeit:
 \newline
Die \emph{Abgabe dieses Aufgabenblattes} kann bis Montag, den 07.03.2011, um 12:00 Uhr im Briefkasten 
\glqq Algebra 1\grqq \; erfolgen. Die Rückgabe der Abgaben und deren Besprechung erfolgt am darauf folgenden Freitag 
im Klausurenkurs.
\newline 
Der \textbf{Klausurenkurs} findet täglich von Montag, den 07.03.2011, bis Freitag, den 11.03.2011, 
jeweils von 08:15 bis 17:15 Uhr im Hörsaal 1004/T statt. Zusätzlich stehen die Seminarräume 2001/T
und 2002/T zur Verfügung. Die Leitung des Klausurenkurses obliegt Herrn Arturo Mancino. Eine Anmeldung zum Klausurenkurs ist
nicht erforderlich. 
\newline
Die \textbf{Klausur} (zu welcher eine Anmeldung nötig ist) zur Algebra I von Professor Nieper-Wißkirchen findet am Samstag, den 
19.03.2011, in den Hörsälen HS I und HS II im Gebäude C statt. Die Prüfungszeit zur Bearbeitung der Klausur  
erstreckt sich von 09:00 bis 12:00 Uhr. Als Hilfsmittel sind ausschließlich Stift und Papier, das Vorlesungsskript sowie die 
Vorlesungsmitschrift zugelassen. 



\begin{exercise}(4 Punkte)\newline
    Seien \(x_1\), \dots, \(x_4\) wie bei der Herleitung der Formel (5.7) im Skript.
    Zeige, daß \(x_2^2 = 5\) und \(x_3^4 = -20 \sqrt{-1} - 15\) gilt. Warum ist es problematisch,
    eine fünfte primitive Einheitswurzel durch den Wurzelausdruck
    \begin{equation}
        \frac 1 4 \left(-1 + \sqrt[4]{20 \sqrt{-1} - 15} + \sqrt{5} + \sqrt[4]{-20 \sqrt{-1} - 15}\right)
    \end{equation}
    anzugeben?
\end{exercise}


\begin{exercise}(4 Punkte)\newline
    Sei \(x\) durch Wurzeln ausdrückbar. Sei \(x'\) galoissch konjugiert zu \(x\).
    Zeige, daß \(x'\) ebenfalls durch Wurzeln ausdrückbar ist, und zwar durch
    denselben Wurzelausdruck wie \(x\).
\end{exercise}

\begin{exercise}(4 Punkte)\newline 
    Sei \(x\) eine algebraische Zahl, deren galoissch Konjugierte
    durch \(x_1 = x\), \(x_2\), \dots, \(x_n\) gegeben sind. Zeige, daß
    \(x\) genau dann konstruierbar ist, wenn der Grad eines zu
    \(x_1\), \dots, \(x_n\) primitiven Elementes über den rationalen Zahlen
    durch eine Zweierpotenz gegeben ist.
\end{exercise}

\begin{exercise}(4 Punkte)\newline 
    Sei \(n\) eine positive natürliche Zahl. Zeige, daß eine primitive \(n\)-te Einheitswurzel
    durch Wurzeln ausgedrückt werden kann, deren Exponenten höchstens das Maximum von \(2\) und
    \(\frac{n - 1}2\) sind.
\end{exercise}


\begin{exercise}(2 Punkte)\newline
    Gib ein Beispiel für ein normiertes separables Polynom \(f(X)\) über den
    rationalen Zahlen an, so daß die Galoissche Gruppe der Nullstellen von \(f(X)\)
    über \(\mathbb Q(\sqrt[3]{2})\) gleich der Galoisschen Gruppe der Nullstellen
    von \(f(X)\) über den rationalen Zahlen ist.
\end{exercise}

\begin{exercise}(3 Punkte)\newline
    Gib ein Beispiel für ein normiertes separables Polynom \(f(X)\) über den
    rationalen Zahlen an, so daß die Galoissche Gruppe der Nullstellen von \(f(X)\)
    über \(\mathbb Q(\sqrt[3]{2})\) eine Untergruppe vom Index \(3\)
    in der Galoisschen Gruppe der Nullstellen von \(f(X)\) über den rationalen Zahlen ist. Ist diese 
    Untergruppe auch ein Normalteiler?
\end{exercise}


\begin{exercise}(3 Punkte)\newline
    Zeige, daß die Dieder-Gruppe \(\mathrm D_4\) eine Untergruppe, aber kein Normalteiler vom Index \(3\)
    in der symmetrischen Gruppe \(\mathrm S_4\) ist.
\end{exercise}



\begin{exercise}(3 Punkte)\newline
    Zeige, daß die Gleichung \(X^5 - 23X + 1 = 0\) über den rationalen Zahlen
    nicht auflösbar ist.
\end{exercise}



\begin{exercise}(3 Punkte)\newline
    Sei \(f(X)\) ein normiertes separables Polynom mit rationalen Zahlen, welches mindestens
    eine nicht reelle Nullstelle besitzt. Zeige, daß die Galoissche Gruppe 
    der Nullstellen \(x_1\), \dots, \(x_n\) von \(f(X)\) mindestens ein
    Element der Ordnung \(2\) besitzt.
\end{exercise}



\begin{exercise}(2+2+2+2+2 Punkte) \textbf{Quickies}:
   \begin{itemize}
    \item[(a)]     Zeige, daß jede nicht triviale Gruppe mindestens zwei Normalteiler besitzt.
     \item[(b)]     Zeige, daß das Zentrum einer Gruppe \(G\) ein Normalteiler in derselben ist.    
     \item[(c)]  Ist die symmetrische Gruppe \(\mathrm S_5\) in fünf Ziffern einfach?    
     \item[(d)]     Gib ein Beispiel für ein normiertes irreduzibles Polynom \(f(X)\) fünften Grades
        über den rationalen Zahlen an, so daß die Gleichung \(f(X) = 0\)
        auflösbar ist.
      \item[(e)]      Sei \(p\) eine Primzahl. Sei \(G\) eine Untergruppe der symmetrischen Gruppe
      \(\mathrm S_p\) in \(p\)-Ziffern. Enthalte \(G\) einen \(p\)-Zykel und eine Transposition.
      Zeige, daß \(G\) schon die volle symmetrische Gruppe \(\mathrm S_p\) ist.
    \end{itemize}
\end{exercise}



\end{document} 