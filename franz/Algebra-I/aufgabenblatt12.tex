\documentclass{algsheet}
    \usepackage{amssymb} 
\usepackage[protrusion=true,expansion=false]{microtype}

%%%%%%%%%%%%%%%%%%%%%%%%%%%%%%%%%%%%%%%%%%%%%%%%%%%%%%%%%%%%%%%%%%%%%%%%%%%%%%%%%%%%%%%%%%%%%%%%%%%%%%%%%%%%%%%%%%%%%%%%%%%%%%%%%%%%%%%%%%%%%%%%%%%%%%%%%%%%

   \lecture{Algebra I\\ \small{Einführung in die Algebra}}
        
        \semester{Wintersemester 2010/2011}
        \sheet{12.\ Aufgabenblatt}
        \author{Dipl.-Math.~Franz Vogler}
        \date{18.~Januar 2011}
        
        %/usr/local/share/texmf/tex/algmacros.sty
    \usepackage[arrow,curve,matrix]{xy}    
        \begin{document}
                \maketitle



\begin{exercise}(2 Punkte)\newline
    Sei \(G\) eine Gruppe. Ist \(i\) eine negative ganze Zahl, so setzen wir
    \(\sigma^{i} = (\sigma^{-1})^{-i}\). Außerdem ist \(\sigma^0 = \id\). Zeige, daß
    \(\sigma^i \circ \sigma^j = \sigma^{i + j}\) für zwei beliebige ganze Zahlen \(i\) und \(j\).
    
    Warum reicht es, diese Aussage für den Fall zu beweisen, daß \(G\) eine volle Permutationsgruppe ist?
\end{exercise}


\begin{exercise}(2 Punkte)\newline
    Sei \(G\) eine Gruppe. Sei \(H\) eine Untergruppe von \(G\), und sei \(K\) eine Untergruppe von \(H\).
    Warum ist \(K\) auch eine Untergruppe von \(G\)?
\end{exercise}



\begin{exercise}(1 Punkt)\newline
    Gibt es in der Permutationsgruppe \(\mathrm S_5\) eine Untergruppe mit \(80\) Elementen?    
\end{exercise}

\begin{exercise}(2 Punkte)\newline
    Sei \(f(X)\) eine normiertes separables Polynom über den rationalen Zahlen. Sei \(t\) ein
    primitives Element zu den Nullstellen (mit Vielfachheit) \(x_1\), \dots, \(x_n\) von \(f(X)\).
    Zeige, daß der Grad von \(t\) über den rationalen Zahlen ein Teiler von \(n!\) ist.
\end{exercise}



\begin{exercise}(3 + 3 Punkte)
   \begin{itemize}
    \item[\textbf{(a)}]     Sei \(G\) eine Gruppe. Sei \(\sigma \in G\) von Ordnung \(n\). Sei \(m\) eine ganze Zahl. Zeige, daß
    die Ordnung von \(\sigma^m\) durch \(\frac n d\) gegeben ist, wobei \(d\) der größte gemeinsame Teiler
    von \(n\) und \(m\) ist.

   \item[\textbf{(b)}]  Sei \(n\) eine natürliche Zahl. 
                Berechne die Ordnungen der Elemente der zyklischen Gruppe \(\mathrm C_n\).
   \end{itemize}
\end{exercise}



\begin{exercise}(2 Punkte)\newline
    Gib die Faktorisierung von \(X^3 + X^2 + X + 1\) in irreduzible Polynome über den rationalen Zahlen an.
\end{exercise}



\begin{exercise}(2 + 2 + 2 Punkte)
        \begin{itemize}
    \item[\textbf{(x)}]    
                   Berechne das Kreisteilungspolynom \(\Phi_3(X)\).
     \item[\textbf{(y)}] 
                  Berechne das Kreisteilungspolynom \(\Phi_6(X)\).
     \item[\textbf{(z)}]
                   Berechne das Kreisteilungspolynom \(\Phi_9(X)\).
        \end{itemize} 
\end{exercise}







\begin{exercise}(2 Punkte)\newline
    Sei \(p\) eine Primzahl. Zeige, daß der Binomialkoeffizient \(\binom{p^2} p\) durch \(p\), aber nicht
    durch \(p^2\) teilbar ist.
\end{exercise}



\begin{exercise}(5 Punkte)\newline
    Seien \(d\) und \(n\) natürliche Zahlen. Sei \(X = \{\upzeta_n, \upzeta_n^2, \dotsc, 
          \upzeta_n^{n - 1},
    \upzeta_n^n\}\) die Menge der \(n\)-ten Einheitswurzeln. Zeige, daß
    \[
        \sigma_d\colon X \to X,\quad \zeta \mapsto \zeta^d
    \]
    genau dann eine Bijektion von \(X\) ist, wenn \(d\) teilerfremd zu \(n\) ist.
\end{exercise}

\begin{exercise}(6 Punkte)\newline
    Gib alle primitiven Wurzeln modulo \(11\) an.
\end{exercise}

\begin{exercise}(4 Punkte)\newline
    Sei \(p\) eine Primzahl. Gib die Primfaktorzerlegung von \(X^{p - 1} - 1\) modulo \(p\) an.
\end{exercise}

\begin{exercise}(3 Punkte)\newline
    Sei \(n\) eine natürliche Zahl. Gib alle Erzeuger von \(\mathrm C_n\) an.
\end{exercise}




\end{document} 