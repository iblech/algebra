\documentclass{algsheet}
    \usepackage{amssymb} 


%%%%%%%%%%%%%%%%%%%%%%%%%%%%%%%%%%%%%%%%%%%%%%%%%%%%%%%%%%%%%%%%%%%%%%%%%%%%%%%%%%%%%%%%%%%%%%%%%%%%%%%%%%%%%%%%%%%%%%%%%%%%%%%%%%%%%%%%%%%%%%%%%%%%%%%%%%%%

   \lecture{Algebra I\\ \small{Einführung in die Algebra}}
        
        \semester{Wintersemester 2010/2011}
        \sheet{2.\ Aufgabenblatt}
        \author{Dipl.-Math.~Franz Vogler}
        \date{26.~Oktober 2010}
        
        %/usr/local/share/texmf/tex/algmacros.sty
    \usepackage[arrow,curve,matrix]{xy}    
        \begin{document}
                \maketitle
NEUER ABGABETERMIN:\newline
Die Lösungen zu den Aufgaben sind Dienstag, den 02. November \emph{bis} 09:00 Uhr in den \glqq Algebra 1\grqq-- Briefkasten
im Erdgeschoß  von Gebäude L einzuwerfen.

\begin{exercise}(4 Punkte)\newline
    Sei \(z\) eine Lösung der Polynomgleichung
    \[
        X^3 -  \sqrt{2 - \sqrt[3] 4} \, X^2 + 3 = 0.     
    \]
    Gib eine normierte Polynomgleichung mit rationalen Koeffizienten an,
    welche \(z\) als Lösung besitzt.
\end{exercise}

\begin{exercise}(2 Punkte)\newline
    Berechne die Inverse der komplexen Zahl $(4 + 3 \, \mathrm i)$.
\end{exercise}

\begin{exercise}(3 Punkte)\newline
    Sei \(z \neq 0\) eine komplexe Zahl, deren Real- und Imaginärteil rationale
    Zahlen sind. Zeige, daß \(z^{-1}\) ebenfalls rationalen Real- und Imaginärteil
    hat.
\end{exercise}

\begin{exercise}(2 Punkte)\newline
    Zeige, daß der Realteil einer komplexen Zahl \(z\) durch
    \(\frac 1 2 (z + \overline z)\)
    und daß der Imaginärteil durch \(\frac 1 {2 i} (z - \overline z)\)
    gegeben ist.
\end{exercise}

\begin{exercise}(3 Punkte)\newline
    Zeige formal die zuerst von Rafael Bombelli\footnote{Rafael Bombelli,
        1526--1572, italienischer Mathematiker} gefundene
    Gleichheit
    \[
        (2 \pm \sqrt{-1})^3 = 2 \pm \sqrt{- 121}
    \]
    und diskutiere, welche Vorzeichen der Quadratwurzeln jeweils zu wählen sind.
\end{exercise}

\begin{exercise}(3 Punkte)\newline
    Sei \(z\) eine invertierbare komplexe Zahl. Folgere die Gleichheit
    \[  
        \overline{z}^{-1} = \overline{z^{-1}}
    \]
    aus der Multiplikativität der komplexen Konjugation.
\end{exercise}

\begin{exercise}(4 Punkte)\newline
    Zeige, daß für zwei reelle Zahlen $a$ und $b$ genau dann die Wurzel
    $\sqrt{a^2 + b^2}$ nahe bei Null ist, wenn sowohl $|a|$ und $|b|$
    nahe bei Null sind. Zeige also:
    \[    \forall \epsilon > 0\, \exists \delta > 0  \colon 
            \sqrt{a^2 + b^2} < \delta  \implies |a|, |b| < \epsilon,
    \]
      \[
        \forall \epsilon > 0\, \exists \delta > 0  \colon 
            |a|, |b| < \delta  \implies \sqrt{a^2  + b^2} < \epsilon,
    \]
    wobei \(\delta\) und \(\epsilon\) reelle Zahlen sind.
\end{exercise}

\begin{exercise}(2 Punkte)\newline
    Interpretiere die Multiplikation mit der imaginären Einheit \(\mathrm i\)
    als geometrische Operation in der Gaußschen Zahlenebene.
\end{exercise}

\begin{exercise}(6 Punkte)\newline
    Erkläre, warum \(\left(\begin{smallmatrix}
                               \cos \alpha & - \sin \alpha\\
                               \sin \alpha & \cos \alpha
                           \end{smallmatrix}\right)\) eine Drehung um den Winkel \(\alpha\) um
    den Ursprung der Gaußschen Zahlenebene beschreibt. Folgere sodann
    die Additionstheoreme
    \begin{align*}
        \cos (\alpha_1 + \alpha_2) & = \cos \alpha_1 \, \cos \alpha_2
            - \sin \alpha_1 \, \sin \alpha_2,
        \\
        \sin (\alpha_1 + \alpha_2) & = \cos \alpha_1 \, \sin \alpha_2
            + \sin \alpha_1 \, \cos \alpha_2
    \end{align*}
    aus der bekannten Formel für das Produkt von Matrizen.
\end{exercise}

\begin{exercise}(4 Punkte)\newline
    Zeige, daß die Polynomgleichung \(X^2 + X + 1 = 0\) in den reellen Zahlen
    keine Lösung besitzt.
\end{exercise}

\begin{exercise}(6 Punkte)\newline
    Konstruiere einen minimalen Zahlbereich \(\set R(\omega)\), welcher die
    reellen Zahlen und eine Lösung \(\omega\) der Polynomgleichung \(X^2 + X + 1 = 0\)
    enthält und in welchem Addition und Multiplikation definiert sind, welche
    Addition und Multiplikation reeller Zahlen fortsetzen und für die ebenfalls
    die einschlägigen Gesetze  der Arithmetik gelten. Zeige, daß \(\omega^3 = 1\)
    in \(\set R(\omega)\) gilt und daß es in \(\set R(\omega)\) eine Lösung
    der Gleichung \(X^2 + 1 = 0\) gibt.
\end{exercise}

\end{document}

