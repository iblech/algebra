\documentclass{algsheet}
    \usepackage{amssymb} 
\usepackage[protrusion=true,expansion=false]{microtype}

%%%%%%%%%%%%%%%%%%%%%%%%%%%%%%%%%%%%%%%%%%%%%%%%%%%%%%%%%%%%%%%%%%%%%%%%%%%%%%%%%%%%%%%%%%%%%%%%%%%%%%%%%%%%%%%%%%%%%%%%%%%%%%%%%%%%%%%%%%%%%%%%%%%%%%%%%%%%

   \lecture{Algebra I\\ \small{Einführung in die Algebra}}
        
        \semester{Wintersemester 2010/2011}
        \sheet{4.\ Aufgabenblatt}
        \author{Dipl.-Math.~Franz Vogler}
        \date{09.~November 2010}
        
        %/usr/local/share/texmf/tex/algmacros.sty
    \usepackage[arrow,curve,matrix]{xy}    
        \begin{document}
                \maketitle




Der neue Termin für Übungsgruppe 6 ist ab dieser Woche Donnerstag 12:15--13:45 (Raum 2006/L1).


\begin{exercise}(1 Punkt)\newline
    Sei \(z_n\) eine konvergente Folge komplexer Zahlen. Warum ist
    \(\lim\limits_{n \to \infty} |z_n|
    = |\lim\limits_{n \to \infty} z_n|\)?
\end{exercise}

\begin{exercise}(4 Punkte)\newline
   Sei \(f(z)\) eine komplexe Polynomfunktion, das heißt sei
    \(f(z) = a_n z^n + a_{n - 1} z^{n - 1} + \dotsb + a_1 z + a_0\) für jede
    komplexe Zahl \(z\) mit festen komplexen Koeffizienten \(a_0\), \dots, \(a_n\).
    Zeige, daß \(f(z)\) bezüglich des komplexen Betrages stetig ist, das heißt
    zeige: 
    \[
        \forall C > 0\, \forall \epsilon > 0\, \exists \delta > 0\,
        \forall z, z'\ \text{mit}\ |z|, |z'| \leq C\colon |z - z'| < \delta
        \implies |f(z) - f(z')| < \epsilon,
    \]
    wobei \(C\), \(\epsilon\) und \(\delta\) reelle Zahlen und \(z\) und \(z'\) zwei
    komplexe Zahlen sind.\end{exercise}

\begin{exercise}(2 Punkte)\newline
    Sei \(q\) eine komplexe Zahl. Beweise die Formel
    \[\sum_{k = 0}^{n - 1} q^k = \frac{q^n-1}{q-1}.\]
\end{exercise}

\begin{exercise}(2 Punkte)\newline
    Gib eine Abschätzung des Betrages aller komplexen Lösungen von
    \(X^4 - 2 X^3 + 5 X^2 - 4 X + 5\) von oben an.
\end{exercise}

\begin{exercise}(4 Punkte)\newline
    Sei \(X^n + a_{n - 1} X^{n - 1} + \dotsb + a_1 X + a_0=0\) eine normierte
    Polynomgleichung mit komplexen Koeffizienten. Zeige, daß jede komplexe
    Lösung \(z\) in der abgeschlossenen Scheibe vom Radius
    $1+\max\left\{|a_0|,\dotsc, |a_{n-1}|\right\}$ 
    liegt.

\end{exercise}

\begin{exercise}(3 Punkte)\newline
    Im Beweis des Fundamentalsatzes der Algebra taucht die Zahl \(3\)
    immer wieder auf. Kann sie durch eine andere Zahl ersetzt werden?
\end{exercise}





\begin{exercise}(3 Punkte)\newline
    Seien \(f(X)\) und \(g(X)\) zwei Polynome mit \(\deg f(X) \leq n\) und
    \(\deg g(X) \leq m\). Zeige, daß \(\deg (f(X) + g(X)) \leq \max\{n, m\}\)
    und \(\deg (f(X) \cdot g(X)) \leq n + m\).
\end{exercise}

\begin{exercise}(1 Punkt)\newline
    Gib ein Beispiel für ein Polynom \(f(X)\) und zwei algebraische Zahlen
    \(x\) und \(y\) an, so daß \(f(x \cdot y) \neq f(x) \cdot f(y)\).
\end{exercise}

\begin{exercise}(2 Punkte)\newline
    Ist \(X + \sqrt 2\) ein Teiler von \(X^3 - 2 X\)?
\end{exercise}

\begin{exercise}(3 Punkte)\newline
    Ein Polynom \(f(X)\) mit ganzzahligen Koeffizienten teile ein weiteres
    Polynom \(g(X)\) mit ganzzahligen Koeffizienten. Zeige, daß für jede ganze
    Zahl \(n\) die ganze Zahl \(f(n)\) ein Teiler der ganzen Zahl \(g(n)\) ist.
\end{exercise}

\begin{exercise}(3 Punkte)\newline
    Besitzt das Polynom \(X^7 + 11 X^3 - 33 X + 22\) einen Teiler der Form
    \((X - a) (X - b)\), wobei \(a\) und \(b\) rationale Zahlen sind?
\end{exercise}

\begin{exercise}(2 Punkte)\newline
    Seien \(p(X)\), \(q(X)\), \(f(X)\) und \(g(X)\) Polynome mit algebraischen
    Koeffizienten. Teile das Polynom \(d(X)\) die beiden Polynome \(f(X)\) und
    \(g(X)\). Zeige, daß \(d(X)\) dann auch das Polynom \(p(X) \cdot f(X)
    + q(X) \cdot g(X)\) teilt.
\end{exercise}

\begin{exercise}(3 Punkte)\newline
    Seien \(f(X) = 3 X^4 - X^3 + X^2 - X + 1\) und \(g(X) = X^3 - 2 X + 1\).
    Gib Polynome \(q(X)\) und \(r(X)\) mit \(f(X) = q(X) g(X) + r(X)\) an, so
    daß \(\deg r(X) < \deg g(X)\).
\end{exercise}

\begin{exercise}(2 Punkte)\newline
    Inwiefern kann ein Polynom in den zwei Unbestimmten \(X\) und \(Y\) als
    Polynom in einer Unbestimmten \(Y\), dessen Koeffizienten Polynome in
    \(X\) sind, aufgefaßt werden?    
    Wie läßt sich diese Aussage auf Polynome mit mehr als zwei Unbestimmten
    verallgemeinern?
\end{exercise}

\begin{exercise}(2 Punkte)\newline
    Schreibe \(\frac 1 {\sqrt 2 + 5 \sqrt 3}\) als polynomiellen Ausdruck
    mit rationalen Koeffizienten in \(\sqrt 2\) und \(\sqrt 3\).
\end{exercise}

\begin{exercise}(4 Punkte)\newline
    Sei \(t\) eine komplexe Zahl, so daß \(\set Q[t] = \set Q(t)\). Zeige, daß
    \(t\) algebraisch ist.
\end{exercise}





\end{document} 