\documentclass{algsheet}
    \usepackage{amssymb}    
%\input{praeambel}     

\usepackage{ucs}
\usepackage[utf8x]{inputenc}
\usepackage[T1]{fontenc}
\usepackage{lmodern}
\usepackage{german}
\usepackage{makeidx}
\usepackage{amsmath}
\usepackage{amsthm}
\usepackage{amsfonts}
\usepackage{amssymb}
\usepackage{mathtools}
\usepackage[Euler]{upgreek}
\usepackage{prettyref}
\usepackage[draft,german]{varioref} % XXX draft mode.
\usepackage{tikz}
\usepackage{graphicx}
\usepackage[unicode,pdftex,pagebackref,colorlinks]{hyperref}
\usepackage[all]{hypcap} % XXX has to go after hyperref
\usepackage[shortalphabetic,backrefs,abbrev]{amsrefs} % XXX has to go after hyperref


\usetikzlibrary{calc,through}
\mathtoolsset{showonlyrefs}



%%%%%%%%%%%%%%%%%%%%%%%%%%%%%%%%%%%%%%%%%%%%%%%%%%%%%%%%%%%%%%%%%%%%%%%%%%%%%%%%%%%%%%%%%%%%%%%%%%%%%%%%%%%%%%%%%%%%%%%%%%%%%%%%%%%%%%%%%%%%%%%%%%%%%%%%%%%%

   \lecture{Algebra I\\ \small{Einführung in die Algebra}}
        
        \semester{Wintersemester 2010/2011}
        \sheet{1.\ Aufgabenblatt}
        \author{Dipl.-Math.~Franz Vogler}
        \date{19.~Oktober 2010}
        
        %/usr/local/share/texmf/tex/algmacros.sty
    \usepackage[arrow,curve,matrix]{xy}    
        \begin{document}
                \maketitle


Willkommen zur Lehrveranstaltung Algebra I im Wintersemester 2010/11!
\newline

Die Anmeldung zu den Übungen erfolgt unter
\newline
http://www.math.uni-augsburg.de/alg/lehre/wintersemester-2010-11/algebra-i/anmeldung-zu-den-ubungen

\newline 

Die Übungen beginnen DIESE WOCHE. Der Abgabetermin der schriftlich zu bearbeitenden Aufgaben ist jeweils der Dienstag 08:00 Uhr beginnend mit dem 26.10.2010. Ort der Abgabe ist der mit
 \glqq Algebra 1\grqq\; gekennzeichnete Briefkasten im Erdgeschoss von Gebäude L. Die Abgaben bitte DEUTLICH mit dem eigenen
Namen, der Übungsgruppe und dem Namen des Übungsleiters versehen. Weitere Einzelheiten zum Ablauf der Übungen geben die Tutoren in der ersten Übung bekannt.
\newline

Die Aufgaben 1--6 werden in dieser Woche gemeinsam in der Übung gerechnet. Die Aufgaben 7--13 sind bis kommenden Montag schriftlich zu bearbeiten und abzugeben. 

\begin{exercise} 
 Zeige, daß \(\sqrt 3\) keine rationale Zahl ist.
\end{exercise}





\begin{exercise}
Zeige, daß \(\sqrt{12}\) keine rationale Zahl ist.
\end{exercise}



\begin{exercise}
    Zeige, daß \(\sqrt[3]{25}\) keine rationale Zahl ist.
\end{exercise}



\begin{exercise}
 Finde eine Polynomgleichung mit ganzzahligen Koeffizienten, welche die ganzen Zahlen $x_1,x_2,\dots,x_{n-1},x_n$ als Nullstellen besitzt.  
\end{exercise}



\begin{exercise}
    Zeige, daß eine ganze Zahl \(a\) genau dann eine
    \(n\)-te Wurzel in den rationalen Zahlen besitzt, wenn
    \(a\) eine \(n\)-te Wurzel in den ganzen Zahlen besitzt.

\end{exercise}




\begin{exercise}
    Zeige, daß jede rationale Lösung
    einer normierten Polynomgleichung mit ganzzahligen
    Koeffizenten eine ganze Zahl ist.

\end{exercise}




\begin{exercise}(5 Punkte)\newline
    Sei
    \[
        X^n + a_{n - 1} X^{n - 1} + \dotsb + a_1 X + a_0 = 0
    \]
    eine normierte Polynomgleichung mit ganzzahligen Koeffizenten. Zeige, daß
    jede ganzzahlige Lösung ein Teiler von \(a_0\) sein muß.
\end{exercise}




\begin{exercise}(5 Punkte)\newline
    Zeige, daß jede normierte Polynomgleichung ungeraden Grades mit
    rationalen Koeffizienten in den reellen Zahlen eine
    Lösung besitzt.
\end{exercise}



\begin{exercise}(5 Punkte)\newline
    Gib eine normierte Polynomgleichung vierten Grades mit
    rationalen Koeffizienten an, welche in den reellen Zahlen keine
    Lösung hat.
\end{exercise}




\begin{exercise}(5 Punkte)\newline
    Gib eine normierte Polynomgleichung fünften Grades mit
    rationalen Koeffizienten an,
    welche als einzige Nullstelle die Zahl \(1\) hat.
\end{exercise}



\begin{exercise}(5 Punkte)\newline
 Gib eine Polynomgleichung mit ganzzahligen Koeffizienten an, die $\sqrt[7]{3+\sqrt[3] 4}$ als Lösung besitzt.
\end{exercise}




\begin{exercise}(5 Punkte)\newline
    Gib eine Polynomgleichung mit ganzzahligen Koeffizienten an,
    welche die reelle Zahl \(\cos 15^\circ\) als Lösung besitzt.
\end{exercise}



\input{cal_dreieck_graphik}
\begin{exercise}(10 Punkte)\newline
    Eugenio Calabi\footnote{Eugenio Calabi, 1923--, italienisch-amerikanischer
    Mathematiker} hat ein nicht gleichseitiges, gleichschenkliges Dreieck
    gefunden, in welchem sich drei gleich große, größte Quadrate gemäß
    Abbildung 1 einschreiben lassen. Zeige, daß das Verhältnis
    \(x\) der längsten zu einer der beiden kürzeren Seiten die Gleichung
    \(2 X^3 - 2 X^2 - 3 X + 2 = 0\) erfüllt.
\end{exercise}













\end{document} 