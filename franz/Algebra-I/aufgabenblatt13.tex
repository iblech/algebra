\documentclass{algsheet}
    \usepackage{amssymb} 
\usepackage[protrusion=true,expansion=false]{microtype}

%%%%%%%%%%%%%%%%%%%%%%%%%%%%%%%%%%%%%%%%%%%%%%%%%%%%%%%%%%%%%%%%%%%%%%%%%%%%%%%%%%%%%%%%%%%%%%%%%%%%%%%%%%%%%%%%%%%%%%%%%%%%%%%%%%%%%%%%%%%%%%%%%%%%%%%%%%%%

   \lecture{Algebra I\\ \small{Einführung in die Algebra}}
        
        \semester{Wintersemester 2010/2011}
        \sheet{13.\ Aufgabenblatt}
        \author{Dipl.-Math.~Franz Vogler}
        \date{25.~Januar 2011}
        
        %/usr/local/share/texmf/tex/algmacros.sty
    \usepackage[arrow,curve,matrix]{xy}    
        \begin{document}
                \maketitle





\begin{exercise}(3 Punkte)\newline %1
    Gib alle \(n \in \{1, \dotsc, 100\}\) an, für die ein regelmäßiges
    \(n\)-Eck mit Zirkel und Lineal konstruierbar ist.
\end{exercise}

\begin{exercise}(3 Punkte)\newline
    Gib eine Konstruktionsvorschrift für das regelmäßige \(15\)-Eck an.
\end{exercise}

\begin{exercise}(4 Punkte)\newline
    Sei \(n\) eine natürliche Zahl. Zeige, daß
    \begin{equation}
        F_{n + 1} = 2 + F_n F_{n - 1} \dotsm F_0.
    \end{equation}
\end{exercise}

\begin{exercise}(2 Punkte)\newline
    Zeige, daß \(F_m\) und \(F_n\) für \(m \neq n\) teilerfremd sind.
    Folgere, daß es unendlich viele Primzahlen gibt.
\end{exercise}

\begin{exercise}(2 Punkte)\newline
    Eine \emph{Mersennesche\footnote{%
    Martin Mersenne, 1588--1648, französischer Theologie, Mathematiker
    und Musiktheoretiker} Primzahl} ist eine Primzahl der Form
    \(M_n = 2^n - 1\), wobei \(n\) eine natürliche Zahl ist. Zeige, daß
    \(M_n\) höchstens dann eine Mersennesche Primzahl ist, wenn
    \(n\) selbst eine Primzahl ist. Zeige allgemeiner, daß \(M_n\) von
    \(M_d\) geteilt wird, wenn \(d\) ein positiver Teiler von \(n\) ist. 
\end{exercise}


\begin{exercise}(5 Punkte)\newline
    Sei \(p\) eine Primzahl.
    Sei \(x\) eine algebraische Zahl vom Grad \(p^n\), so daß deren galoissch Konjugierte
    \(x_1 = x\), \(x_2\), \dots, \(x_{p^n}\) alle in \(x\) rational sind. Zeige, daß
    die Galoissche Gruppe zu \(x_1\), \dots, \(x_{p^n}\) im Zentrum ein Element \(\sigma\)
    der Ordnung \(p\) besitzt. Sei \(y\) ein primitives Element zu
    \(e_1(x_1, \sigma \cdot x_1, \dotsc, \sigma^{p - 1} \cdot x_1)\), \dots,
    \(e_p(x_p, \sigma \cdot x_p, \dotsc, \sigma^{p - 1} \cdot x_p)\). Zeige,
    daß \(y\) den Grad \(p^{n - 1}\) über den rationalen Zahlen besitzt.
    \newline
    (Hinweis: Laß Dich für den zweiten Teil der Aufgabe vom Beweis zu Proposition 4.34 inspirieren.)
\end{exercise}

\begin{exercise}(4 Punkte)\newline
    Sei \(n\) eine natürliche Zahl,
    und sei \(f(X)\) ein irreduzibles, normiertes, separables Polynom vom Grad \(2^n\).
    Seien \(x_1\), \dots, \(x_{2^n}\) die Lösungen von \(f(X)\), deren
    Galoissche Gruppe die Ordnung \(2^n\) habe. Folgt dann, daß \(x_1\) eine konstruierbare
    Zahl ist?
\end{exercise}






\begin{exercise}(4 Punkte)\newline
    Zeige, daß \(\sqrt 2 + \sqrt 7\) und \(- \sqrt 2 - \sqrt 7\) über den rationalen Zahlen galoissch konjugiert
    sind, aber nicht über \(\sqrt 7\).
\end{exercise}

\begin{exercise}(4 Punkte)\newline
    Sei \(t \in \set Q(\sqrt 2)\). Zeige, daß \(t\) entweder eine rationale
    Zahl ist oder daß \(\set Q(t) = \set Q(\sqrt 2)\). Folgere, daß \(\set Q(\sqrt 2)\)
    über den rationalen Zahlen keine echte Zwischenerweiterung besitzt.
\end{exercise}

\begin{exercise}(3 Punkte)\newline
    Seien \(K\) und \(L\) zwei Koeffizientenbereiche, so daß \(K\) eine Zwischenerweiterung
    von \(L\) ist. Sei \(x\) eine algebraische Zahl. Zeige, daß ein galoissch Konjugiertes
    von \(x\) über \(L\) auch ein galoissch Konjugiertes von \(x\) über \(K\)
    ist.
\end{exercise}

\begin{exercise}(3 Punkte)\newline
    Sei \(f(X)\) in normiertes separables Polynom über einem Koeffizientenbereich \(K\), und
    seien \(x_1\), \dots, \(x_n\) seine Nullstellen. Sei \(y\) eine Zahl aus \(K(x_1, \dotsc, x_n)\).
    Zeige, daß die Galoissche Gruppe \(H\) von \(x_1\), \dots, \(x_n\) über \(K(y)\) aus denjenigen
    Permutationen \(\sigma\)
    der Galoisschen Gruppe \(G\) von \(x_1\), \dots, \(x_n\) über \(K\) besteht, unter welchen \(y\) invariant
    bleibt, das heißt, für die \(\sigma \cdot y = y\) gilt.
\end{exercise}





\end{document} 