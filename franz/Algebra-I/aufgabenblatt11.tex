\documentclass{algsheet}
    \usepackage{amssymb} 
\usepackage[protrusion=true,expansion=false]{microtype}

%%%%%%%%%%%%%%%%%%%%%%%%%%%%%%%%%%%%%%%%%%%%%%%%%%%%%%%%%%%%%%%%%%%%%%%%%%%%%%%%%%%%%%%%%%%%%%%%%%%%%%%%%%%%%%%%%%%%%%%%%%%%%%%%%%%%%%%%%%%%%%%%%%%%%%%%%%%%

   \lecture{Algebra I\\ \small{Einführung in die Algebra}}
        
        \semester{Wintersemester 2010/2011}
        \sheet{11.\ Aufgabenblatt}
        \author{Dipl.-Math.~Franz Vogler}
        \date{11.~Januar 2011}
        
        %/usr/local/share/texmf/tex/algmacros.sty
    \usepackage[arrow,curve,matrix]{xy}    
        \begin{document}
                \maketitle





\begin{exercise}(2 Punkte)\newline
    Zeige, daß die alternierende Gruppe \(\mathbf A_n\) eine Untergruppe der
    symmetrischen Gruppe \(\mathbf S_n\) ist.
\end{exercise}



\begin{exercise}(2 Punkte)\newline
    Seien \(\sigma\) und \(\tau\) Symmetrien der Nullstellen \(x_1\), \dots, \(x_n\) eines normierten
    separablen Polynoms. Zeige, daß \(\tau \cdot (\sigma \cdot x) = (\tau \circ \sigma) \cdot x\) für jede Nullstelle
    $x=x_j$ gilt. 
\end{exercise}



\begin{exercise}(2 Punkte)\newline
    Seien \(x_1\), \dots, \(x_n\) Nullstellen eines normierten separablen Polynoms \(f(X)\) über den rationalen
    Zahlen. Seien \(x_1\) und \(x_2\) nicht galoissch konjugiert. Zeige, daß keine Symmetrie
    \(\sigma\) der Nullstellen \(x_1\), \dots, \(x_n\) von \(f(X)\) existiert, so daß
    \(x_2 = \sigma \cdot x_1\).
\end{exercise}




\begin{exercise}(3 Punkte)\newline
    Sei \(f(X)\) ein quadratisches Polynom über den rationalen Zahlen mit Nullstellen \(x_1\) und \(x_2\),
    welche wir als verschieden annehmen. Berechne \(\mathrm{Gal}_{\set Q}(x_1, x_2)\) in Abhängigkeit der
    Diskriminante von \(f(X)\).
\end{exercise}



\begin{exercise}(3 Punkte)\newline
    Sei \(f(X)\) ein irreduzibles normiertes Polynom dritten Grades über den rationalen Zahlen mit
    Nullstellen \(x_1\), \(x_2\) und \(x_3\). Sei \(x_1\) kein primitives Element zu \(x_1\), \(x_2\)
    und \(x_3\). Zeige, daß die Galoissche Gruppe zu den Nullstellen \(x_1\), \(x_2\) und \(x_3\) von
    \(f(X)\) genau sechs Elemente hat.
\end{exercise}

\begin{exercise}(2 Punkte)\newline
    Sei \(f(X)\) ein normiertes separables Polynom vom Grade \(n\). Sei \(t\) ein primitives Element über den
    rationalen Zahlen zu allen Nullstellen von \(f(X)\). Zeige, daß der Grad von \(t\) höchstens \(n!\)
    beträgt.
\end{exercise}






\begin{exercise}(4 Punkte)\newline
    Seien \(x_1\), \dots, \(x_n\) die Nullstellen eines normierten separablen
    Polynoms mit rationalen Koeffizienten. Seien \(z\) und \(w\) zwei in
    \(x_1\), \dots, \(x_n\) rationale Zahlen. Zeige, daß für jede Symmetrie
    \(\sigma\) von \(x_1\), \dots, \(x_n\) die Gleichheiten 
    \begin{align*}
        \sigma \cdot (z + w) & = \sigma \cdot z + \sigma \cdot w\\
        \shortintertext{und}
        \sigma \cdot (z w) & = (\sigma \cdot z) \, (\sigma \cdot w)
    \end{align*}
    gelten.   
\end{exercise}




\begin{exercise}(3 Punkte)\newline
    Seien \(x_1\), \dots, \(x_n\) die Nullstellen eines normierten separablen Polynoms \(f(X)\) mit rationalen
    Koeffizienten. Zeige, daß die Galoissche Gruppe zu \(x_1\), \dots, \(x_n\) genau dann in der
    alternierenden Gruppe \(\mathbf A_n\) liegt, wenn die Diskriminante von \(f(X)\) eine Quadratwurzel in den
    rationalen Zahlen besitzt.
\end{exercise}




\begin{exercise}(3 Punkte)\newline
    Sei \(f(X_1, \dots, X_n)\) ein nicht verschwindendes Polynom mit rationalen Koeffizienten. Zeige, daß
    ganze Zahlen \(m_1\), \dots, \(m_n\) mit \(f(m_1, \dots, m_n) \neq 0\) existieren.
\end{exercise}

\begin{exercise}(3 Punkte)\newline
    Gib eine Galoissche Resolvente für das Polynom \(f(X) = X^2 + X + 1\) an.
\end{exercise}

\begin{exercise}(2 Punkte)\newline
    Warum ist die Galoissche Resolvente nur für separable Polynome definiert worden?
\end{exercise}

\begin{exercise}(4 Punkte)\newline
    Seien \(x_1\), \dots, \(x_n\) die Nullstellen eines normierten separablen Polynoms \(f(X)\) mit
    rationalen Koeffizienten. Sei \(C\) eine natürliche Zahl mit
    \begin{align*}
        n \cdot \left|{\frac{x_i - x_j}{x_k - x_\ell}}\right| \leq C
    \end{align*}
    für alle \(i\), \(j\), \(k\), \(\ell \in \{1, \dots, n\}\) mit \(k \neq \ell\).
    Zeige, daß
    \begin{align*}
        V(X_1, \dots, X_n) = X_1 + C \cdot X_2 + C^2 \cdot X_3 + \dotsc + C^{n - 1} X_n
    \end{align*}
    eine Galoissche Resolvente zu \(f(X)\) ist.    
\end{exercise}

\begin{exercise}(3 Punkte)\newline
    Schreibe das Polynom \(X_1 X_2 X_3 - X_2 X_3 + X_1^3 X_2^2 X_3^2\) als Polynom in \(X_1\)
    und in den elementarsymmetrischen Funktionen von \(X_1\), \(X_2\) und \(X_3\).
\end{exercise}

\begin{exercise}(4 Punkte)\newline
    Sei \(H(X_1, \dotsc, X_n)\) ein Polynom, welches symmetrisch in \(X_3\), \dots, \(X_n\) ist.
    Zeige, daß sich \(H(X_1, \dotsc, X_n)\)
    als Polynom in \(X_1\), in \(X_2\) und in den elementarsymmetrischen Funktionen von
    \(X_1\), \dots, \(X_n\) schreiben läßt.
    Ist diese Darstellung eindeutig?
\end{exercise}




\end{document} 