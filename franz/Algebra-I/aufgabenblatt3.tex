\documentclass{algsheet}
    \usepackage{amssymb} 


%%%%%%%%%%%%%%%%%%%%%%%%%%%%%%%%%%%%%%%%%%%%%%%%%%%%%%%%%%%%%%%%%%%%%%%%%%%%%%%%%%%%%%%%%%%%%%%%%%%%%%%%%%%%%%%%%%%%%%%%%%%%%%%%%%%%%%%%%%%%%%%%%%%%%%%%%%%%

   \lecture{Algebra I\\ \small{Einführung in die Algebra}}
        
        \semester{Wintersemester 2010/2011}
        \sheet{3.\ Aufgabenblatt}
        \author{Dipl.-Math.~Franz Vogler}
        \date{02.~November 2010}
        
        %/usr/local/share/texmf/tex/algmacros.sty
    \usepackage[arrow,curve,matrix]{xy}    
        \begin{document}
                \maketitle

\begin{exercise}(2 Punkte)\newline
 Ist \(\cos 10^\circ\) eine algebraische Zahl?
\end{exercise}





\begin{exercise}(3 Punkte)\newline
     Zeige, daß die Polynomgleichung \(X^3 - 2 X + 5 = 0\) genau eine reelle
    Lösung \(x\) besitzt. Zeige weiter, daß \(x\) eine invertierbare algebraische
    Zahl ist, und gib eine normierte Polynomgleichung mit rationalen
    Koeffizienten an, welche \(x^{-1}\) als Lösung besitzt.

\end{exercise}





\begin{exercise}(4 Punkte)\newline
    Seien \(x\) und \(y\) algebraische Zahlen, welche Lösungen normierter
    Polynomgleichungen mit rationalen Koeffizienten von Graden \(n\) beziehungsweise
    \(m\) sind. Gib eine Abschätzung des Grades an, den eine normierte
    Polynomgleichung mit rationalen Koeffizienten höchstens haben muß, damit sie
    \(x \cdot y\) als Lösung besitzt.
 \end{exercise}






\begin{exercise}
   (2 Punkte)\newline
    Sei \(x\) eine rationale Zahl, welche zugleich eine ganze algebraische Zahl
    ist. Zeige, daß \(x\) dann sogar eine ganze Zahl ist.
\end{exercise}




\begin{exercise}(2 Punkte)\newline
    Seien \(a_0, \dotsc, a_{n - 1}\) rationale Zahlen. Sei \(z\) eine
    transzendente Zahl. Zeige, daß dann auch \(z^n + a_{n - 1} z^{n - 1} +
    \dotsb + a_1 z + a_0\) eine transzendente Zahl ist.
\end{exercise}




\begin{exercise}
\begin{enumerate}
    \item[(a)](1 Punkt)\newline
    Sei \(\zeta\) eine vierte Einheitswurzel, und sei \(\theta\) eine sechste
    Einheitswurzel. Zeige, daß \(\zeta \cdot \theta\) eine zwölfte Einheitswurzel
    ist.


 \item[(b)]  (3 Punkte)\newline
    Seien \(m\) und \(n\) zwei positive natürliche Zahlen, deren größten
    gemeinsamen Teiler wir mit \((m, n)\) bezeichnen. Sei \(\zeta\) eine
    \(m\)-te und \(\theta\) eine \(n\)-te Einheitswurzel. Zeige, daß
    \(\zeta \cdot \theta\) eine \(k\)-te Einheitswurzel ist, wobei
    \(k = \frac{m n}{(m, n)}\).
\end{enumerate}
\end{exercise}

\begin{exercise}
 \begin{enumerate}
 \item[(a)](1 Punkt)\newline
    Wieviele zehnte Einheitswurzeln \(\zeta\) gibt es, so daß alle
    anderen zehnten Einheitswurzeln eine ganzzahlige Potenz von \(\zeta\)
    sind?


\item[(b)](8 Punkte)\newline
    Sei \(n\) eine natürliche Zahl. Sei \(\upphi(n)\) die Anzahl der zu
    \(n\) teilerfremden natürlichen Zahlen kleiner als \(n\). (Die Funktion
    \(\upphi(n)\) heißt \emph{Eulersche \(\upphi\)-Funktion}.) Zeige, daß die
    Anzahl der \(n\)-ten Einheitswurzeln \(\zeta\), so daß jede andere
    \(n\)-te Einheitswurzel eine ganzzahlige Potenz von \(\zeta\) ist, durch
    \(\upphi(n)\) gegeben ist.
\end{enumerate}
\end{exercise}


\begin{exercise}(3 Punkte)\newline
    Sei eine algebraische Zahl \(z\) in der Form
    \(z = r \cdot \exp({\phi \mathrm i})\) mit einer positiven reellen Zahl \(r\) und
    einer reellen Zahl \(\phi\) gegeben. Zeige, daß \(r\) eine algebraische
    Zahl ist.
\end{exercise}

\begin{exercise}(2 Punkte)\newline
    Gib alle komplexen Lösungen von \(X^6 + 1 = 0\) in der Form \(a + b \, \mathrm i\)
    an, wobei \(a\) und \(b\) jeweils reelle Zahlen sind.
\end{exercise}

\begin{exercise}(2 Punkte)\newline
    Gib eine normierte Polynomgleichung an, deren Lösungen genau die Ecken
    eines regelmäßigen Siebenecks in der komplexen Ebene sind, dessen Zentrum
    der Ursprung der Ebene ist und dessen eine Ecke durch die komplexe
    Zahl \(1 + \frac 1 2 \mathrm i\) gegeben ist.
\end{exercise}

\begin{exercise}
\begin{enumerate}
 \item[(a)]  (1 Punkt)\newline
    Zeige, daß die Gleichung \(X^3 + X^2 + X + 1 = 0\) genau drei komplexe
    Lösungen besitzt, und zwar alle vierten Einheitswurzeln bis auf \(1\).


\item[(b)](3 Punkte)\newline
    Zeige, daß die Gleichung \(X^{n - 1} + X^{n - 2} + \dotsb + X + 1 = 0\)
    genau \(n - 1\) komplexe Lösungen besitzt, und zwar alle \(n\)-ten
    Einheitswurzeln bis auf die \(1\).
\end{enumerate}
\end{exercise}

\begin{exercise}(3 Punkte)\newline
    Folgere die Additionstheoreme für die Sinus- und die Kosinusfunktion aus der
    Identität
    \(\exp({x \mathrm i}) \cdot \exp({y \mathrm i}) = \exp({(x + y) \mathrm i})\).
\end{exercise}




\end{document} 