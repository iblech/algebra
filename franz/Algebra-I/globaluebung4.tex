\documentclass{algsheet}
    \usepackage{amssymb} 
\usepackage[protrusion=true,expansion=false]{microtype}

%%%%%%%%%%%%%%%%%%%%%%%%%%%%%%%%%%%%%%%%%%%%%%%%%%%%%%%%%%%%%%%%%%%%%%%%%%%%%%%%%%%%%%%%%%%%%%%%%%%%%%%%%%%%%%%%%%%%%%%%%%%%%%%%%%%%%%%%%%%%%%%%%%%%%%%%%%%%

   \lecture{Algebra I\\ \small{Einführung in die Algebra}}
        
        \semester{Wintersemester 2010/2011}
        \sheet{Globalübung}
        \author{Dipl.-Math.~Arturo Mancino}
        \date{10.~November 2010}
        
        %/usr/local/share/texmf/tex/algmacros.sty
    \usepackage[arrow,curve,matrix]{xy}    
        \begin{document}
                \maketitle


\begin{exercise}(x Punkte)\newline
     Zeige, daß alle ganzen Gaußschen Zahlen ganze algebraische Zahlen sind.
\end{exercise}

\begin{exercise}(x Punkte)\newline
    Zeige, wie sich eine Strecke der Länge \(\frac 1 3\) aus einer Strecke der
    Länge \(1\) konstruieren läßt.
\end{exercise}

\begin{exercise}(x Punkte)\newline
    Zeige, wie sich eine Strecke der Länge \(\sqrt[4] 7\) aus einer Strecke der
    Länge \(1\) konstruieren läßt.
\end{exercise}

\begin{exercise}(x Punkte)\newline
    Konstruiere einen Winkel von \({15^\circ}\).
\end{exercise}

\begin{exercise}(x Punkte)\newline
    Sei \(z\) eine Lösung von \(X^4 + X^3 + X^2 + X + 1 = 0\). Zeige, daß
    \(z + z^{-1}\) eine Lösung von \(X^2 + X - 1 = 0\) ist.
\end{exercise}

\begin{exercise}(x Punkte)\newline
    Zeige, daß die vier Lösungen von \(X^4 + X^3 + X^2 + X + 1 = 0\) allesamt
    konstruierbare komplexe Zahlen sind.
\end{exercise}

\begin{exercise}(x Punkte)\newline
    Gib eine Konstruktionsanleitung für das regelmäßige Fünfeck.
\end{exercise}

\begin{exercise}(x Punkte)\newline
    Seien \(x\), \(x'\), \(y\), \(y'\) vier komplexe Zahlen mit \(x \neq x'\)
    und \(y \neq y'\). Sei \(K_1\) der Kreis durch \(x'\) und mit Mittelpunkt
    \(x\), und sei \(K_2\) der Kreis durch \(y'\) und mit Mittelpunkt \(y\).
    Seien \(K_1\) und \(K_2\) verschieden. Gib explizite Formeln für den oder die
    beiden Schnittpunkte der Kreise \(K_1\) und \(K_2\) an.
\end{exercise}







\end{document} 