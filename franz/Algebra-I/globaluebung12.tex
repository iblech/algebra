\documentclass{algsheet}
    \usepackage{amssymb} 
\usepackage[protrusion=true,expansion=false]{microtype}

%%%%%%%%%%%%%%%%%%%%%%%%%%%%%%%%%%%%%%%%%%%%%%%%%%%%%%%%%%%%%%%%%%%%%%%%%%%%%%%%%%%%%%%%%%%%%%%%%%%%%%%%%%%%%%%%%%%%%%%%%%%%%%%%%%%%%%%%%%%%%%%%%%%%%%%%%%%%

   \lecture{Algebra I\\ \small{Einführung in die Algebra}}
        
        \semester{Wintersemester 2010/2011}
        \sheet{Globalübung}
        \author{Dipl.-Math.~Arturo Mancino}
        \date{19.~Januar 2011}
        
        %/usr/local/share/texmf/tex/algmacros.sty
    \usepackage[arrow,curve,matrix]{xy}    
        \begin{document}
                \maketitle



\begin{exercise}
    Sei \(K\) ein Koeffizientenbereich. Zeige, daß die Äquivalenz von
    Folgen algebraischer Zahlen \(y_1\), \dots, \(y_n\) und \(z_1\), \dots, \(z_m\)
    über \(K\) in der Tat eine Äquivalenzrelation ist.
\end{exercise}



\begin{exercise}
    Sei \(G\) eine Gruppe. Sei \(\sigma \in G\). Zeige, daß der Zentralisator \(G_\sigma\) von \(\sigma\)
    eine Untergruppe von \(G\) ist.
\end{exercise}



\begin{exercise}
    Sei \(G\) eine Gruppe. Sei \(H\) eine Untergruppe von \(G\). Zeige, daß Kongruenz modulo \(H\) eine
    Äquivalenzrelation auf den Elementen von \(G\) ist.
\end{exercise}


\begin{exercise}
    Sei \(G\) eine Gruppe. Sei \(H\) eine Untergruppe von \(G\), und sei \(K\) eine Untergruppe von \(H\).
    Zeige, daß dann
    \begin{equation}
        [G : K] = [G : H] \cdot [H : K]
    \end{equation}
    gilt.
\end{exercise}



\begin{exercise}
    Sei \(p\) eine Primzahl. Zeige, daß
    \begin{equation}
        \Phi_p(X) = \frac{X^p - 1}{X - 1} = X^{p - 1} + X^{p - 2} + \dotsb + X + 1.
    \end{equation}
\end{exercise}

\begin{exercise}
    Sei \(p\) eine Primzahl. Zeige, daß
    \begin{equation}
        \Phi_p(X + 1) = X^{p - 1} + \binom p 1 X^{p - 2} + \dotsb + \binom p {p - 2} X + \binom p {p - 1}.
    \end{equation}
\end{exercise}

\begin{exercise}
    Zeige mit Hilfe des Eisensteinschen Kriteriums und unabhängig von den Aussagen dieses Abschnittes,
    daß \(X^{p - 1} + X^{p - 2} + \dotsb + X + 1\) über den rationalen Zahlen irreduzibel ist.
\end{exercise}



\begin{exercise}
    Sei \(p\) eine Primzahl. Sei \(r\) eine positive natürliche Zahl. Zeige, daß
    \begin{equation}
        \Phi_{p^r}(X) = \Phi_p(X^{p^{r - 1}}).
    \end{equation}
\end{exercise}

\begin{exercise}
    Sei \(n = p_1^{r_1} \dotsm p_s^{r_s}\) die Primfaktorzerlegung einer natürlichen Zahl, wobei wir
    \(r_1\), \dots, \(r_s > 0\) annehmen. Zeige, daß
    \begin{equation}
        \Phi_n(X) = \Phi_{p_1 \dotsm p_s}\Bigl(X^{p_1^{r_1 - 1} \dotsm p_s^{r_s - 1}}\Bigr).
    \end{equation}
\end{exercise}

\begin{exercise}
    Sei \(n > 1\) eine ungerade natürliche Zahl. Zeige, daß \(\Phi_{2n}(X) = \Phi_n(-X)\).
\end{exercise}

\begin{exercise}
    Sei \(p\) eine Primzahl und \(n\) eine zu \(p\) teilerfremde natürliche Zahl. Zeige, daß
    \begin{equation}
        \Phi_{pn}(X) = \frac{\Phi_n(X^p)}{\Phi_n(X)}.
    \end{equation}
\end{exercise}

\begin{exercise}
    Sei \(p\) eine Primzahl und \(n\) ein positives ganzzahliges Vielfaches von \(p\). Zeige, daß
    \(\Phi_{pn}(X) = \Phi_n(X^p)\).
\end{exercise}




\end{document}
 