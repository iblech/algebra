\documentclass{algsheet}
    \usepackage{amssymb} 
\usepackage[protrusion=true,expansion=false]{microtype}
 \usepackage{magros_3}
%%%%%%%%%%%%%%%%%%%%%%%%%%%%%%%%%%%%%%%%%%%%%%%%%%%%%%%%%%%%%%%%%%%%%%%%%%%%%%%%%%%%%%%%%%%%%%%%%%%%%%%%%%%%%%%%%%%%%%%%%%%%%%%%%%%%%%%%%%%%%%%%%%%%%%%%%%%%

   \lecture{Algebra II}
        
        \semester{Sommersemester 2011}
        \sheet{11.\ Aufgabenblatt}
        \author{Dipl.-Math.~Franz Vogler}
        \date{04.~Juli 2011}
        
        %/usr/local/share/texmf/tex/algmacros.sty
    \usepackage[arrow,curve,matrix]{xy}    
        \begin{document}
                \maketitle



\begin{exercise}(2 Punkte)\newline
    Zeige, daß das Nilradikal eines kommutativen Ringes im Schnitt aller seiner
    Primideale liegt.
\end{exercise}

\begin{exercise}(4 Punkte)\newline
    Sei \(\ideal m\) ein Ideal in einem kommutativen Ringe \(R\).
    Zeige, daß \(R/\ideal m\) genau dann ein Körper ist, wenn $\ideal m$ ein maximales Ideal ist. 
    \newline
    Was gilt im Falle eines Primideals (ohne Beweis!)?
\end{exercise}


\begin{exercise}(4 Punkte)\newline
  Sei \(R\) ein Bézoutscher Bereich und \(s\) ein reguläres Element. Zeige, daß
  \(R[s^{-1}]\) wieder ein Bézoutscher Bereich ist.
\end{exercise}

\begin{exercise}(3+2 Punkte)\vspace{-1ex}
  \begin{itemize}
   \item [(a)] 
    Seien \(\ideal a\) und \(\ideal b\) zwei endlich erzeugte Ideale eines
    Prüferschen Bereiches. Zeige, daß eine Zerlegung \(s_1\), \dots, \(s_n\)
    der Eins von \(R\) existiert, so daß für alle \(i \in \Set{1, \dotsc, n}\)
    die Ideale \(\ideal a[s_i^{-1}]\) und \(\ideal b[s_i^{-1}]\) in \(R[s_i^{-1}]\)
    Hauptideale sind.
   \item [(b)]     Gib eine Zerlegung der Eins \(s_1\), \dots, \(s_n\) des Zahlringes
    \(R = \mathbb Z[\sqrt{-13}]\) an, so daß für alle \(i \in \Set{1, \dotsc, n}\)
    das Ideal \((7, 1 + \sqrt{-13})\) in \(R[s_i^{-1}]\) ein Hauptideal ist.
  \end{itemize}
\end{exercise} 

\begin{exercise}(3 Punkte)\newline
    Sei \(R\) ein kommutativer Ring. Sei \(s_1\), \dots, \(s_n\) eine Zerlegung
    der Eins von \(R\). Sei \(\ideal a\) ein Ideal von \(R\), so daß für alle
    \(i \in \Set{1, \dotsc, n}\) gilt, daß \(\ideal a[s_i^{-1}] = (1)\) als
    Ideale in \(R[s_i^{-1}]\). Zeige, daß dann \(\ideal a\) das Einsideal in
    \(R\) ist.
\end{exercise}

\begin{exercise}(5 Punkte)\newline
    Sei \(R\) ein Prüferscher Bereich. Wir wollen ein nicht verschwindendes
    endlich erzeugtes
    Ideal \(\ideal a\) von \(R\) irreduzibel nennen, wenn für jede
    Zerlegung \(\ideal a = \ideal a_1 \dotsm \ideal \dotsm \ideal a_n\) in
    endlich erzeugte Ideale von \(R\) schon ein \(i \in \Set{1, \dotsc, n}\) mit
    \(\ideal a = \ideal a_i\) existiert. Zeige, daß jedes endlich erzeugte irreduzible 
    Ideal von \(R\) ein Primideal ist.
\end{exercise}

 
\begin{exercise}(4+4+3 Punkte)\vspace{-1ex}
      \begin{itemize}
   \item [($\alpha$)]     Sei \(R\) ein Dedekindscher Bereich. Angenommen, wir haben ein Test, der
    feststellt, ob ein endlich erzeugtes Ideal \(\ideal a\) in \(R\) irreduzibel ist 
    bzw.~gegebenenfalls das Ideal in zwei echte Faktoren zerlegt. Zeige, daß
    sich jedes nicht verschwindende Ideal in \(R\) dann bis auf Reihenfolge
    eindeutig als Produkt von Primidealen schreiben läßt.
   \item [($\beta$)]     Sei \(R\) ein Dedekindscher Bereich. Angenommen, wir haben einen Test, der
    feststellt, ob ein gegebenes Ideal \(\ideal a\) in \(R\) ein maximales
    Ideal ist bzw.~gegebenfalls ein Element liefert, um das \(\ideal a\) zu einem
    echten Ideal erweitert werden kann. Zeige, daß 
    sich jedes nicht verschwindende Ideal in \(R\) dann bis auf Reihenfolge
    eindeutig als Produkt von Primidealen schreiben läßt,
    \item [($\gamma$)]  Gib eine Primidealzerlegung von \(1 + \sqrt{-5}\) im Ringe \(\mathbb Z[\sqrt{-5}]\)
  an.
  \end{itemize}
\end{exercise}


\begin{exercise}\textbf{(Staatsexamens-/Klausuraufgaben)}(5 Zusatzpunkte)\newline
    Sei \(R\) ein Dedekindscher Bereich. Sei \(A \in \Mat_{n, m}(R)\) eine Matrix.
    Zeige, daß der Kern von \(A\) lokal endlich erzeugt ist.
\end{exercise}




\begin{exercise}(4 Punkte)\newline
    Sei \(R\) ein noetherscher kommutativer Ring. Ein Algorithmus produziere
    \(m\) Ketten
    \begin{equation}
           \begin{array}{r@{}X@{}r@{}X@{}r@{}X@{}r@{}}
            \ideal a_{10} & \subseteq \ideal a_{11} & \subseteq \ideal a_{12} & \subseteq \dotsb \\
            & \vdots \\
            \ideal a_{m0} & \subseteq \ideal a_{m1} & \subseteq \ideal a_{m2} & \subseteq \dotsb
        \end{array}
    \end{equation}
    von Idealen von \(R\). Zeige, daß ein \(n\) existiert, so daß
    \(\ideal a_{jn} = \ideal a_{j(n + 1)}\) für alle \(j \in \Set{1, \dotsc, m}\).
\end{exercise}


\begin{exercise}(4 Punkte)\newline
    Sei \(x\) eine Nullstelle des Polynoms \(f(X) = X^4 - X^2 - 3 X + 7\) in den
    algebraischen Zahlen. Sei \(K = \mathbb Q(x)\). Bestimme eine teilweise Faktorisierung
    der Ideale \((14, x + 7)\) und \((35, x - 14)\) in \(\mathcal O_K\).
\end{exercise}

\begin{exercise}(4 Punkte)\newline
    Bestimme eine \(\mathbb Z\)-Basis des Ringes ganzer Zahlen von \(\mathbb Q(\sqrt[3] 4)\).
\end{exercise}  



\end{document}