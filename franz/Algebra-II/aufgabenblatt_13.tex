\documentclass{algsheet}
    \usepackage{amssymb} 
\usepackage[protrusion=true,expansion=false]{microtype}
 \usepackage{magros_3}
%%%%%%%%%%%%%%%%%%%%%%%%%%%%%%%%%%%%%%%%%%%%%%%%%%%%%%%%%%%%%%%%%%%%%%%%%%%%%%%%%%%%%%%%%%%%%%%%%%%%%%%%%%%%%%%%%%%%%%%%%%%%%%%%%%%%%%%%%%%%%%%%%%%%%%%%%%%%

   \lecture{Algebra II}
        
        \semester{Sommersemester 2011}
        \sheet{13.\ Aufgabenblatt}
        \author{Dipl.-Math.~Franz Vogler}
        \date{18.~Juli 2011}
        
        %/usr/local/share/texmf/tex/algmacros.sty
    \usepackage[arrow,curve,matrix]{xy}    
        \begin{document}
                \maketitle





\begin{exercise}\textbf{(Quickies)}: (2+2+2+2 Punkte)\newline
  Sei $R$ ein Hauptidealbereich und $A := K[X]$ der Polynomring in einer Variablen über einem
  Körper $K$.  
  \begin{itemize}
    \item[\textbf(Q1)] Zeige, dass $R$ ein Ring mit größten gemeinsamen
                       Teilern ist.
    \item[\textbf(Q2)] Sei $r \in R$ irreduzibel. Beweise, dass das Ideal $(r) \subset R$ ein 
                       maximales Ideal ist.
    \item[\textbf(Q3)] Sei $\mathfrak a \subset A$ ein endlich erzeugtes Ideal. Inwiefern ist es gerechtfertigt
                       von \emph{dem} Erzeuger von $\mathfrak a$ zu sprechen.
    \item[\textbf(Q4)] Sei $A/\mathfrak a$ ein Körper. Ist \emph{der} Erzeuger von $\mathfrak a$
                       irreduzibel?
  \end{itemize}
 
\end{exercise}



\begin{exercise}(2 Punkte)\newline
    Zeige, daß es außer der trivialen keine weiteren endlichen Untergruppen der
    Einheitengruppe von \(\mathbb Q\) existieren.
\end{exercise}

\begin{exercise}(4 Punkte)\newline
    Sei \(N\) eine natürliche Zahl.
    Sei \(K\) ein faktorieller Körper, der keine endliche Charakteristik kleiner
    oder gleich \(N\) hat. Sei \(L\) über \(K\) eine endliche Körpererweiterung vom
    Grade \(N\). Zeige, daß \(L\) faktoriell ist.
\end{exercise}





\begin{exercise}(3+3 Punkte)\vspace{-1ex}
   \begin{itemize}
      \item[(a)]     Sei \(L \supseteq K\) eine Körpererweiterung. Sei \(x \in L\) über \(K\)
    separabel. Zeige, daß dann auch \(x\) über jeder Zwischenerweiterung \(E\)
    von \(L\) über \(K\) separabel ist.
      \item[(b)]    Sei \(L \supseteq K\) eine Körpererweiterung. Sei \(x \in L\) separabel
    über einer Zwischenerweiterung \(E\) von \(L\) über \(K\). Warum ist \(x\)
    im allgemeinen nicht separabel über \(K\)?
   \end{itemize}
\end{exercise}

\newpage


\begin{exercise}(4+4 Punkte)\vspace{-1ex}
        \begin{itemize}
      \item[(1)]    Sei \(g(X)\) ein normiertes Polynom über einem Körper \(K\) mit \(g'(X) = 0\).
    Warum hat der Körper \(K\) die Charakteristik einer Primzahl?
      \item[(2)]     Sei \(g(X)\) ein normiertes Polynom über einem Körper \(K\) mit \(g'(X) = 0\).
    Warum läßt sich \(g(X) = g_1(X^p)\) für eine Primzahl \(p\) und ein Polynom
    \(g_1 \in K[X]\) schreiben?
   \end{itemize}
\end{exercise}





\begin{exercise}(4 Punkte)\newline
    Sei \(K\) ein Körper der Charakteristik einer Primzahl \(p\).
    Sei \(L\) eine Körpererweiterung von \(K\). Sei \(x \in L\) mit \(K(x) = K(x^p)\).
    Konstruiere ein separables Polynom über \(K\), welches \(x\) als Nullstelle hat.
\end{exercise}

\begin{exercise}(4 Punkte)\newline
    Sei \(E = \mathbb F_3[X]/(X^3 + X^2 + 2)\). Schreibe \(X\) in $E$ als einen in \(X^3\)
    rationalen Ausdruck über \(\mathbb F_3\).
\end{exercise}


\begin{exercise}(5 Punkte)\newline
    Sei \(K\) ein faktorieller Körper. Begründe, warum \(K\) genau dann vollkommen
    ist, wenn jedes irreduzible Polynom über \(K\) separabel ist. Welche Richtung
    gilt noch, wenn wir nicht wissen, ob \(K\) faktoriell ist?
\end{exercise}

\begin{exercise}(4 Punkte)\newline
    Sei \(K_0\), \(K_1\), \(K_2\), \dots eine Folge von Körpern. Seien
    Körperhomomorphismen \(\phi_i\colon K_i \to K_{i + 1}\) gegeben, bezüglich
    derer der gerichtete Limes
    \begin{equation}
        L = \varinjlim_{i \in \mathbb N_0} K_i
    \end{equation}
    gebildet wird. Zeige, daß \(L\) ein Körper ist. 
   \newline    
    Wie kann \(L\) in natürlicher
    Weise als Körpererweiterung für alle Körper \(K_i\) aufgefaßt werden?
\end{exercise}

















\end{document}
