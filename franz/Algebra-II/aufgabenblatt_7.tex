\documentclass{algsheet}
    \usepackage{amssymb} 
\usepackage[protrusion=true,expansion=false]{microtype}
\usepackage{magros_3}
%%%%%%%%%%%%%%%%%%%%%%%%%%%%%%%%%%%%%%%%%%%%%%%%%%%%%%%%%%%%%%%%%%%%%%%%%%%%%%%%%%%%%%%%%%%%%%%%%%%%%%%%%%%%%%%%%%%%%%%%%%%%%%%%%%%%%%%%%%%%%%%%%%%%%%%%%%%%

   \lecture{Algebra II}
        
        \semester{Sommersemester 2011}
        \sheet{7.\ Aufgabenblatt}
        \author{Dipl.-Math.~Franz Vogler}
        \date{06.~Juni 2011}
        
        %/usr/local/share/texmf/tex/algmacros.sty
    \usepackage[arrow,curve,matrix]{xy}    
        \begin{document}
                \maketitle


\begin{exercise}\textbf{(Quickies)}(1+2+2+2+1+3 Punkte)\vspace{-1ex}
 \begin{itemize}
  \item [(Q1)]     Seien \(R\) ein Ring. Zeige, daß das direkte Produkt \(R \times 0\) von
    \(R\) mit dem Nullring
    als Ring kanonisch isomorph zum Ring \(R\) selbst ist.
    \item [(Q2)]     Sei \(R\) ein kommutativer Ring. Gibt es einen kanonischen Isomorphismus
    von Ringen zwischen \(R[X, Y]\) und \(R[X] \times R[Y]\)?
  \item [(Q3)]     Seien \(R\) ein kommutativer Ring und \(n\) eine natürliche Zahl. Zeige,
    daß die Diagonalmatrizen einen Unterring von \(\Mat_n(R)\) bilden, welcher
    als Ring isomorph zum direkten Produkte \(R^n = \underbrace{R \times \dotsb
    \times R}_n\) ist.
   \item [(Q4)]     Schreibe das Ideal \((3, 8, 9)\) in \(\mathbb Z\) als Hauptideal.
   \item [(Q5)]       Sei \(p\) eine Primzahl.
    Bestimme alle endlich erzeugten Ideale von \(\mathbb Z_p\).
   \item [(Q6)]      Bestimme die Nilradikale der kommutativen Ringe \(\mathbb Z/(n)\), \(n \in \mathbb N_0\)
    als Hauptideal.
 \end{itemize}
\end{exercise}



\begin{exercise} (6 Punkte)\newline
    Sei \(R\) ein kommutativer Ring. Zeige, daß folgende Aussagen äquivalent
    sind:
    \begin{enumerate}
    \item
        Es existieren nicht verschwindende Elemente \(e\), \(f\) in \(R\) mit
        \(e f = 0\), \(e^2 = e\), \(f^2 = f\) und \(e + f = 1\).
    \item
        Der Ring \(R\) ist als Ring isomorph zu einem Produkte \(S \times T\)
        kommutativer Ringe \(S\) und \(T\), die jeweils nicht der Nullring sind.
    \end{enumerate}
\end{exercise}


\begin{exercise}(3+3 Punkte)\vspace{-1ex}
  \begin{itemize}
   \item [(a)] Sei \(p\) eine Primzahl und \(\mathbb Z_p\) die Menge all derjenigen rationalen
    Zahlen, in deren vollständig gekürzter Bruchdarstellung der Nenner nicht durch
    \(p\) teilbar ist.
    Zeige, daß \(\mathbb Z_p\) ein Unterring von \(\mathbb Q\) ist und bestimme seine
    Einheitengruppe.
   \item [(b)]     Bestimme die Einheiten der ganzen Gaußschen Zahlen \(\mathbb Z[\iu]\).
  \end{itemize}
\end{exercise}



\begin{exercise}(4 Punkte)\newline
    Sei \(R\) ein Integritätsbereich. Sei \(n \ge 0\).
    Seien \(x_1\), \dots, \(x_n\) Elemente
    von \(R\) mit \(x_1 \dotsm x_n = 0\). Zeige, daß ein \(k\) mit \(x_k = 0\)
    existiert. Was ist im Falle von $n=0$ ?
\end{exercise}




\begin{exercise}(4+2+2+2 Punkte)\newline
    Sei \(\mathbb Q[\sin x, \cos x]\) der kleinste Unterring von
    von \(\mathbb R^{\mathbb R}\), welcher die Funktionen \(\sin x\) und \(\cos x\) 
    und die rationalen Zahlen (aufgefaßt als konstante Funktionen)
    enthält.
\vspace{-1ex}
    \begin{itemize}
  \item [(a)] 
    Zeige, daß \(\mathbb Q[\sin x, \cos x]\) genau aus den Funktionen
    \(f \in \mathbb R^{\mathbb R}\) besteht, welche in der Form
    \begin{equation}
        f(x) = a_0 + \sum_{m = 1}^n (a_m \cos (m x) + b_m \sin (m x))
    \end{equation}
    geschrieben werden können, wobei \(a_0\), \(a_1\), \(a_2\), \dots und
    \(b_1\), \(b_2\), \dots rationale Zahlen sind.
  \item [(b)] 
    Zeige, daß für jedes von Null verschiedene \(f \in \mathbb Q[\sin x, \cos x]\)
    eine eindeutig definierte natürliche Zahl \(\deg f = n\) existiert, so daß
    \begin{equation}
        f(x) = a_0 + \sum_{m = 1}^n (a_m \cos (m x) + b_m \sin (m x))
    \end{equation}
    mit \(a_n \neq 0\) oder \(b_n \neq 0\).
  \item [(c)] 
    Seien \(f\), \(g \in \mathbb Q[\sin x, \cos x]\). Zeige, daß \(\deg (f \cdot g)
    = (\deg f) + (\deg g)\), wenn wir zusätzlich \(\deg 0 = -\infty\) setzen.
  \item[(d)] 
    Zeige, daß \(\mathbb Q[\sin x, \cos x]\) ein Integritätsbereich ist. 
\end{itemize}
\end{exercise}



\begin{exercise}(5 Punkte)\newline
    Sei \(R\) ein kommutativer Ring, der genau zwei endlich erzeugte Ideale
    besitzt. Zeige, daß \(R\) ein Körper ist.
\end{exercise}


\begin{exercise}(4 Punkte)\newline
    Sei \(K\) ein Körper. Zeige, daß \(K\) genau dann von Charakteristik \(n \in \mathbb N\)
    ist, wenn der Kern des (einzigen) Ringhomomorphismus' \(\mathbb Z \to K\) durch
    das Hauptideal \((n)\) gegeben ist.
 \newline
    Was gilt in der Situation $n = 0$ ?
\end{exercise}

\begin{exercise}(5 Punkte)\newline
    Zeige, daß der Restklassenring \(\mathbb F_3[X]/(X^2 + 1)\) ein Körper mit
    \(9\) Elementen ist.
\end{exercise}



\begin{exercise}(5 Punkte)\newline
    Sei \(\phi\colon R \to S\) ein Homomorphismus von Ringen. Sei \(\ideal b\)
    ein Ideal von \(S\). Zeige, daß \(\phi^{-1} \ideal b\) ein Ideal von \(R\)
    ist. Für dieses Ideal schreiben wir auch häufig \(R \cap \ideal b\), wenn
    der Homomorphismus \(\phi\) aus dem Zusammenhange hervorgeht.
   \newline
   Ist das Bild eines Ideals unter einem Ringhomomorphismus wieder ein Ideal?
\end{exercise}



\end{document} 
