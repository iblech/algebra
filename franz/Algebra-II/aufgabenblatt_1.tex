\documentclass{algsheet}
    \usepackage{amssymb} 
 \usepackage{magros}
\usepackage[protrusion=true,expansion=false]{microtype}

%%%%%%%%%%%%%%%%%%%%%%%%%%%%%%%%%%%%%%%%%%%%%%%%%%%%%%%%%%%%%%%%%%%%%%%%%%%%%%%%%%%%%%%%%%%%%%%%%%%%%%%%%%%%%%%%%%%%%%%%%%%%%%%%%%%%%%%%%%%%%%%%%%%%%%%%%%%%

   \lecture{Algebra II}
        
        \semester{Sommersemester 2011}
        \sheet{1.\ Aufgabenblatt}
        \author{Dipl.-Math.~Franz Vogler}
        \date{02.~Mai 2011}
        
        %/usr/local/share/texmf/tex/algmacros.sty
    \usepackage[arrow,curve,matrix]{xy}    
        \begin{document}
                \maketitle


\noindent
Willkommen zur Algebra II im Sommersemester 2011!
\vspace{2ex}
 
Die Übungen beginnen DIESE WOCHE. Zusammen mit dem Übungsleiter werdet Ihr Aufgabenblatt 1 lösen. 
\newline
Aufgabenblatt 2 ist selbstständig zu bearbeiten und kann bis Montag, den 09. Mai, 14:00 Uhr in den Briefkasten im Erdgeschoss von Gebäude L1 mit der Aufschrift \glqq Algebra II\grqq\; eingeworfen werden. Der Abgabetermin für die Aufgaben der folgenden Übungsblätter ist jeweils der auf die Ausgabe folgende Montag bis 15:00 Uhr.   
\newline 
Vergesst bitte nicht Eure Abgaben DEUTLICH mit Eurem eigenen Namen, der Nummer Eurer Übungsgruppe und dem Namen Eures Übungsleiters zu versehen. Weitere Einzelheiten zum Ablauf der Übungen und der Bonuspunkteregelung geben die Tutoren in den Übungen bekannt.
\newline
Auf den kommenden Übungszetteln werden gewisse Aufgaben mit dem Zusatz 
\newline \glqq \textbf{(Staatsexamens-/Klausuraufgaben)}\grqq\; versehen. Diese Aufgaben sind im be\-sonderen Maße für das Staatsexamen in Algebra und die Klausur in Algebra II am Ende des Semesters relevant. Sie werden im Klausurenkurs in den Ferien besprochen werden.  
 

\begin{exercise}
    Gibt es auf der Menge aller rationaler Zahlen eine Gruppenstruktur, so daß
    die Gruppenverknüpfung die Multiplikation ist?
\end{exercise}

\begin{exercise}
    Finde zwei quadratische Matrizen \(A\), \(B\) gleicher Größe über den
    rationalen Zahlen, so daß \(A \cdot B \neq B \cdot A\).
\end{exercise}

\begin{exercise}
    Sei \(G\) eine Gruppe und \(e \in G\) ein Element, so daß \(e \cdot x = x\)
    für alle Elemente \(x\) aus \(G\) gilt. Zeige, daß dann schon \(e = 1\).
\end{exercise}

\begin{exercise}
    Sei \(G\) eine Gruppe. Seien \(a\) und \(b\) zwei Gruppenelemente. Gesucht
    seien Gruppenelemente \(x\), welche die Gleichung
    \begin{equation}
        a x = b
    \end{equation}
    erfüllen. Zeige, daß es genau eine Lösung \(x\) gibt.
\end{exercise}

\begin{exercise}
    Sei \(n\) eine ganze Zahl. Ist für eine allgemeine Gruppe \(G\) die
    Potenzabbildung
    \begin{equation}
        G \to G,\quad x \mapsto x^n
    \end{equation}
    ein Gruppenhomomorphismus?
\end{exercise}

\begin{exercise}
    Sei \(G\) eine Gruppe. Sei weiter \(y\) ein Element in \(G\). Zeige, daß
    die \emph{Konjugation} \(G \to G, x \mapsto y x y^{-1}\) ein
    Gruppenisomorphismus ist.
\end{exercise}

\begin{exercise}
    Sei \(G\) eine Gruppe mit der Eigenschaft, daß \(x^2 = 1\) für jedes
    Element \(x \in G\) gilt. Zeige, daß \(G\) eine abelsche Gruppe ist.
\end{exercise}

\begin{exercise}
    Seien \(G\) und \(H\) zwei zweielementige Gruppen. Zeige, daß \(G\) und
    \(H\) auf genau eine Art und Weise zueinander isomorph sind.
\end{exercise}

\begin{exercise}
    Gib einen kanonischen Isomorphismus zwischen \(\Aut(\CG_4)\) und \(\CG_2\)
    an.
\end{exercise}

\begin{exercise}
    Seien \(G\) und \(H\) zwei abelsche Gruppen. Zeige, daß dann auch
    \(G \times H\) eine abelsche Gruppe ist.
\end{exercise}

\begin{exercise}
    Wie gehabt, bezeichnen wir mit \(\mathrm C_2\) eine zyklische Gruppe der Ordnung
    \(2\) und mit \(C_4\) eine zyklische Ordnung der Ordnung \(4\). Zeigen Sie,
    daß es keinen Isomorphismus zwischen den Gruppen \(\mathrm C_2 \times\mathrm C_2\) und 
    \(\mathrm C_4\) gibt. Folgere, daß es nicht isomorphe endliche Gruppen gleicher
    Ordnung gibt.
\end{exercise}

\begin{exercise}
    Seien \(G\) und \(H\) zwei Gruppen. Seien \(g \in G\) und \(h \in H\).
    Zeige, daß
    \begin{equation}
        1, g h, g h g h, g h g h g h, \dotsc
    \end{equation}
    paarweise verschiedene Elemente von \(G \ast H\) sind, falls \(g \neq 1\)
    und \(h \neq 1\) gelten.
\end{exercise}

\begin{exercise}
    Sei \(G\) eine Gruppe. Gib einen kanonischen Gruppenisomorphismus von
    \(G \ast 1\) nach \(G\) an, wobei \(1\) die triviale Gruppe bezeichnet.
\end{exercise}

\begin{exercise}
    Zeige, daß \(1 1' (-1) (-1')\) in \(\set Z \ast \set Z\) nicht trivial ist.
\end{exercise}
\end{document} 
