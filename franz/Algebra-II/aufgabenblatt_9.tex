\documentclass{algsheet}
    \usepackage{amssymb} 
\usepackage[protrusion=true,expansion=false]{microtype}
 \usepackage{magros_3}
%%%%%%%%%%%%%%%%%%%%%%%%%%%%%%%%%%%%%%%%%%%%%%%%%%%%%%%%%%%%%%%%%%%%%%%%%%%%%%%%%%%%%%%%%%%%%%%%%%%%%%%%%%%%%%%%%%%%%%%%%%%%%%%%%%%%%%%%%%%%%%%%%%%%%%%%%%%%

   \lecture{Algebra II}
        
        \semester{Sommersemester 2011}
        \sheet{9.\ Aufgabenblatt}
        \author{Dipl.-Math.~Franz Vogler}
        \date{20.~Juni 2011}
        
        %/usr/local/share/texmf/tex/algmacros.sty
    \usepackage[arrow,curve,matrix]{xy}    
        \begin{document}
                \maketitle



\begin{exercise}\textbf{(Staatsexamens-/Klausuraufgaben)}(5+5+5+5+5+5 Zusatzpunkte)\vspace{-1ex}
\begin{itemize}
 \item [(a)] Sei \(R\) ein kommutativer Ring. Definiere den Begriff eines gerichteten
    Systems von \(R\)-Algebren.
 \item [(b)] Definiere den gerichteten Limes eines gerichteten Systems von $R$-Algebren.
 \item [(c)] Zeige, dass der gerichtete Limes $\varinjlim A_i$ eines gerichteten Systems $(A_i)_{i\in I}$ von $R$-Algebren wieder eine $R$-Algebra ist.
 \item [(d)] Beweise, daß der gerichtete Limes $\varinjlim A_i$ von $R$-Algebren folgende \emph{universelle Eigenschaft} erfüllt:
             \newline
             Zu jeder $R$-Algebra $B$ zusammen mit einer Familie von $R$-Algebrenhomomorphismen $\psi_i \colon A_i \rightarrow B$, welche für alle $i\leq j \in I$ der Gleichung $\psi_i = \psi_j\circ\phi_{ij}$ genügen ($\phi_{ij}$ bezeichnet hier den Strukturmorphismus des gerichteten Systems $(A_i)_{i\in I}$), gibt es genau einen $R$-Algebrenhomomorphismus $\psi \colon \varinjlim A_i \rightarrow B$.
 \item [(e)] Zeige: $\varinjlim A_i$ ist bis auf kanonische Isomorphie eindeutig.
 \item [(f)] Was bedeudet \glqq kanonisch\grqq\; in Aufgabenteil (e)? 
 \end{itemize}
\end{exercise}




\begin{exercise}(5 Punkte)\newline
   Sei \((R_i)_{i \in I}\) ein gerichtetes System von Ringen mit Limes
    \(R = \varinjlim_{i \in I} R_i\). Zeige, daß ein \(x \in R_i\) genau dann in
    \(R\) invertierbar ist, wenn ein \(j \ge i\) existiert, so daß \(x\) in \(R_j\)
    invertierbar ist.
\end{exercise}





\begin{exercise}(5 Punkte)\newline
  Zeige, daß jeder Ring gerichteter Limes endlich erzeugter \(\set Z\)-Algebren
  ist.
\end{exercise}


\begin{exercise}(5 Punkte)\newline
    Zeige, daß eine Menge \(X\) zusammen mit einer Ordnung genau dann gerichtet
    ist, wenn jede endliche Teilmenge von \(X\) in $X$ eine obere Schranke besitzt.
\end{exercise}


\begin{exercise}(7 Punkte)\newline
    Sei \(R\) ein kommutativer Ring. Wir sagen für eine natürliche Zahl \(r\), 
    daß eine Matrix \(A \in \Mat_{n, m}(R)\)
    \emph{Rang \(r\)} habe, wenn \((\Lambda^r(A)) = (1)\) und \((\Lambda^{r + 1}(A)) = (0)\)
    gelten.
    
    Habe eine Matrix \(A \in \Mat_{n, m}(R)\) Rang \(r\). Zeige dann, daß eine Zerlegung \(f_1\), \dots,
    \(f_N\) der Eins von \(R\) existiert, so daß für alle \(i \in \Set{1, \dotsc,
    N}\) die Matrix \(A\) über \(R[f_i^{-1}]\)
    (damit meinen wir das kanonische Bild von \(A\) in \(\Mat_{n, m}(R[f_i^{-1}])\)) ähnlich
    zu folgender Diagonalmatrix ist:
    \begin{equation}
        \begin{pmatrix}
            1    & 0      & \dots  \\
            0      & 1    & \ddots \\
            \vdots & \ddots & \ddots \\
                   &        &        & 1 \\
                   &        &        &      & 0 \\
                   &        &        &      &   & \ddots
        \end{pmatrix}
        \in \Mat_{n, m}(R[f_i^{-1}]).
    \end{equation}
    Hierbei stehen auf der Diagonalen genau \(r\) Stück Einsen.
        
    Wir können also über jedem Ring jede Matrix zumindest lokal in Gauß--Jordansche Normalform 
    bringen. Warum geht dies im allgemeinen nicht global?
\end{exercise}



\begin{exercise}(4 Punkte)\newline
    Zeige, daß \(3 + 2 \iu\) ein irreduzibles Element in \(\set Z[\iu]\) ist.
\end{exercise}


\begin{exercise}(5 Punkte)\newline
    Sei \(R\) ein Ring mit eindeutiger Primfaktorzerlegung. 
    Sei \(K\) sein Quotientenkörper. Wir fassen den Inhalt eines Polynoms
    über \(R\) als Element von \(K\) auf.
    Zeige, daß sich
    der Inhalt von Polynomen mit Koeffizienten in \(R\) auf genau eine Art und
    Weise auf Polynome mit Koeffizienten in \(K\) fortsetzen läßt, so daß sich
    der Inhalt wie in Hilfssatz 7.88 auf Seite 314 weiterhin multiplikativ verhält.
\end{exercise}

\begin{exercise}(5 Punkte)\newline
    Sei \(R\) ein Integritätsbereich. Sei \(f(X) = a_n X^n + a_{n - 1} X^{n - 1}
    + \dotsb + a_1 X + a_0\) ein Polynom, so daß \(1\) ein größter gemeinsamer
    Teiler aller Koeffizienten von \(f(X)\) ist. Sei \(p\) ein Primelement
    von \(R\), welches den Koeffizienten \(a_n\) nicht teilt, welches die
    Koeffizienten \(a_0\), \dots, \(a_{n - 1}\) teilt und den Koeffizienten
    \(a_0\) nicht im Quadrat teilt. Zeige, daß \(f(X)\) dann in \(R[X]\)
    irreduzibel ist.
\end{exercise}

\begin{exercise}(5 Punkte)\newline
    Sei \(R\) ein kommutativer Ring. Sei \(N\) eine natürliche Zahl. Zeige,
    daß die Einschränkung der Abbildung
    \begin{equation}  
        \phi\colon R[X, Y] \to R[X],\quad f \mapsto f(X, X^N)
    \end{equation}
    auf Polynome, deren Grad in \(X\) kleiner als \(N\) ist, injektiv ist.
\end{exercise}




\end{document} 
