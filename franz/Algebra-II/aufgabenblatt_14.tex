\documentclass{algsheet}
    \usepackage{amssymb} 
\usepackage[protrusion=true,expansion=false]{microtype}
 \usepackage{magros_3}
%%%%%%%%%%%%%%%%%%%%%%%%%%%%%%%%%%%%%%%%%%%%%%%%%%%%%%%%%%%%%%%%%%%%%%%%%%%%%%%%%%%%%%%%%%%%%%%%%%%%%%%%%%%%%%%%%%%%%%%%%%%%%%%%%%%%%%%%%%%%%%%%%%%%%%%%%%%%

   \lecture{Algebra II}
        
        \semester{Sommersemester 2011}
        \sheet{14.\ Aufgabenblatt}
        \author{Dipl.-Math.~Franz Vogler}
        \date{25.~Juli 2011}
        
        %/usr/local/share/texmf/tex/algmacros.sty
    \usepackage[arrow,curve,matrix]{xy}    
        \begin{document}
                \maketitle



\textbf{Abgabe: Aufgabenblatt 14 kann bis zum 24.~August um 14:00 Uhr in den Briefkasten eingeworfen werden.}
\newline
Die Besprechung der Aufgaben und die Rückgabe der Abgaben erfolgt im Klausurenkurs, welcher vom 29.~August bis 02.~September stattfinden wird. 






\begin{exercise}\textbf{(Quickies):}(5+5+5 Zusatzpunkte)\vspace{-1ex}
   \begin{itemize}
       \item[(Q1)]     Wieso ist
    \begin{equation}
        K^{p^{-\infty}}
    \end{equation}
    ein sinnvolles Symbol für den vollkommenen Abschluß eines Körpers positiver
    Charakteristik \(p\)?
       \item[(Q2)] Sei \(K\) ein Körper endlicher Charakteristik. Zeige, daß sein Primkörper der
    kleinste Unterkörper (bezüglich der Inklusionsrelation) von \(K\) ist.
       \item[(Q3)] Gibt es in einem Körper mit \(27\) Elementen einen Unterkörper mit \(9\)
    Elementen?
   \end{itemize}
\end{exercise}



\begin{exercise}(5 Punkte)\newline
    Zeige, daß in einem Körper \(K\) mit \(25\) Elementen eine Quadratwurzel
    \(\sqrt 2\) aus \(2\) existiert. Gib einen Erzeuger der multiplikativen Gruppe
    von \(K\) der Form \(a + b \sqrt 2\) an, wobei \(a\) und \(b \in \mathbb F_5\).
\end{exercise}


\begin{exercise}(6 Punkte)\newline
    Seien \(p\) eine Primzahl und \(n\) und \(d\) positive natürliche
    Zahlen. Sei \(q = p^n\). Sei \(L\) ein Körper mit \(q^d\) Elementen und
    \(K\) sein Unterkörper mit \(q\) Elementen. Zeige, daß die Gruppe der
    Automorphismen von \(L\) als \(K\)-Algebra von \(\mathrm{Frob}^n\) erzeugt wird und
    \(d\) Elemente besitzt.
\end{exercise}




\begin{exercise}\textbf{(Spezialistenaufgabe):}(10 Zusatzpunkte)\newline
    Sei \(R\) ein kommutativer Ring positiver Charakteristik \(p\), das heißt
    \(R\) ist nicht der Nullring und \(p = 0\) in \(R\). Sei
    \begin{equation}
        \varprojlim_{i \in \mathbb N_0} R^{p^i}
    \end{equation}
    die Menge aller Folgen \((x_0, x_1, x_2, \dotsc)\) mit \(x_i \in R\) und
    \(x_{i + 1}^p = x_i\) für alle \(i \in \mathbb N_0\). Durch gliedweise Addition
    und Multiplikation der Folgen wird \(E \coloneqq \varprojlim\limits_i R^{p^i}\) zu
    einem kommutativen Ring. Zeige, daß in \(E\) jedes Element eine \(p\)-te Wurzel
    besitzt.
    
    Zeige, daß unter der Voraussetzung, daß \(R\) ein Körper ist, der Ring \(E\)
    vermöge der Abbildung
    \begin{equation}
        (x_0, x_1, x_2, \dotsc) \mapsto x_0
    \end{equation}
    zu einem vollkommenen Unterkörper von \(R\) wird und mit denjenigen Elementen
    in \(R\) identifiziert werden kann, die alle \(q\)-ten Wurzel besitzen, wobei
    \(q\) eine beliebige \(p\)-Potenz ist.
    
    Im allgemeinen heißt der Ring \(E\) heißt die
    \emph{Vervollkommnung von \(R\)}.
\end{exercise}



\begin{exercise}(5 Punkte)\newline
    Gib die siebenten Wurzeln aller Elemente von \(\mathbb F_7\) an.
\end{exercise}





\begin{exercise}(6 Punkte)\newline
    Sei \(K\) ein Körper positiver Charakteristik \(p\). Zeige, daß \(K\) genau
    dann vollkommen ist, wenn der Frobenius von \(K\) ein Isomorphismus von \(K\)
    auf sich selbst ist.
\end{exercise}


\begin{exercise}(6 Punkte)\newline
    Sei \(L\) eine Körpererweiterung eines Körpers \(K\). Zeige, daß die Menge
    der über \(K\) rein inseparablen Elemente in \(L\) eine Zwischenerweiterung
    von \(L\) über \(K\) ist.
\end{exercise}


\begin{exercise}(6 Punkte)\newline
	Sei \(L\) eine endliche Körpererweiterung eines Körpers \(K\). Sei \(L\)
	über \(K\) sowohl separabel als auch rein inseparabel. Zeige, daß \(L = K\).
\end{exercise}

\begin{exercise}(6 Punkte)\newline
	Sei \(L\) über \(K\) eine endliche Körpererweiterung. Sei \(x \in L\) separabel
	über \(K\) und \(y \in L\) rein inseparabel über \(K\). Zeige, daß
	\(K(x, y) = K(x + y)\).
\end{exercise}



\begin{exercise}\textbf{(Staatsexamens-/Klausuraufgaben)}(10+10 Zusatzpunkte)\vspace{-1ex}
   \begin{itemize}
       \item[(a)] Sei \(L\) über \(K\) eine endliche Körpererweiterung. Die \emph{Norm}
	\begin{equation}
		N_{L/K}(x)
	\end{equation}
	eines Elementes \(x\) in \(L\) über \(K\) ist als die Determinante der
	\(K\)-linearen Abbildung von \(L\) nach \(L\) definiert, die durch Multiplikation mit
	\(x\) gegeben ist. Zeige, daß
	\begin{equation}
		N_{L/K}(x) = \left(\prod_{i = 1}^{[L : K]_\sep} x_i\right)^{[L : K]_\mathrm i},
	\end{equation}
	wobei die \(x_i\) die verschiedenen galoisschen Konjugierten von \(x\) in
	einem algebraisch abgeschlossenen Oberkörper \(\Omega\) von \(K\) sind.
      
       \item[(b)] Sei \(L\) über \(K\) eine endliche Körpererweiterung. Sei \(E\) ein über \(K\)
	endlicher Zwischenkörper. Zeige, daß
	\begin{equation}
		\mathrm{disc}_{L/K} = N_{E/K}(\mathrm{disc}_{L/E}) \cdot \mathrm{disc}_{E/K}^{[L : E]}.
	\end{equation}
   \end{itemize}
\end{exercise}



\end{document}