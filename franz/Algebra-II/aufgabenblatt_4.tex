\documentclass{algsheet}
    \usepackage{amssymb} 
    \usepackage{magros_3}
\usepackage[protrusion=true,expansion=false]{microtype}

%%%%%%%%%%%%%%%%%%%%%%%%%%%%%%%%%%%%%%%%%%%%%%%%%%%%%%%%%%%%%%%%%%%%%%%%%%%%%%%%%%%%%%%%%%%%%%%%%%%%%%%%%%%%%%%%%%%%%%%%%%%%%%%%%%%%%%%%%%%%%%%%%%%%%%%%%%%%

   \lecture{Algebra II}
        
        \semester{Sommersemester 2011}
        \sheet{4.\ Aufgabenblatt}
        \author{Dipl.-Math.~Franz Vogler}
        \date{16.~Mai 2011}
        
        %/usr/local/share/texmf/tex/algmacros.sty
    \usepackage[arrow,curve,matrix]{xy}    
        \begin{document}
                \maketitle


\begin{exercise}\textbf{(Quickies)} (2+2+3 Punkte)\vspace{-1ex}
\begin{itemize}
 \item [\textbf{(Q1)}]   Sei \(n \ge 3\).
    Gib einen Isomorphismus von der Dieder-Gruppe \(\DG_n\) zu einer Gruppe mit
    zwei Erzeugern und zwei Relationen an.
\item [\textbf{(Q2)}]     Zeige, daß jedes direkte Produkt zweier Gruppen auch als halbdirektes
    Produkt angesehen werden kann.
\item [\textbf{(Q3)}]  Eine endliche Gruppe wirke auf einer endlichen Menge. Ist die Länge einer
    Bahn der Operation immer ein Teiler der Gruppenordnung?
\end{itemize}
\end{exercise}



\begin{exercise}(5+4 Punkte)\vspace{-1ex}
\begin{itemize}
 \item [\textbf{(1)}]  Seien \(G\) eine Gruppe, \(N\) ein Normalteiler in \(G\) und \(H\) eine
    beliebige Untergruppe von \(G\). Jedes Element von \(G\) lasse sich als
    Produkt \(n h\) mit \(n \in N\) und \(h \in H\) darstellen. Schließlich
    sei \(N \cap H\) die triviale Gruppe. Gib eine Wirkung von \(H\) auf
    \(N\) an, so daß für das diesbezüglich konstruierte halbdirekte Produkt
    \(N \rtimes H\) gilt, daß
    \begin{equation}
        N \rtimes H \to G,\quad (n, h) \mapsto n h
    \end{equation}
    ein Gruppenisomorphismus ist.
\item [\textbf{(2)}]     Sei \(n \ge 1\). Zeige, daß die orthogonale Gruppe \(\OG_n(\mathbb R)\) isomorph
    zu einem halbdirekten Produkte von \(\SO_n(\mathbb R)\) mit \(\CG_2\) ist.
\end{itemize}    
\end{exercise}


\begin{exercise}(5 Punkte)\newline
    Zeige, daß
    \begin{equation}
        \mathbb R^3 \rtimes \SO_3(\mathbb R) \to \GL_4(\mathbb R),\quad
        (b, A) \mapsto \left(\begin{array}{ccc|c}
            \multicolumn{3}{c|}{A} & b \\
            \hline
            0 & 0 & 0 & 1
        \end{array}\right)
    \end{equation}
    ein injektiver Gruppenhomomorphismus von der Galileischen Gruppe in die allgemeine lineare Gruppe ist.
    Wir können die Galileische Gruppe also auch als Matrizengruppe auffassen.
\end{exercise}



\begin{exercise}(3+4+8 Punkte)\vspace{-1ex}
\begin{itemize}
  \item [($\mathfrak a$)]
    Sei \(f\colon G \to H\) ein Homomorphismus von Gruppen. Sei \(K\) eine
    Untergruppe von \(H\). Zeige, daß das das Urbild \(f^{-1}(K)\) eine Untergruppe
    von \(G\) ist.
\item [($\mathfrak b$)]     Sei \(f\colon G \to H\) ein Homomorphismus von Gruppen. Sei \(N\) ein
    Normalteiler von \(H\). Zeige, daß das das Urbild \(f^{-1}(N)\) ein
    Normalteiler von \(G\) ist.
\item [($\mathfrak c$)]     Seien \(G\) eine endliche Gruppe und \(N\) ein endlicher
    Normalteiler in \(G\). Zeige, daß \(G\) genau dann auflösbar ist, wenn
    \(G/N\) und \(N\) auflösbare Gruppen sind.
\end{itemize}
\end{exercise}



\begin{exercise}(5 Punkte)\newline
    Zeige, daß jede endliche \(p\)-Gruppe (also jede endliche Gruppe, deren
    Ordnung eine Primpotenz ist), auflösbar ist.
\end{exercise}

\begin{exercise}(5 Punkte)\newline
    Sei \(G\) ein endliche Gruppe. Zeige, daß ein größter endlicher auflösbarer
    Normalteiler \(N\) von \(G\) existiert (das heißt jeder endliche auflösbare
    Normalteiler von \(G\) liegt in \(N\)).
\end{exercise}

\begin{exercise}(8 Punkte)\newline
    Zeige: Sind \(N\) und \(N'\) zwei auflösbare Normalteiler einer Gruppe \(G\),
    so ist auch \[N \cdot N' = \Set{n n' \mid n \in N, n' \in N'}\] ein auflösbarer
    Normalteiler von \(G\).
\end{exercise}



	



\end{document} 
