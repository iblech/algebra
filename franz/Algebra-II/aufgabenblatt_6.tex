\documentclass{algsheet}
    \usepackage{amssymb} 
\usepackage[protrusion=true,expansion=false]{microtype}
   \usepackage{magros_3}
%%%%%%%%%%%%%%%%%%%%%%%%%%%%%%%%%%%%%%%%%%%%%%%%%%%%%%%%%%%%%%%%%%%%%%%%%%%%%%%%%%%%%%%%%%%%%%%%%%%%%%%%%%%%%%%%%%%%%%%%%%%%%%%%%%%%%%%%%%%%%%%%%%%%%%%%%%%%

   \lecture{Algebra II}
        
        \semester{Sommersemester 2011}
        \sheet{6.\ Aufgabenblatt}
        \author{Dipl.-Math.~Franz Vogler}
        \date{30.~Mai 2011}
        
        %/usr/local/share/texmf/tex/algmacros.sty
    \usepackage[arrow,curve,matrix]{xy}    
        \begin{document}
                \maketitle



\begin{exercise}(5 Punkte)\newline
    Bestimme die Elementarteiler der Matrix
    \begin{equation}
      M =  \begin{pmatrix}
            2 & 6 & 8 \\
            3 & 1 & 2 \\
            9 & 5 & 4
        \end{pmatrix}.
    \end{equation}
\end{exercise}

\begin{exercise}(2+3 Punkte)\vspace{-1ex}
   \begin{itemize}
    \item [(a)] Geben Sie bis auf Isomorphie alle endlichen abelschen Gruppen der Ordnung
    \(24\) an.
    \item [(b)] Geben Sie bis auf Isomorphie alle endlichen abelschen Gruppen der Ordnung
    \(180\) an.
   \end{itemize}
\end{exercise}



\begin{exercise}(5 Punkte)\newline
    Sei \(A\) eine endliche abelsche Gruppe. Für jede Primzahl \(p\) sei
    \(A[p^\infty]\) die Untergruppe (!) derjenigen Elemente von \(A\), deren
    Ordnung eine \(p\)-Potenz ist. Ist diese Untergruppe nicht trivial, heißt \(p\)
    assoziierte Primzahl zu \(A\) und \(A[p^\infty]\) die \(p\)-primäre
    Komponente von \(A\). Zeige, daß \(A\) nur endliche viele assoziierte Primzahlen
    besitzt und daß \(A\) isomorph zum direkten Produkte ihrer \(p\)-primären
    Komponenten ist.
\end{exercise}

\begin{exercise}(5 Punkte)\newline
    Zeige die Eindeutigkeit der Darstellung in Proposition 6.115 auf Seite 281 des Vor\-lesungsskriptes.
\end{exercise}

\begin{exercise}(5 Punkte)\newline
    Sei \(A\) eine \((n \times m)\)-Matrix mit ganzzahligen Einträgen, deren
    Elementarteiler durch \(d_1\), \(d_2\), \dots, \(d_r\) mit \(d_{i - 1} \mid
    d_i\) gegeben seien. Sei \(i \ge 1\).
    Mit \(\lambda_i\) bezeichnen wir die größten
    gemeinsamen Teiler aller \(i\)-Minoren (das heißt Determinanten von
    \((i \times i)\)-Untermatrizen) von \(A\). Zeige, daß \(\lambda_i = d_1
    d_2 \dotsm d_i\).
\end{exercise}


\newpage

\begin{exercise}\textbf{(Quickies)}(2+2+2 Punkte)\vspace{-1ex}
 \begin{itemize}
  \item [(Q1)]  Sei \(R\) ein kommutativer Ring, welcher nicht der Nullring ist. Sei
    \(n \ge 1\) eine natürliche Zahl. Zeige,
    daß \(\Mat_n(R)\) genau für \(n = 1\) ein kommutativer Ring ist.
  \item [(Q2)] 
    Mit \(\mathbb R^{\mathbb R}\) bezeichnen wir die Menge aller Funktionen von
    \(\mathbb R\) nach \(\mathbb R\). Zeige, daß \(\mathbb R^{\mathbb R}\) zu einem Ring
    wird, wenn wir die Addition und Multiplikation durch die übliche Addition
    und Multiplikation von Funktionen definieren, das heißt durch
    \begin{align}
        f + g&\colon x \mapsto f(x) + g(x),& \\
        f \cdot g&\colon x \mapsto f(x) \cdot g(x)
    \end{align}
    definieren.
  \item [(Q3)] 
    Sei \(R\) ein kommutativer Ring. Sei \(S\) eine \(R\)-Algebra. Zeige, daß
    \begin{equation}
        \phi\colon R \to S,\quad x \mapsto x \cdot 1
    \end{equation}
    ein Ringhomorphismus ist.
 \end{itemize}
\end{exercise}



\begin{exercise}(4+4+6 Punkte)\vspace{-1ex}
 \begin{itemize}
  \item [(i)] 
    Zeige, daß \(\mathcal O_{\mathbb Q(\iu)} = \mathbb Z[\iu]\).
  \item [(ii)] 
    Zeige, daß \(\mathcal O_{\mathbb Q(\sqrt{-3})} = \mathbb Z[\upzeta_3]\).
  \item [(iii)] 
    Sei \(n \ge 1\).
    Sei \(\zeta\) eine primitive \(n\)-te Einheitswurzel.
    Zeige, daß
    \begin{equation}
        \mathcal O_{\mathbb Q(\zeta)} = \mathbb Z[\zeta] = \Set{a_0 + a_1 \zeta
        + \dotsb + a_{n - 1} \zeta^{n - 1} \vert a_0, \dotsc, a_{n - 1} \in
        \mathbb Z}.
    \end{equation}
 \end{itemize}
\end{exercise}


\end{document} 
