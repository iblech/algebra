\documentclass{algsheet}
    \usepackage{amssymb} 
\usepackage[protrusion=true,expansion=false]{microtype}
\usepackage{magros_3}
%%%%%%%%%%%%%%%%%%%%%%%%%%%%%%%%%%%%%%%%%%%%%%%%%%%%%%%%%%%%%%%%%%%%%%%%%%%%%%%%%%%%%%%%%%%%%%%%%%%%%%%%%%%%%%%%%%%%%%%%%%%%%%%%%%%%%%%%%%%%%%%%%%%%%%%%%%%%

   \lecture{Algebra II}
        
        \semester{Sommersemester 2011}
        \sheet{8.\ Aufgabenblatt}
        \author{Dipl.-Math.~Franz Vogler}
        \date{15.~Juni 2011}
        
        %/usr/local/share/texmf/tex/algmacros.sty
    \usepackage[arrow,curve,matrix]{xy}    
        \begin{document}
                \maketitle



\begin{exercise}\textbf{(Quickies)}(3+3 Punkte)\vspace{-1ex}
   \begin{itemize}
    \item [\textbf{(Q1)}]    Zeige, daß der Restklassenring \(\mathbb Z[\iu]/(2)\) genau vier Elemente hat.
    Welche Elemente davon sind regulär?
    \item [\textbf{(Q2)}]    Sei \(R\) ein kommutativer Ring. Gib für eine Einheit \(f \in R\) einen
    kanonischen Ringisomorphismus zwischen \(R\) und \(R[f^{-1}]\) an.
   \end{itemize}
\end{exercise}


\begin{exercise}(5 Punkte)\newline
    Sei \(R\) ein kommutativer Ring. Zeige für ein Element \(f \in R\), daß
    \(R[f^{-1}]\) genau dann der Nullring ist, wenn \(f\) in \(R\) nilpotent ist.
\end{exercise}

\begin{exercise}(5 Punkte)\newline
    Sei \(R\) ein kommutativer Ring. Seien \(f\) ein Element in \(R\) und
    \(n\) eine positive natürliche Zahl.
    Zeige, daß \(R[f^{-1}]\) und \(R[f^{-n}] = R[(f^n)^{-1}]\) kanonisch isomorph
    sind.
\end{exercise}

\begin{exercise}(3+3 Punkte)\vspace{-1ex}
     \begin{itemize}
    \item [\textbf{(a)}]    Sei \(s_1\), \dots, \(s_n\) eine Zerlegung der Eins eines kommutativen Ringes
    \(R\). Zeige, daß zwei Elemente \(f\) und \(g\) von \(R\) genau dann gleich
    sind, wenn sie \emph{lokal gleich} sind, das heißt, wenn
    \(f = g\) in \(R[s_i^{-1}]\) für alle \(i \in \Set{1, \dotsc, n}\) gilt.
  \item [\textbf{(b)}]    Sei \(s_1\), \dots, \(s_n\) eine Zerlegung der Eins eines kommutativen Ringes
    \(R\). Zeige, daß ein Element \(f\) in \(R\) genau dann invertierbar ist,
    wenn es \emph{lokal invertierbar} ist, das heißt, wenn
    \(f\) in \(R[s_i^{-1}]\) für alle \(i \in \Set{1, \dotsc, n}\) invertierbar 
    ist.
   \end{itemize}
\end{exercise}



\newpage


\begin{exercise}(3+3 Punkte)\vspace{-1ex}
     \begin{itemize}
    \item [\textbf{(I)}]     Sei \(R\) ein kommutativer Ring und \(S\) eine multiplikativ abgeschlossene
    Teilmenge von \(R\). Wo gibt es Probleme, wenn die Gleichheit zweier
    Brüche \(\frac a s\) und \(\frac b t\) nach \(S\) einfach durch \(at = bs\)
    definiert wird?
    \item [\textbf{(II)}]  Warum ist es nicht so einfach, Lokalisierungen nicht kommutativer Ringe zu
    definieren?
   \end{itemize}
\end{exercise}







\begin{exercise}(4+4 Punkte)\vspace{-1ex}
\begin{itemize}
 \item [(a)] Sei \(R\) ein kommutativer Ring. Sei \(A \in \Mat_{n, m}(R)\) eine Matrix über
    \(R\). Mit \((\Lambda^k(A))\) bezeichnen wir das von den \(k\)-Minoren von
    \(A\) (das heißt den Determinanten von \((k \times k)\)-Untermatrizen)
    erzeugte Ideal in \(R\). Zeige, daß sich \((\Lambda^k(A))\) nicht ändert,
    wenn \(A\) durch eine zu \(A\) ähnliche Matrix ersetzt wird.
 \item [(b)] Sei \(K\) ein Körper. Sei \(A \in \Mat_{n, m}(K)\) eine Matrix über \(K\).
    Zeige, daß \(A\) genau dann Rang \(r\) hat, wenn \((\Lambda^r(A)) = (1)\)
    und \((\Lambda^{r + 1}(A)) = (0)\) gilt.
\end{itemize}
\end{exercise}







\end{document} 
