\documentclass{algsheet}
    \usepackage{amssymb} 
\usepackage[protrusion=true,expansion=false]{microtype}
 \usepackage{magros_3}
%%%%%%%%%%%%%%%%%%%%%%%%%%%%%%%%%%%%%%%%%%%%%%%%%%%%%%%%%%%%%%%%%%%%%%%%%%%%%%%%%%%%%%%%%%%%%%%%%%%%%%%%%%%%%%%%%%%%%%%%%%%%%%%%%%%%%%%%%%%%%%%%%%%%%%%%%%%%

   \lecture{Algebra II}
        
        \semester{Sommersemester 2011}
        \sheet{10.\ Aufgabenblatt}
        \author{Dipl.-Math.~Franz Vogler}
        \date{27.~Juni 2011}
        
        %/usr/local/share/texmf/tex/algmacros.sty
    \usepackage[arrow,curve,matrix]{xy}    
        \begin{document}
                \maketitle


\begin{exercise}(3+3 Punkte)\vspace{-1ex}
  \begin{itemize}
   \item [(a)]     Bestimme die Primfaktorzerlegung von \(X^4 + 4 Y^4\) im Ringe \(\set Z[X, Y]\).

   \item [(b)]      Zeige, daß das Polynom \(X^2 + Y\) im Ringe \(\set Z[X, Y]\) irreduzibel ist.

  \end{itemize}
\end{exercise}



\begin{exercise}(3+3 Punkte)\vspace{-1ex}
  \begin{itemize}
   \item [(i)]      Sei \(R\) ein Ring mit eindeutiger Primfaktorzerlegung. Sei \(f \in R\)
    ein reguläres Element. Zeige, daß \(R[f^{-1}]\) ein Ring mit eindeutiger
    Primfaktorzerlegung ist.
   \item [(ii)]       Sei \(R\) ein faktorieller Ring. Sei \(f \in R\) ein reguläres Element.
    Zeige, daß \(R[f^{-1}]\) ein faktorieller Ring mit eindeutiger Primfaktorzerlegung ist.
  \end{itemize}
\end{exercise}


\begin{exercise}(5 Punkte)\newline
    Sei \(I\) ein echtes Ideal in einem kommutativen Ringe \(R\) (\emph{echt}
    heißt dabei, daß \(1 \notin I\), das heißt, \(I\) ist eine echte Teilmenge
    von \(R\)). Sei \(f(X)\) ein normiertes Polynom über \(R\), so daß das
    Bild von \(f(X)\) unter dem kanonischen Ringhomomorphismus
    \begin{equation}
        R[X] \to R/I[X],\quad f(X) \mapsto f(X)
    \end{equation}
    irreduzibel ist. Zeigen Sie, daß dann auch \(f(X) \in R[X]\) irreduzibel
    ist.
\end{exercise}


\begin{exercise}(3 Punkte)\newline
    Sei \(R\) ein Integritätsbereich. Zeige, daß das Ideal \((X, Y)\) in \(R[X, Y]\)
    nicht von einem Elemente erzeugt werden kann.
\end{exercise}

\begin{exercise}(3 Punkte)\newline
    Bestimme eine teilweise Faktorisierung der drei ganzen Zahlen
    \(99\), \(1200\) und \(160\).
\end{exercise}

\begin{exercise}(5 Punkte)\newline
    Bestimme einen größten gemeinsamen Teiler der beiden Polynome
    \begin{equation}
      f(X,Y) = X^3 Y^2 - X^2 Y^3 + X Y^3 - Y^4
    \end{equation}
    und
    \begin{equation}
      g(X,Y) = X^4 Y - X^3 Y^2 - X^2 Y^2 + X Y^3
    \end{equation}
    im Ringe \(\set Q[X, Y]\). 
\end{exercise}

\begin{exercise}(6 Punkte)\newline
    Sei \(R = \set Z[Y, X_1, X_2, \dotsc]/I\), wobei \(I\) das durch alle
    Linearkombinationen von \(X_{i + 1} Y - X_i\) mit \(i \ge 1\) gebildete 
    Ideal ist. Zeige, daß \(R\) ein Ring mit größten gemeinsamen Teilern ist,
    welcher nicht die aufsteigende Kettenbedingung für Hauptideale erfüllt.
    Zeige weiter, daß keine teilweise Faktorisierung von \(Y\) und \(X_1\)
    existiert.
\end{exercise}

\begin{exercise}(3 Punkte)\newline
  Seien \(a\), \(b\) und \(c\) drei Elemente in einem Ringe mit größten gemeinsamen
  Teilern. Es teile \(a\) das Produkt von \(b\) und \(c\), und es sei \(1\) ein
  größter gemeinsamer Teiler von \(a\) und \(b\). Zeige, daß dann \(a\) das Element
  \(c\) teilt.
\end{exercise}

\begin{exercise}(4 Punkte)\newline
  Sei \(R\) ein Integritätsbereich, in dem eine teilweise Primfaktorzerlegung
  immer möglich ist. Zeige, daß \(R\) ein Ring mit größten gemeinsamen Teilern ist.
\end{exercise}

\begin{exercise}(4 Punkte)\newline
  Sei \(R\) ein Integritätsbereich, in dem eine teilweise Primfaktorzerlegung
  immer möglich ist. Zeige, daß \(R\) ein Ring mit eindeutiger Primfaktorzerlegung
  ist, wenn wir annehmen können, daß wir einen Irreduzibilitätstest für 
  Elemente aus \(R\) haben.
\end{exercise}

\begin{exercise}\textbf{(Staatsexamens-/Klausuraufgaben)}(6 Zusatzpunkte)\newline
  Seien \(a\) und \(b\) ganze Zahlen. Sei \(\omega\) eine primitive dritte Einheitswurzel.
  Definiere
  \begin{equation}
    N(a + b \, \omega) = a^2 - ab + b^2.
  \end{equation}
  Zeige, daß \(\mathcal O_{\set Q(\omega)} = \set Z[\omega]\) zusammen mit der
  Abbildung \(N\) als Norm ein Euklidischer Ring ist.
\end{exercise}

\begin{exercise}(4 Punkte)\newline
  Seien \(a\) und \(b\) ganze Zahlen. Definiere
  \begin{equation}
    N(a + b \sqrt{-5}) \coloneqq a^2 + 5 b^2.
  \end{equation}
  Zeige, daß \(\set Z[\sqrt{-5}]\) zusammen mit der Abbildung \(N\) als Norm
  kein Euklidischer Ring ist.
\end{exercise}


\end{document}