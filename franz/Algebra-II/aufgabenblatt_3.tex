\documentclass{algsheet}
    \usepackage{amssymb} 
          \usepackage{magros_3}
\usepackage[protrusion=true,expansion=false]{microtype}

%%%%%%%%%%%%%%%%%%%%%%%%%%%%%%%%%%%%%%%%%%%%%%%%%%%%%%%%%%%%%%%%%%%%%%%%%%%%%%%%%%%%%%%%%%%%%%%%%%%%%%%%%%%%%%%%%%%%%%%%%%%%%%%%%%%%%%%%%%%%%%%%%%%%%%%%%%%%

   \lecture{Algebra II}
        
        \semester{Sommersemester 2011}
        \sheet{3.\ Aufgabenblatt}
        \author{Dipl.-Math.~Franz Vogler}
        \date{09.~Mai 2011}
        
        %/usr/local/share/texmf/tex/algmacros.sty
    \usepackage[arrow,curve,matrix]{xy}    
        \begin{document}
                \maketitle


\begin{exercise}(3+3 Punkte)\newline
    Seien \(n\) und \(m\) zwei natürliche Zahlen, deren Minimum mit \(k\)
    bezeichnet sei. 
\begin{itemize}
  \item [(a)] Zeige, daß die Gruppe
    \(G \coloneqq \GL_n(\mathbb Q) \times \GL_m(\mathbb Q)\) vermöge
    \begin{equation}
        G \times \Mat_{n, m}(\mathbb Q) \to \Mat_{n, m}(\mathbb Q),\quad
        ((S, T), A) \mapsto S A T^{-1}
    \end{equation}
    auf der Menge der \((n \times m)\)-Matrizen wirkt.
 \item [(b)] Zeige, daß durch den Rang
    eine wohldefinierte Bijektion
    \begin{equation}
        \rk\colon G \backslash \Mat_{n, m}(\mathbb Q) \to \Set{0, \dotsc, k},
        \quad A \mapsto \rk A
    \end{equation}
    gegeben wird.
\end{itemize}
\end{exercise}






\begin{exercise}(4 Punkte)\newline
    Wirke eine Gruppe \(G\) auf einer Menge \(X\). Zeige, daß die Wirkung von
    \(G\) auf \(X\) genau dann frei ist, wenn die Abbildung
    \begin{equation}    
        G \times X \to X \times X,\quad
        (g, x) \mapsto (gx, x)
    \end{equation}
    injektiv ist.
\end{exercise}




\begin{exercise}(5 Punkte)\newline
    Wirke eine endliche Gruppe auf einer endlichen Menge \(X\). Zeige, daß
    endliche Untergruppen \(H_1\), \dots, \(H_n\) von \(G\) und ein Isomorphismus
    \begin{equation}
        G/H_1 \amalg \dotsb \amalg G/H_n \to X
    \end{equation}
    existieren, wobei jeder einzelne Summand der disjunkten Vereinigung auf der
    linken Seite durch Linkstranslation eine \(G\)-Wirkung bekommt.
\end{exercise}

\begin{exercise}(4 Punkte)\newline
    Eine endliche Gruppe der Ordnung \(91\) wirke auf einer endlichen
    Menge mit \(71\) Elementen.
    Zeige, daß die Operation mindestens einen Fixpunkt hat.
\end{exercise}



\begin{exercise}(4+4 Punkte)\vspace{-1ex}
\begin{itemize}
    \item [($\alpha$)] Sei \(G\) eine Gruppe. Sei \((N_i)_{i \in I}\) eine Familie normaler
    Untergruppen von \(G\). Zeige, daß \(N \coloneqq \bigcap\limits_{i \in I}
    N_i\) wieder ein Normalteiler von \(G\) ist.
\item [($\beta$)]     Sei \(G\) eine Gruppe. Sei \(H\) eine Untergruppe von \(G\). Sei
    \((N_i)_{i \in I}\) die Familie aller Normalteiler von \(G\), welche
    \(H\) enthalten. Zeige, daß \(\bigcap\limits_{i \in I} N_i\) der
    normale Abschluß von \(H\) in \(G\) ist.
\end{itemize}
\end{exercise}

\begin{exercise}(6 Punkte)\newline
    Gib einen ausführlichen Beweis für Proposition 6.67 auf Seite 251 des Vorlesungsskriptes.
\end{exercise}

\begin{exercise}\textbf{(Quickies)} (2+2+3 Punkte)\vspace{-1ex}
\begin{itemize}
 \item [\textbf{(Q1)}]   Sei \(n \ge 3\).
    Gib einen Isomorphismus von der Dieder-Gruppe \(\DG_n\) zu einer Gruppe mit
    zwei Erzeugern und zwei Relationen an.
\item [\textbf{(Q2)}]     Zeige, daß jedes direkte Produkt zweier Gruppen auch als halbdirektes
    Produkt angesehen werden kann.
\item [\textbf{(Q3)}]  Eine endliche Gruppe wirke auf einer endlichen Menge. Ist die Länge einer
    Bahn der Operation immer ein Teiler der Gruppenordnung?
\end{itemize}
\end{exercise}



\begin{exercise}(5 Punkte)\newline
    Seien \(G\) eine Gruppe, \(N\) ein Normalteiler in \(G\) und \(H\) eine
    beliebige Untergruppe von \(G\). Jedes Element von \(G\) lasse sich als
    Produkt \(n h\) mit \(n \in N\) und \(h \in H\) darstellen. Schließlich
    sei \(N \cap H\) die triviale Gruppe. Gib eine Wirkung von \(H\) auf
    \(N\) an, so daß für das diesbezüglich konstruierte halbdirekte Produkt
    \(N \rtimes H\) gilt, daß
    \begin{equation}
        N \rtimes H \to G,\quad (n, h) \mapsto n h
    \end{equation}
    ein Gruppenisomorphismus ist.
\end{exercise}

\begin{exercise}(4 Punkte)\newline
    Sei \(n \ge 1\). Zeige, daß die orthogonale Gruppe \(\OG_n(\mathbb R)\) isomorph
    zu einem halbdirekten Produkte von \(\SO_n(\mathbb R)\) mit \(\CG_2\) ist.
\end{exercise}



\end{document} 
