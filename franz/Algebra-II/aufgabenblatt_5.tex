\documentclass{algsheet}
    \usepackage{amssymb} 
    \usepackage{magros_3}
\usepackage[protrusion=true,expansion=false]{microtype}

%%%%%%%%%%%%%%%%%%%%%%%%%%%%%%%%%%%%%%%%%%%%%%%%%%%%%%%%%%%%%%%%%%%%%%%%%%%%%%%%%%%%%%%%%%%%%%%%%%%%%%%%%%%%%%%%%%%%%%%%%%%%%%%%%%%%%%%%%%%%%%%%%%%%%%%%%%%%

   \lecture{Algebra II}
        
        \semester{Sommersemester 2011}
        \sheet{5.\ Aufgabenblatt}
        \author{Dipl.-Math.~Franz Vogler}
        \date{23.~Mai 2011}
        
        %/usr/local/share/texmf/tex/algmacros.sty
    \usepackage[arrow,curve,matrix]{xy}    
        \begin{document}
                \maketitle





\begin{exercise}(4 Punkte)\newline
    Sei \(p\) eine Primzahl.
    Seien \(G\) eine endliche Gruppe und \(H \subseteq K \subseteq G\) endliche
    Untergruppen. Zeige: Ist \(H\) eine Sylowsche \(p\)-Untergruppe zu \(G\), so
    ist \(H\) auch eine Sylowsche \(p\)-Untergruppe zu \(K\).
\end{exercise}

\begin{exercise}(7 Punkte)\newline
    Sei \(p\) eine Primzahl. Für eine endliche abelsche Gruppe \(G\) sei \(H\) diejenige
    endliche Teilmenge, die aus all jenen Elementen von \(G\) besteht, deren
    Ordnung eine \(p\)-Potenz ist. Zeige, daß \(H\) die einzige Sylowsche
    \(p\)-Untergruppe von \(G\) ist.
\end{exercise}

\begin{exercise}(6 Punkte)\newline
    Gib alle Sylowschen Untergruppen der alternierenden Gruppe \(\AG_4\) an.
\end{exercise}

\begin{exercise}(5 Punkte)\newline
    Zeige, daß jede endliche Gruppe der Ordnung \(30\) einen nicht trivialen
    Sylowschen Normalteiler besitzt.
\end{exercise}

\begin{exercise}(5 Punkte)\newline
    Zeige, daß jede endliche Gruppe der Ordnung \(56\) einen nicht trivialen
    Sylowschen Normalteiler besitzt.
\end{exercise}

\begin{exercise}(8 Punkte)\newline
    Jede endliche Gruppe der Ordnung \(36\) ist nicht einfach.
\end{exercise}

\begin{exercise}(7 Punkte)\newline
    Es seien \(p\) und \(q\) zwei Primzahlen mit \(p < q\), so daß \(p\) kein
    Teiler von \(q - 1\) ist. Zeige, daß jede endliche Gruppe der Ordnung \(p q\)
    zyklisch ist.
\end{exercise}


\begin{exercise}(4 Punkte)\newline
    Zeige, daß die abelsche Gruppe \(\set Z/(2) \times \set Z/(3)\) durch nur ein Element
    erzeugt werden kann. Warum ist dies kein Widerspruch zu Hilfssatz 6.112 auf Seite 279?
\end{exercise}

%\begin{exercise} \textbf{(Staatsexamens-/Klausuraufgaben)}(x Zusatzpunkte)\newline
  %  Sei \(G\) eine Gruppe der Ordnung \(600\). Zeigen Sie, daß \(G\) eine
 %   Untergruppe der Ordnung \(15\) enthält.
%\end{exercise}

\begin{exercise}(5 Punkte)\newline
    Sei \(d\) eine ganze Zahl.
    Sei \(A\) eine quadratische Matrix mit ganzzahligen Einträgen der Größe \(n\). Es sei
    weiter \(u\) ein Vektor mit ganzzahligen Einträgen und \(n\) Zeilen, so daß
    \(A \cdot u = 0\) modulo \(d\) (komponentenweise). Zeige, daß dann auch
    \((\det A) \cdot u = 0\) modulo \(d\).
\end{exercise}



\end{document} 
