\documentclass{algsheet}
    \usepackage{amssymb} 
\usepackage[protrusion=true,expansion=false]{microtype}
 \usepackage{magros_3}
%%%%%%%%%%%%%%%%%%%%%%%%%%%%%%%%%%%%%%%%%%%%%%%%%%%%%%%%%%%%%%%%%%%%%%%%%%%%%%%%%%%%%%%%%%%%%%%%%%%%%%%%%%%%%%%%%%%%%%%%%%%%%%%%%%%%%%%%%%%%%%%%%%%%%%%%%%%%

   \lecture{Algebra II}
        
        \semester{Sommersemester 2011}
        \sheet{12.\ Aufgabenblatt}
        \author{Dipl.-Math.~Franz Vogler}
        \date{11.~Juli 2011}
        
        %/usr/local/share/texmf/tex/algmacros.sty
    \usepackage[arrow,curve,matrix]{xy}    
        \begin{document}
                \maketitle



\begin{exercise}(6 Punkte)\newline
    Sei \(K(x)\) über \(K\) eine endliche Körpererweiterung ungeraden Grades.
    Zeige, daß \(K(x) = K(x^2)\).
\end{exercise}

\begin{exercise}(5 Punkte)\newline
    Finde ein normiertes irreduzibles Polynom zweiten Gerades über \(\mathbb F_2\),
    und gib einen Körper mit vier Elementen an.
\end{exercise}

\begin{exercise}(6 Punkte)\newline
    Sei \(K\) ein Körper.
    Sei \(E\) ein Zwischenkörper von \(K(X)\) über \(K\), der ein echter
    Oberkörper von \(K\) ist, das heißt es liegt ein Element in \(E\), welches
    nicht in \(K\) liegt. Zeige, daß \(X\) algebraisch über \(E\) ist.
\end{exercise}

\begin{exercise}(6 Punkte)\newline
    Sei \(K\) ein Körper. Sei
    \begin{equation}
        y \coloneqq \frac{g(X)}{h(X)} \in K(X),
    \end{equation}
    wobei \(g(X)\) und \(h(X)\) teilerfremde Polynome in \(K[X]\) seien. Sei
    das Maximum \(n\) von \(\deg g(X)\) und \(\deg h(X)\) mindestens \(1\). Zeige,
    daß der Grad von \(K(X)\) über \(K(y)\) gerade \(n\) ist.
\end{exercise}

\begin{exercise}(6 Punkte)\newline
    Seien \(L\) über \(K\) eine Körpererweiterung. Seien \(E\) und \(F\) zwei
    Zwischenkörper von \(L\) über \(K\). Wir nennen \(E\) \emph{linear disjunkt
    von \(F\)}, falls jede endliche Menge von Elementen aus \(E\), welche über
    \(K\) linear unabhängig ist, auch über \(F\) linear unabhängig ist.
    
    Zeige, daß diese Relation zwischen \(E\) und \(F\) symmetrisch in \(E\) und
    \(F\) ist, das heißt also, daß \(E\) genau dann linear disjunkt von \(F\) ist,
    wenn \(F\) linear disjunkt von \(E\) ist.
\end{exercise}



\begin{exercise}(5 Punkte)\newline
    Zerlege das Polynom    	
    \begin{equation}    
        f(X) = X^5 + X^4 + X^3 + X^2 + 1
    \end{equation}
    über \(\mathbb F_3\) in seine irreduziblen Faktoren.
\end{exercise}



\begin{exercise}(6 Punkte)\newline
    Sei \(f(X)\) ein Polynom über einem endlichen Körper \(K\). Zeige, daß eine
    endliche Körpererweiterung \(L\) von \(K\) existiert, über der \(f(X)\)
    in Linearfaktoren zerfällt.
\end{exercise}








\end{document}