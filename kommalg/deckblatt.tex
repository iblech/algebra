\documentclass{uebblatt}

\usepackage[T1]{fontenc} 
\usepackage{datatool}
\usepackage{array}

\pagestyle{empty}

\begin{document}

%\DTLsetseparator{;}           
\DTLloaddb{teilnehmende}{teilnehmende.csv}
\DTLsetnumberchars{.}{,}

\DTLforeach*{teilnehmende}{%
  \MatNo=Matrnr,\Nachname=Nachname,\Vorname=Vorname,\Bonus=Prozent}{%    
  \clearpage
  \begin{center}\Large\bfseries Kommutative Algebra\\ Wintersemester 2015/16\end{center}

  \vspace{3em}

  \begin{tabular}{|l|m{9cm}|}
    \hline
    \textbf{Name} & \textbf{\Nachname, \Vorname}  \\
    \hline
    \textbf{Matrikelnummer} & \MatNo  \\
    \hline
    \textbf{Codewort} & \rule{0pt}{2em} \\
    \hline
  \end{tabular}

  \vspace{3em}

  \begin{tabular}{|l|m{1cm}|m{1cm}|m{1cm}|m{1cm}|m{1cm}|}
    \hline
    \textbf{Aufgabe} & \multicolumn{1}{c|}{1} & \multicolumn{1}{c|}{2} & \multicolumn{1}{c|}{3} & \multicolumn{1}{c|}{4} &
    \multicolumn{1}{c|}{5} \\
    \hline
    \textbf{Punkte} & \rule{0pt}{2em} & \rule{0pt}{2em} & \rule{0pt}{2em} & \rule{0pt}{2em} & \rule{0pt}{2em} \\ 
    \hline
  \end{tabular}

  \vspace{1cm}

  \begin{tabular}{|l|m{1cm}|m{1cm}|m{1cm}|m{1cm}|c|}
    \hline
    \textbf{Aufgabe} & \multicolumn{1}{c|}{6} & \multicolumn{1}{c|}{7} & \multicolumn{1}{c|}{8} & \multicolumn{1}{c|}{9} &
    \multicolumn{1}{c|}{Bonus} \\
    \hline
    \textbf{Punkte} & \rule{0pt}{2em} & \rule{0pt}{2em} & \rule{0pt}{2em} & \rule{0pt}{2em} & \textbf{\Bonus} \\ 
    \hline
  \end{tabular}

  \vspace{3em}

  \begin{tabular}{|l|m{9em}|}
    \hline
    \textbf{Gesamtpunktzahl} & \rule{0pt}{2em} \\
    \hline
    \textbf{Prüfungsnote} & \rule{0pt}{2em} \\
    \hline
  \end{tabular}
  \vspace{3em}

  \begin{itemize}
  \item Überprüfen Sie die Angaben auf dem Deckblatt. Schreiben Sie leserlich ein
  von Ihnen ausgedachtes Codewort in die entsprechende Zeile des Deckblatts.
  Merken Sie sich ihr Codewort, weil wir die Prüfungsergebnisse nach den
  Codeworten herausgeben werden. Tragen Sie ansonsten auf dem Deckblatt nichts
  ein.
  \item Bitte achten Sie auf saubere, stichhaltige und vollständige Begründungen.
  Unleserliche oder nicht nachvollziehbare Bearbeitungen werden nicht korrigiert.
  \item Hilfsmittel außer Schreibmaterialien und einem mit Notizen zweiseitig
  beschriebenen DIN-A4-Blatt sind keine erlaubt.
  \item Täuschungsversuche werden streng geahndet.
  \item Wenn Sie der Meinung sind, einen Fehler in den Klausurunterlagen gefunden
  zu haben, melden Sie sich bitte bei der Aufsicht.
  \item Achten Sie darauf, am Ende alle Ihre bearbeiteten Blätter abzugeben.
  \item Zur Bearbeitung der Klausuraufgaben haben Sie insgesamt $120$ Minuten Zeit.
  \item Die Gesamtpunktzahl ist maximal $100$. In die Gesamtpunktzahl gehen maximal
  $50$ Punkte an Bonuspunkten durch gute Mitarbeit in den Übungen ein
  ($1$ Punkt pro $1\,\%$ gelöster Übungsaufgaben).
  \end{itemize}
}

\end{document}  
