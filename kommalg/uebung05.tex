\documentclass{uebblatt}

\begin{document}

\maketitle{5}{}

\begin{aufgabe}{2+m}{Beispiele für gerichtete Limiten}
\begin{enumerate}
\item Sei~$A$ ein Ring. Zeige, dass~$A[X]$ als~$A$-Modul kanonisch isomorph zum
gerichteten Limes des Systems~$A[X]_{\leq0} \hookrightarrow A[X]_{\leq1}
\hookrightarrow A[X]_{\leq2} \cdots$ ist. Dabei ist~$A[X]_{\leq n}$ der Modul
der Polynome vom Grad~$\leq n$.
\item Sei~$M$ ein~$A$-Modul. Sei~$f \in A$. Zeige, dass~$M[f^{-1}]$ kanonisch
isomorph zum gerichteten Limes des Systems~$M \xrightarrow{f} M \xrightarrow{f}
M \xrightarrow{f} \cdots$ ist.
\end{enumerate}
\end{aufgabe}

\begin{aufgabe}{2+m+2+2}{Gesättigte multiplikativ abgeschlossene Mengen}
Eine multiplikativ abgeschlossene Teilmenge~$S$ eines Rings~$A$ heißt genau dann
\emph{gesättigt}, wenn aus~$xy \in S$ schon~$x \in S$ und~$y \in S$ folgt.
\begin{enumerate}
\item Sei~$f \in A$. Sei~$\iota : A \to A[f^{-1}]$ der
Lokalisierungsmorphismus. Zeige: Für~$x \in A$ ist genau dann~$\iota(x)$
invertierbar, wenn~$f \in \sqrt{(x)}$.
\item Sei~$S \subseteq A$ eine gesättigte multiplikativ abgeschlossene
Teilmenge und~$\iota : A \to S^{-1}A$ die zugehörige Lokalisierung.
Zeige: Für~$x \in A$ ist genau dann~$\iota(x)$ invertierbar,
wenn~$x$ in~$S$ liegt.
Folgere, dass~$S^{-1}A$ genau dann ein lokaler Ring ist, wenn~$0 \not\in S$ und aus~$x + y \in S$
schon~$x \in S$ oder~$y \in S$ folgt.
%\item Zeige: Eine multiplikativ abgeschlossene Teilmenge ist genau dann
%gesättigt, wenn ihr Komplement eine Vereinigung von Primidealen ist.
\item Sei~$S \subseteq A$ multiplikativ abgeschlossen.
Zeige, dass es eine kleinste gesättigte, mul\-ti\-pli\-ka\-tiv abgeschlossene und~$S$
umfassende Teilmenge gibt (die \emph{Sättigung} von~$S$). 
\item Seien~$S$ und~$T$ multiplikativ abgeschlossene Mengen mit~$S \subseteq
T$. Zeige, dass~$S^{-1}A \to T^{-1}A,\,a/s \mapsto a/s$ genau
dann bijektiv ist, wenn~$T$ in der Sättigung von~$S$ liegt.
\end{enumerate}
\end{aufgabe}

\begin{aufgabe}{2+m+2}{Lokale Eigenschaften}
\begin{enumerate}
\item Zeige: Sind alle Halme eines Rings reduziert, so ist auch der Ring selbst
reduziert.
\item Zeige oder widerlege: Sind alle Halme Integritätsbereiche, so
auch der Ring selbst.
\item Sei~$M$ ein~$A$-Modul. Gelte~$M_\mmm = 0$ für all diejenigen maximalen
Ideale~$\mmm$ von~$A$, die über einem bestimmten Ideal~$\aaa$ liegen. Zeige:~$M
= \aaa M$.
\end{enumerate}
\end{aufgabe}

\begin{aufgabe}{2+m}{Injektivität von linearen Abbildungen}
\begin{enumerate}
\item Sei~$A$ ein lokaler Ring. Sei~$A^m \to A^n$ eine lineare Injektion.
Zeige, dass~$m \leq n$.
\item Folgere die Behauptung für beliebige Ringe~$A$ mit~$1 \neq 0$.
\end{enumerate}
\end{aufgabe}

\begin{aufgabe}{m}{Topologischer Abschluss von Punkten}
Sei~$\ppp$ ein Primideal eines Rings~$A$. Zeige, dass in~$\Spec A$ gilt:
%Der Abschluss von~${\{\ppp\}}$ in~$\Spec A$ ist
$\overline{\{\ppp\}} = \{ \qqq \in \Spec A \,|\, \ppp \subseteq \qqq \}$.
\end{aufgabe}

\centering
\emph{Wofür steht das "`B."' in "`Benoît B. Mandelbrot"'?}

\end{document}

* Charakterisierung von flachen Moduln
* equational criterion für flache Moduln
* maximale multiplikativ abgeschlossene Teilmengen
* Vergleich von Lokalisierungen
* S^(-1) M = 0 ==> exists s. sM = 0
* ...
