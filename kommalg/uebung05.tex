\documentclass[entwurf]{uebblatt}

\begin{document}

\maketitle{5}{}

\begin{aufgabe}{}{Beispiele für gerichtete Limiten}
\begin{enumerate}
\item Sei~$M$ ein~$A$-Modul. Sei~$f \in A$. Zeige, dass~$M[f^{-1}]$ kanonisch
isomorph zum gerichteten Limes des Systems~$M \xrightarrow{f} M \xrightarrow{f}
M \xrightarrow{f} \cdots$ ist.
\item Sei~$A$ ein Ring. Zeige, dass~$A[X]$ als~$A$-Modul kanonisch isomorph zum
gerichteten Limes des Systems~$A[X]_{\leq0} \hookrightarrow A[X]_{\leq1}
\hookrightarrow A[X]_{\leq2} \cdots$ ist. Dabei ist~$A[X]_{\leq n}$ der Modul
der Polynome vom Grad~$\leq n$.
\end{enumerate}
\end{aufgabe}

\begin{aufgabe}{}{Gesättigte abgeschlossene Mengen}
Eine multiplikativ abgeschlossene Teilmenge~$S$ eines Rings~$A$ heißt genau dann
\emph{gesättigt}, wenn aus~$xy \in S$ schon~$x \in S$ und~$y \in S$ folgt.
\begin{enumerate}
\item Sei~$f \in A$. Sei~$\iota : A \to A[f^{-1}]$ der
Lokalisierungsmorphismus. Zeige: Für~$x \in A$ ist genau dann~$\iota(x)$
invertierbar, wenn~$f \in \sqrt{(x)}$.
\item Sei~$S \subseteq A$ eine gesättigte multiplikativ abgeschlossene
Teilmenge. Zeige: Genau dann ist~$S^{-1}A$ ein lokaler Ring, wenn aus~$x + y
\in S$ schon~$x \in S$ oder~$y \in S$ folgt.
\item Sei~$S \subseteq A$ multiplikativ abgeschlossen.
Zeige, dass es eine kleinste gesättigte multiplikativ abgeschlossene Teilmenge,
die~$S$ umfasst, gibt (die \emph{Sättigung} von~$S$). 
\item Zeige: Eine multiplikativ abgeschlossene Teilmenge ist genau dann
gesättigt, wenn ihr Komplement eine Vereinigung von Primidealen ist.
\end{enumerate}
\end{aufgabe}

\begin{aufgabe}{}{Lokale Eigenschaften}
\begin{enumerate}
\item Zeige: Sind alle Halme eines Rings reduziert, so ist auch der Ring selbst
reduziert.
\item Zeige oder widerlege: Sind alle Halme eines Rings Integritätsbereiche, so
auch der Ring selbst.
\end{enumerate}
\end{aufgabe}

\begin{aufgabe}{}{Injektivität von linearen Abbildungen}
\begin{enumerate}
\item Sei~$A$ ein lokaler Ring. Sei~$A^m \to A^n$ eine lineare Injektion.
Zeige, dass~$m \leq n$.
\item Zeige die Behauptung für beliebige Ringe~$A$ mit~$1 \neq 0$.
\end{enumerate}
\end{aufgabe}

\begin{verbatim}
* topologische Aussagen
* Charakterisierung von flachen Moduln
* equational criterion für flache Moduln
* M_m = 0 für alle m >= a ==> M = aM
* maximale multiplikativ abgeschlossene Teilmengen
* Vergleich von Lokalisierungen
* S^(-1) M = 0 ==> exists s. sM = 0
* ...
\end{verbatim}

\centering
\emph{Wofür steht das "`B."' in "`Benoît B. Mandelbrot"'?}

\end{document}

