\documentclass[entwurf]{uebblatt}

\begin{document}

\maketitle{8}{}

\begin{aufgabe}{}{Nilpotenz von Potenzreihen über noetherschen Ringen}
Seien~$a_0, a_1, \ldots$ nilpotente Elemente in einem noetherschen Ring. Zeige,
dass die Potenzreihe~$\sum_{n=0}^\infty a_n X^n$ nilpotent ist.
\end{aufgabe}

\begin{aufgabe}{}{Die teilweise Obsoletierung eines Teilgebiets der Mathematik}
Sei~$A$ ein noetherscher Ring. Seien~$B$ eine
endlich erzeugte~$A$-Algebra und~$G$ eine endliche Gruppe von~$A$-Algebrenautomorphismen
von~$B$. Zeige, dass~$B^G \defeq \{x \in B \,|\, \text{$g(x) = x$ für alle~$g
\in G$} \}$ eine endlich erzeugte~$A$-Algebra ist.
\end{aufgabe}

\begin{aufgabe}{}{Lokalität der Noetherianität}
Zeige oder widerlege: Sind alle Halme eines Rings noethersch, so auch der Ring
selbst.
\end{aufgabe}

\begin{aufgabe}{}{Ein Kriterium für Noetherianität}
\begin{enumerate}
\item Sei~$\ppp$ ein Ideal, das maximal mit der Eigenschaft ist,
nicht endlich erzeugt zu sein. Zeige, dass~$\ppp$ ein Primideal ist.
(Tipp folgt.)
\item Zeige, dass ein Ring, in dem alle Primideale endlich erzeugt sind,
noethersch ist.
\end{enumerate}
\end{aufgabe}

\end{document}
