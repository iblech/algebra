\documentclass[entwurf]{uebblatt}

\begin{document}

\maketitle{8}{}

\begin{aufgabe}{2}{Nilpotenz von Potenzreihen über noetherschen Ringen}
Seien~$a_0, a_1, \ldots$ nilpotente Elemente in einem noetherschen Ring. Zeige,
dass die Potenzreihe~$\sum_{n=0}^\infty a_n X^n$ nilpotent ist.
\end{aufgabe}

\begin{aufgabe}{m}{Die teilweise Obsoletierung eines Teilgebiets der Mathematik}
Sei~$R$ ein noetherscher Ring. Seien~$A$ eine
endlich erzeugte~$R$-Algebra und~$G$ eine endliche Gruppe von~$R$-Algebrenautomorphismen
von~$A$. Zeige, dass die~$A$-Algebra~$A^G$ der~$G$-Invarianten (siehe Blatt~6,
Aufgabe 4) endlich erzeugt ist.
\end{aufgabe}

\begin{aufgabe}{m}{Noethersche Induktion für noethersche Moduln}
Sei~$M$ ein noetherscher Modul. Sei für jeden Untermodul~$U$ eine
Behauptung~$\varphi(U)$ gegeben. Gelte für alle Untermoduln~$U$:
\begin{quote}Wenn~$\varphi(U')$ für alle Untermoduln~$U'
\supsetneq U$ stimmt, dann stimmt auch~$\varphi(U)$.\end{quote}
Zeige, dass~$\varphi(U)$ für alle Untermoduln~$U$ stimmt.
\end{aufgabe}

\begin{aufgabe}{2}{Lokalität der Noetherianität}
Zeige oder widerlege: Sind alle Halme eines Rings noethersch, so auch der Ring
selbst.
\end{aufgabe}

\begin{aufgabe}{m+2+2}{Ein Kriterium für Noetherianität}
\begin{enumerate}
\item Sei~$\aaa$ ein Ideal eines Rings~$A$ und sei~$x \in A$. Sei~$\aaa + (x)$
endlich erzeugt. Zeige, dass es ein endlich erzeugtes Ideal~$\aaa_0$ mit~$\aaa
+ (x) = \aaa_0 + (x)$ gibt.
\item Sei~$\ppp$ ein Ideal, das maximal mit der Eigenschaft ist,
nicht endlich erzeugt zu sein. Zeige, dass~$\ppp$ ein Primideal ist.
\item Zeige, dass ein Ring, in dem alle Primideale endlich erzeugt sind,
noethersch ist.
\end{enumerate}
\end{aufgabe}

\begin{aufgabe}{2}{Ein konkretes Beispiel für die Länge}
Was ist die Länge des~$\ZZ$-Moduls~$\ZZ/(90)$?
\end{aufgabe}

\centering
\emph{Es folgt noch eine Aufgabe (und ein Comic). Ansonsten ist das Blatt
vollständig.}
\par

\end{document}
