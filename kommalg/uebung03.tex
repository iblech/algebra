\documentclass[entwurf]{uebblatt}

\newcommand{\http}{http:/\kern-.2em/\kern-0.03em}

\begin{document}

\maketitle{3}{Abgabe bis zum ???}

\begin{aufgabe}{2+2}{Der formale Potenzreihenring über dem Grundring}
Sei~$A$ ein Ring.
\begin{enumerate}
\item Ist~$\mmm$ ein maximales Ideal in~$\ps{A}$, so gilt~$X \in \mmm$, die
Kontraktion~$\mmm_0 \defeq A \cap m$ ist ein maximales Ideal in~$A$ und~$\mmm$
ist das von~$\mmm_0$ und~$X$ in~$\ps{A}$ erzeugte Ideal.
\item Jedes Primideal von~$A$ ist Kontraktion eines Primideals von~$\ps{A}$.
\end{enumerate}
\end{aufgabe}

\begin{aufgabe}{}{Ein radikales Distributivgesetz}
\begin{enumerate}
\item Zeige anhand eines Gegenbeispiels, das folgende Rechenregel für Ideale in
einem Ring im Allgemeinen \emph{nicht} gilt:
\[ \aaa \cap \sum_i \bbb_i = \sum_i (\aaa \cap \bbb_i). \]
\item Zeige, dass folgende Regel aber durchaus stets gilt:
\[ \sqrt{\aaa} \cap \sqrt{\sum_i \bbb_i} = \sqrt{\sum_i (\aaa \cap \bbb_i)}. \]
\end{enumerate}
\end{aufgabe}

\end{document}
