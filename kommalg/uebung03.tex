\documentclass[entwurf]{uebblatt}

\newcommand{\http}{http:/\kern-.2em/\kern-0.03em}

\begin{document}

\maketitle{3}{Abgabe bis zum ???}

\begin{aufgabe}{2+2}{Der formale Potenzreihenring über dem Grundring}
Sei~$A$ ein Ring.
\begin{enumerate}
\item Ist~$\mmm$ ein maximales Ideal in~$\ps{A}$, so gilt~$X \in \mmm$, die
Kontraktion~$\mmm_0 \defeq A \cap m$ ist ein maximales Ideal in~$A$ und~$\mmm$
ist das von~$\mmm_0$ und~$X$ in~$\ps{A}$ erzeugte Ideal.
\item Jedes Primideal von~$A$ ist Kontraktion eines Primideals von~$\ps{A}$.
\end{enumerate}
\end{aufgabe}

\begin{aufgabe}{2+m+m+2}{Rechnungen mit Idealen}
\begin{enumerate}
\item Sei~$\aaa$ ein Ideal eines Rings~$A$. Finde einen kanonischen
Ringhomomorphismus~$A[X]/\aaa[X] \to (A/\aaa)[X]$ und zeige, dass er ein
Isomorphismus ist.
\item Sei~$\ppp$ ein Primideal eines Rings~$A$. Zeige, dass dann auch~$\ppp[X]$
in~$A[X]$ prim ist.
\item Gilt die analoge Behauptung von~b) auch für maximale Ideale?
\item Untersuche folgende Ideale auf Primalität und Maximalität:
$(2, X)$ in~$\ZZ[X]$ und $([2], [X])$ in~$\ZZ[X]/(X^2 - X + 6)$.
\end{enumerate}
\end{aufgabe}

\begin{aufgabe}{2+1}{Nichtbeispiele für Hauptidealbereiche}
\begin{enumerate}
\item Sei~$A$ ein Ring derart, dass jedes endlich erzeugte Ideal von~$A[X]$ ein
Hauptideal ist. Zeige, dass jedes reguläre Element von~$A$ schon invertierbar
ist.
\item Folgere: $\ZZ[X]$ und~$\QQ[X,Y]$ sind keine Hauptidealbereiche.
\end{enumerate}
\end{aufgabe}

\begin{aufgabe}{m+2}{Ein radikales Distributivgesetz}
\begin{enumerate}
\item Zeige anhand eines Gegenbeispiels, dass die Rechenregel
"`$\aaa \cap \sum_i \bbb_i = \sum_i (\aaa \cap \bbb_i)$"'
für Ideale in
einem Ring im Allgemeinen \emph{nicht} gilt. \emph{Hinweis.} In den
Ringen~$\ZZ$ und~$\QQ[X]$ wirst du keinen Erfolg haben.
\item Zeige, dass folgende Regel durchaus stets gilt:
$\sqrt{\aaa} \cap \sqrt{\sum_i \bbb_i} = \sqrt{\sum_i (\aaa \cap \bbb_i)}$.
\end{enumerate}
\end{aufgabe}

\begin{aufgabe}{2+1}{Der Darstellungssatz von Stone}
Sei~$A$ ein boolscher Ring. Sei~$\Spec A$ die Menge der Primideale von~$A$. Die
Potenzmenge von~$\Spec A$ bildet mit der symmetrischen Differenz als Addition
und dem Schnitt als Multiplikation ebenfalls einen boolschen Ring.
\begin{enumerate}
\item Gib explizit einen Ringhomomorphismus~$A \to \P(\Spec A)$ an.
\item Zeige, dass dieser injektiv ist.
\end{enumerate}
\end{aufgabe}

\end{document}
