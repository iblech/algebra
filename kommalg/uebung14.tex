\documentclass{uebblatt}

\usepackage{wrapfig}

\begin{document}

\maketitle{14}{}

\begin{aufgabe}{2}{Ein Gegenbeispiel zu einer Verstärkung des Krullschen Satzes}
Finde einen noetherschen Ring zusammen mit einem Ideal~$\aaa \neq (1)$
mit~$\bigcap_{n=0}^\infty \aaa^n \neq (0)$.
\end{aufgabe}

\begin{aufgabe}{m+2+2}{Fasern von Ringhomomorphismen}
Sei~$\ppp$ ein Primideal eines Rings~$A$. Sei~$k(\ppp) \defeq A_\ppp/\ppp A_\ppp$
der Restklassenkörper bei~$\ppp$. Sei~$B$ eine endliche~$A$-Algebra.
\begin{enumerate}
\item Zeige, dass die Primideale~$\qqq$
von~$B$ mit~$A \cap \qqq = \ppp$ in kanonischer Eins-zu-Eins-Korrespondenz zu
den Primidealen von~$B_\ppp/\ppp B_\ppp \cong B \otimes_A k(\ppp)$ stehen.
\item Sei~$A$ ein Körper. Zeige, dass~$B$ nur endlich
viele Primideale besitzt.
\item Sei~$B$ endlich über~$A$. Zeige, dass es nur endlich viele
Primideale~$\qqq$ wie in~b) gibt.
\end{enumerate}
\end{aufgabe}

\begin{aufgabe}{2+2+m}{Dimension des Polynomrings im nicht-noetherschen Fall}
Sei~$A$ ein Ring.
\begin{enumerate}
\item Zeige: $\dim A[X] \geq 1 + \dim A$.
% Tipp: B3, A2b)
\item Sei~$\ppp$ ein Primideal von~$A$. Die Primideale~$\qqq$
von~$A[X]$ mit~$A \cap \qqq = \ppp$ stehen in Eins-zu-Eins-Korrespondenz zu den
Primidealen eines gewissens Rings. Welchem? Welche Dimension hat dieser?
\item Zeige: $\dim A[X] \leq 1 + 2 \dim A$.
\end{enumerate}
\end{aufgabe}

\begin{aufgabe}{3}{Beispiele für Poincarésche Reihe und Hilbertsches Polynom}
Berechne die Poincarésche Reihe und das Hilbertsche Polynom des
gewichteten~$K[X,Y]$-Moduls $K[X,Y]/(X^2, XY)$ bezüglich~$\lambda = \dim_K$.
\end{aufgabe}

\begin{aufgabe}{1}{Dualität zwischen symmetrischer und äußerer Algebra}
Sei~$K$ ein Körper. Sei~$S = K[X_1,\ldots,X_n]$ und sei~$E$ die zugehörige \emph{äußere Algebra}
der \emph{antikommutativen Polynome}, wo~$X_i X_i = 0$ und~$X_i X_j =
-X_j X_i$ gilt. Sei~$\lambda = \dim_K$.
Zeige: $\lambda(S, t) \cdot \lambda(E, -t) = 1$.
\end{aufgabe}
% Verwende ohne Beweis, dass dim_K E_k = \binom{n}{k}.

\setlength{\wrapoverhang}{1.3cm}
\setlength{\columnsep}{0.8cm}
\begin{wrapfigure}{r}{0.38\textwidth}
  \vspace{-1em}
  \includegraphics{images/hilbert-scheme-of-points}
\end{wrapfigure}

\begin{aufgabe}{0}{Rationale Binomialkoeffizienten}
\scriptsize
Wir setzen~$\binom{x}{k}
\defeq x (x-1) \cdots (x-k+1) / k! \in \QQ$ für rationale Zahlen~$x$ und
natürliche Zahlen~$k$.
Solche Binomialkoeffizienten
kommen in Taylor-Entwicklungen vieler wichtiger Funktionen vor.
\begin{enumerate}
\item Zeige: Genau dann kommt im gekürzten Nenner einer rationalen Zahl~$a/b$
nicht der Primfaktor~$p$ vor, wenn es eine~$p$-adische Ganzzahl~$u$ mit~$bu = a$ gibt.
\item Verwende die Dichtheit von~$\ZZ$ in~$\ZZ_p$ und die Stetigkeit von
Polynomen über~$\ZZ_p$, um zu folgern: Im gekürzten Nenner eines rationalen
Binomialkoeffizienten~$\binom{x}{k}$ können nur solche Primfaktoren vorkommen, die auch
im gekürzten Nenner von~$x$ vorkommen.\par
\end{enumerate}
\end{aufgabe}


\end{document}

Berechne die Poincarésche Reihe von ...
https://people.kth.se/~laksov/courses/algebradr01/notes/grading2.pdf

Spaß zum Matrizenmodell des Hilbertschemas? Geht vielleicht zu weit.
