\documentclass[entwurf]{uebblatt}

\newcommand{\cont}{\mathrm{cont}}
\newcommand{\http}{http:/\kern-.2em/\kern-0.03em}

\begin{document}

\maketitle{14}{}

\begin{aufgabe}{}{Ein Gegenbeispiel zu einer Verstärkung des Krullschen Satzes}
Finde einen noetherschen Ring zusammen mit einem Ideal~$\aaa \neq (1)$
mit~$\bigcap_{n=0}^\infty \aaa^n \neq (0)$.
\end{aufgabe}

\begin{aufgabe}{3}{Beispiele für Poincarésche Reihe und Hilbertsches Polynom}
Berechne die Poincarésche Reihe und das Hilbertsche Polynom des
gewichteten~$K[X,Y]$-Moduls $K[X,Y]/(X^2, XY)$ bezüglich~$\dim_K$.
% und von noch paar Moduln
\end{aufgabe}

\begin{aufgabe}{1}{Dualität zwischen symmetrischer und äußerer Algebra}
Sei~$K$ ein Körper. Sei~$S = K[X_1,\ldots,X_n]$ und sei~$E$ die zugehörige \emph{äußere Algebra}
der \emph{antikommutativen Polynome}, wo~$X_i X_i = 0$ und~$X_i X_j =
-X_j X_i$ gilt. Sei~$\lambda = \dim_K$.
Zeige: $\lambda(S, t) \cdot \lambda(E, -t) = 1$.
\end{aufgabe}
% Verwende ohne Beweis, dass dim_K E_k = \binom{n}{k}.

\vfill

\begin{aufgabe}{0}{Rationale Binomialkoeffizienten}
\small
Für rationale Zahlen~$x$ und natürliche Zahlen~$k$ setzen wir~$\binom{x}{k}
\defeq x (x-1) \cdots (x-k+1) / k! \in \QQ$. Solche Binomialkoeffizienten
kommen in Taylor-Entwicklungen vieler wichtiger Funktionen vor.
\begin{enumerate}
\item Zeige: Genau dann kommt im gekürzten Nenner einer rationalen Zahl~$a/b$
nicht der Primfaktor~$p$ vor, wenn es eine~$p$-adische Ganzzahl~$u$ mit~$bu = a$ gibt.
\item Verwende die Dichtheit von~$\ZZ$ in~$\ZZ_p$ und die Stetigkeit von
Polynomen über~$\ZZ_p$, um zu folgern: Im gekürzten Nenner eines rationalen
Binomialkoeffizienten~$\binom{x}{k}$ können nur solche Primfaktoren vorkommen, die auch
im gekürzten Nenner von~$x$ vorkommen.
\end{enumerate}
\end{aufgabe}

\centering
\includegraphics{images/hilbert-scheme-of-points}

\end{document}

Berechne die Poincarésche Reihe von ...
https://people.kth.se/~laksov/courses/algebradr01/notes/grading2.pdf

Spaß zum Matrizenmodell des Hilbertschemas? Geht vielleicht zu weit.

Sei A ein Ring. Zeige: 1 + dim A <= dim A[X] <= 1 + 2 dim A.
Teilaufgaben zur Hinführung:
* (für die obere Schranke) Zeige, dass Primideale von A[X],
  die über einem Primideal p von A liegen, in Eins-zu-Eins-Korrespondenz zu den
  Primidealen von k(p)[X] stehen, wobei k(p) = A_p/pA_p.
* (für die untere Schranke) Blatt 3, Aufgabe 2b).
