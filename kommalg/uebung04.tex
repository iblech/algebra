\documentclass[entwurf]{uebblatt}

\begin{document}

\maketitle{4}{}

\begin{aufgabe}{}{Endliche Erzeugung bei Moduln}
\begin{enumerate}
\item Finde ein Erzeugendensystem eines Moduls, sodass keine Teilfamilie eine
Basis ist. Finde eine linear unabhängige Familie in einem Modul, sodass keine
Erweiterung eine Basis ist.
\item Sei~$0 \to M' \to M \to M'' \to 0$ eine kurze exakte Sequenz von Moduln.
Zeige: Sind die beiden äußeren Moduln endlich erzeugt, so auch der mittlere.
\item Sei~$M$ ein endlich erzeugter~$A$-Modul. Sei~$\varphi : M \to A^n$ eine
surjektive lineare Abbildung. Zeige, dass der Kern von~$\varphi$ endlich
erzeugt ist.
% XXX: Tipp/Präzisierung angeben
\end{enumerate}
\end{aufgabe}

\begin{aufgabe}{}{Nullteiler bezüglich des Tensorprodukts}
\begin{enumerate}
\item Zeige, dass~$\ZZ/(2) \otimes_\ZZ \ZZ/(3) = 0$.
\item Sei~$P$ ein endlich erzeugter Modul über einem lokalen Ring~$(A,\mmm,k)$.
Zeige: Ist~$P \otimes_A k = 0$, so auch~$P = 0$.
\item Seien~$M$ und~$N$ endlich erzeugte Moduln über einem lokalen Ring~$A$.
Zeige: Ist~$M \otimes_A N = 0$, so~$M = 0$ oder~$N = 0$.
\end{enumerate}
\end{aufgabe}

\begin{aufgabe}{}{Surjektivität von linearen Abbildungen}
\begin{enumerate}
\item Sei~$M$ und~$N$ Moduln über einem Ring~$A$. Sei~$N$ sogar endlich
erzeugt. Sei~$\aaa$ ein im Jacobsonschen Radikal enthaltenes Ideal von~$A$.
Sei~$\varphi : M \to N$ eine~$A$-lineare Abbildung. Zeige: Ist die
induzierte Abbildung~$M/\aaa M \to N/\aaa N$ surjektiv, so auch~$\varphi$.
\item Sei~$A$ ein Ring mit~$1 \neq 0$. Sei~$A^m \to A^n$ eine lineare
Surjektion. Zeige, dass~$m \geq n$.
\item Sei~$M$ eine~$(n \times m)$-Matrix über einem lokalen Ring, welche
aufgefasst als lineare Abbildung surjektiv ist. Zeige, dass~$M$ ähnlich zu einer
(rechteckigen) Diagonalmatrix mit genau~$n$ Einsern auf der Hauptdiagonale ist.
\end{enumerate}
\end{aufgabe}

\end{document}

\emph{Wofür steht das "`B."' in "`Benoît B. Mandelbrot"'?}
