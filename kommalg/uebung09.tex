\documentclass{uebblatt}

\newcommand{\https}{https:/\kern-.2em/\kern-0.03em}

\begin{document}

\maketitle{9}{}

\begin{aufgabe}{2}{Ganz, endlich und von endlichem Typ}
Sei~$A \to B$ ein Ringhomomorphismus. Zeige, dass~$B$ genau dann endlich
über~$A$ ist, wenn~$B$ von endlichem Typ und ganz über~$A$ ist.
\end{aufgabe}

\begin{aufgabe}{2+m}{Anwendungen der Noether-Normalisierung}
\begin{enumerate}
\item Sei~$A$ eine endlich erzeugte Algebra über einem Körper~$K$ und
sei~$\mmm$ ein maximales Ideal in~$A$. Zeige, dass~$A/\mmm$ eine endliche
Erweiterung von~$K$ ist.
\item Sei~$\phi : A \to B$ ein Homomorphismus endlich erzeugter Algebren über
einem Körper. Zeige, dass das Urbild eines maximalen Ideals unter~$\varphi$ wieder
maximal ist.
\end{enumerate}
\end{aufgabe}

\begin{aufgabe}{2}{Lokalität der Noetherianität, verfeinert}
Sei~$A$ ein Ring, dessen Halme alle noethersch sind. Gelte außerdem, dass
jedes Element~$x \in A \setminus \{0\}$ nur in endlich vielen maximalen Idealen
liegt. Zeige, dass~$A$ noethersch ist.
\end{aufgabe}

\begin{aufgabe}{2}{Ein schlimmes Ideal}
Finde ein Beispiel für ein Ideal~$\aaa$, sodass für kein~$n \geq 0$
die Inklusion~$(\sqrt{\aaa})^n \subseteq \aaa$ gilt.
\end{aufgabe}

\begin{aufgabe}{m+1+m+1+m+1+1+1}{Großer Tag der Gegenbeispiele}
Welche der folgenden Aussagen sind wahr? Kurze Begründung oder Gegenbeispiel!
\renewcommand{\labelenumi}{\arabic{enumi}.}
\begin{enumerate}
\item Das Bild eines Ideals unter einem Ringhomomorphismus ist ein Ideal.
\item Untermoduln endlich erzeugter Moduln sind endlich erzeugt.
\item Unterringe noetherscher Ringe sind noethersch.
\item Sind alle Halme eines Moduls endlich erzeugt, so auch der Modul selbst.
\item Wenn ein Ringhomomorphismus~$\varphi : A \to B$ surjektiv ist, so folgt
für je zwei parallele Ringhomomorphismen~$\alpha, \beta : B \to C$ aus~$\alpha
\circ \varphi = \beta \circ \varphi$ schon~$\alpha = \beta$.
\item Es gilt die Umkehrung von~5.
\item Ein normiertes Polynom vom Grad~$n$ über einem Ring hat höchstens~$n$
Nullstellen.
\item Seien~$f,g,h \in K[X,Y]$ Polynome. Dann gilt~$(f,g) \cap (h) =
(\operatorname{kgV}(f,h), \operatorname{kgV}(g,h))$.
\end{enumerate}
\end{aufgabe}

\centering
\emph{Wenn ein Ring nicht noethersch ist:} \\
\url{http://tiny.cc/no-no-noether}
\par

\end{document}
