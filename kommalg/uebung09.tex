\documentclass[entwurf]{uebblatt}

\newcommand{\https}{https:/\kern-.2em/\kern-0.03em}

\begin{document}

\maketitle{9}{}

\begin{aufgabe}{}{Ganz, endlich und von endlichem Typ}
Sei~$A \to B$ ein Ringhomomorphismus. Zeige, dass~$B$ genau dann endlich
über~$A$ ist, wenn~$B$ von endlichem Typ und ganz über~$A$ ist.
\end{aufgabe}

\begin{aufgabe}{}{Anwendungen der Noether-Normalisierung}
\begin{enumerate}
\item Sei~$A$ eine endlich erzeugte Algebra über einem Körper~$K$ und
sei~$\mmm$ ein maximales Ideal in~$A$. Zeige, dass~$A/\mmm$ eine endliche
Erweiterung von~$K$ ist.
\item Sei~$\phi : A \to B$ ein Homomorphismus endlich erzeugter Algebren über
einem Körper. Zeige, dass das Urbild eines maximalen Ideals unter~$\varphi$ wieder
maximal ist.
\end{enumerate}
\end{aufgabe}

\begin{aufgabe}{}{Lokalität der Noetherianität, verfeinert}
Sei~$A$ ein Ring, dessen Halme alle noethersch sind. Gelte außerdem, dass
jedes Element~$x \in A \setminus \{0\}$ nur in endlich vielen maximalen Idealen
liegt. Zeige, dass~$A$ noethersch ist.
\end{aufgabe}

\begin{aufgabe}{}{XXX}
Finde ein Beispiel für ein Ideal~$\aaa$, sodass für kein~$n \geq 0$
die Inklusion~$(\sqrt{\aaa})^n \subseteq \aaa$ gilt.
\end{aufgabe}

\end{document}

Wahr oder falsch?
* Untermoduln von e.e. Moduln sind e.e.
* Das Bild eines Ideals ist ein Ideal.
