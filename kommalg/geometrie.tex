\documentclass{uebblatt}

\usepackage{framed}

\definecolor{haskell}{RGB}{150,0,255}

\newcommand{\fmini}[2]{%
  \setlength{\fboxrule}{2pt}%
  \setlength{\fboxsep}{-3pt}%
  \textcolor{haskell}{\fbox{\textcolor{black}{\parbox{#1}{\begin{center}#2\end{center}}}}}}

\newcommand{\korr}[2]{
  \begin{center}
    \fmini{0.30\textwidth}{#1}
    \textcolor{haskell}{$\xleftrightarrow{\qquad\quad}$}
    \fmini{0.30\textwidth}{#2}
  \end{center}}

\newcounter{funnummer}
\definecolor{shadecolor}{rgb}{.93,.93,.93}
\NewEnviron{fun}{%
  \refstepcounter{funnummer}%
  \begin{shaded}%
    \textbf{Spiel \& Spaß \thefunnummer.}
    \BODY
  \end{shaded}%
}

\begin{document}

\begin{center}\Large\textbf{\textsf{Geometrische Vorstellung von Ringen}}\end{center}

Sei~$A$ ein Ring (wie immer: kommutativ und mit Eins). Dann gibt es einen
topologischen Raum,~$\Spec A$, mittels dem man sich den Ring anschaulich
vorstellen kann. Auf diese Weise haben viele Konzepte und Resultate aus der
Algebra eine geometrische Entsprechung; und umgekehrt.

Die folgende Definition stellt das zentrale Bindeglied dar.

\korr{Primideale von $A$}{Punkte von $\Spec A$}

Die Topologie auf~$\Spec A$ definiert man wie folgt: Eine Teilmenge von~$\Spec
A$ heißt genau dann \emph{abgeschlossen}, wenn sie von der Form~$V(\aaa) \defeq \{
\ppp \in \Spec A \,|\, \aaa \subseteq \ppp \}$ ist, wobei~$\aaa$ ein beliebiges
Ideal in~$A$ ist. Eine Teilmenge heißt genau dann \emph{offen}, wenn ihr
Komplement abgeschlossen ist.

\begin{fun}Weise nach, dass diese Festlegungen wirklich eine Topologie
auf~$\Spec A$ definieren.\end{fun}

Ein weiterer wichtiger Aspekt der Korrespondenz ist folgender.

\korr{Elemente von $A$}{Funktionen auf $\Spec A$}

Ist~$f$ nämlich ein Element von~$A$ und~$\ppp \in \Spec A$ ein Punkt, so können
wir definieren: Der \emph{Funktionswert} von~$f$ an der Stelle~$\ppp$ ist das
Bild von~$f$ in~$k(\ppp)$. Dabei ist~$k(\ppp)$ der \emph{Restklassenkörper} an
der Stelle~$\ppp$, das ist der Körper~$A_\ppp/\ppp A_\ppp$.

Sei zum Beispiel~$A = \RR[X]$ und~$f = X^2 - 2X + 3$. Sei~$\ppp = (X-4)$. Dann
ist~$k(\ppp)$ kanonisch zu~$\RR$ isomorph, vermöge~$[f/g] \mapsto f(4)/g(4)$.
Unter diesem Isomorphismus ist dann der Funktionswert von~$f$ bei~$\ppp$ gleich
dem traditionell definierten Funktionswert~$f(4)$.

\begin{fun}Was sind die Funktionswerte von~$90 \in \ZZ$ an allen Punkten
aus~$\Spec \ZZ$? Wo hat die Funktion Nullstellen? Hat die Funktion auch
doppelte Nullstellen?\end{fun}

Im Vergleich zu den in der Differentialgeometrie, komplexen Geometrie oder
riemannschen Geometrie untersuchten Räumen hat~$\Spec A$ noch eine
Besonderheit: Nur in den seltensten Fällen ist~$\Spec A$ ein Hausdorffraum.
Insbesondere sind einpunktige Mengen~$\{\ppp\}$ nicht immer abgeschlossen.
Punkte~$\ppp$, für die~$\{\ppp\}$ doch abgeschlossen ist, heißen
\emph{abgeschlossene Punkte}.

\korr{maximale Ideale}{abgeschlossene Punkte}

\korr{Ideale, die über~$\ppp$ sitzen}{Abschluss von~$\{\ppp\}$}

Die abgeschlossenen Punkte von~$\Spec \CC[X,Y]$ sind alle von der
Form~$(X-z,Y-w)$ mit~$z,w \in \CC$; die abgeschlossenen Punkte von~$\Spec
\CC[X,Y]$ stehen also in kanonischer Eins-zu-Eins-Beziehung zu den Punkten des
Vektorraums~$\CC^2$.

In~$\Spec \CC[X,Y]$ gibt es aber auch noch nicht-abgeschlossene Punkte. Zum
Beispiel ist der Punkt~$(Y-X^2)$ nicht abgeschlossen. In seinem Abschluss
liegen (er selbst und) alle Punkte der Form~$(X-z,Y-w)$ mit~$w=z^2$, also alle
Punkte der Normalparabel. Wenn man ein Bild von~$\Spec \CC[X,Y]$ zeichnet, malt
man~$(Y-X^2)$ als eine Art delokalisierte Punktwolke, als Geisterpunkt, der
überall und nirgendwo auf~$V((Y-X^2))$ sitzt.

\begin{fun}Beweise die letzten beiden Korrespondenzen. Sei also~$\ppp$ ein
Primideal. Zeige: Der topologische Abschluss von~$\{\ppp\}$ in~$\Spec A$ ist
gerade die Menge derjenigen Primideale~$\qqq$, die über~$\ppp$ sitzen, also~$\{
\qqq \in \Spec A \,|\, \ppp \subseteq \qqq \}$.\end{fun}

Mehrere Aussagen der Algebra haben schon mit diesem kleinen Wörterbuch eine
geometrische Bedeutung:

\begin{itemize}
\item \emph{Über jedem Primideal~$\ppp$ gibt es ein maximales Ideal~$\mmm$.}

Jeder Punkt enthält einen abgeschlossenen Punkt in seinem Abschluss.
Jeder generische Punkt spezialisiert sich zu mindestens einem abgeschlossenen
Punkt.

\item \emph{Das Primideal~$\ppp$ ist ein minimales Primideal.}

Der Abschluss von~$\ppp$ ist eine irreduzible Komponente von~$\Spec A$.

\item \emph{Der Ring enthält genau ein minimales Primideal.}

Der Raum~$\Spec A$ ist irreduzibel.

\item \emph{Unter jedem Primideal gibt es ein minimales Primideal.}

Jeder Punkt liegt in irgendeiner irreduziblen Komponente.

\item \emph{Der Ring ist lokal.}

Der Raum~$\Spec A$ enthält nur einen einzigen abgeschlossenen Punkt.
\end{itemize}

\begin{fun}Sei~$A = \Spec \CC[X,Y]/(XY)$. Überzeuge dich davon, dass~$\Spec A$
wie ein Achsenkreuz aussieht. Es besteht aus zwei irreduziblen Komponenten, den
beiden Achsen. Zeige, dass~$A$ genau zwei minimale Primideale enthält
(nämlich~$([X])$ und~$([Y])$) und erkläre, wie diese mit den Komponenten
zusammenhängen.
\end{fun}

Mit dem geometrischen Raum~$\Spec A$ kann man sich auch (beliebige, nicht
unbedingt prime) Ideale von~$A$ veranschaulichen. Und zwar kann man ein
Ideal~$\aaa$ dadurch visualisieren, indem man die abgeschlossene
Menge~$V(\aaa)$ betrachtet. \emph{Anschaulich ist das die Menge derjenigen
Punkte, bei denen alle Funktionen aus~$\aaa$ verschwinden.}

Dabei gelten folgende Rechenregeln:

\begin{itemize}
\item $V(0) = \Spec A$. Die Nullfunktion verschwindet an jedem Punkt.
\item $V((1)) = \emptyset$. Die Einsfunktion verschwindet nirgendwo.
\item $V(\aaa + \bbb) = V(\aaa) \cap V(\bbb)$. Punkte, bei denen alle
Funktionen aus~$\aaa + \bbb$ verschwinden, sind genau die, bei denen alle
Funktionen aus~$\aaa$ und alle Funktionen aus~$\bbb$ verschwinden.
\item $V(\aaa \cap \bbb) = V(\aaa \cdot \bbb) = V(\aaa) \cup V(\bbb)$.
\end{itemize}

Die Geometrie ist aber \emph{blind bezüglich Nilpotenz}: Für jedes Ideal~$\aaa$
gilt~$V(\aaa) = V(\sqrt{\aaa})$. Und wenn eine Funktion~$f \in A$ an allen
Punkten verschwindet, heißt das nicht, dass~$f = 0$; es heißt nur, dass~$f$
nilpotent ist.

\end{document}
