\documentclass{uebblatt}
\usepackage{wrapfig}

\newcommand{\http}{http:/\kern-.2em/\kern-0.03em}

\begin{document}

\maketitle{2}{Abgabe bis zum Montag, den 26. Oktober 2015}

\begin{aufgabe}{2+2+1}{Invertierbarkeit und Nilpotenz in Ringen formaler Potenzreihen}
Sei~$A$ ein Ring. Sei~$f = a_0 + a_1 X + a_2 X^2 + \cdots \in
\ps{A}$ eine formale Potenzreihe. Zeige:
\begin{enumerate}
\item Genau dann ist~$f$ eine Einheit in~$\ps{A}$, wenn~$a_0$ in~$A$ invertierbar
ist.
\item Ist~$f$ nilpotent, so sind alle Koeffizienten~$a_0,a_1,\ldots$ nilpotent.
\item Gilt in~b) die Umkehrung? (Es genügt ein plausibles Argument.)
\end{enumerate}
\end{aufgabe}

\begin{aufgabe}{2+2}{Nilradikal und Jacobsonsches Radikal}
Sei~$A$ ein Ring.
\begin{enumerate}
\item Jacobsonsches Radikal und Nilradikal von~$A[X]$ stimmen miteinander überein.
\item Genau dann liegt eine Potenzreihe~$f$ im Jacobsonschen Radikal
von~$\ps{A}$, wenn~$f(0)$ im Jacobsonschen Radikal von~$A$ liegt.
\end{enumerate}
\end{aufgabe}

\begin{aufgabe}{3+1}{Charakterisierung von Wurzelidealen}
Sei~$\aaa$ ein Ideal eines Rings. Zeige, dass~$\aaa$ genau dann mit
seinem Wurzelideal übereinstimmt, wenn~$\aaa$ ein Schnitt von Primidealen ist.
\end{aufgabe}

\begin{aufgabe}{3+1}{Inhalt von Polynomen}
Sei~$A$ ein Ring. Der \emph{Wurzelinhalt} eines Polynoms~$f = a_0 + \cdots +
a_m X^m \in A[X]$ ist das Ideal~$J(f) \defeq \sqrt{(a_0,\ldots,a_m)}$.
\begin{enumerate}
\item Zeige für alle Polynome~$f, g \in A[X]$: $J(fg) = J(f) \cap J(g)$.
\item Ein Polynom heißt genau dann \emph{primitiv}, wenn sein Wurzelinhalt das
Einsideal ist. Folgere: Genau dann ist ein Produkt~$fg$ primitiv, wenn~$f$
und~$g$ es sind.
\end{enumerate}
\end{aufgabe}

\begin{aufgabe}{2+2+1}{Ideale bestehend aus Nullteilern}
\begin{enumerate}
\item Sei~$I$ ein Ideal eines Rings, das nur Nullteiler enthält. Zeige, dass es
in der Partialordnung all derjenigen Ideale, die~$I$ umfassen und nur Nullteiler
enthalten, ein maximales Element gibt.
\item Zeige, dass ein maximales Element wie in~a) stets ein Primideal ist.
\item Folgere: Die Menge der Nullteiler eines Rings ist eine Vereinigung von
Primidealen.
\end{enumerate}
\end{aufgabe}

\centering
\emph{Was bildet eine abelsche Gruppe unter Addition, einen Monoid unter
Multiplikation, erfüllt ein Distributivgesetz und ist verflucht?}
\par

\end{document}

\item Ist~$\mmm$ ein maximales Ideal in~$\ps{A}$, so gilt~$X \in \mmm$, die
Kontraktion~$\mmm_0 \defeq A \cap m$ ist ein maximales Ideal in~$A$ und~$\mmm$
ist das von~$\mmm_0$ und~$X$ in~$\ps{A}$ erzeugte Ideal.
\item Jedes Primideal von~$A$ ist Kontraktion eines Primideals von~$\ps{A}$.
