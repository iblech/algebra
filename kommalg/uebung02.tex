\documentclass{uebblatt}
\usepackage{wrapfig}

\newcommand{\http}{http:/\kern-.2em/\kern-0.03em}

\begin{document}

\maketitle{2}{Abgabe bis zum Montag, den 26. Oktober 2015}

\begin{aufgabe}{}{Invertierbarkeit und Nilpotenz in Ringen formaler Potenzreihen}
Sei~$A$ ein Ring. Sei~$f = a_0 + a_1 X + a_2 X^2 + \cdots \in
\ps{A}$ eine formale Potenzreihe. Zeige:
\begin{enumerate}
\item Genau dann ist~$f$ eine Einheit in~$\ps{A}$, wenn~$a_0$ in~$A$ invertierbar
ist.
\item Ist~$f$ nilpotent, so sind alle Koeffizienten~$a_0,a_1,\ldots$ nilpotent.
(Gilt die Umkehrung?)
\item Ist~$\mmm$ ein maximales Ideal in~$\ps{A}$, so gilt~$X \in \mmm$, die
Kontraktion~$\mmm_0 \defeq A \cap m$ ist ein maximales Ideal in~$A$ und~$\mmm$
ist das von~$\mmm_0$ und~$X$ in~$\ps{A}$ erzeugte Ideal.
\item Jedes Primideal von~$A$ ist Kontraktion eines Primideals von~$\ps{A}$.
\end{enumerate}
\end{aufgabe}

\begin{aufgabe}{}{Charakterisierung von Wurzelidealen}
Sei~$\aaa$ ein Ideal eines Rings. Zeige, dass~$\aaa$ genau dann mit
seinem Wurzelideal übereinstimmt, wenn~$\aaa$ ein Schnitt von Primidealen ist.
\end{aufgabe}

\begin{aufgabe}{}{Inhalt von Polynomen}
Sei~$A$ ein Ring. Sei für ein Polynom~$f = a_0 + \cdots + a_m X^m \in A[X]$ sein
\emph{Wurzelinhalt} das Ideal~$J(f) \defeq \sqrt{(a_0,\ldots,a_m)}$. Zeige für
alle Polynome~$f, g \in A[X]$: $J(fg) = J(f) \cap J(g)$.
\end{aufgabe}

\end{document}
