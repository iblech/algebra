\documentclass[entwurf]{uebblatt}

\newcommand{\http}{http:/\kern-.2em/\kern-0.03em}

\begin{document}

\maketitle{6}{}

\begin{aufgabe}{}{Ein konkretes Beispiel für eine Primärzerlegung}
Sei~$K$ ein Körper. Seien die Ideale~$\ppp_1 = (X,Y)$, $\ppp_2 = (X,Z)$
und~$\mmm = (X,Y,Z)$ von~$K[X,Y,Z]$ gegeben. Sei~$\aaa = \ppp_1 \ppp_2$.
\begin{enumerate}
\item Zeige, dass~$\aaa = \ppp_1 \cap \ppp_2 \cap \mmm^2$ eine minimale
Primärzerlegung von~$\aaa$ ist.
\item Welche Primideale von~$K[X,Y,Z]$ sind zu~$\aaa$
isolierte Primideale, welche eingebettete?
\item Schreibe die assoziierten Primideale in der Form~$\sqrt{(\aaa:f)}$ für
geeignete~$f \in K[X,Y,Z]$.
\end{enumerate}
\end{aufgabe}

\begin{aufgabe}{}{Ein Kriterium für Assoziiertheit}
Sei~$\aaa$ ein Ideal eines Rings~$A$.
Sei~$\ppp$ ein Ideal, das unter allen Idealen der Form~$(\aaa:x)$ mit~$x
\in A$ und~$x \not\in \aaa$ maximal ist.
\begin{enumerate}
\item Zeige, dass~$\ppp$ ein Primideal ist.
\item Sei~$\aaa$ zerlegbar. Zeige, dass~$\ppp$ ein zu~$\aaa$ assoziiertes
Primideal ist.
\end{enumerate}
\end{aufgabe}

\centering
\begin{center}
  \rotatebox{90}{\tiny\sffamily \http http://www.smbc-comics.com/?id=3565}
  \includegraphics[height=8.2cm]{images/smbc-fractions}
\end{center}

\end{document}
