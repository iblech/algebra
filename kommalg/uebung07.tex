\documentclass{uebblatt}
\usepackage{wrapfig}

\newcommand{\http}{http:/\kern-.2em/\kern-0.03em}

\begin{document}

\maketitle{7}{}

\begin{aufgabe}{4}{Ein konkretes Beispiel zur Noether-Normalisierung}
Sei~$K$ ein Körper. Gib eine Zahl~$r \geq 0$ und einen endlichen,
injektiven~$K$-Algebren-Homomorphismus~$K[Y_1,\ldots,Y_r] \to K[A,B]/(AB-1)$
an.
\end{aufgabe}

\begin{aufgabe}{2+m}{Geschenkte Bijektivität}
Sei~$\varphi : M \to M$ ein Endomorphismus eines Moduls~$M$. Zeige:
\begin{enumerate}
\item Ist~$\varphi$ surjektiv und~$M$ noethersch, so ist~$\varphi$ bijektiv.
\item Ist~$\varphi$ injektiv und~$M$ artinsch, so ist~$\varphi$ bijektiv.
\end{enumerate}
\end{aufgabe}

\begin{aufgabe}{2+2+2}{Endlichkeit minimaler Primideale}
\begin{enumerate}
\item Sei~$A$ ein Ring, in dem das Nilradikal Schnitt endlich vieler Primideale
ist. Zeige, dass~$A$ nur endlich viele minimale Primideale besitzt.
\item Zeige, dass das Nilradikal eines artischen Rings Schnitt
endlich vieler Primideale ist.
\item Zeige die Behauptung aus~b) auch für noethersche Ringe.

{\scriptsize\emph{Tipp.} Führe die Betrachtung eines Wurzelideals, das maximal
mit der Eigenschaft ist, nicht endlicher Schnitt von Primidealen zu sein (wieso
existiert ein solches?) zu einem Widerspruch.\par}
\end{enumerate}
\end{aufgabe}

\end{document}

* Witzige Konsequenzen aus N-N?
