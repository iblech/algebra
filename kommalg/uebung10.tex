\documentclass{uebblatt}

\newcommand{\https}{https:/\kern-.2em/\kern-0.03em}

\begin{document}

\maketitle{10}{}

\begin{aufgabe}{}{Eine elementare Charakterisierung der Dimension}
Für ein Ringelement~$x \in A$ sei~$\bbb_x$ das Ideal~$(x) + (\sqrt{(0)}:x)$.
\begin{enumerate}
\item Sei~$\ppp$ ein minimales Primideal. Sei~$x \in A$. Zeige,
dass~$\bbb_x \not\subseteq \ppp$.
\item Sei~$\ppp \subsetneq \qqq$ eine echte Inklusion von Primidealen. Sei~$x
\in \qqq \setminus \ppp$. Zeige, dass~$\bbb_x \subseteq \qqq$.
\item Zeige für~$n \geq 0$: Genau dann gilt~$\dim A \leq n$, wenn~$\dim
A/\bbb_x \leq n - 1$ für alle~$x \in A$.
\end{enumerate}
\end{aufgabe}

\end{document}
