\documentclass[entwurf]{uebblatt}

\newcommand{\https}{https:/\kern-.2em/\kern-0.03em}

\begin{document}

\maketitle{10}{}

\begin{aufgabe}{m+m+1+1}{Erste Schritte mit der Dimension von Ringen}
Bestimme für die folgenden Ringe ihre Dimension. Dabei ist~$K$ ein Körper.
\begin{enumerate}
\item $K$
\item $K[X]$
\item $\ZZ/(90)$
\item $K[X,Y]/(XY)$
\end{enumerate}
\end{aufgabe}

\begin{aufgabe}{}{Artinität über einem Körper}
Sei~$A$ eine endlich erzeugte Algebra über einem Körper~$K$. Zeige, dass~$A$
genau dann als Ring artinsch ist, wenn~$A$ als~$K$-Vektorraum endlich
dimensional ist.
\end{aufgabe}

\begin{aufgabe}{}{Beispiel für den Struktursatz artinscher Ringe}
Schreibe den artischen Ring~$\ZZ/(90)$ als Produkt artinscher lokaler Ringe.
\end{aufgabe}

\begin{aufgabe}{2+2+2+0}{Eine elementare Charakterisierung der Dimension}
Für ein Ringelement~$x \in A$ sei~$\bbb_x$ das Ideal~$(x) + (\sqrt{(0)}:x)$.
\begin{enumerate}
\item Sei~$\ppp$ ein minimales Primideal. Sei~$x \in A$. Zeige,
dass~$\bbb_x \not\subseteq \ppp$.
\item Sei~$\ppp \subsetneq \qqq$ eine echte Inklusion von Primidealen. Sei~$x
\in \qqq \setminus \ppp$. Zeige, dass~$\bbb_x \subseteq \qqq$.
\item Zeige für~$n \geq 0$: Genau dann gilt~$\dim A \leq n$, wenn~$\dim
A/\bbb_x \leq n - 1$ für alle~$x \in A$.
\item Folgere: Ein Ring ist genau dann von Dimension~$\leq n$, wenn für
je $n+1$ Ringelemente~$x_0,\ldots,x_n$ eine Zahl~$r \geq 0$ mit
\[ (x_0 \cdots x_n)^r \in (x_0 \cdots x_{n-1})^r (x_n^{r+1}) +
  (x_0 \cdots x_{n-2})^r (x_{n-1}^{r+1}) + \cdots +
  (x_0)^r (x_1^{r+1}) + (x_0^{r+1}) \]
existiert.
\end{enumerate}
\end{aufgabe}

\end{document}
