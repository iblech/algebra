\documentclass{uebblatt}

\begin{document}

\maketitle{10}{}

\begin{aufgabe}{m+m+1+1}{Erste Schritte mit der Dimension von Ringen}
Bestimme für die folgenden Ringe ihre Dimension. Dabei ist~$K$ ein Körper.
\begin{enumerate}
\item $K$
\item $K[X]$
\item $\ZZ/(90)$
\item $K[X,Y]/(XY)$
\end{enumerate}
\end{aufgabe}

\begin{aufgabe}{3}{Beispiel für den Struktursatz artinscher Ringe}
Schreibe den artischen Ring~$\ZZ/(90)$ als Produkt artinscher lokaler Ringe.
Gib also artinsche lokale Ringe~$A_1,\ldots,A_n$ und einen
Isomorphismus~$\ZZ/(90) \to A_1 \times \cdots \times A_n$ an.
\end{aufgabe}

\begin{aufgabe}{m+2+3+m}{Artinität über einem Körper}
\begin{enumerate}
\item Seien~$R$ ein Ring und~$M$ ein einfacher~$R$-Modul. Zeige, dass~$M$ isomorph
zu einem~$R$-Modul der Form~$R/\mmm$ ist, wobei~$\mmm$ ein maximales Ideal in~$R$
ist.
\end{enumerate}
Sei im Folgenden~$A$ eine endlich erzeugte lokale Algebra über einem
Körper~$K$. Sei~$\mmm$ das maximale Ideal von~$A$.
\begin{enumerate}
\addtocounter{enumi}{1}
\item Sei~$M = M_0 \supsetneq M_1 \supsetneq \cdots \supsetneq M_n = 0$ eine
Kompositionsreihe eines~$A$-Moduls~$M$. Zeige, dass~$M$ als~$K$-Vektorraum
von Dimension~$\dim_K M = n \cdot \dim_K(A/\mmm)$ ist.
\item Zeige, dass~$A$ genau dann als Ring artinsch ist, wenn~$A$
als~$K$-Vektorraum endlich dimensional ist. Zeige weiter, dass in diesem
Fall~$\dim_K A = \ell(A) \cdot \dim_K(A/\mmm)$ gilt.
\item Zeige den ersten Teil der Behauptung aus Teilaufgabe~c) auch für den
Fall, dass~$A$ nicht lokal ist.
\end{enumerate}
\end{aufgabe}
% entfernt verwandt: https://math.berkeley.edu/~gbergman/papers/P_vs_cP.pdf

\begin{aufgabe}{2+m+2+0}{Eine elementare Charakterisierung der Dimension}
Für ein Ringelement~$x \in A$ sei~$\bbb_x$ das Ideal~$(x) + (\sqrt{(0)}:x)$.
\begin{enumerate}
\item Sei~$\ppp$ ein minimales Primideal. Sei~$x \in A$. Zeige,
dass~$\bbb_x \not\subseteq \ppp$.
\item Sei~$\ppp \subsetneq \qqq$ eine echte Inklusion von Primidealen. Sei~$x
\in \qqq \setminus \ppp$. Zeige, dass~$\bbb_x \subseteq \qqq$.
\item Zeige für~$n \geq 0$: Genau dann gilt~$\dim A \leq n$, wenn~$\dim
A/\bbb_x \leq n - 1$ für alle~$x \in A$.
\item Folgere: Ein Ring ist genau dann von Dimension~$\leq n$, wenn für
je $n+1$ Ringelemente~$x_0,\ldots,x_n$ eine Zahl~$r \geq 0$ mit
\[ (x_0 \cdots x_n)^r \in (x_0 \cdots x_{n-1})^r (x_n^{r+1}) +
  (x_0 \cdots x_{n-2})^r (x_{n-1}^{r+1}) + \cdots +
  (x_0)^r (x_1^{r+1}) + (x_0^{r+1}) \]
existiert.
\end{enumerate}
\end{aufgabe}

\centering
\emph{\textbf{Joke 21.5.2.} Old Macdonald had a form; $e_i \wedge e_i = 0$.}

\end{document}
