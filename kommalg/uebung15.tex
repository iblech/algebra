\documentclass{uebblatt}

\begin{document}

\maketitle{1111}{}
\enlargethispage{2em}

\begin{aufgabe}{m}{Eine explizite Beschreibung der adischen Vervollständigung}
Sei~$\aaa = (x_1,\ldots,x_n)$ ein Ideal in einem noetherschen Ring~$A$. Zeige,
dass die Vervollständigung~$\hat A_\aaa$ isomorph
zu~$\psmany{A}{X_1,\ldots,X_n}/(X_1-x_1,\ldots,X_n-x_n)$ ist.
\end{aufgabe}

\begin{aufgabe}{m+m}{Intervalle von Primidealen in noetherschen Ringen}
\begin{enumerate}
\item
Seien~$\ppp \subseteq \qqq$ Primideale in einem
noetherschen Ring. Sei~$(\ppp,\qqq)$ die Menge all derjenigen Primideale~$\rrr$
mit~$\ppp \subsetneq \rrr \subsetneq \qqq$. Zeige, dass~$(\ppp,\qqq)$ entweder
leer oder unendlich ist.
% Falls nicht leer und endlich: Prop. 6.17, um ein Element x zu finden, über
% dem q minimal sitzt. Dann klar (Noether-Voraussetzung nutzen!).
\item Sei~$A$ ein noetherscher Ring in dem alle Primideale in einer einzigen Kette~$\ppp_0
\subsetneq \ppp_1 \subsetneq \cdots \subsetneq \ppp_n$ mit~$n \geq 2$ auftreten.
Zeige: Es gibt ein Element~$x \in A$ mit~$x + 0 \neq x$.
\end{enumerate}
\end{aufgabe}

\begin{aufgabe}{m}{Dimension des Polynomrings im noetherschen Fall}
\begin{enumerate}
\item Sei~$\ppp$ ein Primideal in einem Ring, das minimal mit der Eigenschaft
ist, ein gegebenes zerlegbares Ideal~$\aaa$ zu umfassen. Zeige, dass~$\ppp$
zu~$\aaa$ assoziiert und sogar isoliert ist.
\item Sei~$\ppp$ ein Primideal der Höhe~$r$ in einem noetherschen Ring.
Zeige, dass Elemente~$x_1,\ldots,x_r$ existieren, sodass~$\ppp$ unter allen
Primidealen, die diese Elemente enthalten, minimal ist.
% Satz 46.1
\item Zeige für alle Primideale~$\ppp$ eines noetherschen Rings:~$\height
\ppp[X] = \height \ppp$.
% b) und Folgerung 45.14
\item Sei~$A$ ein Ring in dem die Behauptung von~c) gilt.
Sei~$\qqq \subseteq A[X]$ ein Primideal. Sei~$\ppp \defeq A \cap \qqq$.
Zeige:~$\height \qqq \leq \height \ppp + 1$.
% Induktion über die Höhe von q. Im Induktionsschritt ein q' < q mit um eins
% verminderter Höhe hernehmen. Dran denken, dass p[X] <= q (und p'[X] <= q').
\item Folgere: Für noethersche Ringe~$A \neq 0$ gilt $\dim A[X] = 1 + \dim A$.
\end{enumerate}
\end{aufgabe}

\begin{aufgabe}{m}{Gar nicht mehr erste Schritte mit der Dimension von Ringen}
Berechne die Dimension des Rings~$\CC[X,Y,Z]/(X-Z, X^2+Y^2+Z^2)$.
\end{aufgabe}

\begin{aufgabe}{0}{Der mystische Körper mit einem Element}
\scriptsize
Der~$n$-dimensionale projektive Raum~$\PP^n_k$ über einem Körper~$k$ ist der
Raum der Ursprungsgeraden in~$k^{n+1}$.
\begin{enumerate}
\item Wie viele Punkte enthält~$\PP^n_k$, wenn~$k$ ein Körper mit~$q$
Elementen ist? Gib die Anzahl als Polynom in~$q$ an.
\item Im klassischen Zugang zur Algebra gibt es nichts, was die Bezeichnung
\emph{Körper mit einem Element} verdient hätte. Was passiert, wenn man trotzdem
in der Formel aus~a)~$q \defeq 1$ setzt? Was sollte also ein~$n$-dimensionaler
projektiver Raum über dem Körper mit einem Element sein?
\end{enumerate}
\end{aufgabe}

\centering
\href{http://www.neverendingbooks.org/mumfords-treasure-map}{\includegraphics[scale=0.2]{images/mumfords-treasure-map}} \\[0.5em]
\small
\emph{Mumfords Schatzkarte: die Primideale von~$\ZZ[X]$}

\end{document}

Dimension von { M in K^(n times m) | rk M <= k }.
Kemper, Seite 70.

Was ist die Höhe des Primideals (pX - 1) in Z_p[X]?

http://www.uio.no/studier/emner/matnat/math/MAT4200/h12/mat4200_h12_problems.pdf
http://folk.uio.no/fredrme/Kommalg.pdf (ganz am Ende)

\begin{aufgabe}{m}{Nulldimensionale reguläre lokale Ringe}
Zeige, dass ein Ring genau dann ein nulldimensionaler regulärer lokaler Ring
ist, wenn er ein Körper ist.
\end{aufgabe}
