\documentclass[entwurf]{uebblatt}

\newcommand{\http}{http:/\kern-.2em/\kern-0.03em}

\begin{document}

\maketitle{15}{}

\begin{aufgabe}{3+m}{Intervalle von Primidealen in noetherschen Ringen}
\begin{enumerate}
\item
Seien~$\ppp \subseteq \qqq$ Primideale in einem
noetherschen Ring. Sei~$(\ppp,\qqq)$ die Menge all derjenigen Primideale~$\rrr$
mit~$\ppp \subsetneq \rrr \subsetneq \qqq$. Zeige, dass~$(\ppp,\qqq)$ entweder
leer oder unendlich ist.
\item Sei~$A$ ein noetherscher Ring in dem alle Primideale in einer einzigen Kette~$\ppp_0
\subsetneq \ppp_1 \subsetneq \cdots \subsetneq \ppp_n$ mit~$n \geq 2$ auftreten.
Zeige, dass es in~$A$ ein Element~$x$ mit~$x + 0 \neq x$ gibt.
\end{enumerate}
\end{aufgabe}

\begin{aufgabe}{}{Dimension des Polynomrings im noetherschen Fall}
Sei~$A$ ein noetherscher Ring. Zeige: $\dim A[X] = 1 + \dim A$.
% Teilaufgabe dazu formulieren!
\end{aufgabe}

\end{document}

Dimension von { M in K^(n times m) | rk M <= k }.
Kemper, Seite 70.

\hat A_a = A[[x_1,...,x_n]]/(x_i-a_i) für a = (a_1,...,a_n) und A noethersch?
http://www.math.uchicago.edu/~may/MISC/Topologies.pdf
