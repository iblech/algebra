\documentclass{algblatt}
\loesungenfalse

\geometry{tmargin=2cm,bmargin=2cm,lmargin=3.0cm,rmargin=3.0cm}

\setlength{\titleskip}{0.7em}
\setlength{\aufgabenskip}{1.5em}

\begin{document}

\vspace*{-1.2cm}
\maketitle{2}{Abgabe bis 29. April 2013, 17:00 Uhr}

\begin{aufgabe}{Lösungen polynomieller Gleichungen sind algebraisch}
Sei $z$ eine Lösung der Polynomgleichung
\[ X^3 - \sqrt{2 - \sqrt[3] 4} \, X^2 + 3 = 0. \]
Finde eine normierte Polynomgleichung mit \emph{rationalen} Koeffizienten,
die~$z$ als Lösung hat.
\end{aufgabe}

\begin{aufgabe}{Auf den Spuren Bombellis}
Zeige formal die zuerst von Rafael Bombelli (1526--1572, italienischer
Mathematiker) gefundene
Gleichheit
\[
    (2 \pm \sqrt{-1})^3 = 2 \pm \sqrt{- 121}
\]
und diskutiere, welche Vorzeichen der Quadratwurzeln jeweils zu wählen sind.
\end{aufgabe}

\begin{aufgabeE}{Rechnen mit komplexen Zahlen}
\item
Sei~$z \neq 0$ eine komplexe Zahl, deren Real- und Imaginärteil rationale
Zahlen sind. Zeige, dass~$z^{-1}$ ebenfalls rationalen Real- und Imaginärteil
hat.
\item
Zeige, dass der Realteil einer komplexen Zahl~$z$ durch
\(\frac 1 2 (z + \overline z)\)
und dass der Imaginärteil durch \(\frac 1 {2 \i} (z - \overline z)\)
gegeben ist.
\item
Sei \(z\) eine invertierbare komplexe Zahl. Folgere die Gleichheit
$
    \overline{z}^{-1} = \overline{z^{-1}}
$
aus der Multiplikativität der komplexen Konjugation.
\item
Interpretiere die Multiplikation mit der imaginären Einheit~$\i$
geometrisch.
\end{aufgabeE}

\begin{aufgabe}{Zahlen nahe bei Null}
Zeige, dass für zwei reelle Zahlen $a$ und $b$ genau dann die Wurzel
$\sqrt{a^2 + b^2}$ nahe bei Null ist, wenn sowohl $|a|$ als auch $|b|$
nahe bei Null sind. Zeige also:
\[ \begin{array}{@{}rr@{\ }c@{\ }l@{}}
  \forall \epsilon > 0\ \exists \delta > 0 \colon &
        \sqrt{a^2 + b^2} < \delta & \implies & |a|, |b| < \epsilon \\[0.5em]
    \forall \epsilon > 0\ \exists \delta > 0 \colon &
        |a|, |b| < \delta & \implies & \sqrt{a^2  + b^2} < \epsilon
\end{array}
\]
\vspace{-1.5em}
\end{aufgabe}

\begin{aufgabeE}{Ein neuer Zahlbereich}
\item
Zeige, dass die Gleichung $X^2 + X + 1 = 0$ in den reellen Zahlen
keine Lösung besitzt.
\item
Konstruiere einen minimalen Zahlbereich~$\RR(\omega)$, welcher die
reellen Zahlen und eine Lösung~$\omega$ der Gleichung $X^2 + X + 1 = 0$
enthält und in welchem Addition und Multiplikation so definiert sind, dass sie
die Addition und Multiplikation reeller Zahlen fortsetzen und die
einschlägigen Gesetze der Arithmetik erfüllen.
\item Zeige, dass~$\omega^3 = 1$ in~$\RR(\omega)$ gilt.
\item Finde eine Lösung der Gleichung~$X^2 + 1 = 0$ in~$\RR(\omega)$.
\end{aufgabeE}

\end{document}
\begin{aufgabe}{Drehmatrizen}
Erkläre, warum \(\left(\begin{smallmatrix}
                           \cos \alpha & - \sin \alpha\\
                           \sin \alpha & \cos \alpha
                       \end{smallmatrix}\right)\) eine Drehung um den Winkel \(\alpha\) um
den Ursprung der Gaußschen Zahlenebene beschreibt. Folgere sodann
die Additionstheoreme
\begin{align*}
    \cos (\alpha_1 + \alpha_2) & = \cos \alpha_1 \, \cos \alpha_2
        - \sin \alpha_1 \, \sin \alpha_2
    \\
    \sin (\alpha_1 + \alpha_2) & = \cos \alpha_1 \, \sin \alpha_2
        + \sin \alpha_1 \, \cos \alpha_2
\end{align*}
aus der bekannten Formel für das Produkt von Matrizen.
\end{aufgabe}

\end{document}
