\documentclass{algblatt}
\loesungenfalse

\geometry{tmargin=2cm,bmargin=2cm,lmargin=3.0cm,rmargin=3.0cm}

\setlength{\titleskip}{0.7em}
\setlength{\aufgabenskip}{1.5em}

\begin{document}

\vspace*{-1.2cm}
\maketitle{2}{Abgabe bis 29. April 2013, 17:00 Uhr}

\begin{aufgabe}{Lösungen polynomieller Gleichungen sind algebraisch}
Sei $z$ eine Lösung der Polynomgleichung
\[ X^3 - \sqrt{2 - \sqrt[3]{4}} \, X^2 + 3 = 0. \]
Finde eine normierte Polynomgleichung mit \emph{rationalen} Koeffizienten,
die~$z$ als Lösung hat.
\begin{loesung}
Wir formen um:
\begin{align*}
  &&z^3 - \sqrt{2 - \sqrt[3]{4}} \, z^2 + 3 &= 0 \\
  \Longleftrightarrow&&
  z^3 + 3 &= \sqrt{2 - \sqrt[3]{4}} \, z^2 \\
  \Longrightarrow&&
  (z^3 + 3)^2 &= 2 \, z^4 - \sqrt[3]{4} \, z^4 \\
  \Longleftrightarrow&&
  (z^3 + 3)^2 - 2 \, z^4 &= \sqrt[3]{4} \, z^4 \\
  \Longrightarrow&&
  ((z^3 + 3)^2 - 2 \, z^4)^3 &= 4 \, z^{12} \\
  \Longleftrightarrow&&
  ((z^3 + 3)^2 - 2 \, z^4)^3 - 4 \, z^{12} &= 0
\end{align*}
Die gesuchte Gleichung lautet also
\[ ((X^3 + 3)^2 - 2 \, X^4)^3 - 4 \, X^{12} = 0, \]
denn diese hat~$z$ als Lösung (Probe unnötig, wieso?), ihre Koeffizienten sind
alle rational und sie ist normiert (das ist nicht ganz offensichtlich, wieso
stimmt das?).

\emph{Bemerkung:} Ausmultipliziert wird die Gleichung nicht schöner:
\begin{multline*}
X^{18}-6\,X^{16}+18\,X^{15}+12\,X^{14}-72\,X^{13}+123\,X^{12}+72\,X^{11}-324\,X^{10}+\\540\,X^{9}
+
          108\,X^{8}-648\,X^{7}+1215\,X^{6}-486\,X^{4}+1458\,X^{3}+729=0.
\end{multline*}
\end{loesung}
\end{aufgabe}

\begin{aufgabe}{Auf den Spuren Bombellis}
Zeige formal die zuerst von Rafael Bombelli (1526--1572, italienischer
Mathematiker) gefundene
Gleichheit
\[
    (2 \pm \sqrt{-1})^3 = 2 \pm \sqrt{-121}
\]
und diskutiere, welche Vorzeichen der Quadratwurzeln jeweils zu wählen sind.
\begin{loesung}
Mit der binomischen Formel multiplizieren wir die linke Seite aus:
\[ (2 \pm \i)^3 = 8 \pm 3 \cdot 4 \, \i - 3 \cdot 2 \pm \i^3 =
  2 \pm 11 \, \i. \]
Bombellis Formel stimmt also, wenn man $\sqrt{-1}$ konsistent als~$\i$ (statt
als~$-\i$) und $\sqrt{-121}$ konsistent als~$11 \,\i$ (statt als~$-11\,\i$)
liest und dann entweder auf beiden Seiten~"`$+$"' oder auf beiden Seiten~"`$-$"' für
das~"`$\pm$"'-Zeichen nimmt.
\end{loesung}
\end{aufgabe}

\ifloesungen\newpage\fi
\begin{aufgabe}{Rechnen mit komplexen Zahlen}
\begin{enumerate}
\item
Sei~$z \neq 0$ eine komplexe Zahl, deren Real- und Imaginärteil rationale
Zahlen sind. Zeige, dass~$z^{-1}$ ebenfalls rationalen Real- und Imaginärteil
hat.
\item
Zeige, dass der Realteil einer komplexen Zahl~$z$ durch $\frac{1}{2} (z + \overline z)$
und dass der Imaginärteil durch $\frac{1}{2 \i} (z - \overline z)$ gegeben ist.
\item
Sei~$z$ eine invertierbare komplexe Zahl. Folgere die Gleichheit
$\overline{z}^{-1} = \overline{z^{-1}}$
aus der Multiplikativität der komplexen Konjugation.
\item
Interpretiere die Multiplikation mit der imaginären Einheit~$\i$
geometrisch.
\end{enumerate}
\begin{loesungE}
\item Wir schreiben~$z = a + b \i$ mit~$a,b \in \QQ$. Dann gilt
\[ z^{-1} = \frac{1}{a + b \i} = \frac{a - b \i}{a^2 + b^2} =
  \frac{a}{a^2 + b^2} + \frac{-b}{a^2 + b^2} \i \]
und wir sehen, dass in der Tat Real- und Imaginärteil wieder in~$\QQ$ liegen.
\item Wir schreiben~$z = a + b \i$ mit~$a,b \in \RR$ und rechnen:
\begin{align*}
  \tfrac{1}{2} (z + \overline z) &= \tfrac{1}{2} (a + b \i + a - b \i) = a \\
  \tfrac{1}{2\i} (z - \overline z) &= \tfrac{1}{2\i} (a + b \i - a + b \i) = b
\end{align*}
\item Aus~$z \cdot z^{-1} = 1$ folgt wegen der Multiplikativität der komplexen
Konjugation $\overline z \cdot \overline{z^{-1}} = 1$. Also
ist~$\overline{z^{-1}}$ das Inverse von~$\overline{z}$, das war zu zeigen.
\item Drehung um $90^\circ$ um den Ursprung im Gegenuhrzeigersinn (wieso?). Skizze!
\end{loesungE}
\end{aufgabe}

\begin{aufgabe}{Zahlen nahe bei Null}
Zeige, dass für zwei reelle Zahlen $a$ und $b$ genau dann die Wurzel
$\sqrt{a^2 + b^2}$ nahe bei Null ist, wenn sowohl $|a|$ als auch $|b|$
nahe bei Null sind. Zeige also:
\[ \begin{array}{@{}rr@{\ }c@{\ }l@{}}
  \forall \epsilon > 0\ \exists \delta > 0 \colon &
        \sqrt{a^2 + b^2} < \delta & \implies & |a|, |b| < \epsilon \\[0.5em]
    \forall \epsilon > 0\ \exists \delta > 0 \colon &
        |a|, |b| < \delta & \implies & \sqrt{a^2  + b^2} < \epsilon
\end{array} \]
\vspace{-1.0em}
\begin{loesung}
\begin{enumerate}
\item Sei~$\epsilon > 0$ beliebig. Setze~$\delta := \epsilon$. Gelte~$\sqrt{a^2
+ b^2} < \delta$. Dann folgt
\[ |a| = \sqrt{|a|^2} \leq \sqrt{|a|^2 + |b|^2} < \delta = \epsilon \]
und analog mit~$|b|$.
\item Sei~$\epsilon > 0$ beliebig. Setze~$\delta := \epsilon/\sqrt{2}$.
Gelte~$|a|,|b| < \delta$. Dann folgt
\[ \sqrt{a^2 + b^2} < \sqrt{\delta^2 + \delta^2} = \sqrt{2} \delta = \epsilon. \]
\end{enumerate}
\emph{Bemerkung:} Auf die passenden Wahlen von~$\delta$ kommt man natürlich
nicht im Vornhinein, sondern erst nach erfolgter Abschätzung.
\end{loesung}
\end{aufgabe}

\ifloesungen\newpage\fi
\begin{aufgabe}{Ein neuer Zahlbereich}
\begin{enumerate}
\item
Zeige, dass die Gleichung $X^2 + X + 1 = 0$ in den reellen Zahlen
keine Lösung besitzt.
\item
Konstruiere einen minimalen Zahlbereich~$\RR(\omega)$, welcher die
reellen Zahlen und eine Lösung~$\omega$ der Gleichung $X^2 + X + 1 = 0$
enthält und in welchem Addition und Multiplikation so definiert sind, dass sie
die Addition und Multiplikation reeller Zahlen fortsetzen und die
einschlägigen Gesetze der Arithmetik erfüllen.
\item Zeige, dass~$\omega^3 = 1$ in~$\RR(\omega)$ gilt.
\item Finde eine Lösung der Gleichung~$X^2 + 1 = 0$ in~$\RR(\omega)$.
\end{enumerate}
\begin{loesungE}
\item \emph{Variante 1:} Man verwendet die Mitternachtsformel und sieht, dass
die beiden Lösungen der Gleichung jeweils echt komplex sind:
\[ \frac{-1 \pm \sqrt{1 - 4}}{2} \]
\emph{Variante 2:} Man zeigt durch quadratische Ergänzung, dass für jedes
reelle~$x$ die linke Seite der Gleichung positiv (und daher nicht null) ist:
\[ x^2 + x + 1 = (x + 1/2)^2 - 1/4 + 1 \geq 3/4 > 0. \]

\item Wir versuchen, die Konstruktion der komplexen Zahlen auf die Situation
hier zu übertragen. Dazu überlegen wir zunächst, wie der
Rechenbereich~$\RR(\omega)$ aussähe, wenn er existierte.
Da~$\RR(\omega)$ die reellen Zahlen umfassen soll, muss~$\RR(\omega)$
neben~$\omega$ selbst auch alle Zahlen der Form~$a + b\omega$ enthalten.
Weitere Zahlen sind aber nicht nötig, da Summe und Produkt solcher Zahlen
wieder von dieser Form sind:
\begin{align}
  \label{omegaadd}
  (a + b \omega) + (c + d \omega) &= (a+c) + (b+d) \omega \\
  \label{omegamult}
  (a + b \omega) \cdot (c + d \omega) &= ac + ad\omega + bc\omega + bd\omega^2 =
  (ac-bd) + (ad+bc-bd)\omega
\end{align}
Dabei haben wir die Identität~$\omega^2 = -\omega - 1$ verwendet. Wir haben
also keinen Anlass, noch weitere Zahlen aufzunehmen.

Es liegt somit nahe, den Zahlbereich~$\RR(\omega)$ formal als Menge von Paaren zu
definieren,
\[ \RR(\omega) := \{ (a,b) \,|\, a,b \in \RR \}\!, \]
und die Rechnungen~\eqref{omegaadd},~\eqref{omegamult} als Definition der
Rechenarten zu verwenden.

\emph{Bemerkung:} Wer den Faktorringbegriff kennt, kann kürzer auch definieren:
\[ \RR(\omega) := \RR[X]/(X^2 + X + 1). \]

\item Per Definition gilt~$\omega^2 + \omega + 1 = 0$, also $\omega^2 = -\omega
- 1$. Damit kann man~$\omega^3$ explizit ausrechnen:
\[ \omega^3 = \omega \cdot \omega^2 = -\omega^2 - \omega = -(-\omega - 1) -
\omega = 1. \]
\emph{Bemerkung:} Alternativ kann man auch den Faktor~$(\omega-1)$ vom Himmel
fallen lassen:
\[ 0 = (\omega^2 + \omega + 1) (\omega - 1) = \omega^3 - 1. \]

\item \emph{Variante 1:} Wir untersuchen für alle~$a,b \in \RR$, ob~$x := a + b
\omega$ eine Lösung der Gleichung ist:
\begin{align*}
  & (a + b\omega)^2 + 1 = 0 \\
  \Longleftrightarrow\ &
  (a^2 - b^2 + 1) + (-b^2 + 2ab) \omega = 0 \\
  \Longleftrightarrow\ &
  a^2 - b^2 + 1 = 0 \text{ und } b \, (2a-b) = 0 \\
  \stackrel{?}{\Longleftrightarrow}\ &
  a^2 - b^2 + 1 = 0 \text{ und } 2a = b \\
  \Longleftrightarrow\ &
  1 = 3 a^2 \text{ und } 2a = b \\
  \Longleftrightarrow\ &
  a = \pm 1 / \sqrt{3} \text{ und } b = \pm 2 / \sqrt{3}
\end{align*}
Da wir insbesondere die Richtung "`$\Leftarrow$"' haben, folgt also: Die
Gleichung~$X^2 + 1 = 0$ hat in~$\RR(\omega)$ zwei Lösungen, nämlich
\[ \frac{1}{\sqrt{3}} + \frac{2}{\sqrt{3}} \omega \quad\text{und}\quad
  {-\frac{1}{\sqrt{3}} - \frac{2}{\sqrt{3}} \omega}. \]

\emph{Variante 2:} Wir verwenden die quadratische Ergänzung von oben:
\[ 0 = \omega^2 + \omega + 1 = (\omega + 1/2)^2 + 3/4, \]
also folgt
\[ \frac{4}{3} \left(\omega + \frac{1}{2}\right)^2 = \left(
  \frac{2}{\sqrt{3}} \omega + \frac{1}{\sqrt{3}}
\right)^2 = -1 \]
und man kann ebenfalls die beiden Lösungen ablesen.

\emph{Bemerkung:} Da es umgekehrt in~$\CC$ eine Lösung~$\xi$ der Gleichung
$X^2+X+1=0$ gibt, sieht man, dass der in dieser Aufgabe neu konstruierte
Rechenbereich~$\RR(\omega)$ tatsächlich isomorph zu~$\CC$ ist. Man kann einen
Isomorphismus sogar explizit hinschreiben:
\[ \begin{array}{@{}rcl@{}}
  \RR(\omega) &\longrightarrow& \CC \\
  a+b\omega &\longmapsto& a+b\xi
\end{array} \]
\end{loesungE}
\end{aufgabe}

\end{document}
\begin{aufgabe}{Drehmatrizen}
Erkläre, warum \(\left(\begin{smallmatrix}
                           \cos \alpha & - \sin \alpha\\
                           \sin \alpha & \cos \alpha
                       \end{smallmatrix}\right)\) eine Drehung um den Winkel \(\alpha\) um
den Ursprung der Gaußschen Zahlenebene beschreibt. Folgere sodann
die Additionstheoreme
\begin{align*}
    \cos (\alpha_1 + \alpha_2) & = \cos \alpha_1 \, \cos \alpha_2
        - \sin \alpha_1 \, \sin \alpha_2
    \\
    \sin (\alpha_1 + \alpha_2) & = \cos \alpha_1 \, \sin \alpha_2
        + \sin \alpha_1 \, \cos \alpha_2
\end{align*}
aus der bekannten Formel für das Produkt von Matrizen.
\end{aufgabe}

\end{document}
