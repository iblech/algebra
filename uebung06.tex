\documentclass{algblatt}
\usepackage{xstring}
\IfSubStr{\jobname}{\detokenize{loesung}}{\loesungentrue}{\loesungenfalse}

\geometry{tmargin=2cm,bmargin=2cm,lmargin=2.9cm,rmargin=2.9cm}

%\setlength{\titleskip}{0.7em}
\setlength{\aufgabenskip}{1.6em}

\begin{document}

\vspace*{-1.5cm}
\maketitle{6}{Abgabe bis 27. Mai 2013, 17:00 Uhr}

\begin{aufgabe}{Anwendungen der Diskriminante}
\begin{enumerate}
\item
Sei~$X^3 + p X + q = 0$ eine reduzierte kubische Gleichung mit ganzzahligen
Koeffizienten~$p$ und~$q$. Zeige, dass die Gleichung drei \emph{verschiedene} Lösungen
(in den komplexen Zahlen) besitzt, wenn~$q$ ungerade ist.
\item Sei~$X^n + a_{n-1}X^{n-1} + \cdots + a_1 X + a_0 = 0$ eine normierte
Polynomgleichung mit rationalen Koeffizienten. Zeige, dass sie mindestens eine
nicht reelle Nullstelle besitzt, wenn ihre Diskriminante negativ ist.
\end{enumerate}
\end{aufgabe}

\begin{aufgabe}{Diskriminanten allgemeiner kubischer Gleichungen}
\begin{enumerate}
\item Berechne die Diskriminante der allgemeinen kubischen Gleichung
$X^3 + a X^2 + b X + c = 0$.
\item Zeige, dass~$X^3 - 5\,X^2 + 3\,X + 9 = 0$ höchstens zwei verschiedene
Lösungen hat.
\end{enumerate}
\end{aufgabe}

\begin{aufgabe}{Die Resultante zweier Polynome}
\begin{enumerate}
\item Seien~$f(X)$ und~$g(X)$ zwei normierte Polynome mit Nullstellen (mit
Vielfachheiten)~$x_1,\ldots,x_n$ bzw.~$y_1,\ldots,y_m$. Zeige, dass der
Ausdruck~$R := \prod_{i,j} (x_i - y_j)$ ein Polynom in den
Koeffizienten von~$f(X)$ und den
Koeffizienten von~$g(X)$ ist.
\item Seien~$X^2 + aX + b = 0$ und~$X^2 + cX + d = 0$ zwei quadratische
Gleichungen. Gib einen in~$a$, $b$, $c$ und~$d$ polynomiellen Ausdruck an, der
genau dann verschwindet, wenn die beiden Gleichungen eine gemeinsame Lösung
besitzen.
\end{enumerate}
\end{aufgabe}

\begin{aufgabe}{Transzendente Zahlen}
\begin{enumerate}
\item Sei~$(z_n)$ eine konvergente komplexe Zahlenfolge mit Grenzwert~$z$ und
seien alle Folgenglieder~$z_n$ algebraisch. Ist dann auch~$z$ algebraisch?
\item Ist~$\sqrt[3]{\pi}$ eine algebraische Zahl? Ist~$\pi^3$ algebraisch?
\item Finde eine Folge paarweise verschiedener transzendenter Zahlen.
\end{enumerate}
\end{aufgabe}

\begin{aufgabe}{Triangulatur des Kreises}
Ist folgendes Problem lösbar? Gegeben ein Kreis. Konstruiere nur mit Zirkel und
Lineal ein gleichseitiges Dreieck mit demselben Flächeninhalt.
\end{aufgabe}

Nicht verpassen: \textbf{Gauß-Vorlesung} über Muster bei Primzahlen am 28. Mai
ab 17:00 Uhr im Parktheater Göggingen, mehr Informationen auf
\url{http://xrl.us/gauss2013}.
 
\end{document} 
