\documentclass[12pt,a4paper,ngerman]{scrartcl}

\usepackage{ucs}
\usepackage[utf8x]{inputenc}
\usepackage[ngerman]{babel}
\usepackage{amsmath,amssymb,amscd,amsthm,graphicx}
\usepackage[protrusion=true,expansion=false]{microtype}
\usepackage{lmodern}
\usepackage{hyperref}
\usepackage[usenames,dvipsnames]{xcolor}
\usepackage{tabto}

\setlength\parskip{\medskipamount}
\setlength\parindent{0pt}

\newcommand{\Bild}{\operatorname{im}}
\newcommand{\Kern}{\operatorname{ker}}
\newcommand{\Span}{\operatorname{span}}
\newcommand{\GL}{\mathrm{GL}}
\newcommand{\R}{\mathbb{R}}
\newcommand{\C}{\mathbb{C}}
\newcommand{\I}{\mathrm{i}}
\newcommand{\xra}[1]{\xrightarrow{#1}}

\newenvironment{indentblock}{%
  \list{}{\leftmargin\leftmargin}%
  \item\relax
}{%
  \endlist
}

\theoremstyle{definition}
\newtheorem*{defn}{Definition}
\newtheorem*{bsp}{Beispiel}

\theoremstyle{plain}
\newtheorem*{prop}{Proposition}
\newtheorem*{beh}{Behauptung}

\theoremstyle{remark}
\newtheorem*{bem}{Bemerkung}

%\usepackage{geometry}
%\geometry{tmargin=3cm,bmargin=3cm,lmargin=3cm,rmargin=3cm}

\usepackage{fancyhdr}
\pagestyle{fancy}
\rhead{19. April 2013}
\lhead{Beweistechniken}

\renewcommand{\labelitemi}{--}

\newcommand{\hil}[1]{\textcolor{blue}{#1}}

\begin{document}

\begin{center}\includegraphics[scale=0.8]{saint_curious_george.png}\end{center}
\vfill
\footnotetext{Comic: \url{http://abstrusegoose.com/353}}
\newpage


\section*{Beweistechniken}

Bis man zum eigentlichen Kern eines mathematischen Problems vordringt, muss man
im Allgemeinen mehrere Definitionen entfalten. Das gelingt meist mit der
Technik des direkten Beweisens ("`immer der Nase nach"').

\emph{Beispiel:} Seien $f{:}\, X \to Y$ und $g{:}\, Y \to Z$ zwei Abbildungen. Wir
wollen zeigen:
\[ \text{$f$ injektiv} \,\wedge\, \text{$g$ injektiv} \ \Longrightarrow\ 
  \text{$g \circ f$ injektiv}. \]
Mit den unten beschriebenen Vorlagen lautet ein sehr ausführlicher Beweis wie
folgt:
\begin{indentblock}
Gelte, dass~$f$ und~$g$ injektiv sind. Um dann die Injektivität von~$g
\circ f$ zu zeigen, seien~$x, \tilde x \in X$
beliebig und gelte~$(g \circ f)(x) = (g \circ f)(\tilde x)$. Nach Definition
der Abbildungsverkettung gilt also
\begin{align*}g(f(x)) &= g(f(\tilde x)). \\
\intertext{Da $g$ injektiv ist, folgt daraus}
f(x) &= f(\tilde x). \\
\intertext{Da auch $f$ injektiv ist, folgt daraus wiederum}
x &= \tilde x. \\
\intertext{Das war zu zeigen.}\end{align*}
\end{indentblock}
\vspace{-2em}
Der Kern der Argumentation liegt in den Folgerungen zwischen den drei
abgesetzten Gleichungen, der Vorbau ist aber trotzdem wichtig; insbesondere ist
es wichtig, die Variablen~$x$ und~$\tilde x$ richtig einzuführen.
Setzt man Vertrautheit des Lesers mit den Voraussetzungen der
Angabenstellung voraus, kann man den Beweis etwas kürzer auch so formulieren:
\begin{indentblock}
Seien~$x, \tilde x \in X$ beliebig mit~$(g \circ f)(x) = (g \circ f)(\tilde
x)$. Dann folgt:
\[ g(f(x)) = g(f(\tilde x)) \ \stackrel{\text{$g$ inj.}}{\Longrightarrow}\ 
   f(x) = f(\tilde x) \ \stackrel{\text{$f$ inj.}}{\Longrightarrow}\ 
   x = \tilde x, \]
das war zu zeigen.
\end{indentblock}

\newpage

\subsection*{Direkter Beweis}

\begin{itemize}
\item Um \hil{$A \wedge B$} direkt zu zeigen, muss man sowohl~$A$ als auch~$B$ zeigen.

\emph{Vorlage:} Da \ldots, gilt $A$. Da außerdem \ldots, gilt auch~$B$.

\item Um \hil{$A \vee B$} direkt zu zeigen, muss man zeigen, dass~$A$ oder~$B$ (oder beide
-- das sagt man selten dazu) gelten. Meistens muss man dazu eine
Fallunterscheidung führen.

\begin{tabbing}
  \emph{Vorlage:} \= \kill
  \emph{Vorlage:} \> Wegen $\ldots$ können nur folgende Fälle eintreten: \\
  \> \emph{Fall 1}: Wegen~$\ldots$ gilt dann~$A$. \\
  \> \emph{Fall 2}: Wegen~$\ldots$ gilt dann~$B$.
\end{tabbing}

\item Um~\hil{$\neg A$} direkt zu zeigen, zeigt man, dass die Annahme, dass~$A$ doch stimmt,
zu einem Widerspruch führt.

\emph{Vorlage:} Angenommen, es gilt doch~$A$. Dann \ldots, das kann nicht sein.

\item Um \hil{$A \Rightarrow B$} direkt zu zeigen, setzt man die Gültigkeit von~$A$ voraus
und zeigt dann~$B$. Ob~$A$ tatsächlich stimmt, ist für die Argumentation nicht
relevant, es geht nur um die hypothetische Schlussfolgerung.

\emph{Vorlage:} Gelte~$A$. Dann \ldots, daher gilt~$B$.

\item Um \hil{$A \Leftrightarrow B$} direkt zu zeigen, zeigt man~$A \Rightarrow B$
(die sog. Hinrichtung) und~$B \Rightarrow A$ (die sog. Rückrichtung). Ob
dabei~$A$ und~$B$ tatsächlich stimmen, ist nicht relevant.

\begin{tabbing}
  \emph{Vorlage:} \= \kill
  \emph{Vorlage:} \> "`$\Rightarrow$"': Gelte~$A$. Dann \ldots, daher gilt~$B$. \\
  \> "`$\Leftarrow$"': Gelte~$B$. Dann \ldots, daher gilt~$A$.
\end{tabbing}

Manchmal sind die Beweise der beiden Richtungen so ähnlich, dass man sie zu
einem zusammenfassen kann:

\begin{tabbing}
  \emph{Vorlage:} \= \kill
  \emph{Vorlage:} \> $A$ gilt genau dann, wenn \ldots; das ist genau dann der
Fall, wenn \ldots; \ldots; \\
  \> das ist genau dann der Fall, wenn~$B$ gilt.
\end{tabbing}

Oder kürzer notiert:

\emph{Vorlage:} $A \Leftrightarrow \ldots \Leftrightarrow \ldots
\Leftrightarrow \ldots \Leftrightarrow \ldots \Leftrightarrow B$

\item Um \hil{$\forall x\in X{:}\, A(x)$} direkt zu zeigen, zeigt man, dass für
jedes~$x \in X$ jeweils die Aussage~$A(x)$ gilt.

\emph{Vorlage:} Sei~$x \in X$ beliebig. Da \ldots, gilt~$A(x)$.

Der bei den Auslassungspunkten auszuführende Beweis darf dabei von~$x$ nur
voraussetzen, dass es ein Element der Menge~$X$ ist: Der Beweis muss mit
jedem~$x \in X$ zurechtkommen.

\item Um \hil{$\exists x \in X{:}\, A(x)$} direkt zu zeigen, gibt man explizit ein~$x
\in X$ an, für das~$A(x)$ gilt.

\begin{tabbing}
  \emph{Vorlage:} \= \kill
  \emph{Vorlage:} \> Setze~$x := \ldots$. Dann liegt~$x$ in der Tat in~$X$,
denn \ldots; und wegen \ldots \\
  \> gilt~$A(x)$.
\end{tabbing}

\item Um~\hil{$X \subseteq Y$} direkt zu zeigen, wobei~$X$ und~$Y$ Mengen sind,
zeigt man, dass jedes Element von~$X$ auch in~$Y$ liegt.

\emph{Vorlage:} Sei~$x \in X$ beliebig. Dann \ldots, daher gilt~$x \in Y$.

\item
Um~\hil{$X = Y$} direkt zu zeigen, zeigt man~$X \subseteq Y$ und~$Y \subseteq
X$.

\begin{tabbing}
  \emph{Vorlage:} \= \kill
  \emph{Vorlage:} \> "`$\subseteq$"': Sei~$x \in X$ beliebig. Dann \ldots,
  daher gilt~$x \in Y$. \\
  \> "`$\supseteq$"': Sei~$y \in Y$ beliebig. Dann \ldots, daher gilt~$y \in
  X$.
\end{tabbing}

Manchmal kann man die beiden Teilbeweise auch zu einem kombinieren.

\item Um \hil{$f = g$} direkt zu zeigen, wobei~$f$ und~$g$ beides Abbildungen
$X \to Y$ sind (also dieselbe Definitions- und Zielmenge haben), zeigt
man, dass die beiden Funktionen an allen Stellen ihres Definitionsbereichs
übereinstimmen.

\emph{Vorlage:} Sei~$x \in X$ beliebig. Dann \ldots, daher gilt~$f(x) = g(x)$.
\end{itemize}


\subsection*{Beweis durch Widerspruch}

Um eine Aussage~$A$ zu zeigen, kann man auch zeigen, dass die Annahme von~$\neg
A$ zu einem Widerspruch führt.

\emph{Vorlage:} Angenommen,~$A$ wäre falsch. Dann \ldots, das kann nicht sein.


\subsection*{Beweis durch Kontraposition}

Um eine Implikation der Form~$A \Rightarrow B$ zu zeigen, kann man auch~$\neg B
\Rightarrow \neg A$ zeigen, d.\,h. unter der Voraussetzung von~$\neg B$ einen
Beweis von~$\neg A$ führen.
Das ist häufig dann hilfreich, wenn~$A$ und~$B$ selbst negierte Aussagen sind.

\emph{Beispiel:} Wenn man sich direkt an einem Beweis der Implikation
\begin{align*}\text{$k$ ist undorig} &\ \Longrightarrow\  \text{$k$ ist unfoberant} \\
\intertext{versucht,
wird man durch die vielen Verneinungen vielleicht verwirrt. Möglicherweise ist
es daher einfacher, die Kontraposition}
\text{$k$ ist foberant} &\ \Longrightarrow\  \text{$k$ ist dorig} \\
\intertext{zu zeigen.}\end{align*}

\end{document}
