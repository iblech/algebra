\documentclass{algblatt}
\loesungenfalse

\geometry{tmargin=2cm,bmargin=1.5cm,lmargin=2.6cm,rmargin=2.6cm}

%\setlength{\titleskip}{0.7em}
\setlength{\aufgabenskip}{1.0em}

\begin{document}

\vspace*{-1.5cm}
\maketitle{7}{Abgabe bis 3. Juni 2013, 17:00 Uhr}
\vspace*{-0.3cm}

\begin{aufgabe}{Größte gemeinsame Teiler und kleinste gemeinsame Vielfache}
\begin{enumerate}
\item Seien die Polynome~$f = X^3 + 2\,X^2 + 2\,X + 4$
und~$g = X^2 + 3\,X + 2$ gegeben. Finde Polynome~$p$ und~$q$ mit
$X+2 = p f + q g$.
\item Seien~$f$ und~$g$ zwei normierte Polynome mit rationalen
Koeffizienten. Zeige, dass \emph{genau ein normiertes} Polynom existiert, welches ein
größter gemeinsamer Teiler von~$f$ und~$g$ ist.
\item Seien~$f$ und~$g$ wie in~b).
Gib ein Verfahren zur Berechnung des größten gemeinsamen Teilers
von~$f$ und~$g$ über die Zerlegung von~$f$ und~$g$ in ihre
irreduziblen Faktoren an.
\item Seien~$f$ und~$g$ wie in~b) und~c). Definiere, was man unter dem
\emph{kleinsten gemeinsamen Vielfachen} von~$f$ und~$g$ verstehen sollte,
und gib eine Konstruktionsvorschrift für es an.
\end{enumerate}
\end{aufgabe}

\begin{aufgabe}{Separable Polynome}
\begin{enumerate}
\item Finde eine Polynomgleichung mit rationalen
Koeffizienten, die dieselben Lösungen wie die
Gleichung $X^7-X^6+4\,X^4-4\,X^3+4\,X-4=0$ besitzt, jedoch alle mit
Vielfachheit~1.
\item Konstruiere eine Polynomgleichung, die genau dann
von einer algebraischen Zahl~$a$ erfüllt wird, wenn das Polynom~$f_a(X) := X^3
+ 2a^2 X - a + 6$ nicht separabel ist.
\item Zeige, dass ein normiertes Polynom~$f$ mit rationalen Koeffizienten genau
dann separabel ist, 
wenn der größte gemeinsame Teiler von~$f$
und~$f'$ das konstante Polynom~$1$ ist.
\end{enumerate}

\begin{loesungE}
\item ...
\item Das Polynom~$f_a$ ist genau dann nicht separabel, wenn seine Diskriminante null
ist:
\[ \Delta_{f_a} = -4p^3 - 27q^2 = \cdots =
  -32 \cdot a^6 - 27 a^2 + 324 a - 972 \stackrel{!}{=} 0. \]
Damit haben wir die geforderte Polynomgleichung gefunden.
\item ...
\end{loesungE}
\end{aufgabe}

\begin{aufgabe}{Irreduzible Polynome}
\begin{enumerate}
\item Sind normierte Polynome vom Grad~1 stets irreduzibel über den rationalen
Zahlen?
\item Zeige, dass normierte Polynome vom Grad~2 oder~3 über den rationalen
Zahlen genau dann reduzibel sind, wenn sie mindestens eine rationale Nullstelle
besitzen.
\item Finde ein Polynom mit rationalen Koeffizienten, das keine rationale
Nullstelle besitzt und trotzdem über den rationalen Zahlen reduzibel ist.
\item Zeige, dass das Polynom $X^3 - \frac{3}{2}X^2 + X - \frac{6}{5}$ über den
rationalen Zahlen irreduzibel ist.
\end{enumerate}

\begin{loesungE}
\item Sei~$f(X)$ ein normiertes Polynom vom Grad~2 oder~3 mit rationalen
Koeffizienten. Die Rückrichtung ist klar: Wenn~$f$ eine rationale
Nullstelle~$x$ besitzt, geht die Division von~$f$ durch den Linearfaktor~$X-x$
auf -- also ist~$f$ zerlegbar.

Sei für den Beweis der Hinrichtung eine Zerlegung~$f = g \cdot h$ gegeben. Nach
der Gradvoraussetzung an~$f$ hat dann~$g$ oder~$h$ Grad~1 und ist daher von der
Form~$X-x$ für eine gewisse rationale Zahl~$x$. Also besitzt~$f$ eine rationale
Nullstelle, nämlich~$x$.

\item Das Polynom~$(X^2+1)^2$ ist eines von unzähligen Beispielen.
\end{loesungE}
\end{aufgabe}

\begin{aufgabe}{Prime Polynome}
\begin{enumerate}
\item Ein normiertes Polynom $f$ mit rationalen Koeffizienten heißt
genau dann \emph{prim}, wenn es nicht das Einspolynom ist und folgende
Eigenschaft hat: Immer, wenn~$f$ ein
Produkt~$g \cdot h$ zweier Polynome mit
rationalen Koeffizienten teilt, so teilt~$f$ schon mindestens einen der
beiden Faktoren. Zeige, dass jedes prime Polynom irreduzibel ist.
\item Teile ein über den rationalen Zahlen irreduzibles Polynom~$f$ ein
Produkt~$g_1 \cdots g_n$ von Polynomen mit rationalen Koeffizienten.
Zeige, dass~$f$ dann schon eines der~$g_i$ teilt.
\end{enumerate}

\begin{loesungE}
\item Sei~$f(X)$ ein primes Polynom. Da~$f$ nicht das Einspolynom ist, hat es
mindestens Grad~1 (wieso?). Es bleibt also nur zu zeigen, dass~$f(X) = f(X)$
die \emph{einzige} Zerlegung von~$f$ ist. Sei dazu~$f = g \cdot h$ mit
normierten nichtkonstanten Polynomen~$g(X), h(X)$ mit rationalen Koeffizienten. Dann folgt
insbesondere~$f \mid gh$, also nach Voraussetzung~$f \mid g$ oder~$f \mid h$.
\end{loesungE}
\end{aufgabe}

\begin{aufgabe}{Euklidischer Algorithmus für ganze Zahlen}
Seien~$a$ und~$b$ ganze Zahlen. Zeige, dass es eine ganze Zahl~$d
\geq 0$ gibt, welche ein gemeinsamer Teiler von~$a$ und~$b$ ist, und für die es
weitere ganze Zahlen~$r$ und~$s$ mit~$d = r \cdot a + s \cdot b$ gibt.
\end{aufgabe}
 
\end{document} 

Herausgefallen:
* Algebraischen Zahlen sind ein Integritätsbereich.
* Bewertung wie Logarithmus
