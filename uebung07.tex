\documentclass{algblatt}
\usepackage{xstring}
\IfSubStr{\jobname}{\detokenize{loesung}}{\loesungentrue}{\loesungenfalse}

\geometry{tmargin=2cm,bmargin=2cm,lmargin=2.9cm,rmargin=2.9cm}

%\setlength{\titleskip}{0.7em}
\setlength{\aufgabenskip}{1.4em}

\begin{document}

\vspace*{-1.5cm}
\maketitle{7}{Abgabe bis 3. Juni 2013, 17:00 Uhr}

\begin{aufgabe}{Größte gemeinsame Teiler und kleinste gemeinsame Vielfache}
\begin{enumerate}
\item Seien die Polynome~$f = X^3 - 2\,X^2 + 2\,X - 4$
und~$g = X^2 - 3\,X + 2$ gegeben. Finde Polynome~$p$ und~$q$ mit
$X-2 = p f + q g$.
\item Seien~$f(X)$ und~$g(X)$ zwei normierte Polynome mit rationalen
Koeffizienten. Gib ein Verfahren zur Berechnung des größten gemeinsamen Teilers
von~$f(X)$ und~$g(X)$ über die Zerlegung von~$f(X)$ und~$g(X)$ in ihre
irreduziblen Faktoren an.
\item Seien~$f(X)$ und~$g(X)$ wie in~b). Definiere, was man unter dem
\emph{kleinsten gemeinsamen Vielfachen} von~$f(X)$ und~$g(X)$ verstehen sollte,
und gib eine Konstruktionsvorschrift für es an.
\end{enumerate}
\end{aufgabe}

\begin{aufgabe}{Separabilität}
\begin{enumerate}
\item Zeige, dass ein normiertes Polynom~$f$ mit rationalen Koeffizienten genau
dann separabel ist, 
wenn der größte gemeinsame Teiler von~$f$
und~$f'$ das konstante Polynom~$1$ ist.
\item Finde eine normierte Polynomgleichung mit rationalen
Koeffizienten, die dieselben Lösungen wie die
Gleichung $X^7+X^6-4\,X^4-4\,X^3+4\,X+4=0$ besitzt, jedoch alle mit
Vielfachheit~1.
\item Konstruiere eine Polynomgleichung mit rationalen Koeffizienten, die genau dann
von einer algebraischen Zahl~$a$ erfüllt wird, wenn das Polynom~$f_a(X) := X^3
+ 2a^2 X - a + 5$ nicht separabel ist.
\end{enumerate}
\end{aufgabe}

\begin{aufgabe}{Irreduzible Polynome}
\begin{enumerate}
\item Zeige, dass normierte Polynome vom Grad~2 oder~3 über den rationalen
Zahlen genau dann irreduzibel sind, wenn sie keine rationale Nullstelle
besitzen.
\item Finde ein Polynom mit rationalen Koeffizienten, das keine rationale
Nullstelle besitzt und trotzdem über den rationalen Zahlen reduzibel ist.
\end{enumerate}
\end{aufgabe}

\begin{aufgabe}{Prime Polynome}
Ein normiertes Polynom~$f(X)$ mit rationalen Koeffizienten heißt genau
dann \emph{prim}, wenn es nicht das Einspolynom ist und folgende Eigenschaft
hat: Immer, wenn~$f(X)$ ein Produkt~$g(X) \cdot h(X)$ zweier Polynome mit
rationalen Koeffizienten teilt, so teilt~$f(X)$ schon mindestens einen der
beiden Faktoren.
\begin{enumerate}
\item Zeige, dass jedes prime Polynom irreduzibel ist.
\item Zeige umgekehrt, dass irreduzible Polynome prim sind.
...nicht getreu...
\end{enumerate}
\end{aufgabe}

\begin{aufgabe}{Euklidischer Algorithmus für ganze Zahlen}
Beweise folgenden Satz, etwa durch Imitation des Vorlesungsbeweises für
Polynome: Seien~$a$ und~$b$ ganze Zahlen. Dann existiert eine ganze Zahl~$d
\geq 0$, welche ein gemeinsamer Teiler von~$a$ und~$b$ ist, und für die es
weitere ganze Zahlen~$r$ und~$s$ mit~$d = r \cdot a + s \cdot b$ gibt.
\end{aufgabe}
 
\end{document} 

Herausgefallen:
* Algebraischen Zahlen sind ein Integritätsbereich.
* Polynome vom Grad 1 stets irreduzibel?
* Bewertung wie Logarithmus
* Beweise, dass es ggT's von Polynomen gibt
