\documentclass{algblatt}
\loesungenfalse

\geometry{tmargin=2cm,bmargin=1.5cm,lmargin=2.6cm,rmargin=2.6cm}

%\setlength{\titleskip}{0.7em}
\setlength{\aufgabenskip}{1.0em}

\begin{document}

\vspace*{-1.5cm}
\maketitle{7}{Abgabe bis 3. Juni 2013, 17:00 Uhr}
\vspace*{-0.3cm}

\begin{aufgabe}{Größte gemeinsame Teiler und kleinste gemeinsame Vielfache}
\begin{enumerate}
\item Seien die Polynome~$f = X^3 + 2\,X^2 + 2\,X + 4$
und~$g = X^2 + 3\,X + 2$ gegeben. Finde Polynome~$p$ und~$q$ mit
$X+2 = p f + q g$.
\item Seien~$f$ und~$g$ zwei normierte Polynome mit rationalen
Koeffizienten. Zeige, dass \emph{genau ein normiertes} Polynom existiert, welches ein
größter gemeinsamer Teiler von~$f$ und~$g$ ist.
\item Seien~$f$ und~$g$ wie in~b).
Gib ein Verfahren zur Berechnung des größten gemeinsamen Teilers
von~$f$ und~$g$ über die Zerlegung von~$f$ und~$g$ in ihre
irreduziblen Faktoren an.
\item Seien~$f$ und~$g$ wie in~b) und~c). Definiere, was man unter einem
\emph{kleinsten gemeinsamen Vielfachen} von~$f$ und~$g$ verstehen sollte,
und gib eine Konstruktionsvorschrift für es an.
\end{enumerate}

\begin{loesungE}
\item

\item \emph{Existenz (kurz auch im Skript abgehandelt):} Dank des euklidischen
Algorithmus gibt es ein normiertes Polynom~$d$, welches ein gemeinsamer Teiler
von~$f$ und~$g$ ist und für gewisse weitere Polynome~$p$ und~$q$ die Beziehung
\[ d = p \cdot f + q \cdot g \]
erfüllt. Aus dieser folgt, dass~$d$ sogar ein größter gemeinsamer Teiler ist,
d.\,h. ein Vielfaches jedes anderen gemeinsamen Teilers ist: Ist~$e$ ein
beliebiger gemeinsamer Teiler von~$f$ und~$g$, so ist er auch ein Teiler
von~$pf$ und~$qg$ und somit auch ein Teiler der rechten Seite der Gleichung,
also von~$d$.

\emph{Eindeutigkeit:} Seien~$d$ und~$\widetilde d$ beides normierte
größte gemeinsame Teiler von~$f$ und~$g$. Da~$d$ ein größer gemeinsamer Teiler
ist, folgt~$\widetilde d \mid d$. Umgekehrt folgt~$d \mid \widetilde d$,
da~$\widetilde d$ ein größter gemeinsamer Teiler ist. Da~$d$ und~$\widetilde d$
beide normiert sind, folgt~$d = \widetilde d$.

\item Wir können~$f$ und~$g$ in ihre irreduziblen Faktoren~$p_i$ zerlegen:
\begin{align*}
  f &= \prod_i p_i^{a_i} \\
  g &= \prod_i p_i^{b_i}
\end{align*}
Dabei dürfen die Exponenten auch null sein -- damit tragen wir, ohne die
Notation verkomplizieren zu müssen, dem Umstand Rechnung, dass manche Faktoren
vielleicht nur in einem der beiden Polynome vorkommen. Als Vorschlag für den
größten gemeinsamen Teiler definieren wir
\[ d = \prod_i p_i^{c_i} \]
mit~$c_i := \min\{ a_i, b_i \}$ für alle Indizes~$i$. Klar ist zumindest, dass
dieses Polynom ein gemeinsamer Teiler von~$f$ und~$g$ ist. Zum Nachweis,
dass~$d$ wirklich größter gemeinsamer Teiler ist, sei ein beliebiger
gemeinsamer Teiler~$\widetilde d$ von~$f$ und~$g$ gegeben. Dann folgt (siehe
unten), dass in~$\widetilde d$ nur die Faktoren~$p_i$ (und keine anderen)
vorkommen, und dass diese höchstens mit Vielfachheit~$a_i$ (wegen~$\widetilde d
\mid f$) und zugleich höchstens mit Vielfachheit~$b_i$ (wegen~$\widetilde d
\mid g$) vorkommen. Unter'm Strich kommen sie also höchstens mit
Vielfachheit~$c_i$ vor, also gilt~$\widetilde d \mid $.

Implizit haben wir dabei folgendes allgemeines Lemma benutzt: Sei~$p$ ein irreduzibles
Polynom und gelte~$f \mid g$. Dann ist die Vielfachheit von~$p$ in~$f$
höchstens gleich der Vielfachheit von~$p$ in~$g$.

\item Folgende Definition ist sinnvoll: Ein Polynom~$k$ heißt genau dann
\emph{kleinstes gemeinsames Vielfaches} von~$f$ und~$g$, wenn~$k$ ein
gemeinsames Vielfaches von~$f$ und~$g$ ist und für jedes gemeinsame
Vielfache~$\widetilde k$ von~$f$ und~$g$ gilt:~$k \mid \widetilde k$.
\end{loesungE}
\end{aufgabe}

\begin{aufgabe}{Separable Polynome}
\begin{enumerate}
\item Finde eine Polynomgleichung mit rationalen
Koeffizienten, die dieselben Lösungen wie die
Gleichung $X^7-X^6+4\,X^4-4\,X^3+4\,X-4=0$ besitzt, jedoch alle mit
Vielfachheit~1.
\item Konstruiere eine Polynomgleichung, die genau dann
von einer algebraischen Zahl~$a$ erfüllt wird, wenn das Polynom~$f_a(X) := X^3
+ 2a^2 X - a + 6$ nicht separabel ist.
\item Zeige, dass ein normiertes Polynom~$f$ mit rationalen Koeffizienten genau
dann separabel ist, 
wenn der größte gemeinsame Teiler von~$f$
und~$f'$ das konstante Polynom~$1$ ist.
\end{enumerate}

\begin{loesungE}
\item Wir befolgen das Verfahren des Skripts und berechnen daher zunächst die
normierte Ableitung des Polynoms~$f = X^7-X^6+4\,X^4-4\,X^3+4\,X-4$:
\[ \frac{1}{7} f'(X) = X^6 - \frac{6}{7} X^5 + \frac{16}{7} X^3 - \frac{12}{7}
X^2 + \frac{4}{7}. \]
Der eindeutig bestimmte normierte größte gemeinsame Teiler von~$f$ und~$f'/7$
ist das Polynom~$d(X) = X^3 + 2$, wie eine Nebenrechnung mit dem euklidischen
Algorithmus zeigt, und es gilt~$f(X) = d(X) \cdot \widetilde f(X)$
mit~$\widetilde f(X) = X^4 - X^3 + 2\,X - 2$. Die gesuchte Gleichung ist also
\[ X^4 - X^3 + 2\,X - 2 = 0. \]

\emph{Variante:} Vielleicht schafft man es auch irgendwie, das gegebene
Polynom~$f$ zu faktorisieren: Es gilt~$f(X) = (X-1) \cdot (X^3 + 2)^2$. Dann
sieht man sofort, dass
\[ (X-1) \cdot (X^3 + 2) = 0 \]
die gesuchte Gleichung ist. Diese stimmt mit obiger überein.

\item Das Polynom~$f_a$ ist genau dann nicht separabel, wenn seine Diskriminante null
ist:
\[ \Delta_{f_a} = -4p^3 - 27q^2 = \cdots =
  -32 \cdot a^6 - 27 a^2 + 324 a - 972 \stackrel{!}{=} 0. \]
Damit haben wir die geforderte Polynomgleichung gefunden.

\item Sei~$d(X)$ der normierte größte gemeinsame Teiler
von~$f$ und~$f'$. Dann sind die Nullstellen von~$d(X)$ (in den algebraischen
Zahlen) genau die mehrfachen Nullstellen von~$f(X)$, d.\,h. diejenigen mit
Vielfachheit~$\geq 2$.

"`$\Longrightarrow$"': Da~$f(X)$ nach Voraussetzung keine mehrfachen
Nullstellen besitzt, besitzt~$d(X)$ also keine einzige Nullstelle. Da es
normiert ist, muss es daher gleich dem Einspolynom sein.

"`$\Longleftarrow$"': Da~$d(X)$ nach Voraussetzung keine Nullstellen besitzt,
besitzt~$f(X)$ keinerlei mehrfache Nullstellen.
\end{loesungE}
\end{aufgabe}

\begin{aufgabe}{Irreduzible Polynome}
\begin{enumerate}
\item Sind normierte Polynome vom Grad~1 stets irreduzibel über den rationalen
Zahlen?
\item Zeige, dass normierte Polynome vom Grad~2 oder~3 über den rationalen
Zahlen genau dann reduzibel sind, wenn sie mindestens eine rationale Nullstelle
besitzen.
\item Finde ein Polynom mit rationalen Koeffizienten, das keine rationale
Nullstelle besitzt und trotzdem über den rationalen Zahlen reduzibel ist.
\item Zeige, dass das Polynom $X^3 - \frac{3}{2}X^2 + X - \frac{6}{5}$ über den
rationalen Zahlen irreduzibel ist.
\end{enumerate}

\begin{loesungE}
\item Ja!

\item Sei~$f(X)$ ein normiertes Polynom vom Grad~2 oder~3 mit rationalen
Koeffizienten. Die Rückrichtung ist klar: Wenn~$f$ eine rationale
Nullstelle~$x$ besitzt, geht die Division von~$f$ durch den Linearfaktor~$X-x$
auf -- also ist~$f$ zerlegbar.

Sei für den Beweis der Hinrichtung eine Zerlegung~$f = g \cdot h$ gegeben. Nach
der Gradvoraussetzung an~$f$ hat dann~$g$ oder~$h$ Grad~1 und ist daher von der
Form~$X-x$ für eine gewisse rationale Zahl~$x$. Also besitzt~$f$ eine rationale
Nullstelle, nämlich~$x$.

\item Das Polynom~$(X^2+1)^2$ ist eines von unzähligen Beispielen.

\item Nach Teilaufgabe~b) genügt es, nachzuweisen, dass das gegebene Polynom
keine rationalen Nullstellen besitzt. Äquivalent ist zu zeigen, dass das
mit~$10$ durchmultiplizierte Polynom
\[ 10 \, X^3 - 15 \, X^2 + 10 \, X - 12 \]
keine rationalen Nullstellen besitzt. In vollständig gekürzter Darstellung
müssen Zähler und Nenner solcher Nullstellen Teiler von~$-12$ bzw.~$10$ sein.
Also kommen nur die Möglichkeiten
\begin{align*}
  \text{Zähler} &\in \{ \pm 1, \pm 2, \pm 3, \pm 4, \pm 6, \pm 12 \}, \\
  \text{Nenner} &\in \{ \pm 1, \pm 2, \pm 5, \pm 10 \},
\end{align*}
in Betracht. Probiert man diese Möglichkeiten alle durch (ein paar fallen
wegen fehlender Teilerfremdheit wieder weg), sieht man: Das Polynom hat keine
rationalen Nullstellen.
\end{loesungE}
\end{aufgabe}

\begin{aufgabe}{Prime Polynome}
\begin{enumerate}
\item Ein normiertes Polynom $f$ mit rationalen Koeffizienten heißt
genau dann \emph{prim}, wenn es nicht das Einspolynom ist und folgende
Eigenschaft hat: Immer, wenn~$f$ ein
Produkt~$g \cdot h$ zweier Polynome mit
rationalen Koeffizienten teilt, so teilt~$f$ schon mindestens einen der
beiden Faktoren. Zeige, dass jedes prime Polynom irreduzibel ist.
\item Teile ein über den rationalen Zahlen irreduzibles Polynom~$f$ ein
Produkt~$g_1 \cdots g_n$ von Polynomen mit rationalen Koeffizienten.
Zeige, dass~$f$ dann schon eines der~$g_i$ teilt.
\end{enumerate}

\begin{loesungE}
\item Sei~$f(X)$ ein primes Polynom. Da~$f$ nicht das Einspolynom ist, hat es
mindestens Grad~1 (wieso?). Es bleibt also nur zu zeigen, dass~$f(X) = f(X)$
die \emph{einzige} Zerlegung von~$f$ ist. Sei dazu~$f = g \cdot h$ mit
normierten nichtkonstanten Polynomen~$g(X), h(X)$ mit rationalen Koeffizienten
eine hypothetische Zerlegung.
Dann folgt insbesondere~$f \mid gh$, also wegen der Primalität~$f \mid g$ oder~$f
\mid h$.

Im ersten Fall folgt~$g = f \cdot \widetilde g$ für ein
Polynom~$\widetilde g$. Dieses Polynom muss normiert sein, da~$g$ und~$h$ es
sind. Eingesetzt ergibt sich~$f = f \widetilde g h$; ein Gradvergleich zeigt,
dass dann~$h$ doch konstant ist -- das ist ein Widerspruch. Analog verfährt man
im zweiten Fall.

\item Wir führen einen Induktionsbeweis. Der Induktionsanfang~$n = 0$ ist
witzig: Für ihn müssen wir zeigen, dass aus der Voraussetzung~$f \mid 1$
(leeres Produkt) eine unmögliche Aussage folgt (nämlich, dass es ein~$i \in
\emptyset$ gibt). Da die Voraussetzung~$f \mid 1$ nie erfüllt ist (irreduzible
Polynome haben mindestens Grad~1), ist diese Implikation trivialerweise
erfüllt.

Wenn man mag, kann man die Induktion auch erst bei~$n = 1$ beginnen: Dann ist
der Induktionsanfang klar und bereit weniger Kopfschmerzen.

Für den Beweis des Induktionsschritts~$n \to n + 1$ gelte~$f \mid g_1 \cdots
g_{n+1} = (g_1 \cdots g_n) \cdot g_{n+1}$. Nach Folgerung~3.11 folgt dann~$f
\mid g_1 \cdots g_n$ oder~$f \mid g_{n+1}$. Im zweiten Fall sind wir sofort
fertig, im ersten Fall nach Anwendung der Induktionsvoraussetzung.
\end{loesungE}
\end{aufgabe}

\begin{aufgabe}{Euklidischer Algorithmus für ganze Zahlen}
Seien~$a$ und~$b$ ganze Zahlen. Zeige, dass es eine ganze Zahl~$d
\geq 0$ gibt, welche ein gemeinsamer Teiler von~$a$ und~$b$ ist, und für die es
weitere ganze Zahlen~$r$ und~$s$ mit~$d = r \cdot a + s \cdot b$ gibt.
\end{aufgabe}
 
\end{document} 

Herausgefallen:
* Algebraischen Zahlen sind ein Integritätsbereich.
* Bewertung wie Logarithmus
