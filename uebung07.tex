\documentclass{algblatt}
\usepackage{xstring}
\IfSubStr{\jobname}{\detokenize{loesung}}{\loesungentrue}{\loesungenfalse}

\ifloesungen
  \geometry{tmargin=2cm,bmargin=1.9cm,lmargin=2.32cm,rmargin=2.32cm}
\else
  \geometry{tmargin=2cm,bmargin=1.5cm,lmargin=2.6cm,rmargin=2.6cm}
\fi

%\setlength{\titleskip}{0.7em}
\setlength{\aufgabenskip}{1.0em}

\begin{document}

\vspace*{-1.5cm}
\maketitle{7}{Abgabe bis 3. Juni 2013, 17:00 Uhr}
\vspace*{-0.3cm}

\begin{aufgabe}{Größte gemeinsame Teiler und kleinste gemeinsame Vielfache}
\begin{enumerate}
\item Seien die Polynome~$f = X^3 + 2\,X^2 + 2\,X + 4$
und~$g = X^2 + 3\,X + 2$ gegeben. Finde Polynome~$p$ und~$q$ mit
$X+2 = p f + q g$.
\item Seien~$f$ und~$g$ zwei normierte Polynome mit rationalen
Koeffizienten. Zeige, dass \emph{genau ein normiertes} Polynom existiert, welches ein
größter gemeinsamer Teiler von~$f$ und~$g$ ist.
\item Seien~$f$ und~$g$ wie in~b).
Gib ein Verfahren zur Berechnung des größten gemeinsamen Teilers
von~$f$ und~$g$ über die Zerlegung von~$f$ und~$g$ in ihre
irreduziblen Faktoren an.
\item Seien~$f$ und~$g$ wie in~b) und~c). Definiere, was man unter einem
\emph{kleinsten gemeinsamen Vielfachen} von~$f$ und~$g$ verstehen sollte,
und gib eine Konstruktionsvorschrift für es an.
\end{enumerate}

\begin{loesungE}
\item Wir verwenden natürlich den euklidischen Algorithmus:
\begin{align*}
  f &= (X-1) \cdot g + (3 \, X + 6) \\
  g &= \frac{1}{3} (X+1) \cdot (3\,X+6) + 0
\end{align*}
Dann noch rückwärts auflösen: $3\,X+6 = f - (X-1) \cdot g$, also $X+2 =
\frac{1}{3} f + \frac{1}{3} (-X+1) g$.

\emph{Variante:} Man kann die Neu-Normierung auch schon in jedem Schritt des
Algorithmus vornehmen (diese Variante verwendet der Beweis im Skript):
\begin{align*}
  f &= (X-1) \cdot g + 3 \, (X + 2) \\
  g &= (X+1) \cdot (X+2) + 0
\end{align*}
Dann noch rückwärts auflösen: $X + 2 = \frac{1}{3} f + \frac{1}{3} (-X+1) g$.

\item \emph{Existenz (kurz auch im Skript abgehandelt):} Dank des euklidischen
Algorithmus gibt es ein normiertes Polynom~$d$, welches ein gemeinsamer Teiler
von~$f$ und~$g$ ist und für gewisse weitere Polynome~$p$ und~$q$ die Beziehung
\[ d = p \cdot f + q \cdot g \]
erfüllt. Aus dieser folgt, dass~$d$ sogar ein größter gemeinsamer Teiler ist,
d.\,h. ein Vielfaches jedes anderen gemeinsamen Teilers ist: Ist~$e$ ein
beliebiger gemeinsamer Teiler von~$f$ und~$g$, so ist er auch ein Teiler
von~$pf$ und~$qg$ und somit auch ein Teiler der rechten Seite der Gleichung,
also von~$d$.

\emph{Eindeutigkeit:} Seien~$d$ und~$\widetilde d$ beides normierte
größte gemeinsame Teiler von~$f$ und~$g$. Da~$d$ ein größer gemeinsamer Teiler
ist, folgt~$\widetilde d \mid d$. Umgekehrt folgt~$d \mid \widetilde d$,
da~$\widetilde d$ ein größter gemeinsamer Teiler ist. Da~$d$ und~$\widetilde d$
beide normiert sind, folgt~$d = \widetilde d$ (wieso?).

\item Wir können~$f$ und~$g$ in ihre irreduziblen Faktoren~$p_i$ zerlegen:
\begin{align*}
  f &= \prod_i p_i^{a_i} \\
  g &= \prod_i p_i^{b_i}
\end{align*}
Dabei dürfen die Exponenten auch null sein -- damit tragen wir, ohne die
Notation verkomplizieren zu müssen, dem Umstand Rechnung, dass manche Faktoren
vielleicht nur in einem der beiden Polynome vorkommen. Als Vorschlag für den
größten gemeinsamen Teiler definieren wir
\[ d := \prod_i p_i^{c_i} \]
mit~$c_i := \min\{ a_i, b_i \}$ für alle Indizes~$i$. Klar ist zumindest, dass
dieses Polynom ein gemeinsamer Teiler von~$f$ und~$g$ ist. Zum Nachweis,
dass~$d$ wirklich größter gemeinsamer Teiler ist, sei ein beliebiger
gemeinsamer Teiler~$\widetilde d$ von~$f$ und~$g$ gegeben. Dann folgt (siehe
unten), dass in~$\widetilde d$ nur die Faktoren~$p_i$ (und keine anderen)
vorkommen, und dass diese höchstens mit Vielfachheit~$a_i$ (wegen~$\widetilde d
\mid f$) und zugleich höchstens mit Vielfachheit~$b_i$ (wegen~$\widetilde d
\mid g$) vorkommen. Unter'm Strich kommen sie also höchstens mit
Vielfachheit~$c_i$ vor, also gilt~$\widetilde d \mid d$.

Implizit haben wir dabei folgendes allgemeines Lemma verwendet: Sei~$p$ ein irreduzibles
Polynom und gelte~$f \mid g$ (also~$g = \widetilde g \cdot f$ für ein
Polynom~$\widetilde g$). Dann ist die Vielfachheit von~$p$ in~$f$
höchstens gleich der Vielfachheit von~$p$ in~$g$. Diese Beobachtung folgt
sofort aus der Eindeutigkeit der Zerlegung in irreduzible Faktoren (wieso?).

\item Folgende Definition ist sinnvoll: Ein Polynom~$k$ heißt genau dann
\emph{kleinstes gemeinsames Vielfaches} von~$f$ und~$g$, wenn~$k$ ein
gemeinsames Vielfaches von~$f$ und~$g$ ist und für jedes gemeinsame
Vielfache~$\widetilde k$ von~$f$ und~$g$ gilt:~$k \mid \widetilde k$.

Konstruieren können wir es zum Beispiel durch die Setzung
\[ k := \prod_i p_i^{e_i}, \]
wobei wir die Bezeichnungen der Lösung von Teilaufgabe~c) weiterverwenden,
jetzt aber~$e_i := \max\{ a_i, b_i \}$ setzen. Der Beweis, dass dieses Polynom
tatsächlich ein kleinstes gemeinsames Vielfaches von~$f$ und~$g$ ist, verläuft
völlig analog.

\emph{Konstruktionsvariante:} Man kann auch~$k := f \cdot g / d$ setzen. Dabei geht
die Polynomdivision auf, denn da~$d$ größter gemeinsamer Teiler ist, gibt es
Polynome~$\widetilde f, \widetilde g, p, q$ mit
\begin{align*}
  f &= \widetilde f \cdot d, \\
  g &= \widetilde g \cdot d, \\
  d &= p f + q g,
\end{align*}
weshalb die Darstellungen~$k = \widetilde f \cdot g = f \cdot \widetilde g$
folgen. Über diese ist auch klar, dass~$k$ tatsächlich ein gemeinsames
Vielfaches von~$f$ und~$g$ ist. Sei zum Nachweis der Kleinstheit ein weiteres
gemeinsames Vielfaches~$\widehat k$ gegeben. Dann gibt es Polynome~$\widehat f,
\widehat g$ mit~$\widehat k = \widehat f \cdot f$ und~$\widehat k = \widehat g
\cdot g$. Mit der Bézoutdarstellung von~$d$ folgt
\[ \widehat k = \widehat f \cdot f =
  \widehat f \cdot \widetilde f d =
  \widehat f \widetilde f p f + \widehat f \widetilde f q g =
  \widehat g \widetilde g p f + \widehat f q \cdot k =
  (\widehat g p + \widehat f q) \cdot k, \]
also ist~$k$ in der Tat ein Teiler von~$\widehat k$.

\emph{Definitionsvariante:} Folgende Definition funktioniert ebenfalls, ist aber weniger
schön (da sie nicht nur über Teilbarkeit spricht): Ein Polynom~$k$ heißt genau
dann \emph{kleinstes gemeinsames Vielfaches} von~$f$ und~$g$, wenn~$k$ ein
gemeinsames Vielfaches von~$f$ und~$g$ ist und für jedes gemeinsame
Vielfache~$\widetilde k$ von~$f$ und~$g$ gilt:~$\deg k \leq \deg \widetilde k$.

\emph{Bemerkung für Leute, die wissen, was eine Kategorie ist:} Die eleganten
Definitionen von ggT und kgV sind Beispiele für Definitionen durch sog.
\emph{universelle Eigenschaften}. Tatsächlich kann man ggT und kgV als Produkt
bzw. Koprodukt in einer geeigneten Kategorie von Polynomen realisieren (in der
zwischen zwei Polynomen genau dann ein Morphismus verläuft, falls das erste das
zweite teilt).
\end{loesungE}
\end{aufgabe}

\begin{aufgabe}{Separable Polynome}
\begin{enumerate}
\item Finde eine Polynomgleichung mit rationalen
Koeffizienten, die dieselben Lösungen wie die
Gleichung $X^7-X^6+4\,X^4-4\,X^3+4\,X-4=0$ besitzt, jedoch alle mit
Vielfachheit~1.
\item Konstruiere eine Polynomgleichung, die genau dann
von einer algebraischen Zahl~$a$ erfüllt wird, wenn das Polynom~$f_a(X) := X^3
+ 2a^2 X - a + 6$ nicht separabel ist.
\item Zeige, dass ein normiertes Polynom~$f$ mit rationalen Koeffizienten genau
dann separabel ist, 
wenn der größte gemeinsame Teiler von~$f$
und~$f'$ das konstante Polynom~$1$ ist.
\end{enumerate}

\begin{loesungE}
\item Wir befolgen das Verfahren des Skripts und berechnen daher zunächst die
normierte Ableitung des Polynoms~$f = X^7-X^6+4\,X^4-4\,X^3+4\,X-4$:
\[ \frac{1}{7} f'(X) = X^6 - \frac{6}{7} X^5 + \frac{16}{7} X^3 - \frac{12}{7}
X^2 + \frac{4}{7}. \]
Der eindeutig bestimmte normierte größte gemeinsame Teiler von~$f$ und~$f'/7$
ist das Polynom~$d(X) = X^3 + 2$, wie eine Nebenrechnung mit dem euklidischen
Algorithmus zeigt, und es gilt~$f(X) = d(X) \cdot \widetilde f(X)$
mit~$\widetilde f(X) = X^4 - X^3 + 2\,X - 2$. Die gesuchte Gleichung ist also
\[ X^4 - X^3 + 2\,X - 2 = 0. \]

\emph{Variante:} Man kann auch über die Faktorisierung von~$f$ gehen:
Es gilt~$f(X) = (X-1) \cdot (X^3 + 2)^2$. Dann
sieht man sofort, dass
\[ (X-1) \cdot (X^3 + 2) = 0 \]
die gesuchte Gleichung ist, denn die beiden Faktoren haben jeweils nur einfache
Nullstellen und sie überlappen nicht mit denen des anderen Faktors (das ist
nach dem Abelschen Irreduzibilitätssatz auch allgemein klar). Diese Gleichung
stimmt mit obiger überein.

\item Das Polynom~$f_a$ ist genau dann nicht separabel, wenn seine Diskriminante null
ist:
\[ \Delta_{f_a} = -4p^3 - 27q^2 = \cdots =
  -32 a^6 - 27 a^2 + 324 a - 972 \stackrel{!}{=} 0. \]
Damit haben wir die geforderte Polynomgleichung gefunden.

\item Sei~$d(X)$ der normierte größte gemeinsame Teiler
von~$f$ und~$f'$. Dann sind die Nullstellen von~$d(X)$ (in den algebraischen
Zahlen) genau die mehrfachen Nullstellen von~$f(X)$, d.\,h. diejenigen mit
Vielfachheit~$\geq 2$. Dieser Umstand ist im Skript auf Seite~74 festgehalten:
Gilt~$f(X) = \prod_i (X-x_i)^{a_i}$, so ist~$\operatorname{ggT}(f, f') =
\prod_i (X-x_i)^{a_i - 1}$.

"`$\Longrightarrow$"': Da~$f(X)$ nach Voraussetzung keine mehrfachen
Nullstellen besitzt, besitzt~$d(X)$ also keine einzige Nullstelle. Da es
normiert ist, muss es daher gleich dem Einspolynom sein.

"`$\Longleftarrow$"': Da~$d(X)$ nach Voraussetzung keine Nullstellen besitzt,
besitzt~$f(X)$ keinerlei mehrfache Nullstellen.

\emph{Variante für die Rückrichtung:} Nach Vorlesung ist~$f_0 := f /
\operatorname{ggT}(f, f')$ ein separables Polynom. Da nach Voraussetzung der
Nenner das Einspolynom ist, ist also~$f_0 = f$ ein separables Polynom.
\end{loesungE}
\end{aufgabe}

\ifloesungen\newpage\fi
\begin{aufgabe}{Irreduzible Polynome}
\begin{enumerate}
\item Sind normierte Polynome vom Grad~1 stets irreduzibel über den rationalen
Zahlen?
\item Zeige, dass normierte Polynome vom Grad~2 oder~3 über den rationalen
Zahlen genau dann reduzibel sind, wenn sie mindestens eine rationale Nullstelle
besitzen.
\item Finde ein Polynom mit rationalen Koeffizienten, das keine rationale
Nullstelle besitzt und trotzdem über den rationalen Zahlen reduzibel ist.
\item Zeige, dass das Polynom $X^3 - \frac{5}{2}X^2 + \frac{4}{3}$ über den
rationalen Zahlen irreduzibel ist.
\end{enumerate}

\begin{loesungE}
\item Ja!

\emph{Bemerkung:} Über den ganzen Zahlen ist die Situation komplizierter, wenn
man die Forderung nach Normiertheit fallen lässt: Etwa
gilt das Polynom~$2\,X - 2 = 2 \cdot (X - 1)$ dort als reduzibel. Die
verfeinerte Regel lautet: Ein Polynom vom Grad~1 ist genau dann irreduzibel
über den ganzen Zahlen, wenn es primitiv ist.

\item Sei~$f(X)$ ein normiertes Polynom vom Grad~2 oder~3 mit rationalen
Koeffizienten. Die Rückrichtung ist klar: Wenn~$f$ eine rationale
Nullstelle~$x$ besitzt, geht die Division von~$f$ durch den Linearfaktor~$X-x$
auf -- also ist~$f$ zerlegbar.

Sei für den Beweis der Hinrichtung eine Zerlegung~$f = g \cdot h$ gegeben. Nach
der Gradvoraussetzung an~$f$ hat dann~$g$ oder~$h$ Grad~1 und ist daher von der
Form~$X-x$ für eine gewisse rationale Zahl~$x$. Also besitzt~$f$ eine rationale
Nullstelle, nämlich~$x$.

\emph{Bemerkung:} Im Skript auf Seite~76 wird eine \emph{Zerlegung} eines
normierten Polynoms~$f$ als ein Produkt der Form~$f = f_1 \cdots f_n$
definiert, wobei die~$f_i$ jeweils normiert und mindestens vom Grad~1 sein
sollen.

\item Das Polynom~$(X^2+1)^2$ ist eines von unzähligen Beispielen.

\item Nach Teilaufgabe~b) genügt es nachzuweisen, dass das gegebene Polynom
keine rationalen Nullstellen besitzt. Äquivalent ist zu zeigen, dass das
mit~$6$ durchmultiplizierte Polynom
\[ 6 \, X^3 - 15 \, X^2 + 8 \]
keine rationalen Nullstellen besitzt. In vollständig gekürzter Darstellung
müssen Zähler und Nenner solcher Nullstellen Teiler von~$8$ bzw.~$6$ sein.
Also kommen nur die Möglichkeiten
\begin{align*}
  \text{Zähler} &\in \{ \pm 1, \pm 2, \pm 4, \pm 8 \}, \\
  \text{Nenner} &\in \{ \pm 1, \pm 2, \pm 3, \pm 6 \},
\end{align*}
infrage. Probiert man diese Möglichkeiten alle durch (ein paar fallen
wegen fehlender Teilerfremdheit wieder weg), sieht man: Das Polynom hat keine
rationalen Nullstellen.

\emph{Variante:} Man kann auch das allgemeine Verfahren zur
Irreduzibilitätsprüfung einsetzen (wie in Beispiel~3.17 des Skripts
vorgeführt). Der Inhalt des Polynoms ist~$\frac{1}{6}$. Da es vom Grad~3 ist,
genügt es, einelementige Auswahlen der Nullstellen zu untersuchen -- wir müssen
also prüfen, ob mindestens eine der Nullstellen mit~$6$ multipliziert eine
ganze Zahl gibt. Das ist nicht der Fall:
\begin{align*}
  x_1 &\approx -0{,}651 & x_2 &\approx 0{,}916 & x_3 &\approx \phantom{0}2{,}232 \\
  6 x_1 &\approx -3{,}9 & 6 x_2 &\approx 5{,}5 & x_3 &\approx 13{,}4
\end{align*}
Wenn man Zugriff auf Näherungswerte der Nullstellen hat (und man weiß, dass man
ihnen vertrauen kann), ist dieses Verfahren in der Praxis schneller als obige
Methode über Nullstellenkandidaten.
\end{loesungE}
\end{aufgabe}

\ifloesungen\newpage\fi
\begin{aufgabe}{Prime Polynome}
\begin{enumerate}
\item Ein normiertes Polynom $f$ mit rationalen Koeffizienten heißt
genau dann \emph{prim}, wenn es nicht das Einspolynom ist und folgende
Eigenschaft hat: Immer, wenn~$f$ ein
Produkt~$g \cdot h$ zweier Polynome mit
rationalen Koeffizienten teilt, so teilt~$f$ schon mindestens einen der
beiden Faktoren. Zeige, dass jedes prime Polynom irreduzibel ist.
\item Teile ein über den rationalen Zahlen irreduzibles Polynom~$f$ ein
Produkt~$g_1 \cdots g_n$ von Polynomen mit rationalen Koeffizienten.
Zeige, dass~$f$ dann schon eines der~$g_i$ teilt.
\end{enumerate}

\begin{loesungE}
\item Sei~$f(X)$ ein primes Polynom. Zunächst müssen wir zeigen, dass~$f(X) =
f(X)$ eine Zerlegung ist. Nach Definition von \emph{Zerlegung} müssen wir dazu
nur anmerken, dass~$f$ mindestens Grad~1 hat; das ist erfüllt, da~$f$ nicht das
Einspolynom ist.

Dann bleibt nur noch zu zeigen, dass~$f(X) = f(X)$
die \emph{einzige} Zerlegung von~$f$ ist. Sei dazu~$f = g \cdot h$ mit
normierten nichtkonstanten Polynomen~$g(X), h(X)$ mit rationalen Koeffizienten
eine hypothetische Zerlegung.
Dann folgt insbesondere~$f \mid gh$, also wegen der Primalität~$f \mid g$ oder~$f
\mid h$.

Im ersten Fall folgt~$g = f \cdot \widetilde g$ für ein
Polynom~$\widetilde g$. Dieses Polynom muss normiert sein, da~$g$ und~$h$ es
sind. Eingesetzt ergibt sich~$f = f \widetilde g h$; ein Gradvergleich zeigt,
dass dann~$h$ doch konstant ist -- das ist ein Widerspruch. Analog verfährt man
im zweiten Fall.

\item Wir führen einen Induktionsbeweis. Der Induktionsanfang~$n = 0$ ist
witzig: Für ihn müssen wir zeigen, dass aus der Voraussetzung~$f \mid 1$
(leeres Produkt) eine unmögliche Aussage folgt (nämlich, dass es ein~$i \in
\emptyset$ gibt). Da die Voraussetzung~$f \mid 1$ nie erfüllt ist (irreduzible
Polynome haben mindestens Grad~1), ist diese Implikation trivialerweise
erfüllt.

Wenn man mag, kann man die Induktion auch erst bei~$n = 1$ beginnen: Dann ist
der Induktionsanfang klar und bereit weniger Kopfschmerzen.

Für den Beweis des Induktionsschritts~$n \to n + 1$ gelte~$f \mid g_1 \cdots
g_{n+1} = (g_1 \cdots g_n) \cdot g_{n+1}$. Nach Folgerung~3.11 folgt dann~$f
\mid g_1 \cdots g_n$ oder~$f \mid g_{n+1}$. Im zweiten Fall sind wir sofort
fertig, im ersten Fall nach Anwendung der Induktionsvoraussetzung.

\emph{Variante:} Wir können die Polynome~$g_i$ jeweils in ihre irreduziblen
Faktoren zerlegen. Wegen der Eindeutigkeit der Zerlegung muss dann einer dieser
vielen Faktoren gleich~$f$ sein. Also teilt~$f$ dasjenige~$g_i$, zu dem dieser
Faktor gehört.
\end{loesungE}
\end{aufgabe}

\ifloesungen\newpage\fi
\begin{aufgabe}{Euklidischer Algorithmus für ganze Zahlen}
Seien~$a$ und~$b$ ganze Zahlen. Zeige, dass es eine ganze Zahl~$d
\geq 0$ gibt, welche ein gemeinsamer Teiler von~$a$ und~$b$ ist, und für die es
weitere ganze Zahlen~$r$ und~$s$ mit~$d = r \cdot a + s \cdot b$ gibt.

\begin{loesung}
Wir führen sukzessive Divisionen mit Rest durch, sodass wir folgende
Gleichungen erhalten:
\begin{align*}
  a &= p_1 \cdot b + r_1 \\
  b &= p_2 \cdot r_1 + r_2 \\
  r_1 &= p_3 \cdot r_2 + r_3 \\
  r_2 &= p_4 \cdot r_3 + r_4 \\
  \vdots \\
  r_{n-3} &= p_{n-1} \cdot r_{n-2} + r_{n-1} \\
  r_{n-2} &= p_{n} \cdot r_{n-1} + r_{n} \\
  r_{n-1} &= p_{n+1} \cdot r_{n} + 0
\end{align*}
Dabei soll jeweils~$0 \leq r_i < |r_{i-1}|$ gelten, wobei wir der
Übersichtlichkeit halber~$r_0 := b$ setzen. Wir hören auf, wenn sich als
Rest~$0$ ergibt; den vorletzten Rest~$r_n$ nennen wir kurz~"`$d$"'. Nun sind
noch drei Dinge zu tun:
\begin{enumerate}
\item[1.] Wir müssen zeigen, dass das Verfahren \emph{terminiert}, also
irgendwann zu einem Ende kommt. (Wenn nicht, hätten wir~$d$ nicht wirklich
gefunden!)
\item[2.] Wir müssen zeigen, dass der vorletzte Rest~$d$ tatsächlich ein
gemeinsamer Teiler von~$a$ und~$b$ ist.
\item[3.] Wir müssen zeigen, dass es wirklich die geforderte Bézoutdarstellung
gibt.
\end{enumerate}
Für Punkt~1 müssen wir nur beobachten, dass die Reste~$r_i$ mit jedem Schritt
echt abnehmen, aber nach unten durch~$0$ beschränkt sind. Daher muss
irgendwann (genauer: nach spätestens~$|b|$ Schritten) der Rest~$0$
erreicht werden.

Für Punkt~2 können wir die Gleichungen rückwärts betrachten: Die letzte
Gleichung sagt uns, dass~$d = r_n$ ein Teiler von~$r_{n-1}$ ist. Aus der
vorletzten Gleichung folgt daher, dass~$d$ auch ein Teiler von~$r_{n-2}$ ist.
Analog folgt dann aus der drittletzten Gleichung, dass~$d$ ein Teiler
von~$r_{n-3}$ ist (wieso?). Wenn wir auf dieselbe Art und Weise fortfahren,
erkennen wir schließlich, dass~$d$ ein Teiler von~$b$ und von~$a$ ist.

Für Punkt~3 lösen wir die Gleichungen rückwärts auf: Mithilfe der vorletzten
Gleichung können wir~$d$ als Linearkombination von~$r_{n-2}$ und~$r_{n-1}$
ausdrücken. Die drittletzte Gleichung erlaubt uns wiederum, die Zahl~$r_{n-1}$
als Linearkombinaton von~$r_{n-3}$ und~$r_{n-2}$ zu schreiben; als
Zwischenfazit können wir daher~$d$ als Linearkombination von~$r_{n-3}$
und~$r_{n-2}$ schreiben. Wenn wir auf diese Art und Weise fortfahren, können
wir schlussendlich~$d$ als Linearkombination von~$a$ und~$b$ ausdrücken.

\emph{Bemerkung:} Sollte~$|b| > |a|$ sein, funktioniert das hier gegebene
Verfahren trotzdem: Im ersten Schritt stellt sich die gewohnte Situation von
selbst ein (siehe Beispiel).

\emph{Bemerkung:} Gute Beispiele zum Vorführen bzw. Üben des euklidischen
Algorithmus geben Fibonaccizahlen ab (wieso?), ggf. multipliziert mit einer
gemeinsamen Konstante, damit der größte gemeinsamer Teiler keine langweilige~1
wird. Etwa ergibt sich für~$a = 2 \cdot 21$, $b = 2 \cdot 34$:
\begin{align*}
  a = 42 &= 0 \cdot 68 + 42 \\
  b = 68 &= 1 \cdot 42 + 26 \\
  42 &= 1 \cdot 26 + 16 \\
  26 &= 1 \cdot 16 + 10 \\
  16 &= 1 \cdot 10 + \phantom{0}6 \\
  10 &= 1 \cdot \phantom{0}6 + \phantom{0}4 \\
   6 &= 1 \cdot \phantom{0}4 + \phantom{0}2 \\
   4 &= 2 \cdot \phantom{0}2 + \phantom{0}0.
\end{align*}
\end{loesung}
\end{aufgabe}
 
\end{document} 

Herausgefallen:
* Algebraischen Zahlen sind ein Integritätsbereich.
* Bewertung wie Logarithmus

Was passiert bei der f_a-Aufgabe, wenn man f_0 betrachtet?

Euklid-Aufgabe mit Induktionsbeweis
