\documentclass{algblatt}
\loesungenfalse

\geometry{tmargin=2cm,bmargin=2cm,lmargin=2.9cm,rmargin=2.9cm}

%\setlength{\titleskip}{0.7em}
\setlength{\aufgabenskip}{1.6em}

\begin{document}

\vspace*{-1.5cm}
\maketitle{3}{Abgabe bis 6. Mai 2013, 17:00 Uhr}

\begin{aufgabe}{Beispiele für algebraische Zahlen}
\begin{enumerate}
\item Ist die Zahl~$\cos 10^\circ$ algebraisch?
\item Zeige, dass die Polynomgleichung $X^3 - 2X + 5 = 0$ genau eine reelle
Lösung~$\alpha$ besitzt.
\item Zeige, dass diese Lösung~$\alpha$ invertierbar ist, und finde eine normierte
Polynomgleichung mit rationalen Koeffizienten, die~$\alpha^{-1}$ als Lösung besitzt.
\end{enumerate}
\begin{loesungE}
\item Ja, denn die Zahl~$\cos 10^\circ$ ist der Realteil der komplexen
Zahl~$e^{\pi\i/18}$, und diese ist algebraisch, da sie die Gleichung
\[ X^{18} + 1 = 0 \]
erfüllt (wieso?). Da Realteile algebraischer Zahlen selbst ebenfalls
algebraisch sind, begründet das die Algebraizität von~$\cos 10^\circ$.

\emph{Bemerkung:} Ein ähnliches Argument beweist die Algebraizität des Kosinus
eines jeden rationalen Vielfaches von~$2\pi$.

\emph{Bemerkung:} Über das Additionstheorem~$\cos(30^\circ) = \sqrt{3}/2 = 4
\cos^3(10^\circ) - 3\cos(10^\circ)$ kann man durch Umstellen auch explizit die
Gleichung~$X^6-\frac{3}{2}X^4+\frac{9}{16}X^2-\frac{3}{64} = 0$ erhalten.

\item Wir setzen~$f := X^3-2X+5$. Da~$f(-3) =
-16 < 0 < 1 = f(-2)$, besitzt die Gleichung~$f(X) = 0$ nach Blatt~1, Aufgabe~2
mindestens eine reelle Lösung~$\alpha$ im Intervall~$(-3, -2)$. Mit einer Polynomdivision
durch~$(X-\alpha)$ kann man~$f$ faktorisieren:
\[ f = (X - \alpha) (X^2 + \alpha X + \alpha^2-2). \]
Das verbleibende Polynom hat nun keine weiteren reellen Nullstellen, denn seine
Diskriminante ist negativ:
\[ D = \alpha^2 - 4(\alpha^2-2) = 8 - 3 \alpha^2 \leq 8 - 3 \cdot 2^2 = -4 < 0. \]
\item Die Zahl~$\alpha$ kann nicht Null sein, da Null keine Lösung der
Gleichung~$f(X) = 0$ ist:
\[ f(0) = 0^3 - 2 \cdot 0 + 5 = 5 \neq 0. \]
Also ist~$\alpha$ invertierbar. Für die Zahl~$\alpha^{-1}$ gilt
\[ (\alpha^{-1})^{-3} - 2 (\alpha^{-1})^{-1} + 5 = 0; \]
das ist zwar eine Gleichung, aber keine Polynomgleichung für~$\alpha^{-1}$.
Wenn wir mit~$(\alpha^{-1})^3$ durchmultiplizieren, erhalten wir die
äquivalente Gleichung
\[ 1 - 2 (\alpha^{-1})^2 + 5 (\alpha^{-1})^3 = 0. \]
Also ist~$\alpha^{-1}$ Lösung der normierten Polynomgleichung mit rationalen
Koeffizienten
\[ X^3 - \frac{2}{5} X^2 + \frac{1}{5} = 0. \]
\end{loesungE}
\end{aufgabe}

\begin{aufgabe}{Produkt algebraischer Zahlen}
\begin{enumerate}
\item Seien~$x$ und~$y$ Zahlen mit~$x^3-x+1=0$ und~$y^2-2=0$.
Finde eine normierte Polynomgleichung mit rationalen
Koeffizienten, die die Zahl~$x \cdot y$ als Lösung besitzt.
\item Der \emph{Grad} einer algebraischen Zahl~$z$ ist der kleinstmögliche Grad
einer normierten Polynomgleichung mit rationalen Koeffizienten, die~$z$ als Lösung
besitzt. Finde eine Abschätzung für den Grad des Produkts zweier algebraischer
Zahlen in Abhängigkeit der Grade der Faktoren.
\end{enumerate}
\begin{loesungE}
\item Es ist unnötig, nach den Zahlen~$x$ und~$y$ aufzulösen. Stattdessen
können wir direkt das Verfahren aus Proposition~1.3 des Skripts verwenden, wir
setzen also $c_{ij} := x^i y^j$ für~$i = 0,1,2$ und~$j = 0,1$ und rechnen:
\begin{align*}
  xy \cdot c_{00} &= xy \cdot x^0 y^0 = xy = c_{11} \\
  xy \cdot c_{01} &= xy \cdot x^0 y^1 = xy^2 = 2x = 2 c_{10} \\
  xy \cdot c_{10} &= xy \cdot x^1 y^0 = x^2y = c_{21} \\
  xy \cdot c_{11} &= xy \cdot x^1 y^1 = x^2y^2 = 2x^2 = 2 c_{20} \\
  xy \cdot c_{20} &= xy \cdot x^2 y^0 = x^3y = (x-1)y = c_{11} - c_{01} \\
  xy \cdot c_{21} &= xy \cdot x^2 y^1 = x^3y^2 = 2x^3 = 2(x-1) = 2 c_{10} - 2 c_{00}
\end{align*}
In Matrixform:
\[
  xy \cdot
  \begin{pmatrix}c_{00}\\c_{01}\\c_{10}\\c_{11}\\c_{20}\\c_{21}\end{pmatrix} =
  \underbrace{\begin{pmatrix}
    0&0&0&1&0&0 \\
    0&0&2&0&0&0 \\
    0&0&0&0&0&1 \\
    0&0&0&0&2&0 \\
    0&-1&0&1&0&0 \\
    -2&0&2&0&0&0
  \end{pmatrix}}_{=:\,A}
  \begin{pmatrix}c_{00}\\c_{01}\\c_{10}\\c_{11}\\c_{20}\\c_{21}\end{pmatrix}
\]
Also ist~$xy$ als Eigenwert dieser Matrix Nullstelle ihres charakteristischen
Polynoms
\[ p(X) = \det(X I - A) = \cdots = X^6 - 4\,X^4 + 4\,X^2 - 8 \]
und erfüllt somit die Gleichung~$p(X) = 0$.

\begin{scriptsize}\emph{Bemerkung:} Bei anderer Anordnung der~$c_{ij}$ erhält
man die Beziehungen
\[
  xy \cdot
  \begin{pmatrix}c_{00}\\c_{10}\\c_{20}\\c_{01}\\c_{11}\\c_{21}\end{pmatrix} =
  \begin{pmatrix}
    0&0&0&0&1&0 \\
    0&0&0&0&0&1 \\
    0&0&0&-1&1&0 \\
    0&2&0&0&0&0 \\
    0&0&2&0&0&0 \\
    -2&2&0&0&0&0
  \end{pmatrix}
  \begin{pmatrix}c_{00}\\c_{10}\\c_{20}\\c_{01}\\c_{11}\\c_{21}\end{pmatrix}\!,
  \quad
  xy \cdot
  \begin{pmatrix}c_{00}\\c_{01}\\c_{10}\\c_{20}\\c_{11}\\c_{21}\end{pmatrix} =
  \begin{pmatrix}
    0&0&0&0&1&0 \\
    0&0&2&0&0&0 \\
    0&0&0&0&0&1 \\
    0&-1&0&0&1&0 \\
    0&0&0&2&0&0 \\
    -2&0&2&0&0&0
  \end{pmatrix}
  \begin{pmatrix}c_{00}\\c_{01}\\c_{10}\\c_{20}\\c_{11}\\c_{21}\end{pmatrix}\!.
\]
Das charakteristische Polynom ist jeweils dasselbe.\end{scriptsize}

\item Sei~$x$ eine algebraische Zahl vom Grad~$n$ und~$y$ eine algebraische
Zahl vom Grad~$m$. Nach Proposition~1.3 des Skripts erhält man eine
Polynomgleichung für das Produkt~$xy$, indem man aus den Zahlen~$xy \cdot
c_{ij}$, wobei~$c_{ij} := x^i y^j$ und~$i = 0,\ldots,n-1$, $j = 0,\ldots,m-1$,
eine Matrix baut und deren charakteristisches Polynom bestimmt. Da diese Matrix
eine~$(nm \times nm)$-Matrix ist, hat das charakteristische Polynom Grad~$nm$.
Also ist der Grad des Produkts höchstens~$nm$.

\emph{Bemerkung:} Diese Abschätzung ist \emph{scharf}, d.\,h. es gibt
tatsächlich Fälle, bei denen der Grad des Produkts genau gleich dem Produkt der
Grade der Faktoren ist (etwa bei~$x = \sqrt[3]{2}$, $y = \sqrt{3}$ -- der Grad
von~$x$ ist~3, der von~$y$ ist~2 und der von~$x \cdot y$ ist~6). Es gibt aber
auch Fälle, bei denen der Produktgrad deutlich unter der Schranke aus der
Abschätzung bleibt (etwa bei~$x = \sqrt[71]{2}$, $y = 1/x$).

\emph{Bemerkung:} Für den Grad der Summe algebraischer Zahlen gilt dieselbe
Abschätzung.
\end{loesungE}
\end{aufgabe}

\ifloesungen\newpage\fi
\begin{aufgabe}{Eigenschaften algebraischer Zahlen}
\begin{enumerate}
\item Zeige, dass das komplex Konjugierte einer jeden algebraischen Zahl algebraisch
ist.
\item Zeige, dass der Betrag einer jeden algebraischen Zahl algebraisch ist.
\item Zeige, dass rationale ganz algebraische Zahlen schon ganzzahlig sind.
\item Sei~$f$ ein normiertes Polynom vom Grad mindestens~1 mit rationalen Koeffizienten
und~$z$ eine transzendente Zahl. Zeige, dass dann auch~$f(z)$ eine
transzendente Zahl ist.
\end{enumerate}
\begin{loesungE}
\item Da~$z$ algebraisch ist, ist~$z$ Lösung einer normierten Polynomgleichung
\[ X^n + a_{n-1} X^{n-1} + \cdots + a_1 X + a_0 = 0 \]
mit rationalen Koeffizienten, d.\,h. es gilt
$z^n + a_{n-1} z^{n-1} + \cdots + a_1 z + a_0 = 0$.
Damit folgt (wieso?)
\[ 0 = \overline{z^n + a_{n-1} z^{n-1} + \cdots + a_1 z + a_0} =
  \overline{z}^n + a_{n-1} \overline{z}^{n-1} + \cdots + a_1 \overline{z} + a_0, \]
also ist~$\overline{z}$ Lösung derselben Gleichung und damit als algebraisch
entlarvt.

\emph{Bemerkung:} Die Erkenntnis aus dieser Aufgabe kann man als griffige
Merkregel formulieren: Lösungen von Polynomgleichungen mit reellen
Koeffizienten treten stets in komplex-konjugierten Paaren auf. Für allgemeine
Polynomgleichungen stimmt das nicht.

\item Sei~$z$ eine algebraische Zahl. Dann gilt
\[ |z|^2 = z \overline{z}. \]
Da mit~$z$ auch~$\overline{z}$ algebraisch ist und das Produkt algebraischer
Zahlen algebraisch ist, ist die rechte Seite dieser Identität algebraisch. Der
Betrag von~$z$ ist also als eine der Lösungen der Gleichung mit algebraischen
Koeffizienten
\[ X^2 - z \overline{z} = 0 \]
ebenfalls algebraisch.
\item Sei~$z$ eine rationale ganz-algebraische Zahl. Dann erfüllt~$z$ also eine
normierte Polynomgleichung mit ganzzahligen Koeffizienten. Nach Blatt~0,
Aufgabe~3b) ist~$z$ daher schon ganzzahlig.
\item Angenommen,~$y := f(z)$ wäre algebraisch. Dann gibt es ein normiertes
Polynom~$g$ mit rationalen Koeffizienten, sodass~$y$ die Gleichung
\[ g(Y) = 0 \]
erfüllt, sodass also~$g(f(z)) = 0$ ist. Setzt man~$h := g \circ f$ -- das ist
wieder ein normiertes Polynom mit rationalen Koeffizienten (wieso?) -- sieht
man, dass~$z$ Lösung der Gleichung~$h(X) = 0$ ist. Das ist ein Widerspruch zur
Transzendenz von~$z$.

\emph{Explizitere Variante:} Angenommen,~$y := f(z) = z^n + a_{n-1} z^{n-1} + \cdots +
a_1 z + a_0$ wäre algebraisch. Dann erfüllt~$y$ eine normierte Polynomgleichung
mit rationalen Koeffizienten:
\[ y^m + b_{m-1} y^{m-1} + \cdots + b_1 y + b_0 = 0. \]
Setzt man obige Darstellung von~$y$ in diese Gleichung ein, erhält man eine
normierte Polynomgleichung mit rationalen Koeffizienten, die~$z$ als Lösung
hat. Das ist ein Widerspruch zur Transzendenz von~$z$.
\end{loesungE}
\end{aufgabe}

\ifloesungen\newpage\fi
\begin{aufgabe}{Spielen mit Einheitswurzeln}
\begin{enumerate}
\item Finde alle komplexen Lösungen der Gleichung~$X^6 + 1 = 0$.
\item Finde eine Polynomgleichung, deren Lösungen genau die Ecken
desjenigen re\-gel\-mä\-ßi\-gen Siebenecks in der komplexen Zahlenebene sind, dessen Zentrum
der Ursprung der Ebene ist und das deine Lieblingszahl als eine Ecke besitzt.
\item Zeige, dass die Gleichung~$X^{n-1} + X^{n-2} + \cdots + X + 1 = 0$
genau~$n-1$ Lösungen besitzt, und zwar alle~$n$-ten Einheitswurzeln bis auf
die~$1$.
\item Sei~$\zeta$ eine $n$-te und~$\vartheta$ eine~$m$-te Einheitswurzel.
Zeige, dass~$\zeta \cdot \vartheta$ eine~$k$-te Einheitswurzel ist, wobei~$k$
das kleinste gemeinsame Vielfache von~$n$ und~$m$ ist.
\end{enumerate}
\begin{loesungE}
\item Bezeichne~$\xi$ eine primitive sechste Einheitswurzel, etwa~$\xi =
e^{2\pi\i / 6}$. Eine Lösung der Gleichung ist~$\i$. Daher sind die
insgesamt sechs Lösungen der Gleichung durch
\[ \i,\quad \xi \i,\quad \xi^2 \i,\quad \xi^3 \i,\quad \xi^4 \i,\quad \xi^5 \i \]
gegeben (wieso?).

\emph{Bemerkung:} Man kann auch die Faktorisierung~$X^{12} - 1 = (X^6 - 1)
\cdot (X^6 + 1)$ ausnutzen. An dieser erkennt man nämlich sofort, dass
die Lösungen von~$X^6 + 1 = 0$ einfach genau die zwölften Einheitswurzeln sind,
die keine sechsten Einheitswurzeln sind.

\item Sei~$\heartsuit$ meine Lieblingszahl. Dann tut's die Gleichung~$X^7 -
\heartsuit^7 = 0$ (wieso?).

\emph{Bemerkung:} Wenn man möchte, kann man die Gleichung auch ausfaktorisiert
hinschreiben. Sei dazu~$\xi$ eine primitive siebte Einheitswurzel, etwa~$\xi =
e^{2\pi\i / 7}$. Dann ist obige Gleichung äquivalent zu
\[ \prod_{k=0}^6 (X - \xi^k \cdot \heartsuit) = 0. \]

\emph{Bemerkung:} Für die meisten Wahlen von~$\heartsuit$ kann es keine
Polynomgleichung mit \emph{reellen} Koeffizienten geben, die genau die sieben
Ecken als Lösungen besitzt. Denn jede solche Gleichung würde mit~$\heartsuit$
auch das komplex Konjugierte~$\overline{\heartsuit}$ als Lösung besitzen, das
ist aber im Allgemeinen keine der Ecken.

\item Sei~$x$ eine beliebige komplexe Zahl. Dann gilt:
\begin{align*}
  && x^{n-1} + x^{n-2} + \cdots + x + 1 &= 0 \\
  \stackrel{?}{\Longleftrightarrow} &&
    x^{n-1} + x^{n-2} + \cdots + x + 1 &= 0 \ \wedge\  x \neq 1 \\
  \Longleftrightarrow &&
    (x - 1) \cdot (x^{n-1} + x^{n-2} + \cdots + x + 1) &= 0 \ \wedge\  x \neq 1 \\
  \Longleftrightarrow &&
    x^n - 1 &= 0 \ \wedge\  x \neq 1 \\
  \Longleftrightarrow &&
    \omit\rlap{\text{$x$ ist eine der~$n$-ten Einheitswurzeln, aber nicht
    die~$1$.}}
\end{align*}
Da wir durchgängig Äquivalenzumformungen verwendet haben, zeigt diese
Überlegung tatsächlich die Behauptung.

\emph{Bemerkung:} Bei einem~"`$\Rightarrow$"'-Schritt können Scheinlösungen
entstehen, bei einem~"`$\Leftarrow$"'-Schritt können Lösungen verloren gehen.

\item Da~$k$ ein Vielfaches von~$n$ ist, gilt~$\zeta^k = 1$. Analog
gilt~$\vartheta^k = 1$. Daher folgt:
\[ (\zeta \cdot \vartheta)^k = \zeta^k \cdot \vartheta^k = 1 \cdot 1 = 1. \]
\end{loesungE}
\end{aufgabe}

\ifloesungen\newpage\fi
\begin{aufgabe}{Primitive Einheitswurzeln}
Eine~$n$-te Einheitswurzel~$\zeta$ heißt genau dann \emph{primitiv}, wenn
\emph{jede}~$n$-te Einheitswurzel eine ganzzahlige Potenz von~$\zeta$ ist.
Sei~$\varphi(n)$ die Anzahl der zu~$n$ teilerfremden Zahlen
in~$\{1,\ldots,n\}$.
\begin{enumerate}
\item Kläre ohne Verwendung von~b): Welche der vierten Einheitswurzeln sind
primitiv?
\item Zeige, dass es genau~$\varphi(n)$ primitive~$n$-te
Einheitswurzeln gibt.
\end{enumerate}
\begin{loesungE}
\item Insgesamt gibt es vier vierte Einheitswurzeln:
\[ 1,\quad \i,\quad -1,\quad -\i. \]
Von diesen sind~$\i$ und~$-\i$ primitiv: Denn die Potenzen von~$\i$ geben
gerade diese vier Zahlen, und für~$-\i$ stimmt es auch. Die anderen beiden
Wurzeln sind aber nicht primitiv: Denn die Potenzen von~$1$ sind nur~$1$
selbst, und die von~$-1$ sind nur~$\pm 1$.

\item Sei~$\xi := e^{2\pi\i/n}$. Dann wollen wir untersuchen, wann eine
beliebige~$n$-te Einheitswurzel~$\xi^a$ primitiv ist:
\begin{align*}
  & \text{$\xi^a$ primitiv} \\
  \Longleftrightarrow\ &
    \text{jede $n$-te Einheitswurzel ist Potenz von~$\xi^a$} \\
  \Longleftrightarrow\ &
    \text{speziell~$\xi$ ist Potenz von~$\xi^a$} \\
  \Longleftrightarrow\ &
    \exists m \in \ZZ{:}\ 
    (\xi^a)^m = \xi \\
  \Longleftrightarrow\ &
    \exists m \in \ZZ{:}\ 
    am \equiv 1 \mod n \\
  \Longleftrightarrow\ &
    \text{$a$ und $n$ sind zueinander teilerfremd}
\end{align*}
Das zeigt die Behauptung. (Wieso gelten die Äquivalenzaussagen?)

\emph{Bemerkung:} Eine abstraktere Argumentation ist folgende. Die Gruppe
der~$n$-ten Einheitswurzeln (bzgl. der Multiplikation) ist (unkanonisch)
isomorph zu~$\ZZ/(n)$ (bzgl. der Addition). Die Teilmenge der \emph{Erzeuger}
in~$\ZZ/(n)$ (das sind hier die doch bzgl. der Multiplikation invertierbaren
Elemente) entspricht unter dieser Korrespondenz gerade der Menge der
primitiven~$n$-ten Einheitswurzeln. Da es bekanntlich genau~$\varphi(n)$ bzgl. der
Multiplikation invertierbare Elemente in~$\ZZ/(n)$ gibt, zeigt das die
Behauptung.
\end{loesungE}
\end{aufgabe}

\end{document}

\begin{exercise}(3 Punkte)\newline
    Folgere die Additionstheoreme für die Sinus- und die Kosinusfunktion aus der
    Identität
    \(\exp({x \mathrm i}) \cdot \exp({y \mathrm i}) = \exp({(x + y) \mathrm i})\).
\end{exercise}
