\documentclass{algblatt}
\loesungenfalse

\geometry{tmargin=2cm,bmargin=2cm,lmargin=2.9cm,rmargin=2.9cm}

%\setlength{\titleskip}{0.7em}
\setlength{\aufgabenskip}{1.6em}

\begin{document}

\vspace*{-1.5cm}
\maketitle{6}{Abgabe bis 27. Mai 2013, 17:00 Uhr}

\begin{aufgabe}{Anwendungen der Diskriminante}
\begin{enumerate}
\item
Sei~$X^3 + p X + q = 0$ eine reduzierte kubische Gleichung mit ganzzahligen
Koeffizienten~$p$ und~$q$. Zeige, dass die Gleichung drei \emph{paarweise verschiedene} Lösungen
(in den komplexen Zahlen) besitzt, wenn~$q$ ungerade ist.
\item Sei~$X^n + a_{n-1}X^{n-1} + \cdots + a_1 X + a_0 = 0$ eine normierte
Polynomgleichung mit rationalen Koeffizienten. Zeige, dass sie mindestens eine
nicht reelle Nullstelle besitzt, wenn ihre Diskriminante negativ ist.
\end{enumerate}

\begin{loesungE}
\item Die Diskriminante~$\Delta = -4p^3 - 27q^2$ ist nicht null, da der
Term~$-4p^3$ gerade, aber~$27q^2$ ungerade ist.
\item Wenn~$x_1,\ldots,x_n$ die Lösungen der Gleichungen mit Vielfachheiten
sind, so ist~$\Delta = \prod_{i < j} (x_i-x_j)^2$. Wenn nun alle~$x_i$ reell
wären, wäre~$\Delta \geq 0$.
\end{loesungE}
\end{aufgabe}

\begin{aufgabe}{Diskriminanten allgemeiner kubischer Gleichungen}
\begin{enumerate}
\item Berechne die Diskriminante der allgemeinen kubischen Gleichung
$X^3 + a X^2 + b X + c = 0$.
\item Zeige, dass~$X^3 - 5\,X^2 + 7\,X - 3 = 0$ höchstens zwei verschiedene
Lösungen hat.
\end{enumerate}

\begin{loesungE}
\item Wenn wir~$Y := X + \frac{a}{3}$ und
\begin{align*}
  p &:= b - \frac{a^2}{3} &
  q &:= \frac{2a^3 - 9ab + 27c}{27}
\end{align*}
setzen, lässt sich die gegebene Gleichung äquivalent als
\[ Y^3 + pY + q = 0 \]
schreiben. Die Lösungen für~$Y$ dieser Gleichung sind von den Lösungen der
originalen Gleichung um~$a/3$ verschoben -- das ändert aber die Diskriminante
nicht, da in ihr nur die Differenzen der Lösungen eingehen. Somit ist die
Diskriminante der gegebenen Gleichung gleich der Diskriminante der reduzierten
Gleichung, also gleich
\[ \Delta = -4p^3 - 27q^2 = \cdots = a^2b^2 - 4b^3 - 4a^3c - 27c^2 + 18abc. \]

\item Die Diskriminante dieser Gleichung ist null:
\[ \Delta = (-5)^2 \cdot 7^2 - 4 \cdot 7^3 - 4 \cdot (-5)^3 \cdot (-3) - 27 \cdot
(-3)^2 + 18\cdot(-5)\cdot7\cdot(-3) = 0. \]

\emph{Variante:} Man errät die Lösung~1 und findet dann nach einer
Polynomdivision die weiteren Nullstellen. Dann ergibt sich die
Faktorisierung~$X^3 - 5\,X^2 + 7\,X - 3 = (X-1)^2 \cdot (X-3)$.
\end{loesungE}
\end{aufgabe}

\begin{aufgabe}{Transzendente Zahlen}
\begin{enumerate}
\item Sei~$(z_n)$ eine konvergente komplexe Zahlenfolge mit Grenzwert~$z$ und
seien alle Folgenglieder~$z_n$ algebraisch. Ist dann auch~$z$ algebraisch?
\item Ist~$\sqrt[3]{\pi}$ eine algebraische Zahl? Ist~$\pi^3$ algebraisch?
\item Finde eine Folge paarweise verschiedener transzendenter Zahlen.
\end{enumerate}

\begin{loesungE}
\item Das ist in den seltensten Fällen der Fall. Etwa kann man
\[ z_n := 3{,}\text{die ersten~$n$ Nachkommaziffern von~$\pi$} \]
setzen. Dann sind alle Folgenglieder algebraisch (sogar rational), aber der
Grenzwert~$\pi$ ist nicht algebraisch.

Ein anderes Beispiel ist
\[ z_n := \left(1 + \frac{1}{n}\right)^n. \]
Dann sind ebenfalls alle Folgenglieder algebraisch (sogar rational), aber der
Grenzwert~$e$ ist nicht algebraisch.

\emph{Bemerkung:} Tatsächlich gilt [in klassischer Logik] eine Art Umkehrung
der Aufgabe: Jede komplexe Zahl ist Grenzwert einer Folge algebraischer Zahlen.
(Es ist sogar jede komplexe Zahl Grenzwert einer Folge von Zahlen der
Form~$x+\i y$ mit~$x,y \in \QQ$.) Man sagt auch, dass die algebraischen Zahlen
\emph{dicht} in~$\CC$ liegen.

\item Nein: Wäre~$\sqrt[3]{\pi}$ algebraisch, so wäre auch~$(\sqrt[3]{\pi})^3 =
\pi$ algebraisch.

Ebenso ist~$\pi^3$ nicht algebraisch: Wäre~$\pi^3$ algebraisch, so wäre
auch~$\sqrt[3]{\pi^3} = \pi$ algebraisch (da Wurzeln algebraischer Zahlen stets
algebraisch sind).

\item Man kann etwa $z_n := \pi + n$ setzen. Die Folgenglieder sind paarweise
verschieden (klar) und jeweils transzendent (wieso?).
\end{loesungE}
\end{aufgabe}

\begin{aufgabe}{Triangulatur des Kreises}
Ist folgendes Problem lösbar? Gegeben ein Kreis. Konstruiere nur mit Zirkel und
Lineal ein gleichseitiges Dreieck mit demselben Flächeninhalt.

\begin{loesung}
Sei~$r$ der Radius des Kreises und~$a$ die Seitenlänge eines flächengleichen
gleichseitigen Dreiecks. Da nach dem Satz des Pythagoras die Höhe des Dreiecks
durch~$h = \frac{\sqrt{3}}{2} a$ gegeben ist, gilt dann also die Beziehung
\[ A_\bigcirc = \pi r^2 = \frac{\sqrt{3}}{4} a^2 = A_\triangle. \]
Somit ist die Seitenlänge~$a$ die Zahl
\[ a = \sqrt{\frac{4}{\sqrt{3}} \pi} \cdot r. \]
Für die meisten Werte von~$r$, etwa~$r = 1$, ist die Seitenlänge~$a$ daher
nicht algebraisch (wieso?) und somit nicht konstruierbar: Wäre sie es, könnte
man die Strecke beim Ursprung abtragen und so die Zahl~$a$ konstruieren -- aber
transzendente Zahlen sind nicht konstruierbar. Im Allgemeinen ist das Problem
also nicht lösbar.

\emph{Geometrische Alternativlösung:}
Aus jedem gleichseitigen Dreieck kann man ein flächengleiches Rechteck
konstruieren, indem man es längs einer Höhe aufschneidet und eine der
entstehenden Hälften längs der Diagonale mit der anderen Hälfte verklebt
(Skizze!).

Ferner kann man aus jedem Rechteck ein flächengleiches Quadrat konstruieren:
\begin{center}
  \scalebox{0.5}{\input{quadrat-aus-rechteck.pspdftex}}
\end{center}
Dabei garantiert der Höhensatz, dass das konstruierte Quadrat tatsächlich
denselben Flächeninhalt hat wie das Rechteck.

Wäre also das gegebene Problem immer lösbar, wäre auch das Problem der
Quadratur des Kreises immer möglich. Das ist bekanntermaßen aber nicht der
Fall.
\end{loesung}
\end{aufgabe}


\begin{aufgabe}{Die Resultante zweier Polynome}
\begin{enumerate}
\item Seien~$f(X)$ und~$g(Y)$ zwei normierte Polynome mit Nullstellen (mit
Vielfachheiten)~$x_1,\ldots,x_n$ bzw.~$y_1,\ldots,y_m$. Zeige, dass der
Ausdruck~$R := \prod_{i,j} (x_i - y_j)$ ein Polynom in den
Koeffizienten von~$f(X)$ und den
Koeffizienten von~$g(Y)$ ist.
\item Seien~$X^2 + aX + b = 0$ und~$Y^2 + cY + d = 0$ zwei quadratische
Gleichungen. Gib einen in~$a$, $b$, $c$ und~$d$ polynomiellen Ausdruck an, der
genau dann verschwindet, wenn die beiden Gleichungen eine gemeinsame Lösung
besitzen.
\end{enumerate}

\begin{loesungE}
\item Die Grundidee ist einfach: Der gegebene Ausdruck~$R$ ist symmetrisch in
den~$x_i$ und separat symmetrisch in den~$y_j$. Mit zweimaliger Anwendung des
Hauptsatzes über die symmetrischen Funktionen folgt daher, dass~$R$ ein
polynomieller Ausdruck in den elementarsymmetrischen Funktionen der~$x_i$ und
in denen der~$y_j$ ist. Diese sind nach dem Satz von Vieta bis auf Vorzeichen
durch die Koeffizienten von~$f$ bzw.~$g$ gegeben.

Wenn man etwas präziser verstehen möchte, auf welche Polynome man den Hauptsatz
anwendet, muss man etwas ausholen.

Zunächst betrachten wir noch nicht speziell die gegebenen
Polynome~$f$ und~$g$. Stattdessen definieren wir allgemein ein Polynom
\[ P := \prod_{i,j} (X_i - Y_j), \]
man beachte die Großbuchstaben auf der rechten Seite. Dieses Polynom ist
offenkundig in den~$X_i$ und separat in den~$Y_j$ symmetrisch. Unser Ziel ist
es nun, dieses Polynom als Polynom in den elementarsymmetrischen
Funktionen~$e_i(X_1,\ldots,X_n)$ und~$\widetilde e_j(Y_1,\ldots,Y_m)$ zu schreiben. Das
erreichen wir in zwei Schritten. Zur besseren Lesbarkeit verwenden wir die
Abkürzung~"`$\vec X$"' für~$X_1,\ldots,X_n$ und analog für~$Y_1,\ldots,Y_m$.

\newcommand{\symm}{\mathrm{symm}}%
\emph{Schritt 1:} Wir fassen~$P$ als Polynom
in~$(\ZZ[\vec Y]_\symm)[\vec X]$ auf. Dabei meinen wir mit~"`$\ZZ[\vec
Y]_\symm$"' den Rechenbereich der in~$Y_1,\ldots,Y_m$ \emph{symmetrischen}
Polynome. So aufgefasst, ist es symmetrisch (in seinen nunmehr einzigen
Variablen, den~$X_i$), womit Satz~2.12 der
Vorlesung uns garantiert, dass es genau ein Polynom~$H \in (\ZZ[\vec
Y]_\symm)[E_1,\ldots,E_n]$ mit
\[ P = H(e_1(\vec X), \ldots, e_n(\vec X)) \]
gibt. Die Koeffizienten von~$H$ stammen dabei aus demselben Rechenbereich wie
die von~$P$, nach unserer Auffassung also~$\ZZ[\vec Y]_\symm$; konkret handelt
es sich bei den Koeffizienten von~$H$ also um in den~$Y_j$ symmetrische
Polynome.

\emph{Schritt 2:} Das Polynom~$H$ können wir auch als Polynom
aus~$(\ZZ[E_1,\ldots,E_n])[\vec Y]$ auffassen; so aufgefasst, ist es in
seinen nunmehr einzigen Variablen, den~$Y_j$, symmetrisch. Damit können wir
abermals Satz~2.12 der Vorlesung
anwenden: Es gibt genau ein Polynom~$L \in
(\ZZ[E_1,\ldots,E_n])[\widetilde E_1,\ldots,\widetilde E_m]$ mit
\[ H = L(\widetilde e_1(\vec Y), \ldots, \widetilde e_m(\vec Y)). \]
Dabei bezeichnen wir zur besseren Unterscheidung die elementarsymmetrischen
Funktionen in den~$Y_j$ mit~$\widetilde e_1(\vec Y), \ldots, \widetilde e_m(\vec Y)$.

\emph{Zwischenfazit:} Zusammenfassend gilt
\begin{equation}\label{zwischenfazit}
  P(X_1,\ldots,X_n, Y_1,\ldots,Y_m) =
  L(e_1(\vec X),\ldots,e_n(\vec X), \widetilde e_1(\vec Y),\ldots,\widetilde e_m(\vec Y)).
\end{equation}
Die Notation auf der rechten Seite bedeutet dabei, dass wir in~$L$ für die
Variablen~$E_i$ jeweils die~$e_i(\vec X)$ und für die Variablen~$\widetilde E_j$ jeweils
die~$\widetilde e_j(\vec Y)$ einsetzen. Unser obiges Ziel ist also erreicht.

Jetzt betrachten wir speziell die Polynome~$f = X^n + a_{n-1}X^{n-1} + \cdots +
a_1X + a_0$ und~$g = Y^m + b_{m-1}Y^{m-1} + \cdots + b_1Y + b_0$ mit ihren
Nullstellen~$x_1,\ldots,y_n$ bzw.~$y_1,\ldots,y_m$. Nach dem Vietaschen Satz
gelten die Beziehungen
\begin{align*}
  e_i(x_1,\ldots,x_n) &= (-1)^i a_{n-i} \\
  \widetilde e_j(y_1,\ldots,y_m) &= (-1)^j b_{m-j}.
\end{align*}
Setzen wir also in Gleichung~\eqref{zwischenfazit} für die Platzhalter~$X_1,\ldots,X_n$ die
tatsächlichen Nullstellen~$x_1,\ldots,x_n$ und für~$Y_1,\ldots,Y_m$ die
Nullstellen~$y_1,\ldots,y_m$ ein, erhalten wir
\[ R = P(x_1,\ldots,x_n,y_1,\ldots,y_m) =
  L(\pm a_{n-1},\ldots,\pm a_0, \pm b_{m-1},\ldots,\pm b_0). \]
Also ist~$R$ in der Tat ein in den Koeffizienten von~$f$ und~$g$ polynomieller
Ausdruck.

\emph{Bemerkung:} Der Ausdruck ist sogar \emph{universell} -- das Polynom~$L$,
das die Form des Ausdrucks vorgibt, hängt nur von den Graden~$n$ und~$m$, aber
nicht von den konkreten Koeffizienten~$a_i$, $b_j$ ab.

\item In Erinnerung an Teilaufgabe~a) definieren wir
\[ R := (x_1 - y_1) \cdot (x_1 - y_2) \cdot (x_2 - y_1) \cdot (x_2 - y_2), \]
wobei~$x_1,x_2$ und~$y_1,y_2$ die Lösungen der ersten bzw. zweiten Gleichung
seien. Dieser Ausdruck ist genau dann null, wenn die beiden Gleichungen
gemeinsame Lösungen besitzen. Jetzt müssen wir ihn noch als Polynom in den
Koeffizienten schreiben -- Teilaufgabe~a) verleiht uns die Gewissheit, dass das
möglich ist. Zur konkreten Ausführung nutzen wir die Beziehungen aus dem
Vietaschen Satz,
\begin{align*}
  b &= x_1 x_2 & d &= y_1 y_2, \\
  a &= -(x_1+x_2) & c &= -(y_1 + y_2),
\end{align*}
und rechnen:
\begin{align*}
  R &= (x_1 - y_1) (x_1 - y_2) (x_2 - y_1) (x_2 - y_2) \\
  &= (x_1^2 - x_1y_2 - x_1y_1 + y_1y_2) (x_2^2 - x_2y_2 - x_2y_1 + y_1y_2) \\
  &= (x_1^2 + cx_1 + d) (x_2^2 + cx_2 + d) \\
  &= x_1^2 x_2^2 + c x_1^2 x_2 + d x_1^2 + c x_1 x_2^2 + c^2 x_1 x_2 +
    c d x_1 + d x_2^2 + cd x_2 + d^2 \\
  &= b^2 + bcx_1 + dx_1^2 + bcx_2 + c^2b + cdx_1 + dx_2^2 + cdx_2 + d^2 \\
  &= b^2 - abc + d(x_1^2 + x_2^2 + 2x_1x_2) - 2x_1x_2d - acd + c^2b + d^2 \\
  &= a^2d - abc - acd + b^2 + bc^2 - 2bd + d^2.
\end{align*}

\emph{Variante:} Statt der Beziehungen aus dem Satz von Vieta kann man auch die
explizite Darstellung der Lösungen durch die Mitternachtsformel verwenden und
dann die Ausmultipliziererei beginnen. Man erhält dasselbe Ergebnis.

\emph{Variante, die zu wenig zeigt:} Man kann mit der Mitternachtsformel die
Lösungen der beiden Gleichungen angeben und diese gleichsetzen -- die
beiden~"`$\pm$"'-Zeichen sollen dabei unabhängig voneinander sein:
\begin{align*}
  && x_{1,2} &= y_{1,2} \\
  \Longleftrightarrow &&
    -a \pm_1 \sqrt{a^2 - 4b} &= -c \pm_2 \sqrt{c^2 - 4d} \\
  \Longleftrightarrow &&
    \pm_1\sqrt{a^2 - 4b} \mp_2\sqrt{c^2 - 4d} &= a - c \\
  \Longrightarrow &&
    a^2 - 4b + 2 \pm_1 \mp_2 \sqrt{a^2-4b}\sqrt{c^2-4d} + c^2 - 4d &= a^2 - 2ac + c^2 \\
  \Longleftrightarrow &&
    \pm_1 \mp_2 \sqrt{a^2-4b}\sqrt{c^2-4d} &= 2b + 2d - ac \\
  \Longrightarrow &&
    (a^2 - 4b) \cdot (c^2 - 4d) &= (2b + 2d - ac)^2 \\
  \Longleftrightarrow &&
    a^2d - abc - acd + b^2 + bc^2 - 2bd + d^2 &= 0
\end{align*}
Dann erhält man zwar denselben Ausdruck wie oben, kann sich jedoch nicht sicher
sein, dass er \emph{nur} dann verschwindet, wenn die beiden Gleichungen eine
gemeinsame Lösung besitzen: Denn in der Herleitung kamen nicht
ausschließlich~"`$\Leftrightarrow$"'-Pfeile vor. Man kann sich aber sicher
sein, dass der Ausdruck \emph{zumindest} dann verschwindet, wenn es gemeinsame
Lösungen gibt, denn jeder Schritt funktionierte zumindest in der
Richtung~"`$\Rightarrow$"'.

\emph{Bemerkung, die Neugierde wecken soll:} Man kann den gesuchten Ausdruck
auch als Determinante einer gewissen Matrix schreiben:
\[ R = \det\begin{pmatrix}
  1 & a & b & 0 \\
  0 & 1 & a & b \\
  1 & c & d & 0 \\
  0 & 1 & c & d
\end{pmatrix}\!. \]
Das hat einen tieferen Grund.
\end{loesungE}
\end{aufgabe}

Nicht verpassen: \textbf{Gauß-Vorlesung} über Muster bei Primzahlen am 28. Mai
ab 17:00 Uhr im Parktheater Göggingen, mehr Informationen auf
\url{http://xrl.us/gauss2013}.
 
\end{document} 


TODO: Beste Formulierung von Aufgabe 5a finden!
