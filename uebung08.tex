\documentclass{algblatt}
\loesungenfalse

\geometry{tmargin=2cm,bmargin=2cm,lmargin=2.7cm,rmargin=2.7cm}

%\setlength{\titleskip}{0.7em}
\setlength{\aufgabenskip}{1.4em}

\begin{document}

\vspace*{-1.5cm}
\maketitle{8}{Abgabe bis 10. Juni 2013, 17:00 Uhr}

\begin{aufgabe}{Irreduzibilitätstest}
Bestimme numerisch die Nullstellen von~$f(X) = X^4 - 12\,X^2 + 1$ bis auf
wenige Stellen nach dem Komma, und nutze diese Information um zu zeigen,
dass~$f(X)$ über den rationalen Zahlen irreduzibel ist.

\begin{loesung}
Die Nullstellen von~$f(X)$ sind näherungsweise
\begin{align*}
  x_1 &\approx -3{,}452, &
  x_2 &\approx -0{,}290, &
  x_3 &\approx 0{,}290, &
  x_4 &\approx 3{,}452.
\end{align*}
Da~$f(X)$ normiert und ganzzahlige Koeffizienten hat, können wir nun alle
Auswahlen der Nullstellen durchgehen und jeweils prüfen, ob die
elementarsymmetrischen Funktionen in diesen Nullstellen ganze Zahlen ergeben.

\emph{Linearfaktoren:} Keine der Nullstellen ist ganzzahlig, also kann kein
Linearfaktor abspalten.

\emph{Quadratische Faktoren:} Für jede zweielementige Auswahl der Nullstellen
sind stets nicht beide elementarsymmetrischen Funktionen in den Nullstellen
ganzzahlig:
\begin{align*}
  e_1(x_1,x_2) &\approx -3{,}7 &
  e_2(x_1,x_2) &\approx \phantom{+}1{,}0 \\
  e_1(x_1,x_3) &\approx -3{,}2 &
  e_2(x_1,x_3) &\approx -1{,}0 \\
  e_1(x_1,x_4) &\approx \phantom{+}0{,}0 &
  e_2(x_1,x_4) &\approx -11{,}9 \\
  e_1(x_2,x_3) &\approx \phantom{+}0{,}0 &
  e_2(x_2,x_3) &\approx -0{,}1 \\
  e_1(x_2,x_4) &\approx \phantom{+}3{,}2 &
  e_2(x_2,x_4) &\approx -1{,}0 \\
  e_1(x_3,x_4) &\approx \phantom{+}3{,}7 &
  e_2(x_3,x_4) &\approx \phantom{+}1{,}0
\end{align*}

\emph{Kubische Faktoren:} Kann es nicht geben, da die komplementären Faktoren
Linearfaktoren wären.
\end{loesung}
\end{aufgabe}

\begin{aufgabe}{Inhalt von Polynomen}
Sei~$f(X)$ ein normiertes Polynom mit rationalen Koeffizienten.
\begin{enumerate}
\item Zeige, dass der Inhalt von~$f$ genau dann ganzzahlig ist, wenn alle
Koeffizienten von~$f$ ganzzahlig sind.
\item Zeige, dass der Inhalt von~$f$ das Inverse des Leitkoeffizienten
von~$\widetilde f$ ist.
\item Zeige, dass der Inhalt von~$f$ das Inverse einer ganzen Zahl ist.
...nicht getreu...
\end{enumerate}

\begin{loesungE}
\end{loesungE}
\end{aufgabe}

\begin{aufgabe}{Kongruenzrechnungen}
\begin{enumerate}
\item Sei~$n$ eine ganze Zahl und seien~$a,a', b,b'$ ganze Zahlen mit~$a \equiv
a'$ und~$b \equiv b'$ modulo~$n$. Rechne explizit nach, dass dann auch~$a+b
\equiv a'+b'$ modulo~$n$.
\item Sei~$a$ eine ganze Zahl mit~$a \equiv 1$ modulo~$5$. Für welche
Exponenten~$k$ ist~$a^k \equiv 2$ modulo~$5$?
\item Finde zwei Inverse von~$8$ modulo~$35$.
\end{enumerate}

\begin{loesungE}
\item Gelte~$a \equiv a'$ und~$b \equiv b'$ modulo~$n$, d.\,h.~$a - a'$ und~$b
- b'$ sind jeweils durch~$n$ teilbar. Dann ist auch
\[ (a+b) - (a'+b') = (a-a') + (b-b') \]
durch~$n$ teilbar, d.\,h. es gilt~$a+b \equiv a'+b'$.

\item Für keinen einzigen Exponenten ist das erfüllt. Denn wenn~$a \equiv 1$
modulo~$5$, so sind auch~$a^2$, $a^3$ usw. jeweils kongruent zu~$1$ modulo~$5$.
Aber~$2$ ist nicht kongruent zu~$1$ modulo~$5$.

\item Ein Inverses ist~$22$, denn~$8 \cdot 22 = 176 = 5 \cdot 35 + 1 \equiv 1$
modulo~$35$. Ein anderes ist~$-13$.
\end{loesungE}
\end{aufgabe}

\begin{aufgabe}{Reduktion modulo einer Primzahl}
Sei~$f(X)$ ein normiertes Polynom mit ganzzahligen Koeffizienten. Beweise oder
widerlege: Ist~$f(X)$ modulo einer Primzahl reduzibel, so ist~$f(X)$ auch über
den rationalen Zahlen reduzibel.

\begin{loesung}
Das stimmt nicht. Etwa ist das Polynom~$f(X) = X^2 + 1$ über der Primzahl~$2$
reduzibel, denn es gilt~$f(X) = X^2 + 1 \equiv (X + 1)^2 \mod 2$ -- aber
bekanntermaßen ist~$f(X)$ nicht über den rationalen Zahlen reduzibel.

\emph{Bemerkung:} Die Umkehrung stimmt aber schon, siehe Proposition~3.23 im
Skript.
\end{loesung}
\end{aufgabe}

\begin{aufgabe}{Irreduzibilitätstest nach Leopold Kronecker}
\begin{enumerate}
\item Seien~$b_0,\ldots,b_m$ von Null verschiedene ganze Zahlen. Zeige, dass es
nur endlich viele Polynome~$g(X)$ vom Grad~$\leq m$ mit ganzzahligen Koeffizienten
gibt, sodass für alle~$i = 0,\ldots,m$ die ganze Zahl~$g(i)$ ein Teiler
von~$b_i$ ist.
\item Sei~$f(X)$ ein primitives Polynom vom Grad~$n$ mit ganzzahligen
Koeffizienten und~$f(i) \neq 0$ für alle~$0 \leq i \leq \frac{n}{2}$. Zeige,
dass es nur endlich viele Polynome~$g(X), h(X)$ mit ganzzahligen Koeffizienten
und~$f = g \cdot h$ gibt.
\item Verwende Teilaufgabe~b), um ein Verfahren anzugeben, das von einem
primitiven Polynom~$f(X)$ mit ganzzahligen Koeffizienten feststellt, ob es
über den rationalen Zahlen reduzibel oder irreduzibel ist.
\end{enumerate}

\begin{loesungE}
\item Jede der Zahlen~$b_i$ besitzt nur endlich viele Teiler, da sie nicht null
ist. Daher gibt es nur endlich viele Tupel~$(g_0,\ldots,g_m)$ mit~$g_i \mid
b_i$ für alle~$i = 0,\ldots,m$. Wegen des Satzes über die Eindeutigkeit der
Polynominterpolation gibt es für jedes dieser Tupel nur genau ein Polynom~$g(X)$ mit~$g(i)
= g_i$ für alle~$i = 0,\ldots,m$.

\item Gelte~$f(X) = g(X) \cdot h(X)$.
Nach Aufgabe~4e) von Blatt~4 sind dann
für jedes~$i$ mit~$0 \leq i \leq \frac{n}{2}$ die Zahlen~$g(i)$ und~$h(i)$
jeweils Teiler von~$b_i := f(i)$. Mindestens einer der beiden Faktoren hat
Grad~$\leq \frac{n}{2}$; daher folgt mit Teilaufgabe~a), dass es nur endlich
viele Möglichkeiten für ihn gibt. Der andere Faktor ist aus dem ersten sowieso
eindeutig bestimmt. Das zeigt zusammengenommen die Behauptung.

\item Von einem gegebenen primitiven Polynom~$f(X)$ vom Grad~$n$ mit ganzzahligen
Koeffizienten können wir zunächst prüfen, ob eine der Zahlen~$f(i)$ für~$0 \leq i
\leq \frac{n}{2}$ null ist. Wenn ja, ist~$f(X)$ sicherlich reduzibel. Wenn
nein, können wir die endlich vielen Teiler der Zahlen~$f(i)$ bestimmen, durch
Polynominterpolation jeweils ein Polynom~$g(X)$ konstruieren und prüfen, ob die
Polynomdivision von~$f(X)$ durch~$g(X)$ glatt in einem ganzzahligen Polynom
aufgeht. Wenn ja, ist~$f(X)$ reduzibel, da wir einen abspaltenden Faktor
gefunden haben. Wenn die Division so nie aufgeht, ist~$f(X)$ über den ganzen
Zahlen und wegen seiner Primitivität auch über den rationalen Zahlen
irreduzibel.
\end{loesungE}
\end{aufgabe}

\end{document}

Herausgefallen: Inv'barkeit modulo n
