\documentclass{algblatt}
\usepackage{xstring}
\IfSubStr{\jobname}{\detokenize{loesung}}{\loesungentrue}{\loesungenfalse}

\geometry{tmargin=2cm,bmargin=2cm,lmargin=2.7cm,rmargin=2.7cm}

%\setlength{\titleskip}{0.7em}
\setlength{\aufgabenskip}{1.4em}

\begin{document}

\vspace*{-1.5cm}
\maketitle{8}{Abgabe bis 10. Juni 2013, 17:00 Uhr}

\begin{aufgabe}{Numerischer Irreduzibilitätstest}
Bestimme numerisch die Nullstellen von~$f(X) = X^4 - 12\,X^2 + 1$ bis auf
wenige Stellen nach dem Komma, und nutze diese Information um zu zeigen,
dass~$f(X)$ über den rationalen Zahlen irreduzibel ist.
\end{aufgabe}

\begin{aufgabe}{Inhalt von Polynomen}
Sei~$f(X)$ ein nicht verschwindendes Polynom mit rationalen Koeffizienten. Zeige, dass der
Inhalt von~$f$\ldots
\begin{enumerate}
\item \ldots genau dann ganzzahlig ist, wenn alle
Koeffizienten von~$f$ ganzzahlig sind.
\item \ldots das Inverse des Leitkoeffizienten
von~$\widetilde f$ ist, wenn~$f$ normiert ist.
\item \ldots das Inverse einer ganzen Zahl ist,
wenn~$f$ normiert ist. Gilt auch die Umkehrung?
\end{enumerate}
\end{aufgabe}

\begin{aufgabe}{Kongruenzrechnungen}
\begin{enumerate}
\item Sei~$n$ eine ganze Zahl und seien~$a,a', b,b'$ ganze Zahlen mit~$a \equiv
a'$ und~$b \equiv b'$ modulo~$n$. Rechne explizit nach, dass dann auch~$a+b
\equiv a'+b'$ modulo~$n$.
\item Sei~$n$ eine ganze Zahl und sei~$a$ eine zu~$n$ teilerfremde ganze Zahl.
Zeige, dass für ganze Zahlen~$b, b'$ mit~$a b \equiv a b' \equiv 1$ modulo~$n$
folgt, dass~$b \equiv b'$ modulo~$n$.
\item Sei~$a$ eine ganze Zahl mit~$a \equiv 1$ modulo~$5$. Für welche
Exponenten~$k$ ist~$a^k \equiv 2$ modulo~$5$?
\item Finde zwei Inverse von~$8$ modulo~$35$.
\end{enumerate}
\end{aufgabe}

\begin{aufgabe}{Reduktion modulo einer Primzahl}
Sei~$f(X)$ ein normiertes Polynom mit ganzzahligen Koeffizienten. Beweise oder
widerlege: Ist~$f(X)$ modulo einer Primzahl reduzibel, so ist~$f(X)$ auch über
den rationalen Zahlen reduzibel.
\end{aufgabe}

\begin{aufgabe}{Irreduzibilitätstest nach Leopold Kronecker}
\begin{enumerate}
\item Seien~$b_0,\ldots,b_m$ von Null verschiedene ganze Zahlen. Zeige, dass es
nur endlich viele Polynome~$g(X)$ vom Grad~$\leq m$ mit ganzzahligen Koeffizienten
gibt, sodass für alle~$i = 0,\ldots,m$ die ganze Zahl~$g(i)$ ein Teiler
von~$b_i$ ist.
\item Sei~$f(X)$ ein primitives Polynom vom Grad~$n$ mit ganzzahligen
Koeffizienten und~$f(i) \neq 0$ für alle~$0 \leq i \leq \frac{n}{2}$. Zeige,
dass es nur endlich viele Polynome~$g(X), h(X)$ mit ganzzahligen Koeffizienten
und~$f = g \cdot h$ gibt.
\item Verwende Teilaufgabe~b), um ein Verfahren anzugeben, das von einem
primitiven Polynom~$f(X)$ mit ganzzahligen Koeffizienten feststellt, ob es
über den rationalen Zahlen reduzibel oder irreduzibel ist.
\end{enumerate}
\end{aufgabe}

\end{document}
