\documentclass{algblatt}
\loesungenfalse

\geometry{tmargin=2cm,bmargin=2cm,lmargin=2.7cm,rmargin=2.7cm}

%\setlength{\titleskip}{0.7em}
\setlength{\aufgabenskip}{1.4em}

\begin{document}

\vspace*{-1.5cm}
\maketitle{8}{Abgabe bis 10. Juni 2013, 17:00 Uhr}

\begin{aufgabe}{Numerischer Irreduzibilitätstest}
Bestimme numerisch die Nullstellen von~$f(X) = X^4 - 12\,X^2 + 1$ bis auf
wenige Stellen nach dem Komma, und nutze diese Information um zu zeigen,
dass~$f(X)$ über den rationalen Zahlen irreduzibel ist.

\begin{loesung}
Die Nullstellen von~$f(X)$ sind näherungsweise
\begin{align*}
  x_1 &\approx -3{,}452, &
  x_2 &\approx -0{,}290, &
  x_3 &\approx 0{,}290, &
  x_4 &\approx 3{,}452.
\end{align*}
Da~$f(X)$ normiert ist und ganzzahlige Koeffizienten hat, können wir nun alle
Auswahlen der Nullstellen durchgehen und jeweils prüfen, ob die
elementarsymmetrischen Funktionen in diesen Nullstellen ganze Zahlen ergeben.

\emph{Linearfaktoren:} Keine der Nullstellen ist ganzzahlig, also kann kein
Linearfaktor abspalten.

\emph{Quadratische Faktoren:} Für jede zweielementige Auswahl der Nullstellen
sind stets nicht beide elementarsymmetrischen Funktionen in den Nullstellen
ganzzahlig:
\begin{align*}
  e_1(x_1,x_2) &\approx -3{,}7 &
  e_2(x_1,x_2) &\approx \phantom{+}1{,}0 \\
  e_1(x_1,x_3) &\approx -3{,}2 &
  e_2(x_1,x_3) &\approx -1{,}0 \\
  e_1(x_1,x_4) &\approx \phantom{+}0{,}0 &
  e_2(x_1,x_4) &\approx -11{,}9 \\
  e_1(x_2,x_3) &\approx \phantom{+}0{,}0 &
  e_2(x_2,x_3) &\approx -0{,}1 \\
  e_1(x_2,x_4) &\approx \phantom{+}3{,}2 &
  e_2(x_2,x_4) &\approx -1{,}0 \\
  e_1(x_3,x_4) &\approx \phantom{+}3{,}7 &
  e_2(x_3,x_4) &\approx \phantom{+}1{,}0
\end{align*}
Tatsächlich muss man nicht alle dieser Auswahlen prüfen -- es genügen die
ersten drei. Denn die letzten drei sind einfach die Komplementärauswahlen zu
den ersten drei.

\emph{Kubische Faktoren:} Kann es nicht geben, da die komplementären Faktoren
Linearfaktoren wären.
\end{loesung}
\end{aufgabe}

\begin{aufgabe}{Inhalt von Polynomen}
Sei~$f(X)$ ein nicht verschwindendes Polynom mit rationalen Koeffizienten. Zeige, dass der
Inhalt von~$f$\ldots
\begin{enumerate}
\item \ldots genau dann ganzzahlig ist, wenn alle
Koeffizienten von~$f$ ganzzahlig sind.
\item \ldots das Inverse des Leitkoeffizienten
von~$\widetilde f$ ist, wenn~$f$ normiert ist.
\item \ldots das Inverse einer ganzen Zahl ist,
wenn~$f$ normiert ist. Gilt auch die Umkehrung?
\end{enumerate}

\begin{loesung}
\newcommand{\leit}[1]{\mathrm{Leitkoeff}(#1)}%
Es gilt die Beziehung~$\leit{\widetilde f} = c^{-1} \cdot \leit{f}$,
wobei~$\widetilde f$ ein Polynom mit ganzzahligen Koeffizienten ist, das zudem
primitiv ist. Diese Beziehung werden wir wiederholt verwenden.
\begin{enumerate}
\item "`$\Longrightarrow$"': Es gilt~$f = c \cdot \widetilde f$, also ist~$f$ als
Produkt von Polynomen aus~$\ZZ[X]$ selbst ein Polynom aus~$\ZZ[X]$.

"`$\Longleftarrow$"': Wenn~$f$ ganzzahlige Koeffizienten hat, ist~$c$ der
größte gemeinsame Teiler der Koeffizienten von~$f$ (wieso?) und daher ganzzahlig.

\item Wenn~$f$ normiert ist, gilt~$\leit{\widetilde f} = c^{-1} \cdot
\leit{f} = c^{-1} \cdot 1$, also ist~$c$ das Inverse der Zahl~$\leit{\widetilde f}$.

\item Mit Teilaufgabe~b) folgt sofort die Behauptung, denn der Leitkoeffizient
von~$\widetilde f$ ist ganzzahlig. Die Umkehrung gilt überhaupt nicht: Eines
von unzähligen Gegenbeispielen ist~$f(X) = 2\,X + 3$, dessen Inhalt~$1$ ist,
das aber nicht normiert ist. Ein anderes Gegenbeispiel ist~$f(X) = \frac{1}{3}
X$, dessen Inhalt~$\frac{1}{3}$ ist.
\end{enumerate}
\end{loesung}
\end{aufgabe}

\begin{aufgabe}{Kongruenzrechnungen}
\begin{enumerate}
\item Sei~$n$ eine ganze Zahl und seien~$a,a', b,b'$ ganze Zahlen mit~$a \equiv
a'$ und~$b \equiv b'$ modulo~$n$. Rechne explizit nach, dass dann auch~$a+b
\equiv a'+b'$ modulo~$n$.
\item Sei~$n$ eine ganze Zahl und sei~$a$ eine zu~$n$ teilerfremde ganze Zahl.
Zeige, dass für ganze Zahlen~$b, b'$ mit~$a b \equiv a b'$ modulo~$n$
folgt, dass~$b \equiv b'$ modulo~$n$.
\item Sei~$a$ eine ganze Zahl mit~$a \equiv 1$ modulo~$5$. Für welche
Exponenten~$k$ ist~$a^k \equiv 2$ modulo~$5$?
\item Finde zwei ganze Zahlen, die modulo~$35$ invers zu~$8$ sind.
\end{enumerate}

\begin{loesungE}
\item Gelte~$a \equiv a'$ und~$b \equiv b'$ modulo~$n$, d.\,h.~$a - a'$ und~$b
- b'$ sind jeweils durch~$n$ teilbar. Dann ist auch
\[ (a+b) - (a'+b') = (a-a') + (b-b') \]
durch~$n$ teilbar, d.\,h. es gilt~$a+b \equiv a'+b'$.

\emph{Bemerkung:} Etwas abstrakter formuliert hat man in dieser Aufgabe
gezeigt, dass die Addition auf dem Restklassenring~$\ZZ/(n)$ wohldefiniert ist.

\item Da~$a$ und~$n$ zueinander teilerfremd sind, gibt es eine
Bézoutdarstellung der Form~$1 = pa + qn$ für gewisse ganze Zahlen~$p$ und~$q$.
Modulo~$n$ gilt daher~$1 \equiv p a$ (d.\,h.~$p$ ist ein Inverses für~$a$
modulo~$n$). Wenn man die gegebene Kongruenz~$ab \equiv ab'$ auf beiden Seiten
mit~$p$ multipliziert, erhält man~$pab \equiv pab'$, also~$b \equiv b'$.

\item Für keinen einzigen Exponenten ist das erfüllt. Denn wenn~$a \equiv 1$
modulo~$5$, so sind auch~$a^2$, $a^3$ usw. jeweils kongruent zu~$1$ modulo~$5$.
Aber~$2$ ist nicht kongruent zu~$1$ modulo~$5$.

\emph{Variante:} Wenn~$a \equiv 1$ modulo~$5$, gibt es eine ganze Zahl~$p$
mit~$a - 1 = 5p$. Dann kann man mit dem binomischen Lehrsatz das
Produkt~$(a-1)^k$ explizit ausmultiplizieren und sehen, dass es modulo~$5$
kongruent zu~$1$ ist.

\item Ein Inverses ist~$22$, denn~$8 \cdot 22 = 176 = 5 \cdot 35 + 1 \equiv 1$
modulo~$35$. Ein anderes ist~$-13$.

\emph{Bemerkung:} Allgemein kann man ein Inverses einer ganzen Zahl~$a$
modulo~$n$ stets so berechnen (vgl. Proposition~3.20): Zunächst bestimmt man
eine Bézoutdarstellung~$d = p \cdot a + q \cdot n$ des größten gemeinsamen
Teilers von~$a$ und~$n$, $d \geq 1$. Ist~$d \neq 1$, so ist~$a$ modulo~$n$
nicht invertierbar. Andernfalls folgt~$1 \equiv p \cdot a + 0$ modulo~$n$, also
ist~$p$ ein Inverses zu~$a$ modulo~$n$. Eine mögliche Bézoutdarstellung hier
ist~$1 = (-13) \cdot 8 + 3 \cdot 35$.
\end{loesungE}
\end{aufgabe}

% XXX: Wäre besser einfach als Teil e) der vorherigen Aufgabe gewesen!
% Dann hätte das Blatt nur vier Aufgaben gehabt, und man müsste nicht diese
% kurze Aufgabe völlig überbewerten.
\begin{aufgabe}{Reduktion modulo einer Primzahl}
Sei~$f(X)$ ein normiertes Polynom mit ganzzahligen Koeffizienten. Beweise oder
widerlege: Ist~$f(X)$ modulo einer Primzahl reduzibel, so ist~$f(X)$ auch über
den rationalen Zahlen reduzibel.

\begin{loesung}
Das stimmt nicht. Etwa ist das Polynom~$f(X) = X^2 + 1$ über der Primzahl~$2$
reduzibel, denn es gilt~$f(X) = X^2 + 1 \equiv (X + 1)^2 \mod 2$ -- aber
bekanntermaßen ist~$f(X)$ nicht über den rationalen Zahlen reduzibel.

\emph{Bemerkung:} Die Umkehrung stimmt aber schon, siehe Proposition~3.23 im
Skript.
\end{loesung}
\end{aufgabe}

\begin{aufgabe}{Irreduzibilitätstest nach Leopold Kronecker}
\begin{enumerate}
\item Seien~$b_0,\ldots,b_m$ von Null verschiedene ganze Zahlen. Zeige, dass es
nur endlich viele Polynome~$g(X)$ vom Grad~$\leq m$ mit ganzzahligen Koeffizienten
gibt, sodass für alle~$i = 0,\ldots,m$ die ganze Zahl~$g(i)$ ein Teiler
von~$b_i$ ist.
\item Sei~$f(X)$ ein primitives Polynom vom Grad~$n$ mit ganzzahligen
Koeffizienten und~$f(i) \neq 0$ für alle~$0 \leq i \leq \frac{n}{2}$. Zeige,
dass es nur endlich viele Polynome~$g(X), h(X)$ mit ganzzahligen Koeffizienten
und~$f = g \cdot h$ gibt.
\item Verwende Teilaufgabe~b), um ein Verfahren anzugeben, das von einem
primitiven Polynom~$f(X)$ mit ganzzahligen Koeffizienten feststellt, ob es
über den rationalen Zahlen reduzibel oder irreduzibel ist.
\end{enumerate}

\begin{loesungE}
\item Jede der Zahlen~$b_i$ besitzt nur endlich viele Teiler, da sie nicht null
ist. Daher gibt es nur endlich viele Tupel~$(g_0,\ldots,g_m)$ mit~$g_i \mid
b_i$ für alle~$i = 0,\ldots,m$. Wegen des Satzes über die Eindeutigkeit der
Polynominterpolation gibt es für jedes dieser Tupel nur genau ein
Polynom~$g(X)$ vom Grad~$\leq m$ mit~$g(i)
= g_i$ für alle~$i = 0,\ldots,m$.

\item Gelte~$f(X) = g(X) \cdot h(X)$.
Nach Aufgabe~4e) von Blatt~4 sind dann
für jedes~$i$ mit~$0 \leq i \leq \frac{n}{2}$ die Zahlen~$g(i)$ und~$h(i)$
jeweils Teiler von~$b_i := f(i)$. Mindestens einer der beiden Faktoren hat
Grad~$\leq \frac{n}{2}$; daher folgt mit Teilaufgabe~a), dass es nur endlich
viele Möglichkeiten für ihn gibt. Der andere Faktor ist aus dem ersten sowieso
eindeutig bestimmt. Das zeigt zusammengenommen die Behauptung.

\item Von einem gegebenen primitiven Polynom~$f(X)$ vom Grad~$n$ mit ganzzahligen
Koeffizienten können wir zunächst prüfen, ob eine der Zahlen~$f(i)$ für~$0 \leq i
\leq \frac{n}{2}$ null ist. Wenn ja, ist~$f(X)$ sicherlich reduzibel. Wenn
nein, können wir die endlich vielen Teiler der Zahlen~$f(i)$ bestimmen, durch
Polynominterpolation jeweils ein Polynom~$g(X)$ konstruieren und prüfen, ob die
Polynomdivision von~$f(X)$ durch~$g(X)$ glatt in einem ganzzahligen Polynom
aufgeht. Wenn ja, ist~$f(X)$ reduzibel, da wir einen abspaltenden Faktor
gefunden haben. Wenn die Division so nie aufgeht, ist~$f(X)$ über den ganzen
Zahlen und wegen seiner Primitivität auch über den rationalen Zahlen
irreduzibel.

\emph{Bemerkung:} Das hier vorgestellte Verfahren ist in der Praxis relativ
untauglich. Besser ist es, spezialisierte Algorithmen zu verwenden; einen davon
(bei weitem nicht der beste) haben wir in der Vorlesung kennengelernt
(Auswahlen von Nullstellen betrachten). All diese besseren Verfahren verwenden
aber die reellen oder komplexen Zahlen. Der Vorteil des Verfahrens dieser
Aufgabe ist, dass es ausschließlich mit der Betrachtung rationaler Zahlen
auskommt. Das ist von einem theoretischen Standpunkt aus wünschenswert: Denn in
der Formulierung der Frage nach der Irreduzibilität eines primitiven Polynoms
mit ganzzahligen Koeffizienten treten die reellen oder komplexen Zahlen ja gar
nicht auf.
\end{loesungE}
\end{aufgabe}

\end{document}
