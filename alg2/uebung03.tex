\documentclass{algblatt}
\usepackage{xstring}
\IfSubStr{\jobname}{\detokenize{loesung}}{\loesungentrue}{\loesungenfalse}


\begin{document}

\maketitle{3}{Abgabe bis 5. November 2013, 17:00 Uhr}

\begin{aufgabe}{2+2}{Länge von Bahnen}
Sei~$G$ eine endliche Gruppe. Wirke~$G$ auf einer endlichen Menge~$X$.
\begin{enumerate}
\item Zeige oder widerlege: Die Länge einer beliebigen Bahn der Operation ist
ein Teiler der Gruppenordnung.
\item[S b)]
Sei speziell~$|G| = 91$ und~$|X| = 71$. Zeige, dass die Operation mindestens
einen Fixpunkt besitzt.
\end{enumerate}
\end{aufgabe}

\begin{aufgabe}{3+1}{Freie Wirkungen}
\begin{enumerate}
\item Zeige, dass eine Wirkung einer Gruppe~$G$ auf einer Menge~$X$
genau dann \emph{frei} ist (das bedeutet, dass alle
Standgruppen~$G_x$ triviale Gruppen sind), wenn folgende Abbildung injektiv
ist:
\[ G \times X \longrightarrow X \times X, \quad
  (g,x) \mapsto (gx, x) \]
\item Welche Länge haben Bahnen freier Wirkungen stets?
\end{enumerate}
\end{aufgabe}

\begin{aufgabe}{2+2}{Wirkung vermöge Konjugation}
\begin{enumerate}
\item Bestimme den Zentralisator von~$(1,2) \in \SSS_4$.
\item Seien~$H$ und~$H'$ zueinander konjugierte Untergruppen einer Gruppe~$G$.
Finde einen Gruppenisomorphismus~$H \to H'$.
\end{enumerate}
\end{aufgabe}

\begin{aufgabe}{1+3}{Klassifikation von Wirkungen}
\begin{enumerate}
\item Wirke~$G$ auf Mengen~$Y_1,\ldots,Y_n$. Wie kann man unter Einbeziehung
der gegebenen Operationen die disjunkt-gemachte
Vereinigung~$Y_1 \amalg \cdots \amalg Y_n$ mit der Struktur einer~$G$-Wirkung versehen?
\item
Wirke eine endliche Gruppe~$G$ auf einer endlichen Menge~$X$. Zeige, dass
endliche Untergruppen~$H_1,\ldots,H_n \subseteq G$ zusammen mit einem Isomorphismus
von~$G$-Wirkungen
\[ G/H_1 \amalg \cdots \amalg G/H_n \longrightarrow G \]
existieren. Dabei wirkt~$G$ auf den Summanden~$G/H_i$ jeweils durch
Linkstranslation.
\end{enumerate}
\end{aufgabe}

\begin{aufgabe}{4}{Schnitt von Normalteilern}
Sei~$(N_i)_{i \in I}$ eine Familie normaler Untergruppen
einer Gruppe~$G$. Zeige, dass~$N := \cap_{i \in I} N_i$ wieder ein Normalteiler von~$G$
ist.
\end{aufgabe}

\end{document}
