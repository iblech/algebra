\documentclass{algblatt}
\usepackage{xstring}
\IfSubStr{\jobname}{\detokenize{loesung}}{\loesungentrue}{\loesungenfalse}


\begin{document}

\maketitle{5}{Abgabe bis 19. November 2013, 17:00 Uhr}

\begin{aufgabeE}{2+2}{Beispiele von Sylowschen Untergruppen}
\item
Bestimme alle Sylowschen Untergruppen der alternierenden Gruppe~$\AAA_4$.
\item
Sei~$G$ eine endliche abelsche Gruppe. Sei~$H \subseteq G$ die endliche
Teilmenge all derjenigen Elemente von~$G$, deren Ordnung eine~$p$-Potenz ist.
Zeige, dass~$H$ eine Untergruppe von~$G$ und sogar die einzige
Sylowsche~$p$-Untergruppe ist.
\end{aufgabeE}

\begin{aufgabeE}{1+3}{Grundlagen zu Sylowschen Untergruppen}
\item Seien~$G$ eine endliche Gruppe und~$H \subseteq K \subseteq G$ endliche
Untergruppen. Sei ferner~$H$ eine Sylowsche~$p$-Untergruppe zu~$G$. Zeige,
dass~$H$ dann auch eine Sylowsche~$p$-Untergruppe zu~$K$ ist.
\item Sei~$G$ eine endliche Untergruppe und~$H \subseteq G$ eine Untergruppe,
deren Ordnung eine~$p$-Potenz ist. Zeige: Die Untergruppe~$H$ ist genau dann
eine Sylowsche~$p$-Untergruppe zu~$G$, wenn~$H$ maximal unter
allen~$p$-Untergruppen von~$G$ ist.
\end{aufgabeE}

\begin{aufgabeE}{3+3}{Existenz nichttrivialer Normalteiler}
\item[S a)] Zeige, dass jede Gruppe der Ordnung~$30$ einen
Sylowschen Normalteiler besitzt.
\item[S b)] Zeige, dass jede Gruppe der Ordnung~$56$ einen
Sylowschen Normalteiler besitzt.
\end{aufgabeE}

\begin{aufgabeE}{3+3}{Weitere Anwendungen der Sylowschen Sätze}
\item[S a)] Zeige, dass jede endliche Gruppe der Ordnung~$36$ nicht einfach ist.
\item[S b)] Seien~$p$ und~$q$ Primzahlen mit~$p < q$ und~$p \nmid q-1$. Zeige,
dass jede Gruppe der Ordnung~$pq$ zyklisch ist.
\end{aufgabeE}

\end{document}
