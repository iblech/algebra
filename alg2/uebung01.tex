\documentclass{algblatt}
\loesungenfalse


\begin{document}

\maketitle{1}{Abgabe bis 22. Oktober 2013, 17:00 Uhr}

\begin{aufgabe}{Abstrakte Beispiele für Gruppenhomomorphismen}
\begin{enumerate}
\item Sei~$n$ eine ganze Zahl und~$G$ eine Gruppe. Ist dann die
Potenzabbildung~$G \to G,\ g \mapsto g^n$ stets ein Gruppenhomomorphismus?
\item Seien~$G$ und~$H$ Gruppen. Finde einen Gruppenisomorphismus~$G \times H \to H \times G$.
\item Seien~$x$ und~$y$ Elemente einer Gruppe~$G$. Finde einen
Gruppenisomorphismus~$\phi : G \to G$ mit~$\phi(xy) = yx$.
\end{enumerate}
\end{aufgabe}

\begin{aufgabeE}{Gruppen mit zwei Elementen}
\item Seien~$G$ und~$H$ zweielementige Gruppen. Zeige, dass~$G$ und~$H$ auf
genau eine Art und Weise isomorph sind.
\item Zeige: $\Aut(\CCC_4) \cong \CCC_2$.
\end{aufgabeE}

\begin{aufgabe}{Nicht-isomorphe Gruppen gleicher Ordnung}
Zeige, dass es zwar eine Bijektion, nicht aber einen Gruppenisomorphismus
zwischen~$\CCC_4$ und~$\CCC_2 \times \CCC_2$ gibt.

\emph{Bemerkung.} Mit dem Hauptsatz über endlich präsentierte abelsche Gruppen
werden wir dieses Beispiel später konzeptioneller verstehen.
\end{aufgabe}

\begin{aufgabe}{Freies Produkt von Gruppen}
\begin{enumerate}
\item Das Element~$1\,1'\,(-1)\,(-1)' \in \ZZ * \ZZ$ ist nicht offensichtlich
gleich dem Einselement. Zeige, dass dieser erste Eindruck korrekt ist.
\item Seien~$G$ und~$H$ Gruppen. Seien~$g \in G$, $h \in H$ jeweils nicht das
jeweilige Einselement. Zeige, dass die Elemente
$1, gh, ghgh, ghghgh, \ldots$
paarweise verschiedene Elemente von~$G * H$ sind.
\item Seien~$G$ und~$H$ Gruppen, die beide ein vom Einselement verschiedenes
Element besitzen. Zeige, dass~$G * H$ nicht abelsch ist.
\item Sei~$G$ eine Gruppe. Gib einen kanonischen Gruppenisomorphismus~$G * 1 \to G$ an,
wobei~$1$ die triviale Gruppe bezeichnet.

\tiny
\item \emph{Für Teilnehmer des Pizzaseminars:} Zeige, dass das freie
Produkt das Koprodukt in der Kategorie der Gruppen ist.
\end{enumerate}
\end{aufgabe}

\end{document}
