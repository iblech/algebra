\documentclass{algblatt}
\usepackage{color}

\newenvironment{indentblock}{%
  \list{}{\leftmargin\leftmargin}%
  \item\relax
}{%
  \endlist
}

\definecolor{grey}{rgb}{0.9,0.9,0.9}
\newcommand{\hint}[1]{\rotatebox{180}{\vbox{\textcolor{grey}{#1}}}}

\begin{document}

\begin{center}\Large \textsf{\textbf{Hinweise zu den Übungsaufgaben in Algebra II}}\end{center}
\vspace{1em}

\section*{Übungsblatt 3}

\paragraph{Aufgabe 1.} Teilaufgabe~a) hat etwas mit Standgruppen zu tun. Für
Teilaufgabe~b) ist interessant, was Fixpunkte mit Bahnen zu tun haben. (Was
sind Fixpunkte denn überhaupt, und was sind Bahnen?) Welche Gleichung der
Vorlesung ist also vermutlich anwendbar?

\paragraph{Aufgabe 4.} Ein beliebiges Element der disjunkt-gemachten
Vereinigung $Y_1 \amalg \cdots \amalg Y_n$ ist ein Paar~$(i,y)$, wobei~$i \in
\{ 1,\ldots,n \}$ ein Index und~$y$ ein Element der entsprechenden Menge~$Y_i$ ist.
Ein \emph{Isomorphismus von~$G$-Wirkungen} ist per Definition eine
bijektive~$G$-äquivariante Abbildung. Für Teilaufgabe~b) ist es hilfreich,~$X$
in Bahnen zu zerlegen.

\paragraph{Aufgabe 5.}
Diese Aufgabe benötigt aber nur die Definition von
Normalteilern und das Verständnis der mengentheoretischen Schreibweise: Die
Menge~$N$ besteht aus all den Elementen von~$G$, welche in allen~$N_i$ liegen. Über die
Größe von~$I$ kann nichts vorausgesetzt werden. Wer mag, kann aber zuerst den
Fall des Schnitts zweier Normalteiler behandeln; der allgemeine Fall verläuft
ähnlich.


\section*{Übungsblatt 4}

\paragraph{Aufgabe 1.} Ein \emph{kleinster Normalteiler, welcher~$H$ umfasst,}
ist per Definition ein Normalteiler~$N$ in~$G$, welcher~$H$ umfasst und welcher
folgende Eigenschaft hat:
Für jeden beliebigen Normalteiler~$N'$ in~$G$, welcher~$H$ umfasst, gilt~$N
\subseteq N'$.

\paragraph{Aufgabe 2.} In beiden Teilaufgaben geht es nicht um
Umkehrfunktionen, sondern um Urbildmengen.

\paragraph{Aufgabe 3.} Die Gruppe~$\GL_n(\RR)$ ist die Menge der
invertierbaren~$(n \times n)$-Matrizen, mit der Matrixmultiplikation als
Gruppenverknüpfung. Die Untergruppe~$\OOO_n(\RR)$ ist die Teilmenge der
orthogonalen Matrizen. Eine Matrix~$A$ heißt genau dann
\emph{orthogonal}, wenn das Produkt~$A^t A$ die
Einheitsmatrix ist. Orthogonale Matrizen haben als Determinante stets~$\pm 1$.
Die Untergruppe~$\SO_n(\RR)$ ist die Teilmenge solcher
orthogonalen Matrizen, deren Determinante~$+1$ ist. Für die Determinante gilt
die Rechenregel~$\det(AB) = \det(A) \cdot \det(B)$.

Bei Teilaufgabe~b) muss man sich zunächst überlegen, ob man~$\SO_n(\RR)$
auf~$\CCC_2$ oder umgekehrt wirken lassen möchte (nur eine Variante
funktioniert), und wie diese Wirkung explizit aussehen soll. Wie bei
Teilaufgabe~c) die Gruppe~$\SO_3(\RR)$ auf~$\RR^3$ wirkt, ist im Skript
angegeben (Beispiel 6.76).

Im Staatsexamen ist das halbdirekte Produkt immer wieder wichtig, um die öfter
vorkommenden Aufgaben der Art \emph{Geben Sie eine nicht-abelsche Gruppe der
Ordnung~2012 an.} zu lösen.

\emph{Für Teilnehmer des Pizzaseminars:} Findet ihr eine kategorielle
Beschreibung des halbdirekten Produkts? (So, wie man das direkte Produkt auch
als terminales Objekt in der Kategorie der Möchtegern-Produkte beschreiben
kann.)

\hint{%
Die Diagonalmatrix, die oben links eine~$-1$ stehen und deren restliche
Diagonaleinträge mit~$+1$ besetzt sind, spielt bei Teilaufgabe~b) eine Rolle.
Wer mich anschreibt, bekommt weitere Tipps.}

\paragraph{Aufgabe 4.} Eine endliche Gruppe heißt genau dann \emph{$p$-Gruppe},
wenn die Anzahl ihrer Elemente eine~$p$-Potenz ist.

\hint{%
Eine nichttriviale~$p$-Gruppe besitzt stets ein Element der Ordnung~$p$ in
ihrem Zentrum. Das Kriterium aus~a) ist für~b) nützlich.}

\paragraph{Aufgabe 5.} Ein \emph{größter endlicher auflösbarer Normalteiler}
ist per Definition ein Normalteiler~$N$ in~$G$, welcher selbst endlich und auflösbar
ist und folgende Eigenschaft hat: Für jeden beliebigen endlichen auflösbaren
Normalteiler~$N'$ in~$G$ gilt~$N' \subseteq N$.

\hint{Für~b) ist~a) nützlich.}


\section*{Übungsblatt 4}

Auf dem gesamten Übungsblatt bezeichnet~"`$p$"' stets eine Primzahl.

\paragraph{Aufgabe 1.} Konventionsgemäß ist die Zahl~$1$ eine~$p$-Potenz. (Wieso
ist das sinnvoll und für Teilaufgabe~b) wichtig?)

\paragraph{Aufgabe 2.} Eine \emph{$p$-Untergruppe von~$G$} ist per Definition
eine Untergruppe von~$G$, deren Ordnung eine~$p$-Potenz ist.
Eine Untergruppe~$H$ heißt per Definition genau dann
\emph{maximal unter allen~$p$-Untergruppen von~$G$}, wenn sie selbst
eine~$p$-Untergruppe von~$G$ ist und außerdem folgende Eigenschaft hat: Ist~$K
\subseteq G$ eine beliebige~$p$-Untergruppe mit~$H \subseteq K$, so gilt
schon~$H = K$.

Eine \emph{maximale} $p$-Untergruppe ist also etwas anderes als eine
\emph{größte} $p$-Untergruppe!

\hint{%
Ein Beispiel zu einem ganz anderem Thema soll den Unterschied verdeutlichen:
Unter den Mengen
\[ \emptyset, \quad \{a\}, \quad \{a,b\}, \quad \{a,b,c\}, \quad \{d\}, \quad \{d,e\}, \quad
\{d,f\} \]
gibt es keine größte, aber drei maximale: Nämlich~$\{a,b,c\}$, $\{d,e\}$
und~$\{d,f\}$. Ferner gibt es eine kleinste (nämlich~$\emptyset$). Diese ist
auch minimal. Ergänzt man noch die Menge~$\{a,b,c,d,e,f\}$, so ändert sich die
Situation: Diese neue Menge ist jetzt die einzige Menge, die maximal ist.
Außerdem ist sie die größte.
Für eine der Richtungen der Behauptung der Aufgabe hilft der erste Sylowsche Satz.}

\paragraph{Aufgabe 3.} Captain Obvious bittet mich, folgenden Tipp zu
verbreiten: Die Sylowschen Sätze könnten helfen.

In Kürze werde ich hier ein vollständiges Beispiel für eine andere Gruppenordnung
online stellen. Die Aufgaben~3 und~4 sind typische Staatsexamensaufgaben und
nicht ganz leicht, wenn man das Standardvorgehen nicht kennt. Auf Anfrage gebe
ich gerne weitere Tipps.

\hint{%
Fallunterscheidung nach der Anzahl gewisser Sylowscher Untergruppen treffen.
Übersicht über mögliche Elementordnungen anlegen.}

\paragraph{Aufgabe 4.}
Die zweite Voraussetzung an die beiden Primzahlen ist, dass~$p$ kein Teiler
von~$q-1$ ist.

\hint{%
Für Teilaufgabe~a) hilft es vielleicht, die Abbildung
\[ G \longrightarrow \Aut(M), \quad g \longmapsto \mathrm{conj}_g \]
zu betrachten. Dabei bezeichnet~$M$ die Menge der Sylowschen~$3$-Untergruppen
der gegebenen Gruppe~$G$ und~$\Aut(M)$ die Menge der Bijektionen~$M \to M$. Die
Bijektion~$\mathrm{conj}_g$ schickt eine Sylowsche~$3$-Untergruppe~$H$ auf~$g H
g^{-1}$.
}

\end{document}
