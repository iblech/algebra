\documentclass{algblatt}
\usepackage{color}

\newenvironment{indentblock}{%
  \list{}{\leftmargin\leftmargin}%
  \item\relax
}{%
  \endlist
}

\definecolor{grey}{rgb}{0.9,0.9,0.9}
\newcommand{\hint}[1]{\rotatebox{180}{\vbox{\textcolor{grey}{#1}}}}

\begin{document}

\begin{center}\Large \textsf{\textbf{Hinweise zu den Übungsaufgaben in Algebra II}}\end{center}
\vspace{1em}

\section*{Übungsblatt 3}

\paragraph{Aufgabe 1.} Teilaufgabe~a) hat etwas mit Standgruppen zu tun. Für
Teilaufgabe~b) ist interessant, was Fixpunkte mit Bahnen zu tun haben. (Was
sind Fixpunkte denn überhaupt, und was sind Bahnen?) Welche Gleichung der
Vorlesung ist also vermutlich anwendbar?

\paragraph{Aufgabe 4.} Ein beliebiges Element der disjunkt-gemachten
Vereinigung $Y_1 \amalg \cdots \amalg Y_n$ ist ein Paar~$(i,y)$, wobei~$i \in
\{ 1,\ldots,n \}$ ein Index und~$y$ ein Element der entsprechenden Menge~$Y_i$ ist.
Ein \emph{Isomorphismus von~$G$-Wirkungen} ist per Definition eine
bijektive~$G$-äquivariante Abbildung. Für Teilaufgabe~b) ist es hilfreich,~$X$
in Bahnen zu zerlegen.

\paragraph{Aufgabe 5.}
Diese Aufgabe benötigt aber nur die Definition von
Normalteilern und das Verständnis der mengentheoretischen Schreibweise: Die
Menge~$N$ besteht aus all den Elementen von~$G$, welche in allen~$N_i$ liegen. Über die
Größe von~$I$ kann nichts vorausgesetzt werden. Wer mag, kann aber zuerst den
Fall des Schnitts zweier Normalteiler behandeln; der allgemeine Fall verläuft
ähnlich.


\section*{Übungsblatt 4}

\paragraph{Aufgabe 1.} Ein \emph{kleinster Normalteiler, welcher~$H$ umfasst,}
ist per Definition ein Normalteiler~$N$ in~$G$, welcher~$H$ umfasst und welcher
folgende Eigenschaft hat:
Für jeden beliebigen Normalteiler~$N'$ in~$G$, welcher~$H$ umfasst, gilt~$N
\subseteq N'$.

\paragraph{Aufgabe 2.} In beiden Teilaufgaben geht es nicht um
Umkehrfunktionen, sondern um Urbildmengen.

\paragraph{Aufgabe 3.} Die Gruppe~$\GL_n(\RR)$ ist die Menge der
invertierbaren~$(n \times n)$-Matrizen, mit der Matrixmultiplikation als
Gruppenverknüpfung. Die Untergruppe~$\OOO_n(\RR)$ ist die Teilmenge der
orthogonalen Matrizen. Eine Matrix~$A$ heißt genau dann
\emph{orthogonal}, wenn das Produkt~$A^t A$ die
Einheitsmatrix ist. Orthogonale Matrizen haben als Determinante stets~$\pm 1$.
Die Untergruppe~$\SO_n(\RR)$ ist die Teilmenge solcher
orthogonalen Matrizen, deren Determinante~$+1$ ist. Für die Determinante gilt
die Rechenregel~$\det(AB) = \det(A) \cdot \det(B)$.

Bei Teilaufgabe~b) muss man sich zunächst überlegen, ob man~$\SO_n(\RR)$
auf~$\CCC_2$ oder umgekehrt wirken lassen möchte (nur eine Variante
funktioniert), und wie diese Wirkung explizit aussehen soll. Wie bei
Teilaufgabe~c) die Gruppe~$\SO_3(\RR)$ auf~$\RR^3$ wirkt, ist im Skript
angegeben (Beispiel 6.76).

Im Staatsexamen ist das halbdirekte Produkt immer wieder wichtig, um die öfter
vorkommenden Aufgaben der Art \emph{Geben Sie eine nicht-abelsche Gruppe der
Ordnung~2012 an.} zu lösen.

\emph{Für Teilnehmer des Pizzaseminars:} Findet ihr eine kategorielle
Beschreibung des halbdirekten Produkts? (So, wie man das direkte Produkt auch
als terminales Objekt in der Kategorie der Möchtegern-Produkte beschreiben
kann.)

\hint{%
Die Diagonalmatrix, die oben links eine~$-1$ stehen und deren restliche
Diagonaleinträge mit~$+1$ besetzt sind, spielt bei Teilaufgabe~b) eine Rolle.
Wer mich anschreibt, bekommt weitere Tipps.}

\paragraph{Aufgabe 4.} Eine endliche Gruppe heißt genau dann \emph{$p$-Gruppe},
wenn die Anzahl ihrer Elemente eine~$p$-Potenz ist.

\hint{%
Eine nichttriviale~$p$-Gruppe besitzt stets ein Element der Ordnung~$p$ in
ihrem Zentrum. Das Kriterium aus~a) ist für~b) nützlich.}

\paragraph{Aufgabe 5.} Ein \emph{größter endlicher auflösbarer Normalteiler}
ist per Definition ein Normalteiler~$N$ in~$G$, welcher selbst endlich und auflösbar
ist und folgende Eigenschaft hat: Für jeden beliebigen endlichen auflösbaren
Normalteiler~$N'$ in~$G$ gilt~$N' \subseteq N$.

\hint{Für~b) ist~a) nützlich.}


\section*{Übungsblatt 4}

Auf dem gesamten Übungsblatt bezeichnet~"`$p$"' stets eine Primzahl.

\paragraph{Aufgabe 1.} Konventionsgemäß ist die Zahl~$1$ eine~$p$-Potenz. (Wieso
ist das sinnvoll und für Teilaufgabe~b) wichtig?)

\paragraph{Aufgabe 2.} Eine \emph{$p$-Untergruppe von~$G$} ist per Definition
eine Untergruppe von~$G$, deren Ordnung eine~$p$-Potenz ist.
Eine Untergruppe~$H$ heißt per Definition genau dann
\emph{maximal unter allen~$p$-Untergruppen von~$G$}, wenn sie selbst
eine~$p$-Untergruppe von~$G$ ist und außerdem folgende Eigenschaft hat: Ist~$K
\subseteq G$ eine beliebige~$p$-Untergruppe mit~$H \subseteq K$, so gilt
schon~$H = K$.

Eine \emph{maximale} $p$-Untergruppe ist also etwas anderes als eine
\emph{größte} $p$-Untergruppe!

\hint{%
Ein Beispiel zu einem ganz anderem Thema soll den Unterschied verdeutlichen:
Unter den Mengen
\[ \emptyset, \quad \{a\}, \quad \{a,b\}, \quad \{a,b,c\}, \quad \{d\}, \quad \{d,e\}, \quad
\{d,f\} \]
gibt es keine größte, aber drei maximale: Nämlich~$\{a,b,c\}$, $\{d,e\}$
und~$\{d,f\}$. Ferner gibt es eine kleinste (nämlich~$\emptyset$). Diese ist
auch minimal. Ergänzt man noch die Menge~$\{a,b,c,d,e,f\}$, so ändert sich die
Situation: Diese neue Menge ist jetzt die einzige Menge, die maximal ist.
Außerdem ist sie die größte.
Für eine der Richtungen der Behauptung der Aufgabe hilft der erste Sylowsche Satz.}

\paragraph{Aufgabe 3.} Captain Obvious bittet mich, folgenden Tipp zu
verbreiten: Die Sylowschen Sätze könnten helfen.

An dieser Stelle hatte ich ein vollständiges Schema versprochen, jedoch muss
das bis nach der Besprechung warten, da ein solches zu viel vorwegnehmen würde.
Auf Anfrage gebe ich aber trotzdem gerne weitere Tipps.

\hint{%
Ein Beispiel zur Überlappungsfrage: Eine Untergruppe mit~$2^2 \cdot 3^5$
Elementen kann nur im Identitätselement mit einer Untergruppe von~$7^2 \cdot
11^3$ überlappen (wieso?). Eine Untergruppe mit~$2^2 \cdot 3^5$ Elementen kann
mit einer Untergruppe von~$3^2 \cdot 11^3$ Elementen in höchstens~$3^2$ Elementen
überlappen (wieso?).}

\hint{%
Leider bleiben bei anderen Gruppenordnungen meistens mehrere Möglichkeiten für
die Anzahl der Sylowschen~$p$-Untergruppen übrig. In diesen Fällen hilft es
manchmal, für den hypothetischen Fall, dass alle~$n_p > 1$ sind, eine Übersicht
über die Elemente der Gruppe anzulegen: Stets gibt es das neutrale Element.
Ferner gibt es für jede Sylowsche Untergruppe jeweils entsprechend viele
weitere Elemente. Das Identitätselement haben aber all diese Untergruppen
gemeinsam und darf daher nicht mehrfach gezählt werden. Außerdem können sich
Sylowsche Untergruppen zur selben Primzahl nichttrivial überlappen. Sylowsche
Untergruppen zu verschiedenen Primzahlen haben aber stets nur das
Identitätselement gemeinsam (wieso?). Mit diesen Überlegungen kann man versuchen, einen
Widerspruch herzuleiten: Die Elementübersicht muss zeigen, dass es mehr
Elemente geben müsste, als faktisch in der Gruppe vorhanden sind.}

\hint{%
Hier ein Beispiel. Sei~$G$ eine Gruppe mit~$|G| = 84 = 2^2 \cdot 3 \cdot
7$ Elementen. Dann muss die Anzahl~$n_7$ der Sylowschen~$7$-Untergruppen ein
Teiler von~$2^2 \cdot 3$ und modulo~$7$ kongruent zu~$1$ sein. An positiven
Teilern gibt es nur~$1, 2, 3, 4, 12$. Daher muss~$n_7 = 1$ sein. Es gibt also
genau eine Sylowsche~$7$-Untergruppe, und diese muss daher ein Normalteiler
sein.}

\paragraph{Aufgabe 4.}
Die zweite Voraussetzung an die beiden Primzahlen ist, dass~$p$ kein Teiler
von~$q-1$ ist.

\hint{%
Für Teilaufgabe~a) hilft es vielleicht, die Abbildung
\[ G \longrightarrow \Aut(M), \quad g \longmapsto \mathrm{conj}_g \]
zu betrachten. Dabei bezeichnet~$M$ die Menge der Sylowschen~$3$-Untergruppen
der gegebenen Gruppe~$G$ und~$\Aut(M)$ die Menge der Bijektionen~$M \to M$. Die
Bijektion~$\mathrm{conj}_g$ schickt eine Sylowsche~$3$-Untergruppe~$H$ auf~$g H
g^{-1}$.
}


\section*{Übungsblatt 5}

\paragraph{Aufgabe 1.} Ein~$i$-Minor ist die Determinante einer (nicht
notwendigerweise zusammenhängenden)~$(i \times i)$-Untermatrix. Etwa sind
die~$2$-Minoren der Matrix
\[ \begin{pmatrix}
  1 & 2 & 3 \\
  4 & 5 & 6
\end{pmatrix} \]
die Zahlen~$1\cdot5-4\cdot2$, $1\cdot6-4\cdot3$ und~$2\cdot6-5\cdot3$. Je nach
Konvention gehören die Negativen dieser Zahlen auch noch zu den~$2$-Minoren;
für welche Konvention man sich entscheidet, spielt bei dieser Aufgabe aber
keine Rolle, da es sowieso nur um den größten gemeinsamen Teiler
der~$i$-Minoren geht.

\hint{%
Unter den Transformationen der Vorlesung, die man benötigt, um eine Matrix in
smithsche Normalform zu überführen, ändern sich zwar die~$i$-Minoren, nicht
aber der größte gemeinsame Teiler aller~$i$-Minoren. Dieses Faktum ist ggf. zu
beweisen. Ein eleganter Beweis ist mit Techniken des äußeren Kalküls (siehe
etwa die jetzige~LA-I-Vorlesung) möglich, andere Beweisansätze gibt es aber
sicher auch.}

Für Teilaufgabe~b) kann man das Verfahren aus der Vorlesung (mit Zeilen- und
Spaltentransformationen) oder Teilaufgabe~a) verwenden.

\paragraph{Aufgabe 2.} Bei beiden Teilaufgaben ist also eine Liste von
abelschen Gruppen der jeweiligen Ordnung gesucht, sodass jede abelsche Gruppe
dieser Ordnung isomorph zu einer der Gruppen auf der Liste ist und sodass keine
zwei verschiedenen Gruppen der Liste zueinander isomorph sind. Ohne
Unterstützung mit Vorlesungswissen ist die Aufgabe schwer.

\paragraph{Aufgabe 3.} Die Notation in der Angabe ist etwas seltsam, hat aber
einen guten Grund. Wie dem Text zu entnehmen ist, gilt
\[ A[p^\infty] := \{ x \in A \,|\, \text{$\ord(x)$ ist eine~$p$-Potenz} \}
\subseteq A. \]
Bei der Besprechung von Blatt~5 haben wir gesehen, wie man diese Menge auch
geringfügig einfacher beschreiben kann. Für Teilaufgabe~b) ist ein geeigneter
Isomorphismus
\[ A[p_1^\infty] \times \cdots \times A[p_r^\infty] \longrightarrow A \]
zu finden (anzugeben). Auch muss nachgerechnet werden, dass die gefundene
Abbildung tatsächlich ein Gruppenhomomorphismus ist und bijektiv ist.

\paragraph{Aufgabe 4.} Der Ring~$\ZZ_{(p)}$ ist nicht zu verwechseln mit dem
Restklassenring~$\ZZ/(p)$. Bitte rechnet nicht alle Ringaxiome nach, sondern
nur die Unterringaxiome: Die neutralen Elemente bezüglich Addition und
Multiplikation müssen enthalten sein, die Summe und das Produkt zweier Elemente
muss wieder enthalten sein und Negative von Elementen müssen wieder enthalten sein.

\paragraph{Aufgabe 5.} Es gilt $\ZZ[\zeta] = \{ a_0 + a_1 \zeta + \cdots +
a_{n-1} \zeta^{n-1} \,|\, a_0,\ldots,a_{n-1} \in \ZZ \}$, dieser Umstand muss
nicht nachgewiesen werden.

\hint{Zu Teilaufgabe~b): Die Techniken des üblichen Beweises, dass~$\O_\QQ =
\ZZ$, lassen sich auf diesen Fall übertragen. Ein genauerer Hinweis wird noch
folgen.}


\section*{Übungsblatt 6}

\paragraph{Aufgabe 1.}
Teilaufgabe~a) lautet ausformuliert wie folgt:
Sei~$\varphi : R \to S$ ein Homomorphismus von Ringen. Sei~$\bb \subseteq
S$ ein Ideal in~$S$. Zeige, dass~$\phi^{-1}(\bb) = \{ x \in R \,|\, \phi(x) \in
\bb \} \subseteq R$ ein Ideal in~$R$ ist.
Die Behauptung in Teilaufgabe~b) (welche falsch ist) wäre, dass für ein
Ideal~$\aa \subseteq R$ die Menge~$\phi(\aa) = \{ \phi(x) \,|\, x \in \aa \}
\subseteq S$ ein Ideal von~$S$ ist.

\hint{%
Für Teilaufgabe~b) genügt ein Gegenbeispiel.}

\paragraph{Aufgabe 2.}
Falls ihr euch wundert, welches Ideal von~$\ZZ$ nicht endlich erzeugt ist: In
klassischer Logik gibt es kein solches. (Bonusaufgabe: Beweise das.)

\hint{%
Bitte überseht bei Teilaufgabe~c) kein Ideal. Es sind insgesamt zwei.}

\hint{%
Jedes endlich erzeugte Ideal von~$\ZZ$ ist sogar ein
Hauptideal, kann also von einem einzigen Element erzeugt werden. Ein explizites
Beispiel: Es gilt~$(12,15) = \{ 12a + 15b \,|\, a,b \in \ZZ \} = (3)$.}

\paragraph{Aufgabe 3.} Die Lösung zu Teilaufgabe~c) lässt
sich einfacher aufschreiben, wenn man folgende Charakterisierung verwendet
(welche nicht bewiesen werden muss): Ein Ring~$R$ ist genau dann der Nullring,
wenn~$1 = 0 \in R$.

\hint{%
Ein Beispiel für Teilaufgabe~b): Für~$n = 4$
gilt~$\sqrt{(0)} = (2) \subseteq \ZZ/(4)$.}

\paragraph{Aufgabe 4.} Die Eindeutigkeit des Ringhomomorphismus muss nicht
bewiesen werden. Achtet aber darauf, den Homomorphismus explizit genug anzugeben.

\paragraph{Aufgabe 5.} Bonusfrage: Wie kann man sich~$S \times T$ geometrisch
vorstellen, wenn man geometrische Vorstellungen von~$S$ und~$T$ kennt?

\hint{%
Für die Richtung a) $\Rightarrow$ b) kann man~$S = (e) \subseteq R$ setzen. Mit
den Operationen von~$R$ wird das zu einem Ring, allerdings mit einem anderen
Einselement.}


\section*{Übungsblatt 8}

\paragraph{Aufgabe 1.} "`$f = g$ in~$R[s_i^{-1}]$"' bedeutet, dass die
Brüche~$f/1$ und~$g/1$ als Elemente von~$R[s_i^{-1}]$ gleich sind. Was das
wiederum bedeutet, steht bei der Definition der Lokalisierung im Skript. Bei
Teilaufgabe~b) ist mit "`dem Bild von~$f$ in~$R[s_i^{-1}]$"' das Element~$f/1
\in R[s_i^{-1}]$ gemeint.

\hint{%
Teilaufgabe~b) kann man so anpacken, indem man erstmal ausschreibt, was die
Voraussetzungen sind: Lokal sind Inverse geben, die haben eine bestimmte Form.
Die Inversen sind wirklich Inverse (erfüllen also eine entsprechende Gleichung
die auf~"`$=1$"' endet), das nach kann man Definition in einer Gleichung
über~$R$ umwandeln. Dann mit~"`o.\,B.\,d.\,A."''s etwas Ordnung in den
Index-Dschungel bringen und den~"`$1^N$"'-Trick der Vorlesung verwenden.
Alternativ kann man auch ein bestimmtes Lemma der Vorlesung zu Hilfe nehmen,
dann tauscht man ein paar Rechnungen gegen ein paar allgemeine Überlegungen
ein.
}

\paragraph{Aufgabe 2.} Wenn euch die Definition des gerichteten Limes
im Skript zu ungenau ist, hier eine ausführlichere Definition: Sei ein
gerichtetes System~$(R_i)_{i \in I}$ von Ringen gegeben. Dieses umfasst also
eine bestimmte gerichtete Menge~$I$, für jeden Index~$i \in I$ jeweils einen
Ring~$R_i$ und in der Notation unterdrückte Ringhomomorphismen~$\phi_{ij} : R_i
\to R_j$ für jedes Paar~$(i,j)$ mit~$i \preceq j$. Diese Ringhomomorphismen
müssen für~$i \preceq j \preceq k$ die Gleichung
\[ \phi_{jk} \circ \phi_{ij} = \phi_{ik} : R_i \to R_k \]
erfüllen. Als Menge ist dann der gerichtete Limes~$R :=
\varinjlim_{i \in I} R_i$ durch
\[ R := \Bigl(\coprod_{i \in I} R_i\Bigr)/{\sim} \]
gegeben. Ein beliebiges Element von~$R$ hat also die Form
\[ [\langle i, x\rangle], \]
wobei~$i$ ein Index aus~$I$ und~$x$ ein Element aus dem entsprechenden
Ring~$R_i$ ist. Die Äquivalenzrelation ist durch die Forderung
\[ \langle i, x \rangle \sim \langle j, y \rangle
  \quad:\Longleftrightarrow\quad
  \exists k \in I, i \preceq k, j \preceq k{:}\ \phi_{ik}(x) = \phi_{jk}(y) \]
festgelegt. Ein Element von~$R$ wird also repräsentiert durch ein Element
aus einem der~$R_i$, wobei zwei solche Elemente genau dann als äquivalent
zählen, wenn ihr Bild in einem Ring~$R_k$ mit~$i,j \preceq k$ übereinstimmt.
Die~$\phi$'s stammen aus dem Datum des gerichteten Systems, von dem man den
Limes nimmt.

Die Addition ist wie folgt definiert: Seien~$[\langle i, x \rangle], [\langle
j, y\rangle]$ Elemente von~$R$. Da~$I$ gerichtet ist, gibt es eine gemeinsame
obere Schanke für~$i$ und~$j$, also ein Element~$k \in I$ mit~$i \preceq k$
und~$j \preceq k$. Die Summe der beiden Elemente ist dann als~$[\langle k,
\phi_{ik}(x) + \phi_{jk}(x)]$ definiert. Man kann nachrechnen, dass dieses
Ergebnis nicht von den getroffenen Wahlen (insgesamt drei Stück: den Wahlen der
beiden Repräsentanten und die Wahl von~$k$) abhängt. (Ihr müsst das aber nicht
machen, die Beweislast liegt dafür bei der Vorlesung.) Man addiert also, indem
man die Repräsanten in einen gemeinsamen Ring überführt und dort addiert. Die
Multiplikation funktioniert völlig analog.

In Teilaufgabe~a) meint "`$x \in R_i$ in~$R_j$ invertierbar"',
dass~$\phi_{ij}(x) \in R_j$ invertierbar ist.

\emph{Bonusaufgabe:} Wieso ist wichtig, dass man von einer gerichteten Menge
fordert, dass sie bewohnt ist (also ein Element enthält)?

\hint{%
Für Teilaufgabe~b) kann es sinnvoll sein, die Gesamtheit aller
(als~$\ZZ$-Algebra) endlich erzeugten Unterringe des vorgegebenen Rings zu
betrachten.}

% x in R_j invbar

\paragraph{Aufgabe 3.} Im Skript ist ein Schema-F-Verfahren beschrieben,
mit dem man Teilaufgabe~c) lösen kann. Vergesst nicht, die Irreduzibilität der
gefundenen Faktoren nachzuweisen. Die Kriterien aus Aufgabe~4 könnten dafür und
für Teilaufgabe~b) hilfreich sein.

\paragraph{Aufgabe 4.} In Teilaufgabe~b) lautet die Voraussetzung an~$f(X)$ wie
folgt: Wenn man~$f(X) = X^n + a_{n-1}X^{n-1} + \cdots + a_1X + a_0 \in R[X]$
schreibt, so ist vorausgesetzt, dass~$\overline{f}(X) = X^n + [a_{n-1}]X^{n-1}
+ \cdots + [a_1]X + [a_0] \in (R/I)[X]$ irreduzibel ist.

Wer sich fragt, wann die seltsame Bedingung an~$I$ erfüllt ist: Der
Faktorring~$R/I$ ist genau dann ein Integritätsbereich, wenn~$I$ ein Primideal
ist. Was das ist, lernen wir nächste Woche.

\hint{%
Für Polynome \emph{über Integritätsbereichen} gilt die übliche Gradformel (wieso?).
}

\paragraph{Aufgabe 5.} In Spiegelschrift folgende manche Lemmas, die hilfreich
sein könnten.

\hint{%
Sei ein Bruch aus~$R[f^{-1}]$ gegeben. Dann kann man den Zähler in~$R$ in
irreduzible Elemente zerlegen. Diese werden in~$R[f^{-1}]$ jedoch im
Allgemeinen nicht irreduzibel sein: Manche werden invertierbar werden! Die
gehören also nicht zur gesuchten Zerlegung des Bruchs in Irreduzible
über~$R[f^{-1}]$.
}

\hint{%
Lemma: Sei ein Element~$x \in R$ irreduzibel und kein Teiler von~$f$. Dann
ist~$x$ auch in~$R[f^{-1}]$ irreduzibel. (Wieso?)}

\hint{%
Lemma: Ein Element~$x \in R$ ist genau dann in~$R[f^{-1}]$ invertierbar, wenn~$x$ ein
Teiler einer gewissen Potenz~$f^n$, $n \geq 0$, ist. (Wieso?)}


\paragraph{Aufgabe 6.} Wer möchte, kann mit dieser Aufgabe mehr als~100\,\% der
Übungspunkte erreichen oder diese interessante Aufgabe zugunsten anderer
Aufgaben bearbeiten. [Es bleibt aber dabei, dass für die 1,0 nicht~100\,\% der
Übungspunkte benötigt werden.] Es darf verwendet werden, dass sich das Ideal
der~$i$-Minoren unter Basiswechsel (d.\,h. unter Multiplikation mit
invertierbaren Matrizen von links und von rechts) nicht ändert. Für
Teilaufgabe~d) folgt ein genauerer Hinweis auf Anfrage per Mail.
% XXX

Matrizen~$A, B$ gleicher Dimension heißen genau dann \emph{zueinander ähnlich},
wenn es invertierbare Matrizen~$R, S$ passender Größe mit~$B = R A S$ gibt.

Ist der zugrundeliegende Ring sogar ein Körper, so führt die Rangdefinition der
Übungs\-auf\-ga\-be auf die bekannte Rangdefinition aus der linearen Algebra.

Eine~$(n \times m)$-Matrix besitzt keinerlei~$i$-Minoren für~$i > n$ und für~$i
> m$. Das Ideal, das von solchen~$i$-Minoren erzeugt wird, ist daher das
Nullideal.

Nur zur Information: Ein Beispiel für einen lokalen Ring ist~$\ZZ_{(p)}$,
wobei~$p$ eine Primzahl ist. Ferner ist jeder Körper ein lokaler Ring. Der Ring
$\ZZ$ selbst ist dagegen kein lokaler Ring. Der Ring~$K[X,Y]_{(X-a,Y-b)} :=
S^{-1}K[X,Y]$ mit~$S := K[X,Y] \setminus (X-a,Y-b) = \{ f(X,Y) \in K[X,Y]
\,|\, f(a,b) \neq 0 \}$ ist ein geometrisch motiviertes Beispiel für einen
lokalen Ring: Seine Elemente sind \emph{Keime} "`guter Funktionen"' auf~$K^2$
-- das sind Funktionen, die nur auf einer kleinen offenen Umgebung um~$(a,b)$ definiert
sein müssen.

\hint{%
Für Teilaufgabe~c) muss man mit verschachtelten Zerlegungen der Eins umgehen.
Ich möchte nicht, dass ihr euch in lauter Technik verliert -- im Digicampus ist
ein allgemeines Lemma festgehalten, dass ihr verwenden könnt.
}

\hint{%
Zu Teilaufgabe~b): Der Fall~$r = 0$ lässt sich kurz erledigen, wieso? Im Fall~$r = 1$
ist mindestens ein Matrixeintrag invertierbar (wieso?). Diesen kann man dann
mit elementaren Zeilen- und Spaltenumformungen nach oben links bringen
(wieso?). Wie geht es dann weiter? Vielleicht ist folgende allgemeine
Beobachtung nützlich: Wenn das von den~$i$-Minoren erzeugte Ideal das Einsideal
ist, so ist für alle~$j < i$ auch das von den~$j$-Minoren erzeugte Ideal das
Einsideal ist.}

\hint{%
Bei Teilaufgabe~a) lässt sich ein Beispiel über~$R = \ZZ$ finden.
}


\section*{Übungsblatt 9}

\paragraph{Aufgabe 2.} Die erste Teilaufgabe ist so gedacht, dass man
\emph{keine} vollständigen Faktorisierungen in irreduzible Elemente bestimmt,
sondern sich mit teilweisen Faktorisierungen begnügt. Für die zweite
Teilaufgabe steckt im Beweis der Vorlesung, dass der Polynomring über einem
ggT-Ring wieder ein ggT-Ring ist (Proposition~7.97), ein explizites
Verfahren, was man hier einsetzen kann.

\paragraph{Aufgabe 3.} In Teilaufgabe~a) meint~"`$d/c$"' das eindeutig
bestimmte Element~$v \in R$ mit~$vc = d$. (Wieso existiert ein solches?)
Allgemein heißt ein Element~$u$ genau dann \emph{größter gemeinsamer Teiler}
zweier Elemente~$x$ und~$y$, wenn
\begin{itemize}
\item $u$ ein Teiler von~$x$ und von~$y$ ist und
\item für jeden gemeinsamen Teiler~$\widetilde u$ von~$x$ und~$y$ gilt,
dass~$\widetilde u$ ein Teiler von~$u$ ist.
\end{itemize}

\hint{%
Betrachtet für Teilaufgabe~b) den größten gemeinsamen Teiler von~$ac$
und~$bc$.}

\paragraph{Aufgabe 4.} Die Gleichheit~$\O_{\QQ(\omega)} = \ZZ[\omega]$ muss
nicht nachgerechnet werden. Der so erhaltene Ring heißt auch \emph{Ring der
Eisenstein-Zahlen}.

\paragraph{Aufgabe 5.} Im Digicampus findet ihr nützliche Rechenregeln für
Ideale und Ringisomorphismen.

Nur zur Information: Mit klassischer Logik lässt sich auch die Umkehrung der
Aussage in Teilaufgabe~a) zeigen: Ein Element, dass in allen Primidealen eines
Rings enthalten ist, ist tatsächlich schon nilpotent. Wer mag, kann sich daran
versuchen; in unserer Vorlesung haben wir aber nicht die nötige Technologie,
um den Beweis einfach aussehen zu lassen.

\end{document}
