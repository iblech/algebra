\documentclass{algblatt}
\usepackage{color}

\newenvironment{indentblock}{%
  \list{}{\leftmargin\leftmargin}%
  \item\relax
}{%
  \endlist
}

\definecolor{grey}{rgb}{0.9,0.9,0.9}
\newcommand{\hint}[1]{\rotatebox{180}{\vbox{\textcolor{grey}{#1}}}}

\begin{document}

\begin{center}\Large \textsf{\textbf{Hinweise zu den Übungsaufgaben in Algebra II}}\end{center}
\vspace{1em}

\section*{Übungsblatt 3}

\paragraph{Aufgabe 1.} Teilaufgabe~a) hat etwas mit Standgruppen zu tun. Für
Teilaufgabe~b) ist interessant, was Fixpunkte mit Bahnen zu tun haben. (Was
sind Fixpunkte denn überhaupt, und was sind Bahnen?) Welche Gleichung der
Vorlesung ist also vermutlich anwendbar?

\paragraph{Aufgabe 4.} Ein beliebiges Element der disjunkt-gemachten
Vereinigung $Y_1 \amalg \cdots \amalg Y_n$ ist ein Paar~$(i,y)$, wobei~$i \in
\{ 1,\ldots,n \}$ ein Index und~$y$ ein Element der entsprechenden Menge~$Y_i$ ist.
Ein \emph{Isomorphismus von~$G$-Wirkungen} ist per Definition eine
bijektive~$G$-äquivariante Abbildung. Für Teilaufgabe~b) ist es hilfreich,~$X$
in Bahnen zu zerlegen.

\paragraph{Aufgabe 5.}
Diese Aufgabe benötigt aber nur die Definition von
Normalteilern und das Verständnis der mengentheoretischen Schreibweise: Die
Menge~$N$ besteht aus all den Elementen von~$G$, welche in allen~$N_i$ liegen. Über die
Größe von~$I$ kann nichts vorausgesetzt werden. Wer mag, kann aber zuerst den
Fall des Schnitts zweier Normalteiler behandeln; der allgemeine Fall verläuft
ähnlich.


\section*{Übungsblatt 4}

\paragraph{Aufgabe 1.} Ein \emph{kleinster Normalteiler, welcher~$H$ umfasst,}
ist per Definition ein Normalteiler~$N$ in~$G$, welcher~$H$ umfasst und welcher
folgende Eigenschaft hat:
Für jeden beliebigen Normalteiler~$N'$ in~$G$, welcher~$H$ umfasst, gilt~$N
\subseteq N'$.

\paragraph{Aufgabe 2.} In beiden Teilaufgaben geht es nicht um
Umkehrfunktionen, sondern um Urbildmengen.

\paragraph{Aufgabe 3.} Die Gruppe~$\GL_n(\RR)$ ist die Menge der
invertierbaren~$(n \times n)$-Matrizen, mit der Matrixmultiplikation als
Gruppenverknüpfung. Die Untergruppe~$\OOO_n(\RR)$ ist die Teilmenge der
orthogonalen Matrizen. Eine Matrix~$A$ heißt genau dann
\emph{orthogonal}, wenn das Produkt~$A^t A$ die
Einheitsmatrix ist. Orthogonale Matrizen haben als Determinante stets~$\pm 1$.
Die Untergruppe~$\SO_n(\RR)$ ist die Teilmenge solcher
orthogonalen Matrizen, deren Determinante~$+1$ ist. Für die Determinante gilt
die Rechenregel~$\det(AB) = \det(A) \cdot \det(B)$.

Bei Teilaufgabe~b) muss man sich zunächst überlegen, ob man~$\SO_n(\RR)$
auf~$\CCC_2$ oder umgekehrt wirken lassen möchte (nur eine Variante
funktioniert), und wie diese Wirkung explizit aussehen soll. Wie bei
Teilaufgabe~c) die Gruppe~$\SO_3(\RR)$ auf~$\RR^3$ wirkt, ist im Skript
angegeben (Beispiel 6.76).

Im Staatsexamen ist das halbdirekte Produkt immer wieder wichtig, um die öfter
vorkommenden Aufgaben der Art \emph{Geben Sie eine nicht-abelsche Gruppe der
Ordnung~2012 an.} zu lösen.

\emph{Für Teilnehmer des Pizzaseminars:} Findet ihr eine kategorielle
Beschreibung des halbdirekten Produkts? (So, wie man das direkte Produkt auch
als terminales Objekt in der Kategorie der Möchtegern-Produkte beschreiben
kann.)

\hint{%
Die Diagonalmatrix, die oben links eine~$-1$ stehen und deren restliche
Diagonaleinträge mit~$+1$ besetzt sind, spielt bei Teilaufgabe~b) eine Rolle.
Wer mich anschreibt, bekommt weitere Tipps.}

\paragraph{Aufgabe 4.} Eine endliche Gruppe heißt genau dann \emph{$p$-Gruppe},
wenn die Anzahl ihrer Elemente eine~$p$-Potenz ist.

\hint{%
Eine nichttriviale~$p$-Gruppe besitzt stets ein Element der Ordnung~$p$ in
ihrem Zentrum. Das Kriterium aus~a) ist für~b) nützlich.}

\paragraph{Aufgabe 5.} Ein \emph{größter endlicher auflösbarer Normalteiler}
ist per Definition ein Normalteiler~$N$ in~$G$, welcher selbst endlich und auflösbar
ist und folgende Eigenschaft hat: Für jeden beliebigen endlichen auflösbaren
Normalteiler~$N'$ in~$G$ gilt~$N' \subseteq N$.

\hint{Für~b) ist~a) nützlich.}

\end{document}
