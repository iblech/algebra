\documentclass{algblatt}
\loesungenfalse


\begin{document}

\maketitle{2}{Abgabe bis 28. Oktober 2013, 17:00 Uhr}

\begin{aufgabe}{2+2}{Untergruppen und Nebenklassen}
\begin{enumerate}
\item
Gib alle Links- und Rechtsnebenklassen in der symmetrischen Gruppe~$\SSS_3$
modulo~$H := \{ \id, (2,3) \}$ an.
\item Seien $H_1$ und~$H_2$ Untergruppen einer Gruppe~$G$. Gelte~$G = H_1 \cup
H_2$. Existiere ein Element~$g \in G$ mit~$g \not\in H_1$. Zeige, dass dann schon~$G
= H_2$.
\end{enumerate}
\end{aufgabe}

\begin{aufgabe}{3+1}{Ordnung von Gruppenelementen}
\begin{enumerate}
\item Sei~$\phi : G \to H$ ein Gruppenhomomorphismus. Sei~$x \in G$ ein Element
endlicher Ordnung. Zeige, dass die Ordnung von~$\phi(x) \in H$ ebenfalls
endlich ist, und zwar ein Teiler der Ordnung von~$x$.
\item Seien~$x$ und~$y$ Elemente einer Gruppe~$G$. Sei die Ordnung von~$xy$
endlich. Zeige, dass auch die Ordnung von~$yx$ endlich ist.
\end{enumerate}
\end{aufgabe}

\begin{aufgabe}{2+2+2+2}{Zyklische Gruppen}
\begin{enumerate}
\item Sei~$G$ eine Gruppe von Primzahlordnung. Zeige, dass~$G$ genau
zwei endliche Untergruppen besitzt.
\item Sei~$G$ eine zyklische Gruppe. Zeige, dass~$G$ abelsch ist.
\item[S c)] Zeige, dass die additive Gruppe der rationalen Zahlen nicht zyklisch ist.
\item[S d)] Sei~$G$ eine endliche Gruppe. Sei~$\Aut(G)$ zyklisch.
Zeige, dass~$G$ abelsch ist.
\end{enumerate}
\end{aufgabe}

\begin{aufgabe}{2+2}{Beispiele für Wirkungen}
\begin{enumerate}
\item Sei~$\phi : G \to H$ ein Gruppenhomomorphismus. Zeige, dass durch
\[ G \times H \longrightarrow H, \quad (g,h) \longmapsto g \bullet h := \phi(g)\,h
\]
eine Wirkung von~$G$ auf~$H$ definiert wird. Zeige weiter, dass dies die
einzige Wirkung von~$G$ auf~$H$ ist, bezüglich der~$\phi$ zu
einer~$G$-äquivarianten Abbildung wird, wenn man~$G$ auf den Urbildbereich durch
Linkstranslation wirken lässt.
\item[S b)] Zeige, dass die Gruppe~$G := \GL_n(\QQ) \times \GL_m(\QQ)$ vermöge
\[ G \times \QQ^{n \times m} \longrightarrow \QQ^{n \times m}, \quad
  ((S,T), A) \longmapsto SAT^{-1} \]
auf der Menge der~$(n \times m)$-Matrizen wirkt.
\end{enumerate}
\end{aufgabe}

\end{document}
