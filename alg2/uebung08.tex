\documentclass{algblatt}
\loesungenfalse


\begin{document}

\maketitle{8}{Abgabe bis 10. Dezember 2013, 17:00 Uhr}

\begin{aufgabe}{2+2}{Lokale Gleichheit und lokale Invertierbarkeit}
Sei~$s_1,\ldots,s_n$ eine Zerlegung der Eins eines kommutativen Rings~$R$.
\begin{enumerate}
\item Zeige, dass Elemente~$f, g \in R$ genau dann gleich sind, wenn sie
\emph{lokal gleich} sind, das heißt, wenn~$f = g$ in~$R[s_i^{-1}]$ für alle~$i
\in \{ 1,\ldots, n \}$ gilt.
\item Zeige, dass ein Element~$f \in R$ genau dann invertierbar ist, wenn es
\emph{lokal invertierbar} ist, das heißt, wenn für jedes~$i \in
\{1,\ldots,n\}$ das Bild von~$f$ in~$R[s_i^{-1}]$ invertierbar ist.
\end{enumerate}
\end{aufgabe}

\begin{aufgabeE}{2+2}{Spiel und Spaß mit dem gerichteten Limes}
\item Sei~$(R_i)_{i \in I}$ ein gerichtetes System von Ringen mit Limes~$R =
\varinjlim_{i \in I} R_i$. Zeige, dass ein Element~$x \in R_i$ genau dann
in~$R$ invertierbar ist, wenn ein~$j \succeq i$ existiert, sodass~$x$ in~$R_j$
invertierbar ist.

\item Zeige, dass jeder Ring gerichteter Limes endlich erzeugter~$\ZZ$-Algebren
ist.
\end{aufgabeE}

\begin{aufgabeE}{1+1+2}{Beispiele für Primfaktorzerlegungen}
\item[S a)] Zeige, dass~$3 + 2\i \in \ZZ[\i]$ irreduzibel ist.
\item[S b)] Zeige, dass~$X^2 + Y \in \ZZ[X,Y]$ irreduzibel ist.
\item[S c)] Bestimme die Primfaktorzerlegung von~$X^4 + 4\,Y^4 \in \ZZ[X,Y]$.
\end{aufgabeE}

\begin{aufgabeE}{2+2}{Allgemeine Irreduzibilitätskriterien}
\item Sei~$f(X) = a_n X^n + a_{n-1} X^{n-1} + \cdots + a_1 X + a_0$ ein Polynom
über einem Integritätsbereich~$R$. Sei Eins ein größter gemeinsamer Teiler der
Koeffizienten von~$f(X)$. Sei~$p \in R$ ein Primelement,
welches~$a_0,\ldots,a_{n-1}$ teilt, $a_n$ nicht teilt und~$a_0$ nicht im
Quadrat teilt. Zeige, dass~$f(X)$ in~$R[X]$ irreduzibel ist.
\item Sei~$I \subsetneq R$ ein echtes Ideal eines kommutativen Rings~$R$.
Sei~$f(X) \in R[X]$ ein normiertes Polynom, das über~$R/I$ irreduzibel ist.
Zeige, dass~$f(X)$ dann auch als Element von~$R[X]$ irreduzibel ist.
\end{aufgabeE}

\begin{aufgabe}{2+2}{Lokalisierung weg von einem Element}
Sei~$f$ ein reguläres Element eines Integritätsbereichs~$R$.
\begin{enumerate}
\item Sei~$R$ sogar ein Ring mit eindeutiger Primfaktorzerlegung. Zeige,
dass~$R[f^{-1}]$ dann ebenfalls ein Ring mit eindeutiger Primfaktorzerlegung
ist.
\item Sei~$R$ faktoriell. Zeige, dass dann~$R[f^{-1}]$ ebenfalls faktoriell
ist.
\end{enumerate}
\end{aufgabe}

\begin{center}
\emph{Für eine spannende Bonus-Aufgabe bitte wenden.}
\end{center}
\newpage

\begin{aufgabe}{1+2+1+2}{Lokale Gauß--Jordansche Normalform}
Eine~$(n \times m)$-Matrix über einem beliebigen kommutativen Ring~$R$
heißt genau dann \emph{vom Rang~$r$}, wenn das von den~$r$-Minoren von~$A$
erzeugte Ideal das Einsideal und das von den~$(r+1)$-Minoren erzeugte Ideal das
Nullideal ist.
\begin{enumerate}
\item Finde ein Beispiel für eine Matrix über einem Ring, die in diesem Sinn
keinen Rang besitzt.
\end{enumerate}
Ein kommutativer Ring~$R$ heißt \emph{lokaler Ring}, falls
\[ \forall x,y \in R{:}\quad \text{$x + y$ invertierbar}
\quad\Longrightarrow\quad
\text{$x$ invertierbar oder $y$ invertierbar}. \]
\begin{enumerate}
\addtocounter{enumi}{1}
\item Sei~$A$ eine~$(n \times m)$-Matrix vom Rang~$r$ über einem lokalen Ring~$R$.
Zeige, dass~$A$ eine \emph{Gauß--Jordansche Normalform} besitzt,
also ähnlich zu einer rechteckigen Diagonalmatrix
mit genau~$r$ Einsern und sonst nur Nullern auf der Hauptdiagonale ist.
\item Sei nun~$R$ wieder ein beliebiger kommutativer Ring. Seien~$x,y \in R$ derart, dass die
Summe~$x+y$ in~$R$ invertierbar ist. Zeige, dass es eine Zerlegung der Eins
von~$R$ gibt, sodass in den lokalisierten Ringen jeweils~$x$ oder~$y$
invertierbar ist. Gewissermaßen ist also jeder Ring \emph{lokal} ein lokaler
Ring.
\item Sei~$A$ eine~$(n \times m)$-Matrix vom Rang~$r$ über einem beliebigen
kommutativen Ring~$R$. Zeige, dass~$A$ \emph{lokal} eine
Gauß--Jordansche Normalform besitzt, dass es also eine
Zerlegung~$s_1,\ldots,s_n$ der Eins von~$R$ gibt, sodass~$A$ für jedes~$i$
über~$R[s_i^{-1}]$ ähnlich zu einer solchen Diagonalmatrix ist.
\end{enumerate}
\end{aufgabe}

\end{document}
