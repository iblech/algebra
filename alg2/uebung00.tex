\documentclass{algblatt}
\usepackage{xstring}
\IfSubStr{\jobname}{\detokenize{loesung}}{\loesungentrue}{\loesungenfalse}


\begin{document}

\maketitle{0}{
  \vbox{Präsenzblatt, Besprechung in der ersten Übung, keine Abgabe \\[1.6em]

  \begin{minipage}{0.93\textwidth}
  \setlength\parskip{\medskipamount}
  Willkommen zur \emph{Algebra II} im Wintersemester 2013/2014!

  Die Übungen beginnen in der ersten Woche am 17. Oktober. In der ersten Übung
  besprechen wir dieses Präsenzblatt, dessen Lösung nicht abgegeben werden
  muss. Eine Anmeldung zu den Übungen ist nicht erforderlich.

  Übungsblatt~1 ist dann schriftlich zu bearbeiten und bis zum 23.~Oktober im
  üblichen Briefkasten abzugeben. Euch allen ist das Pamphlet
  \url{http://xrl.us/uebungsblatt} zur Bedeutung von Übungsblättern bekannt.
  \end{minipage}
}}

\begin{aufgabe}{Grundlagen}
\begin{enumerate}
\item Gibt es auf der Menge der rationalen Zahlen eine Gruppenstruktur mit der
gewöhnlichen Multiplikation als Gruppenverknüpfung?
\item Finde zwei quadratische Matrizen~$A, B$ gleicher Größe mit~$A \cdot B
\neq B \cdot A$.

\emph{Bonusaufgabe:} Finde~$A$ und~$B$ so, dass zusätzlich~$A,B \neq 0$,
aber~$A \cdot B = 0$ gilt.
\end{enumerate}
\end{aufgabe}

\begin{aufgabe}{Allgemeines zu Gruppen}
Sei~$G$ eine Gruppe.
\begin{enumerate}
\item Sei~$e \in G$ ein Element mit~$e \cdot g = g$ für alle~$g \in G$. Zeige,
dass~$e = 1$.
\item Seien~$a,b,g \in G$. Zeige: $a = b \Longleftrightarrow g \cdot a = g
\cdot b$.
\item Seien~$a,b \in G$. Zeige, dass es genau ein Gruppenelement~$x \in G$
mit~$a \cdot x = b$ gibt.
\end{enumerate}
\end{aufgabe}

\begin{aufgabe}{Ein Kriterium für Kommutativität}
\begin{enumerate}
\item Sei~$G$ eine Gruppe mit~$g^2 = 1$ für jedes~$g \in G$. Zeige, dass~$G$
abelsch ist.
\item Finde ein Beispiel für Gruppe, die abelsch ist, aber das Kriterium aus~a)
nicht erfüllt.
\end{enumerate}
\end{aufgabe}

\begin{aufgabe}{Konjugation als Isomorphismus}
Sei~$x$ ein festes Element einer Gruppe~$G$. Zeige, dass die Abbildung
$G \to G, g \mapsto x g x^{-1}$, die \emph{Konjugation mit~$g$}, ein
Gruppenisomorphismus ist.
\end{aufgabe}

\end{document}
