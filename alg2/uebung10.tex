\documentclass{algblatt}
\loesungenfalse

\setlength{\aufgabenskip}{1em}

\begin{document}

\maketitle{10}{Abgabe bis 7. Januar 2014, 17:00 Uhr}

\begin{aufgabeE}{2+3}{Praktische Arbeit mit Idealen}
\item Finde eine Zerlegung~$1 = s_1 + \cdots + s_n$ von~$R = \ZZ[\sqrt{-13}]$,
sodass für alle~$i \in \{ 1,\ldots,n \}$ das Ideal~$(7, 1 + \sqrt{-13})
\subseteq R[s_i^{-1}]$ ein Hauptideal ist.
\item Sei~$x$ eine Nullstelle des Polynoms~$X^4-X^2-3\,X+7$ in den
algebraischen Zahlen. Sei~$K = \QQ(x)$. Bestimme eine teilweise Faktorisierung
der Ideale~$(14,x+7)$ und~$(35,x-14)$ in~$\O_K$.
\end{aufgabeE}

\begin{aufgabe}{2+2}{Irreduzible Ideale}
Ein nicht verschwindendes Ideal~$\aa$ eines Rings~$R$ heißt genau dann
\emph{irreduzibel}, wenn es für jede Zerlegung~$\aa = \aa_1 \cdots \aa_n$ in
endlich erzeugte Ideale ein~$i \in \{1,\ldots,n\}$ mit~$\aa = \aa_i$ gibt.
\begin{enumerate}
\item Zeige, dass in prüferschen Bereichen endlich erzeugte irreduzible Ideale prim sind.
\item Zeige, dass in prüferschen Bereichen Zerlegungen von Idealen in endlich
erzeugte Primideale
bis auf Reihenfolge eindeutig sind (sofern sie existieren).
\end{enumerate}
\end{aufgabe}

\begin{aufgabe}{2+2}{Algorithmen zur Idealfaktorisierung}
Sei~$R$ ein dedekindscher Bereich. Beschreibe unter den folgenden
Voraussetzungen jeweils ein Verfahren, das ein gegebenes nicht verschwindendes
endlich erzeugte Ideal von~$R$ in Primideale faktorisiert, also als Produkt von
Primidealen schreibt.
\begin{enumerate}
\item Wir haben einen Test, der von einem gegebenen nicht verschwindenden endlich erzeugten Ideal feststellt,
ob es irreduzibel ist, und es gegebenenfalls in zwei echte Faktoren zerlegt
(oder angibt, dass es das Einsideal ist).
\item Wir haben einen Test, der von einem gegebenen nicht verschwindenden endlich erzeugten Ideal
feststellt, ob es maximal ist, und gegebenenfalls ein Element liefert, um das
es zu einem echten Ideal echt erweitert werden kann (oder angibt, dass das
Ideal schon das Einsideal war).
\end{enumerate}
\end{aufgabe}

\begin{aufgabe}{3}{Gemeinsame Stoppstellen mehrerer Ketten}
Sei~$R$ ein noetherscher kommutativer Ring. Ein Algorithmus produziere~$m$
Ketten von endlich erzeugten Idealen in~$R$:
\[ \begin{array}{@{}ccccccc@{}}
  \aa_{10} &\subseteq& \aa_{11} &\subseteq& \aa_{12} &\subseteq& \cdots \\
  \vdots &         & \vdots &         & \vdots &          \\
  \aa_{m0} &\subseteq& \aa_{m1} &\subseteq& \aa_{m2} &\subseteq& \cdots
\end{array} \]
Zeige, dass es eine Stelle~$n$ gibt, sodass~$\aa_{jn} = \aa_{j(n+1)}$ für alle~$j \in
\{ 1,\ldots,m \}$ gilt.
\end{aufgabe}

\enlargethispage{1cm}

\begin{aufgabe}{4}{Beispiel für eine Ganzheitsbasis}
Bestimme eine~$\ZZ$-Basis von~$\O_{\QQ(\sqrt[3]{4})}$.
\end{aufgabe}

\end{document}
