\documentclass{algblatt}
\usepackage{xstring}
\IfSubStr{\jobname}{\detokenize{loesung}}{\loesungentrue}{\loesungenfalse}


\begin{document}

\maketitle{15}{Abgabetermin entscheidet ihr}

\begin{aufgabe}{4}{Charakterisierung von Normalität}
Sei~$L$ eine endliche Körpererweiterung eines Körpers~$K$. Zeige, dass~$L$
genau dann über~$K$ normal ist, wenn jedes über~$K$ irreduzible Polynom~$f(X)
\in K[X]$, welches in~$L$ eine Nullstelle besitzt, über~$L$ schon in
Linearfaktoren zerfällt.
\end{aufgabe}

\begin{aufgabe}{4}{Normalität in Körpertürmen}
Sei~$L$ eine endliche Körpererweiterung eines Körpers~$K$. Sei~$E$ eine
endliche Zwischenerweiterung. Zeige anhand eines Beispiels, dass~$L$ über~$K$
im Allgemeinen nicht normal ist, auch wenn~$L$ über~$E$ und~$E$ über~$K$
normale Erweiterungen sind.
\end{aufgabe}

\begin{aufgabe}{4}{Beispiel für eine nicht-galoissche Erweiterung}
Zeige, dass~$\FF_p(T)$ über~$\FF_p(T^p)$ keine galoissche Erweiterung ist.
\end{aufgabe}

\begin{aufgabe}{4}{Ein Kriterium von Artin}
Sei~$L$ eine endliche Körpererweiterung eines Körpers~$K$. Zeige, dass~$L$
genau dann galoissch über~$K$ ist, wenn eine endliche Untergruppe~$G$ der
Automorphismen von~$L$ (also der Gruppe aller Körperisomorphismen von~$L$
nach~$L$) existiert, sodass
\[ K = L^G := \{ x \in L \,|\, \text{$\sigma x = x$ für alle $\sigma \in G$}
\}. \]
\end{aufgabe}

\end{document}
