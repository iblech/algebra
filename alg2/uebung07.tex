\documentclass{algblatt}
\loesungenfalse


\begin{document}

\maketitle{7}{Abgabe bis 3. Dezember 2013, 17:00 Uhr}

\begin{aufgabeE}{1+1}{Urbilder und Bilder von Idealen}
\item Zeige, dass Urbilder von Idealen unter Ringhomomorphismen wieder Ideale
sind.
\item Zeige, dass Bilder von Idealen unter Ringhomomorphismen im Allgemeinen
aber keine Ideale sind.
\end{aufgabeE}

\begin{aufgabe}{1+2+1}{Beispiele für Ideale}
Skizziere alle endlich erzeugten Ideale von folgenden Ringen zusammen mit ihren
Inklusionsbeziehungen:
\begin{enumerate}
\item $\ZZ$.
\item[S b)] $\ZZ_{(p)}$ (aus Blatt 6, Aufgabe 4), wobei~$p$ eine Primzahl ist.
\item $K$, wobei~$K$ ein beliebiger Körper ist.
\end{enumerate}
\end{aufgabe}

\begin{aufgabeE}{2+2+2}{Nilpotente und reguläre Elemente}
\item Zeige, dass der Restklassenring~$\ZZ[\i]/(2)$ genau vier Elemente hat.
Welche Elemente sind regulär?
\item Sei~$n \geq 0$. Bestimme das Nilradikal von~$\ZZ/(n)$.
\item Sei~$R$ ein kommutativer Ring. Sei~$f \in R$. Zeige: Der Ring~$R[f^{-1}]$ ist
genau dann der Nullring, wenn~$f$ in~$R$ nilpotent ist.
\end{aufgabeE}

\begin{aufgabe}{1+3}{Charakteristik von Körpern}
Sei~$K$ ein Körper.
\begin{enumerate}
\item Gib den eindeutig bestimmten Ringhomomorphismus~$\epsilon : \ZZ \to K$
explizit an.
\item Zeige, dass~$K$ genau dann von Charakteristik~$n$ ist, wenn~$\ker
\epsilon = (n)$.
\end{enumerate}
\end{aufgabe}

\begin{aufgabe}{4}{Geometrische Komponenten}
Sei~$R$ ein kommutativer Ring. Zeige, dass folgende Aussagen äquivalent sind:
\begin{enumerate}
\item Es gibt~$e, f \in R$ mit~$e \neq 0, f \neq 0, ef = 0, e^2 = e, f^2 = f$
und~$e + f = 1$.
\item Es gibt kommutative Ringe~$S$ und~$T$, die jeweils nicht der Nullring
sind, mit~$R \cong S \times T$.
\end{enumerate}
\end{aufgabe}

\end{document}
