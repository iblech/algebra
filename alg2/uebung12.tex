\documentclass{algblatt}
\usepackage{xstring}
\IfSubStr{\jobname}{\detokenize{loesung}}{\loesungentrue}{\loesungenfalse}

\setlength{\aufgabenskip}{1em}

\begin{document}

\maketitle{12}{Abgabe bis 21. Januar 2014, 17:00 Uhr}

\begin{aufgabeE}{2+2+2}{Beispielrechnungen in endlicher Charakteristik}
\item Finde von jedem Element aus~$\FF_7$ all seine siebten Wurzeln.
\item Sei~$E := \FF_3[X]/(X^3 + X^2 + 2)$. Schreibe~$\alpha := [X] \in E$ als einen
in~$\alpha^3$ rationalen Ausdrück über~$\FF_3$.
\item Sei~$K$ ein Körper der Charakteristik~$p$. Sei~$x$ ein Element einer
Körpererweiterung mit~$K(x) = K(x^p)$. Konstruiere ein separables Polynom
über~$K$, das~$x$ als Nullstelle besitzt.
\end{aufgabeE}

\begin{aufgabe}{2+2}{Vererbung von Separabilität}
Sei~$L \supseteq E \supseteq K$ ein Turm von Körpererweiterungen. Zeige oder
widerlege folgende Aussagen:
\begin{enumerate}
\item Ist ein Element aus~$L$ über~$K$ separabel, so auch über~$E$.
\item Ist ein Element aus~$L$ über~$E$ separabel, so auch über~$K$.
\end{enumerate}
\end{aufgabe}

\begin{aufgabe}{2+2}{Kriterium für Separabilität}
Sei~$N \geq 0$ eine natürliche Zahl und~$K$ ein Körper von Charakteristik
größer als~$N$.
\begin{enumerate}
\item Sei~$f(X)$ ein Polynom vom Grad~$\leq N$ über~$K$. Zeige:
Ist $f(X)$ irreduzibel, so auch separabel.
\item Sei ferner~$K$ faktoriell und~$L$ eine endliche Erweiterung vom Grad~$N$.
Zeige, dass~$L$ faktoriell ist.
\end{enumerate}
\end{aufgabe}

\begin{aufgabe}{2+1}{Gerichteter Limes von Körpern}
Sei~$(K_i)_{i \in I}$ ein gerichtetes System von Ringen. Seien alle~$K_i$ sogar
Körper.
\begin{enumerate}
\item Zeige, dass der gerichtete Limes~$L := \varinjlim_{i \in I} K_i$ ein
Körper ist.
\item Zeige, dass~$L$ in kanonischer Art und Weise als Körpererweiterung eines
jeden~$K_i$ aufgefasst werden kann.
\end{enumerate}
\end{aufgabe}

\begin{aufgabe}{2+1}{Vollkommene Körper}
Sei~$K$ ein faktorieller Körper.
\begin{enumerate}
\item Zeige, dass~$K$ genau dann vollkommen ist, wenn jedes irreduzible Polynom
über~$K$ auch separabel über~$K$ ist.
\item Welche Richtung lässt sich noch zeigen, wenn wir nicht voraussetzen
wollen, dass~$K$ faktoriell
ist?
\end{enumerate}
\end{aufgabe}

\end{document}
