\documentclass{algblatt}
\loesungenfalse


\begin{document}

\maketitle{14}{Abgabe bis 4. Februar 2014, 17:00 Uhr}

\begin{aufgabe}{??}{Allgemeines zu reiner Inseparabilität}
Sei~$L$ ein Körpererweiterung von~$K$.
\begin{enumerate}
\item Zeige, dass die Menge der über~$K$ rein inseparablen Elemente in~$L$ eine
Zwischenerweiterung von~$L$ über~$K$ ist.
\item Sei~$L$ sowohl separabel als auch rein inseparabel über~$K$. Zeige,
dass~$L = K$.
\item Sei ein über~$K$ separables Element~$x \in L$ und ein über~$K$ rein
inseparables Element~$y \in L$ gegeben. Zeige: $K(x,y) = K(x+y)$.
\end{enumerate}
\end{aufgabe}

\begin{aufgabe}{??}{Norm und Diskriminante}
Sei~$L$ eine endliche Körpererweiterung von~$K$.
\begin{enumerate}
\item Seien die~$x_i$ die galoissch Konjugierten eines Elements~$x \in L$ in
einem algebraisch abgeschlossenen Oberkörper~$\Omega \subseteq K$. Zeige:
\[ N_{L/K}(x) = \left(\prod_{i=1}^{\gra{L}{K}_\text{s}}
x_i\right)^{\gra{L}{K}_\text{i}}. \]
\item Sei~$E$ ein über~$K$ endlicher Zwischenkörper. Zeige:
\[ \disc_{L/K} = N_{E/K}(\disc_{L/E}) \cdot (\disc_{E/K})^{\gra{L}{E}}. \]
\end{enumerate}
\end{aufgabe}

\begin{aufgabe}{??}{Erster Gehversuch mit transzendenten Erweiterungen}
Sei~$L = \QQ(X)$ und~$E = \QQ(X^3 - 2, X^6 - X^2 - 1)$.
\begin{enumerate}
\item Finde ein primitives Element von~$E$ über~$\QQ$.
\item Zeige, dass~$L$ eine endliche Erweiterung von~$E$ ist. Was ist der Grad?
\end{enumerate}
\end{aufgabe}

Eine weitere Aufgabe wird noch folgen.

\end{document}
