\documentclass{algblatt}
\loesungenfalse


\begin{document}

\maketitle{14}{Abgabe bis 4. Februar 2014, 17:00 Uhr}

\begin{aufgabe}{2+2+2}{Allgemeines zu reiner Inseparabilität}
Sei~$L$ ein Körpererweiterung von~$K$.
\begin{enumerate}
\item Zeige, dass die Menge der über~$K$ rein inseparablen Elemente in~$L$ eine
Zwischenerweiterung von~$L$ über~$K$ ist.
\item Sei~$L$ sowohl separabel als auch rein inseparabel über~$K$. Zeige,
dass~$L = K$.
\item Sei ein über~$K$ separables Element~$x \in L$ und ein über~$K$ rein
inseparables Element~$y \in L$ gegeben. Zeige: $K(x,y) = K(x+y)$.
\end{enumerate}
\end{aufgabe}

\begin{aufgabe}{3+3}{Norm und Diskriminante}
Sei~$L$ eine endliche Körpererweiterung von~$K$.
\begin{enumerate}
\item Seien die~$x_i$ die galoissch Konjugierten eines Elements~$x \in L$ in
einem algebraisch abgeschlossenen Oberkörper~$\Omega \subseteq K$. Zeige:
\[ N_{L/K}(x) = \left(\prod_{i=1}^{\gra{L}{K}_\text{s}}
x_i\right)^{\gra{L}{K}_\text{i}}. \]
\item Sei~$E$ ein über~$K$ endlicher Zwischenkörper. Zeige:
\[ \disc_{L/K} = N_{E/K}(\disc_{L/E}) \cdot (\disc_{E/K})^{\gra{L}{E}}. \]
\end{enumerate}
\end{aufgabe}

\begin{aufgabe}{2+2}{Erster Gehversuch mit transzendenten Erweiterungen}
Sei~$L = \QQ(X)$ und~$E = \QQ(X^3 - 2, X^6 - X^2 - 1)$.
\begin{enumerate}
\item Finde ein primitives Element von~$E$ über~$\QQ$.
\item Zeige, dass~$L$ eine endliche Erweiterung von~$E$ ist. Was ist der Grad?
\end{enumerate}
\end{aufgabe}

\begin{aufgabe}{2+2}{Beispiele für Spannoperationen}
Eine \emph{Spannoperation} auf einer Menge~$S$ ist eine Vorschrift, die jeder
endlichen Teilmenge~$\{x_1,\ldots,x_n\}$ von~$S$ eine gewisse
Teilmenge~$\langle x_1,\ldots,x_n \rangle \subseteq S$ zuordnet, sodass gewisse
natürliche Axiome erfüllt sind (siehe Hinweisblatt).
\begin{enumerate}
\item Sei~$V$ ein endlich-dimensionaler Vektorraum über einem Körper~$K$. Für
je endlich viele Vektoren~$v_1,\ldots,v_n \in V$ sei~$\langle v_1,\ldots,v_n
\rangle$ ihr linearer Spann. Zeige, dass diese Vorschrift eine Spannoperation
auf~$V$ definiert.
\item Sei~$L$ über~$K$ eine Körpererweiterung. Für je endlich viele
Elemente~$x_1,\ldots,x_n \in L$ sei~$\langle x_1,\ldots,x_n \rangle$ die Teilmenge
der Elemente aus~$L$, welche über~$K(x_1,\ldots,x_n)$ algebraisch sind. Zeige,
dass diese Vorschrift eine Spannoperation auf~$L$ definiert.
\end{enumerate}
\end{aufgabe}

\end{document}
