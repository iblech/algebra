\documentclass{algblatt}
\loesungenfalse


\begin{document}

\maketitle{4}{Abgabe bis 12. November 2013, 17:00 Uhr}

\begin{aufgabe}{2}{Normaler Abschluss}
Sei~$H$ eine Untergruppe einer Gruppe~$G$. Sei~$(N_i)_{i \in I}$ die Familie
all derjenigen Normalteiler von~$G$, welche~$H$ umfassen. Zeige,
dass~$\bigcap_{i \in I} N_i$ der normale Abschluss von~$H$ in~$G$ ist.
\end{aufgabe}

\begin{aufgabeE}{2+2+2}{Beispiele für halbdirekte Produkte}
\item Zeige, dass jedes direkte Produkt zweier Gruppen auch als halbdirektes
Produkt angesehen werden kann.
\item Zeige, dass die orthogonale Gruppe~$\OOO_n(\RR)$ isomorph zu einem
halbdirekten Produkt von~$\SO_n(\RR)$ mit~$\CCC_2$ ist.
\item Zeige, dass folgende Abbildung ein injektiver Gruppenhomomorphismus ist.
\[ \RR^3 \rtimes \SO_3(\RR) \longrightarrow \GL_4(\RR),\quad
  (b, A) \longmapsto \left(\begin{array}{ccc|c}
    \multicolumn{3}{c|}{A} & b \\\hline
    0 & 0 & 0 & 1
  \end{array}\right) \]
\end{aufgabeE}
\vspace{-1em}

\begin{aufgabe}{1+1}{Urbilder unter Gruppenhomomorphismen}
Sei~$f : G \to H$ ein Gruppenhomomorphismus.
\begin{enumerate}
\item Sei~$K \subseteq H$ eine Untergruppe. Zeige, dass~$f^{-1}[K]$
eine Untergruppe von~$G$ ist.
\item Sei~$K \subseteq H$ ein Normalteiler. Zeige, dass~$f^{-1}[N]$
ein Normalteiler von~$G$ ist.
\end{enumerate}
\end{aufgabe}

\begin{aufgabeE}{4+2}{Ein Kriterium für Auflösbarkeit}
\item Sei~$G$ eine endliche Gruppe und~$N$ ein endlicher Normalteiler in~$G$.
Zeige, dass~$G$ genau dann auflösbar ist, wenn~$G/N$ und~$N$ es sind.
\item[S b)] Zeige, dass jede endliche~$p$-Gruppe auflösbar ist.
\end{aufgabeE}

\begin{aufgabeE}{2+2}{Auflösbare Normalteiler}
\item Seien~$N$ und~$N'$ auflösbare Normalteiler eine Gruppe~$G$. Zeige,
dass~$N \cdot N' := \{ n n' \,|\, n \in N, n' \in N' \}$ eine Untergruppe und
sogar ein auflösbarer Normalteiler von~$G$ ist.
\item Sei~$G$ eine endliche Gruppe. Zeige, dass ein größter endlicher
auflösbarer Normalteiler von~$G$ existiert.
\tiny
\item \emph{Für Teilnehmer des Pizzaseminars:} Gib kategorielle
Interpretationen der Konstruktionen aus~a) und~b) in der durch die Halbordnung
der auflösbaren Normalteiler induzierten Kategorie.
\end{aufgabeE}

\end{document}
