\documentclass{algblatt}
\loesungenfalse


\begin{document}

\maketitle{9}{Abgabe bis 17. Dezember 2013, 17:00 Uhr}

\begin{aufgabe}{3}{Ein Gegenbeispiel}
Sei~$R$ ein Integritätsbereich. Zeige, dass das Ideal~$(X,Y) \subseteq R[X,Y]$
kein Hauptideal ist.
\end{aufgabe}

\begin{aufgabeE}{2+2}{Praxis zu größten gemeinsamen Teilern}
\item Bestimme eine teilweise Faktorisierung der Zahlen $99$, $1200$ und $160$.
\item Bestimme einen größten gemeinsamen Teiler der folgenden Polynome
in~$\QQ[X,Y]$:
\[ X^3Y^2 - X^2Y^3 + XY^3 - Y^4, \quad X^4Y - X^3Y^2 - X^2Y^2 + XY^3. \]
\end{aufgabeE}
\vspace{-1em}

\begin{aufgabeE}{1+1+2}{Theorie zu größten gemeinsamen Teilern}
\item Seien~$a,b,c$ Elemente eines Integritätsbereichs. Sei~$c$ regulär und
existiere ein größter gemeinsamer Teiler~$d$ von~$ac$ und~$bc$. Zeige, dass~$d$
durch~$c$ teilbar ist und dass~$d/c$ ein größter gemeinsamer Teiler für~$a$
und~$b$ ist.
\item Seien~$a,b,c$ Elemente eines Rings mit größten gemeinsamen Teilern.
Sei~$a$ ein Teiler von~$b \cdot c$ und sei die Eins ein größter gemeinsamer
Teiler von~$a$ und~$b$. Zeige, dass~$a$ auch ein Teiler von~$c$ ist.
\item Sei~$R$ ein Integritätsbereich, in dem eine teilweise Faktorisierung
immer möglich ist. Zeige, dass~$R$ ein Ring mit größten gemeinsamen Teilern
ist.
\end{aufgabeE}

\begin{aufgabeE}{2+2}{Beispiele und Nichtbeispiele für euklidische Ringe}
\item[S a)] Sei~$\omega$ eine primitive dritte Einheitswurzel. Zeige, dass der
Ring~$\O_{\QQ(\omega)} = \ZZ[\omega]$ zusammen mit der Norm~$N : a+b\omega
\mapsto a^2 - ab + b^2$ ein euklidischer Ring ist.
\item[S b)] Zeige, dass der Ring~$\ZZ[\sqrt{-5}]$ zusammen mit der Norm~$N :
a+b\sqrt{-5} \mapsto a^2 + 5b^2$ \emph{kein} euklidischer Ring ist.
\end{aufgabeE}

\begin{aufgabeE}{1+2+2}{Primideale und maximale Ideale}
\item Zeige, dass ein nilpotentes Element eines kommutativen Rings~$R$ in allen
Primidealen von~$R$ liegt.
\item Sei~$\mm \subseteq R$ ein Ideal eines kommutativen Rings~$R$. Zeige,
dass~$\mm$ genau dann ein maximales Ideal ist, wenn der Faktorring~$R/\mm$ ein
Körper ist.
\item[S c)] Ist das Ideal~$(2, X) \subseteq \ZZ[X]/(X^2-X+6)$ maximal? Ist es ein
Primideal?
\end{aufgabeE}

\end{document}
