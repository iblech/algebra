\documentclass{algblatt}
\loesungenfalse

\setlength{\aufgabenskip}{1em}

\begin{document}

\maketitle{11}{Abgabe bis 14. Januar 2014, 17:00 Uhr}

\begin{aufgabe}{4}{Körpererweiterungen ungeraden Grads}
Sei~$K(x)$ über einem Körper~$K$ eine Körpererweiterung ungeraden Grads. Zeige,
dass~$K(x) = K(x^2)$.
\end{aufgabe}

\begin{aufgabe}{2+2}{Beispiele mit endlichen Körpern}
\begin{enumerate}
\item
Finde ein normiertes irreduzibles Polynom zweiten Grades über~$\FF_2$ und gib
einen Körper mit vier Elementen an.
\item Zerlege das Polynom~$X^5 + X^4 + X^3 + X^2 + 1$ über~$\FF_3$ in
irreduzible Faktoren.
\end{enumerate}
\end{aufgabe}

\begin{aufgabe}{1+3}{Beispiele mit dem Körper der rationalen Funktionen}
\begin{enumerate}
\item Sei~$K$ ein Körper. Sei~$E$ ein Zwischenkörper von~$K(X)$ über~$K$.
Sei~$u$ ein Element von~$E$, das nicht in~$K$ liegt. Zeige, dass~$X$
algebraisch über~$E$ ist.
\item Sei~$K$ ein Körper und seien~$g(X), h(X) \in K[X]$ teilerfremde Polynome.
Sei ferner~$h(X) \neq 0$ und~$n := \max\{ \deg g(X), \deg h(X) \} \geq 1$.
Zeige, dass der Grad von~$K(X)$ über~$K(y)$ gerade~$n$ ist, wobei~$y :=
\tfrac{g(X)}{h(X)} \in K(X)$.
\end{enumerate}
\end{aufgabe}

\begin{aufgabe}{4}{Lineare Disjunktheit}
Sei~$L$ über~$K$ eine Körpererweiterung. Eine Zwischenerweiterung~$E$ heißt
genau dann \emph{linear disjunkt} von einer weiteren Zwischenerweiterung~$F$,
falls jede endliche Familie von Elementen aus~$E$, welche über~$K$ linear
unabhängig ist, auch über~$F$ linear unabhängig ist.

Zeige: Ist~$E$ linear disjunkt von~$F$, so ist auch~$F$ linear disjunkt von~$E$.
\end{aufgabe}

\begin{aufgabe}{4}{Die Kronecker-Konstruktion}
\begin{enumerate}
\item Folgt noch.
\item Sei~$f(X)$ ein Polynom über einem endlichen Körper~$K$. Zeige, dass eine
Kör\-per\-erwei\-te\-rung~$L$ von~$K$ existiert, über der~$f(X)$ in Linearfaktoren
zerfällt.
\end{enumerate}
\end{aufgabe}

\end{document}
