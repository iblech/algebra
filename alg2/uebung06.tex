\documentclass{algblatt}
\loesungenfalse


\begin{document}

\maketitle{6}{Abgabe bis 26. November 2013, 17:00 Uhr}

\begin{aufgabeE}{2+2}{Elementarteiler von Matrizen}
\item Sei~$M$ eine ganzzahlige~$(n \times m)$-Matrix. Seien~$d_1,\ldots,d_r$
die Elementarteiler von~$M$. Sei~$\lambda_i$ der größte gemeinsamer Teiler
aller~$i$-Minoren. Zeige: $\lambda_i = d_1 \cdots d_i$.
\item Bestimme die Elementarteiler folgender Matrix:
\[ \begin{pmatrix}2 & 6 & 8 \\ 3 & 1 & 2 \\ 9 & 5 & 4\end{pmatrix} \]
\end{aufgabeE}
\vspace{-0.5em}

\begin{aufgabeE}{2+2}{Klassifikation endlicher abelscher Gruppen}
\item[S a)] Bestimme bis auf Isomorphie alle abelschen Gruppen der Ordnung~$24$.
\item[S b)] Bestimme bis auf Isomorphie alle abelschen Gruppen der Ordnung~$180$.
\end{aufgabeE}

\begin{aufgabe}{1+3}{Zerlegung in $p$-primäre Komponenten}
Sei~$A$ eine endliche abelsche Gruppe. Eine Primzahl~$p$ heißt genau dann
\emph{assoziierte Primzahl zu~$A$}, wenn die Untergruppe~$A[p^\infty]$
derjenigen Elemente von~$A$, deren Ordnung eine~$p$-Potenz ist, nichttrivial
ist. In diesem Fall heißt~$A[p^\infty]$ \emph{$p$-primäre Komponente
von~$A$}.
\begin{enumerate}
\item Zeige, dass~$A$ nur endlich viele assoziierte Primzahlen besitzt.
\item Zeige, dass~$A$ isomorph zum direkten Produkt der~$p$-primären
Komponenten von~$A$ ist.
\end{enumerate}
\end{aufgabe}

\begin{aufgabe}{2+2}{Lokalisierung nach einer Primzahl}
Sei~$p$ eine Primzahl und~$\ZZ_p$ die Menge all derjenigen rationalen Zahlen,
in deren vollständig gekürzter Bruchdarstellung der Nenner nicht durch~$p$
teilbar ist.
\begin{enumerate}
\item Zeige, dass~$\ZZ_p$ ein Unterring von~$\QQ$ ist.
\item Welche Elemente von~$\ZZ_p$ sind invertierbar?
\end{enumerate}
\end{aufgabe}

\begin{aufgabeE}{2+2}{Beispiele für Ganzheitsringe}
\item[S a)] Zeige: $\O_{\QQ(\i)} = \ZZ[\i]$.
\item[S b)] Sei~$\zeta$ eine primitive~$n$-te Einheitswurzel. Zeige:
$\O_{\QQ(\zeta)} = \ZZ[\zeta]$.
\end{aufgabeE}

\end{document}
