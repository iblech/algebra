\documentclass{algblatt}
\usepackage{multicol}
\loesungenfalse

\geometry{tmargin=2.0cm,bmargin=2.0cm,lmargin=2.25cm,rmargin=2.25cm}

%\setlength{\titleskip}{0.7em}
\setlength{\aufgabenskip}{1.0em}

\begin{document}

\vspace*{-1.5cm}
\maketitle{13}{Abgabe bis 15. Juli 2013, 17:00 Uhr}

\begin{aufgabe}{Konstruierbare~$n$-Ecke}
\begin{enumerate}
\item Für welche~$n \in \{ 1,\ldots,100 \}$ ist ein regelmäßiges~$n$-Eck
mit Zirkel und Lineal konstruierbar?
\item Gib eine Konstruktionsvorschrift für das regelmäßige~$15$-Eck an.
\end{enumerate}
\end{aufgabe}

\begin{aufgabe}{Fermatsche und Mersennesche Primzahlen}
\begin{enumerate}
\item Zeige für alle natürlichen Zahlen~$n \geq 0$: $F_{n+1} = 2 + F_n F_{n-1}
\cdots F_0$.
\item Zeige, dass~$F_m$ und~$F_n$ für~$m \neq n$ teilerfremd sind. Folgere
daraus, dass es unendlich viele Primzahlen gibt.
\item Eine \emph{Mersennesche Zahl} ist eine Zahl der Form~$M_n = 2^n -
1$. Zeige, dass~$M_n$ höchstens dann eine Primzahl ist, wenn~$n$ eine Primzahl
ist.
\item Zeige allgemeiner, dass~$M_n$ von~$M_d$ geteilt wird, wenn~$d$ ein
positiver Teiler von~$n$ ist.
\end{enumerate}
\end{aufgabe}

\begin{aufgabe}{Hauptsatz der Galoistheorie}
Bestimme alle Untergruppen der galoisschen Gruppe der Nullstellen des
Polynoms~$X^4 + 1$ und die zugehörigen Zwischenerweiterungen.
\end{aufgabe}

\begin{aufgabe}{Relative galoissch Konjugierte}
\begin{enumerate}
\item Finde zwei algebraische Zahlen, die über~$\QQ$ galoissch konjugiert sind,
über~$\QQ(\sqrt{3})$ aber nicht.
\item Seien~$K$ und~$L$ Koeffizientenbereiche mit~$L \supseteq K \supseteq
\QQ$ und~$x$ eine algebraische Zahl. Zeige, dass ein galoissch Konjugiertes
von~$x$ über~$L$ auch ein galoissch Konjugiertes von~$x$ über~$K$ ist.
\end{enumerate}
\end{aufgabe}

\begin{aufgabe}{Relative Galoisgruppen}
\begin{enumerate}
\item Finde ein normiertes separables Polynom mit rationalen Koeffizienten,
sodass die galoissche Gruppe seiner Nullstellen über~$\QQ$ gleich der
über~$\QQ(\sqrt[3]{5})$ ist.
\item Sei~$f \in K[X]$ ein normiertes separables Polynom und~$x_1,\ldots,x_n$
seine Nullstellen. \\ Sei~$y \in K(x_1,\ldots,x_n)$. Zeige:
\[ \Gal_{K(y)}(x_1,\ldots,x_n) = \{ \sigma \in \Gal_K(x_1,\ldots,x_n) \,|\,
\sigma \cdot y = y \}. \]
\end{enumerate}
\end{aufgabe}

\begin{aufgabe}{Zentrum einer Galoisgruppe}
Sei~$p$ eine Primzahl und~$x$ eine algebraische Zahl vom Grad~$p^n$. Seien alle
galoissch Konjugierten~$x_1 = x, x_2, \ldots, x_{p^n}$ von~$x$ in~$x$ rational.
\begin{enumerate}
\item Zeige, dass das Zentrum der galoisschen Gruppe der~$x_1,\ldots,x_{p^n}$ ein
Element~$\sigma$ der Ordnung~$p$ enthält.
\item Sei~$\sigma$ eine Permutation wie in~a) und~$y$ ein primitives Element
zu den Zahlen~$e_i(x_1,\sigma\cdot x_1,\ldots,\sigma^{p-1}\cdot x_1)$, $i =
1,\ldots,p$. Zeige, dass~$y$ vom Grad~$p^{n-1}$ ist.
\end{enumerate}
\end{aufgabe}

\end{document}
