\documentclass{algblatt}
\loesungenfalse

\geometry{tmargin=2cm,bmargin=2cm,lmargin=2.9cm,rmargin=2.9cm}

%\setlength{\titleskip}{0.7em}
\setlength{\aufgabenskip}{1.6em}

\begin{document}

\vspace*{-1.5cm}
\maketitle{4}{Abgabe bis 13. Mai 2013, 17:00 Uhr}

\begin{aufgabe}{Lage der Lösungen von Polynomengleichungen}
Sei $X^n + a_{n - 1} X^{n - 1} + \cdots + a_1 X + a_0 = 0$ eine normierte
Polynomgleichung mit komplexen Koeffizienten. Zeige, dass jede komplexe
Lösung~$z$ höchstens die Entfernung
$1+\max\{|a_0|,\ldots,|a_{n-1}|\}$ zum Ursprung hat.
\begin{loesung}
Sei~$R := |a_{n-1}| + \cdots + |a_0|$. Dann gilt für
alle~$x \in \RR$ mit~$|x| \geq R + 1$ folgende Hilfsüberlegung (wieso?):
\begin{align*}
  \left|a_{n-1} \frac{1}{x} + \cdots + a_1 \frac{1}{x^{n-1}} + a_0
  \frac{1}{x^n}\right| &\stackrel{\triangle}{\leq}
  |a_{n-1}| \frac{1}{|x|} + \cdots + |a_1| \frac{1}{|x|^{n-1}} + |a_0|
  \frac{1}{|x|^n} \\
  &\leq R \cdot \frac{1}{|x|}
  \leq \underbrace{\frac{R}{R+1}}_{=:\,q} < 1
\end{align*}
Somit gilt (wieso?) für alle~$x \geq R + 1$
\[ |f(x)| = |x|^n \left|1 + a_{n-1} \frac{1}{x} + \cdots + a_0 \frac{1}{x^n}\right|
  \geq |x|^n \left(1 - q\right) > 0, \]
insbesondere also~$f(x) \neq 0$.
\end{loesung}
\end{aufgabe}

\begin{aufgabe}{Stetigkeit von Polynomfunktionen}
Sei~$f:\CC\to\CC, z \mapsto a_nz^n + a_{n-1}z^{n-1} + \cdots + a_1z + a_0$ eine
Polynomfunktion mit Koeffizienten~$a_0,\ldots,a_n \in \CC$. Zeige, dass~$f$ in
folgendem starken Sinn stetig ist:
\[
  \forall C > 0\ 
  \forall \epsilon > 0\ 
  \exists \delta > 0 \ 
  \forall \text{$z,z' \in \CC$ mit $|z|,|z'| \leq C$}{:}\ \ 
  |z-z'| < \delta \Longrightarrow |f(z) - f(z')| < \epsilon
\]
\vspace{-2.0em}
\begin{loesung}
Sei~$R > 0$ beliebig. Sei~$\epsilon > 0$ beliebig. Wir setzen
\[ \delta := \epsilon \cdot \left[n R^{n-1} \cdot \Biggl(\sum_{i=0}^n |a_i| +
1\Biggr)\right]^{-1}, \]
dann gilt für alle~$z, w \in \CC$ mit $|z|, |w| \leq C$ und~$|z - w| < \delta$
zunächst die Abschätzung
\begin{align*}
  |z^i - w^i| &=
    |z - w| \cdot |z^{i-1} + z^{i-2}w + z^{i-3}w^2 + \cdots + zw^{i-2} + w^{i-1}| \\
  &\leq |z-w| \cdot (R^{i-1} + \cdots + R^{i-1}) \\
  &\leq n R^{n-1} |z-w|
\intertext{und daher folgt}
  |f(z) - f(w)| &=
    \Biggl|\sum_{i=0}^{n-1} a_i (z^i - w^i)\Biggr| \leq
    \sum_{i=0}^{n-1} |a_i| |z^i - w^i| \\
  &\leq
    \sum_{i=0}^{n-1} |a_i| \cdot n R^{n-1} |z-w| =
    n R^{n-1} \left(\sum_{i=0}^{n-1} |a_i|\right) \cdot |z - w| < \epsilon.
\end{align*}
\end{loesung}
\end{aufgabe}

\begin{aufgabe}{Rechenregeln}
\begin{enumerate}
\item Seien~$f$ und~$g$ Polynome mit~$\deg f \leq n$ und~$\deg g \leq m$.
Zeige, dass $\deg(f+g) \leq \max\{n,m\}$ und~$\deg(fg) \leq n+m$.
\item Beweise oder widerlege: Für alle Polynome~$f$ und Zahlen~$x,y$ gilt~$f(xy) = f(x) f(y)$.
\item Sei~$q$ eine komplexe Zahl ungleich Eins. Zeige: $\sum_{k=0}^n q^k =
(q^{n+1}-1)\,/\,(q-1).$
\end{enumerate}
\begin{loesungE}
\item Da~$\deg f \leq n$, gibt es Koeffizienten~$a_0,\ldots,a_n \in \CC$
sodass $f = \sum_{i=0}^n a_i X^i$. Analog gibt es Koeffizienten~$b_0,\ldots,b_m
\in \CC$ mit~$g = \sum_{j=0}^m b_j X^j$. Dann sehen wir: In der Summe~$f+g$
sind die Koeffizienten aller Monomome vom Grad~$> \max\{n,m\}$ und im
Produkt~$fg$ die aller Monome vom Grad~$> nm$ null. Das zeigt die Behauptung.

\emph{Bemerkung:} Wenn man von den Koeffizienten nicht entscheiden kann, ob sie
null oder nicht null sind (wie bei allgemeinen reellen oder komplexen Zahlen
der Fall), ist der Grad eines Polynoms keine wohldefinierte natürliche Zahl
(wieso?). Dem zusammengesetzten Ausdruck "`$\deg f \leq n$"' kann man aber
trotzdem einen Sinn verleihen, nämlich dass alle Koeffizienten von~$f$ zu
Monomen mit Grad echt größer als~$n$ null sind. In diesem Sinn ist die Aufgabe
zu verstehen.

\item Die Behauptung gilt fast nie. Ein einfaches Gegenbeispiel ist
\[ f(X) := X + 1, \quad x := 0, \quad y := 1, \]
denn dann ist
\[ f(xy) = f(0) = 1 \neq 2 = 1 \cdot 2 = f(0) f(1). \]

\item Wir rechnen:
\begin{align*}
  (q-1) \cdot (1 + q + q^2 + \cdots + q^n) &=
    \phantom{1\mathop{-}{}} q + q^2 + q^3 + \cdots + q^n + q^{n+1} \\
  &-1 - q - q^2 - q^3 - \cdots - q^n \\
  &= q^{n+1} - 1.
\end{align*}
Nach Division durch~$q-1$ steht die zu zeigende Identität da.
\end{loesungE}
\end{aufgabe}

\begin{aufgabe}{Teiler von Polynomen}
\begin{enumerate}
\item Ist~$X+\sqrt{2}$ ein Teiler von~$X^3-2X$?
\item Besitzt~$X^7 + 11 X^3 - 33 X + 22$ einen Teiler der
Form~$(X-a)(X-b)$ mit~$a,b \in \QQ$?
\item Sei $f = 3 X^4 - X^3 + X^2 - X + 1$ und $g = X^3 - 2 X + 1$. \\
Finde Polynome $q$ und $r$ mit $f = q g + r$ und
$\deg r < \deg g$.
\item Sei~$d$ ein gemeinsamer Teiler zweier Polynome~$f$ und~$g$ und seien~$p$
und~$q$ weitere Polynome.
Zeige, dass~$d$ dann auch ein Teiler von~$pf + qg$ ist.
\item Seien~$f$, $g$ und~$h$ Polynome mit ganzzahligen Koeffizienten und~$f =
gh$. Zeige, dass für jede ganze Zahl~$n$ die ganze Zahl~$g(n)$ ein Teiler
von~$f(n)$ ist.
\end{enumerate}
\begin{loesungE}
\item \emph{Variante 1:} Ja, denn~$-\sqrt{2}$ ist eine Nullstelle von~$X^3 -
2X$ (wieso?).

\emph{Variante 2:} Ja, denn es gilt:
$X^3 - 2X = X (X^2 - 2) = X (X - \sqrt{2}) (X + \sqrt{2})$.

\item Nach Blatt~0, Aufgabe~3b) und Blatt~1, Aufgabe~1 können
rationale Nullstellen des gegebenen Polynoms nur Teiler von~22 sein. Einsetzen
zeigt aber, dass keine der Zahlen
\[ \pm 1, \quad \pm 2, \quad \pm 11, \quad \pm 22 \]
Nullstellen sind. Also besitzt das Polynom keinerlei rationale Nullstellen und
daher auch keine Teiler der Form~$(X-a)(X-b)$ mit~$a,b \in \QQ$.

\item Polynomdivision liefert
\begin{align*}
  q &= 3X - 1, \\
  r &= 7X^2 - 6X + 2.
\end{align*}

\item Nach Voraussetzung gibt es Polynome~$\tilde f$ und~$\tilde g$ mit~$f = d
\tilde f$ und~$g = d \tilde g$. Damit folgt
\[ pf + qg = pf\tilde f + qd\tilde g = d \cdot (p\tilde f + q\tilde g), \]
also ist~$d$ tatsächlich ein Teiler von~$pf+qg$.

\item Für jede ganze Zahl~$n$ folgt $f(n) = g(n) \cdot h(n)$ (wieso?). Da~$h(n)$ eine
ganze Zahl ist (wieso?), zeigt das schon die Behauptung.
\end{loesungE}
\end{aufgabe}

\begin{aufgabe}{Polynomielle Ausdrücke}
\begin{enumerate}
\item Schreibe~$\frac{1}{\sqrt{2} + 5\sqrt{3}}$ als polynomiellen Ausdruck
in~$\sqrt{2}$ und~$\sqrt{3}$ mit rat. Koeffizienten.
\item Sei~$z$ eine komplexe Zahl mit~$\QQ(z) = \QQ[z]$. Zeige, dass~$z$
algebraisch ist.
\item Inwiefern kann man ein Polynom in zwei Unbestimmten~$X$ und~$Y$ als
Polynom in einer einzigen Unbestimmten~$Y$, dessen Koeffizienten
Polynome in~$X$ sind, auffassen?
\end{enumerate}
\begin{loesungE}
\item Wir bedienen uns desselben Tricks, den man auch beim Dividieren durch
komplexe Zahlen verwendet:
\[ \frac{1}{\sqrt{2} + 5\sqrt{3}} =
  \frac{\sqrt{2} - 5\sqrt{3}}{(\sqrt{2} + 5\sqrt{3}) (\sqrt{2} - 5\sqrt{3})} =
  \frac{\sqrt{2} - 5\sqrt{3}}{2 - 25 \cdot 3} =
  \frac{-1}{73} \sqrt{2} + \frac{5}{73} \sqrt{3}. \]

\item Wir beweisen die Behauptung zunächst für den Fall, dass~$z \neq 0$. Dann
ist nämlich $1/z$ ein Element von~$\QQ(z)$ und daher auch von~$\QQ[z]$; also
gibt es ein Polynom~$f(X)$ mit rationalen Koeffizienten und
$\frac{1}{z} = f(z)$. Dieses Polynom kann nicht das Nullpolynom sein (wieso?)
und hat daher mindestens Grad~0. Die Zahl~$z$ ist also Lösung der
Polynomgleichung
\[ f(X) \cdot X - 1 = 0 \]
mit rationalen Koeffizienten. Diese ist nichttrivial (wegen der Multiplikation
mit~$X$ ist ihr Grad mindestens~1) und enttarnt daher nach Normierung~$z$ als
algebraisch.

Nun wollen wir den allgemeinen Fall behandeln. In klassischer Logik ist das
einfach, denn da ist~$z$ null oder nicht null; im ersten Fall ist~$z$ sowieso
algebraisch, im zweiten Fall haben wir das gerade gesehen. Intuitionistisch ist
diese Fallunterscheidung nicht zulässig, trotzdem können wir den Beweis retten:
Denn auch konstruktiv gilt
\[ |z| > 0 \quad\text{oder}\quad |z| < 1. \]
Im ersten Fall folgt~$z \neq 0$ und daher die Algebraizität nach obigem
Argument. Im zweiten Fall ist~$z' := z + 1$ nicht null; wegen~$\QQ(z) =
\QQ(z')$ und~$\QQ[z] = \QQ[z']$ (wieso?) zeigt obige Argumentation, dass~$z'$
algebraisch ist. Also ist auch~$z = z' - 1$ algebraisch.
\end{loesungE}
\end{aufgabe}

\begin{aufgabe}{Beweis des Fundamentalsatzes}
Im Beweis des Fundamentalsatzes der Algebra tritt die Zahl~3 immer wieder auf.
Kann sie durch eine kleinere Zahl~$3-\epsilon$ ersetzt werden?
\end{aufgabe}

\end{document}

\begin{exercise}(1 Punkt)\newline
    Sei \(z_n\) eine konvergente Folge komplexer Zahlen. Warum ist
    \(\lim\limits_{n \to \infty} |z_n|
    = |\lim\limits_{n \to \infty} z_n|\)?
\end{exercise}

\begin{exercise}(2 Punkte)\newline
    Gib eine Abschätzung des Betrages aller komplexen Lösungen von
    \(X^4 - 2 X^3 + 5 X^2 - 4 X + 5\) von oben an.
\end{exercise}

\end{document} 
