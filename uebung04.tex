\documentclass{algblatt}
\usepackage{xstring}
\IfSubStr{\jobname}{\detokenize{loesung}}{\loesungentrue}{\loesungenfalse}

\geometry{tmargin=2cm,bmargin=2cm,lmargin=2.9cm,rmargin=2.9cm}

%\setlength{\titleskip}{0.7em}
\setlength{\aufgabenskip}{1.6em}

\begin{document}

\vspace*{-1.5cm}
\maketitle{4}{Abgabe bis 13. Mai 2013, 17:00 Uhr}

\begin{aufgabe}{Lage der Lösungen von Polynomgleichungen}
Sei $X^n + a_{n - 1} X^{n - 1} + \cdots + a_1 X + a_0 = 0$ eine normierte
Polynomgleichung mit komplexen Koeffizienten. Zeige, dass jede komplexe
Lösung~$z$ höchstens die Entfernung
$1+\max\{|a_0|,\ldots,|a_{n-1}|\}$ zum Ursprung hat.
\end{aufgabe}

\begin{aufgabe}{Stetigkeit von Polynomfunktionen}
Sei~$f:\CC\to\CC, z \mapsto a_nz^n + a_{n-1}z^{n-1} + \cdots + a_1z + a_0$ eine
Polynomfunktion mit Koeffizienten~$a_0,\ldots,a_n \in \CC$. Zeige, dass~$f$ in
folgendem starken Sinn stetig ist:
\[
  \forall R > 0\ 
  \forall \epsilon > 0\ 
  \exists \delta > 0 \ 
  \forall \text{$z,w \in \CC$ mit $|z|,|w| \leq R$}{:}\ \ 
  |z-w| < \delta \Longrightarrow |f(z) - f(w)| < \epsilon
\]
\vspace{-2.0em}
\end{aufgabe}

\begin{aufgabe}{Rechenregeln}
\begin{enumerate}
\item Seien~$f$ und~$g$ Polynome mit komplexen Koeffizienten und~$\deg f \leq
n$ und~$\deg g \leq m$.
Zeige, dass $\deg(f+g) \leq \max\{n,m\}$ und~$\deg(fg) \leq n+m$.
\item Beweise oder widerlege: Für alle Polynome~$f$ und Zahlen~$x,y$ gilt~$f(xy) = f(x) f(y)$.
\item Sei~$q$ eine komplexe Zahl ungleich Eins. Zeige: $\sum_{k=0}^n q^k =
(q^{n+1}-1)\,/\,(q-1).$
\end{enumerate}
\end{aufgabe}

\begin{aufgabe}{Teiler von Polynomen}
\begin{enumerate}
\item Ist~$X+\sqrt{2}$ ein Teiler von~$X^3-2X$?
\item Besitzt~$X^7 + 11\,X^3 - 33\,X + 22$ einen Teiler der
Form~$(X-a)(X-b)$ mit~$a,b \in \QQ$?
\item Sei $f = 3\,X^4 - X^3 + X^2 - X + 1$ und $g = X^3 - 2\,X + 1$. \\
Finde Polynome $q$ und $r$ mit $f = q g + r$ und
$\deg r < \deg g$.
\item Sei~$d$ ein gemeinsamer Teiler zweier Polynome~$f$ und~$g$ und seien~$p$
und~$q$ weitere Polynome.
Zeige, dass~$d$ dann auch ein Teiler von~$pf + qg$ ist.
\item Seien~$f$, $g$ und~$h$ Polynome mit ganzzahligen Koeffizienten und~$f =
g \cdot h$. Zeige, dass für jede ganze Zahl~$n$ die ganze Zahl~$g(n)$ ein Teiler
von~$f(n)$ ist.
\end{enumerate}
\end{aufgabe}

\begin{aufgabe}{Polynomielle Ausdrücke}
\begin{enumerate}
\item Schreibe~$\frac{1}{\sqrt{2} + 5\sqrt{3}}$ als polynomiellen Ausdruck
in~$\sqrt{2}$ und~$\sqrt{3}$ mit rat. Koeffizienten.
\item Sei~$z$ eine komplexe Zahl mit~$\QQ(z) = \QQ[z]$. Zeige, dass~$z$
algebraisch ist.
\item Inwiefern kann man ein Polynom in zwei Unbestimmten~$X$ und~$Y$ als
Polynom in einer einzigen Unbestimmten~$Y$, dessen Koeffizienten
Polynome in~$X$ sind, auffassen?
\end{enumerate}
\end{aufgabe}

\begin{aufgabe}{Beweis des Fundamentalsatzes}
Im Beweis des Fundamentalsatzes der Algebra tritt die Zahl~3 immer wieder auf.
Kann sie durch eine kleinere Zahl~$3-\epsilon$ ersetzt werden?
\end{aufgabe}

\end{document}

\begin{exercise}(1 Punkt)\newline
    Sei \(z_n\) eine konvergente Folge komplexer Zahlen. Warum ist
    \(\lim\limits_{n \to \infty} |z_n|
    = |\lim\limits_{n \to \infty} z_n|\)?
\end{exercise}

\begin{exercise}(2 Punkte)\newline
    Gib eine Abschätzung des Betrages aller komplexen Lösungen von
    \(X^4 - 2 X^3 + 5 X^2 - 4 X + 5\) von oben an.
\end{exercise}

\end{document} 
