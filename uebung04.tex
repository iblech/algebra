\documentclass{algblatt}
\usepackage{xstring}
\IfSubStr{\jobname}{\detokenize{loesung}}{\loesungentrue}{\loesungenfalse}

\geometry{tmargin=2cm,bmargin=2cm,lmargin=2.85cm,rmargin=2.85cm}

%\setlength{\titleskip}{0.7em}
\setlength{\aufgabenskip}{1.4em}

\begin{document}

\vspace*{-1.5cm}
\maketitle{4}{Abgabe bis 13. Mai 2013, 17:00 Uhr}

\begin{aufgabe}{Lage der Lösungen von Polynomgleichungen}
Sei $X^n + a_{n - 1} X^{n - 1} + \cdots + a_1 X + a_0 = 0$ eine normierte
Polynomgleichung mit komplexen Koeffizienten. Zeige, dass jede komplexe
Lösung~$z$ höchstens die Entfernung
$1+\max\{|a_0|,\ldots,|a_{n-1}|\}$ zum Ursprung hat.
\begin{loesung}
Sei~$R := \max\{|a_0|,\ldots,|a_{n-1}|\}$. Dann gilt für alle~$z \in \CC$
mit~$|z| > 1 + R$
\begin{align*}
  |f(z)| &= |z^n + a_{n-1}z^{n-1} + \cdots + a_1 z + a_0| \\
  &\geq |z^n| - |a_{n-1}z^{n-1} + \cdots + a_1 z + a_0| \\
  &\geq |z^n| - \left(|a_{n-1}| |z|^{n-1} + \cdots + |a_1| |z| + |a_0|\right) \\
  &\geq |z^n| - R \cdot \left(|z|^{n-1} + \cdots + |z| + 1\right) \\
  &= |z^n| - R \cdot \frac{|z|^n-1}{|z|-1} \\
  &> |z^n| - R \cdot \frac{|z|^n-1}{1+R-1} \\
  &= |z^n| - (|z|^n - 1) \\
  &= 1 > 0,
\end{align*}
insbesondere also~$f(z) \neq 0$. Somit hat jede Lösung der Polynomgleichung
Betrag $\leq 1 + R$.

\emph{Schlechtere Variante.}
Für alle alle~$z \in \CC$ mit~$|z| > 1 + R$ gilt folgende Hilfsüberlegung (wieso?):
\begin{align*}
  \left|a_{n-1} \frac{1}{z} + \cdots + a_1 \frac{1}{z^{n-1}} + a_0
  \frac{1}{z^n}\right| &\stackrel{\triangle}{\leq}
  |a_{n-1}| \frac{1}{|z|} + \cdots + |a_1| \frac{1}{|z|^{n-1}} + |a_0|
  \frac{1}{|z|^n} \\
  &\leq R \cdot \sum_{i=1}^n \left(\frac{1}{|z|}\right)^i
  \leq R \cdot \sum_{i=1}^\infty \left(\frac{1}{|z|}\right)^i \\
  &= R \cdot \frac{|z|^{-1}}{1 - |z|^{-1}} =
  \underbrace{\frac{R}{|z| - 1}}_{=:\,q} < \frac{R}{(1+R)-1} = 1
\end{align*}
Dabei haben wir die abgeschnittene geometrische Reihe mit der Formel
\[ \sum_{i=1}^\infty p^n = \frac{1}{1 - p} - 1 =
  \frac{p}{1-p} \]
behandelt. Somit gilt (wieso?) für alle~$|z| > 1 + R$
\[ |f(z)| = |z|^n \left|1 + a_{n-1} \frac{1}{z} + \cdots + a_0 \frac{1}{z^n}\right|
  \geq |z|^n \left(1 - q\right) > 0, \]
insbesondere also~$f(z) \neq 0$.

Die Idee ist also: Über die Terme der Ordnung echt kleiner als~$n$ können wir
wenig aussagen. Aber diese Terme werden vom Monom~$z^n$ dominiert -- ausnutzen
können wir das dadurch, indem wir es ausklammern.

\emph{Bemerkung:} Mit nur wenig Mehraufwand sieht man, dass die Lösungen sogar
in der \emph{offenen} Kreisscheibe mit Radius~$1 + R$ liegen müssen: Bei der
ersten Lösung hat man nämlich für~$|z| \geq 1 + R$ trotzdem noch~$|f(z)| \geq 1
> 0$, und bei der zweiten hat man beim Übergang von der endlichen zur
> unendlichen geometrischen Reihe auch schon~"`$<$"' statt nur~"`$\leq$"'.

\emph{Bemerkung:} Die in dieser Aufgabe gegebene Schranke ist \emph{asymptotisch
scharf}: Etwa ist für~$a > 0$ die Schranke zur Polynomgleichung
\[ \left(X - a\right) \left(X + \frac{a}{1+a}\right) =
  X^2 - \frac{a^2}{1 + a} X - \frac{a^2}{1+a} = 0 \]
durch
\[ 1 + R = 1 + \frac{a^2}{1 + a} \]
gegeben. Für~$a \to \infty$ nähert sich diese dem tatsächlichen größten
Abstand einer Lösung zum Ursprung -- also~$a$ -- beliebig genau an:
\[ 1 + R - a = \frac{1}{1 + a} \xrightarrow{a \to \infty} 0. \]
\end{loesung}
\end{aufgabe}

\begin{aufgabe}{Stetigkeit von Polynomfunktionen}
Sei~$f:\CC\to\CC, z \mapsto a_nz^n + a_{n-1}z^{n-1} + \cdots + a_1z + a_0$ eine
Polynomfunktion mit Koeffizienten~$a_0,\ldots,a_n \in \CC$. Zeige, dass~$f$ in
folgendem starken Sinn stetig ist:
\[
  \forall R > 0\ 
  \forall \epsilon > 0\ 
  \exists \delta > 0 \ 
  \forall \text{$z,w \in \CC$ mit $|z|,|w| \leq R$}{:}\ \ 
  |z-w| < \delta \Longrightarrow |f(z) - f(w)| < \epsilon
\]
\vspace{-2.0em}
\begin{loesung}
Sei~$R > 0$ beliebig. Sei~$\epsilon > 0$ beliebig. Wir setzen
\[ \delta := \epsilon \cdot \left[n R^{n-1} \cdot \Biggl(\sum_{i=0}^n |a_i| +
1\Biggr)\right]^{-1}. \]
Wegen der Addition von~1 kann hier keine Division durch Null auftreten.
Ohne Einschränkung der Allgemeinheit sei außerdem~$R \geq 1$ (dann ist die
Funktion~$t \mapsto R^t$ monoton steigend).
Dann gilt für alle~$z, w \in \CC$ mit $|z|, |w| \leq R$ und~$|z - w| < \delta$
zunächst die Abschätzung
\begin{align*}
  |z^i - w^i| &=
    |z - w| \cdot |z^{i-1} + z^{i-2}w + z^{i-3}w^2 + \cdots + zw^{i-2} + w^{i-1}| \\
  &\leq |z-w| \cdot (R^{i-1} + \cdots + R^{i-1}) \\
  &\leq i R^{i-1} \cdot |z-w| \leq n R^{n-1} \cdot |z-w|
\intertext{für alle~$i = 0, \ldots, n$ und daher folgt}
  |f(z) - f(w)| &=
    \Biggl|\sum_{i=0}^n a_i (z^i - w^i)\Biggr| \leq
    \sum_{i=0}^n |a_i| |z^i - w^i| \\
  &\leq
    \sum_{i=0}^n |a_i| \cdot n R^{n-1} \cdot |z-w| =
    n R^{n-1} \left(\sum_{i=0}^n |a_i|\right) \cdot |z - w| < \epsilon.
\end{align*}

\emph{Bemerkung:} Man kann in der Hilfsüberlegung auch auf die
letzte Abschätzung~$i R^{i-1} \leq n R^{n-1}$ verzichten. Dann kann man~$\delta$
sogar etwas kleiner wählen (das ist besser), nämlich als
\[ \delta = \epsilon \cdot \left[
  \sum_{i=0}^n i |a_i| R^{i-1} + 1
\right]^{-1}, \]
und muss auch nicht die Einschränkung~$R \geq 1$ annehmen:
\begin{align*}
  |f(z) - f(w)| &=
    \Biggl|\sum_{i=0}^n a_i (z^i - w^i)\Biggr| \leq
    \sum_{i=0}^n |a_i| |z^i - w^i| \\
  &\leq
    \sum_{i=0}^n |a_i| \cdot i R^{i-1} \cdot |z-w| =
    \left(\sum_{i=0}^n i |a_i| R^{i-1}\right) \cdot |z-w| < \epsilon.
\end{align*}

\emph{Bemerkung:} In Worten lautet die Behauptung, dass die Polynomfunktion~$f$
auf jeder abgeschlossenen Kreisscheibe um den Ursprung gleichmäßig stetig ist.
Der Vorteil der hier gegebenen Lösung gegenüber einem Verweis auf eine
Analysis-Vorlesung ist, dass wir die $\epsilon$-Abhängigkeit von~$\delta$
explizit angeben konnten.
\end{loesung}
\end{aufgabe}

\begin{aufgabe}{Rechenregeln}
\begin{enumerate}
\item Seien~$f$ und~$g$ Polynome mit komplexen Koeffizienten und~$\deg f \leq
n$ und~$\deg g \leq m$.
Zeige, dass $\deg(f+g) \leq \max\{n,m\}$ und~$\deg(fg) \leq n+m$.
\item Beweise oder widerlege: Für alle Polynome~$f$ und Zahlen~$x,y$ gilt~$f(xy) = f(x) f(y)$.
\item Sei~$q$ eine komplexe Zahl ungleich Eins. Zeige: $\sum_{k=0}^n q^k =
(q^{n+1}-1)\,/\,(q-1).$
\end{enumerate}
\begin{loesungE}
\item Da~$\deg f \leq n$, gibt es Koeffizienten~$a_0,\ldots,a_n \in \CC$
sodass $f = \sum_{i=0}^n a_i X^i$. Analog gibt es Koeffizienten~$b_0,\ldots,b_m
\in \CC$ mit~$g = \sum_{j=0}^m b_j X^j$. Dann sehen wir: In der Summe~$f+g$
sind die Koeffizienten aller Monomome vom Grad~$> \max\{n,m\}$ und im
Produkt~$fg$ die aller Monome vom Grad~$> nm$ null. Das zeigt die Behauptung.

\emph{Bemerkung:} Abgerundet wird die Aufgabe durch Beispiele, wo man sieht,
wie sich bei der Addition die höchsten Potenzen wegheben oder nicht wegheben.

\emph{Bemerkung:} Wenn man von den Koeffizienten nicht entscheiden kann, ob sie
null oder nicht null sind (wie bei allgemeinen reellen oder komplexen Zahlen
der Fall), ist der Grad eines Polynoms keine wohldefinierte natürliche Zahl
(wieso?). Dem zusammengesetzten Ausdruck "`$\deg f \leq n$"' kann man aber
trotzdem einen Sinn verleihen, nämlich dass alle Koeffizienten von~$f$ zu
Monomen mit Grad echt größer als~$n$ null sind. In diesem Sinn ist die Aufgabe
zu verstehen.

\item Die Behauptung gilt fast nie. Ein einfaches Gegenbeispiel ist
\[ f(X) := X + 1, \quad x := 0, \quad y := 1, \]
denn dann ist
\[ f(xy) = f(0) = 1 \neq 2 = 1 \cdot 2 = f(0) f(1). \]

Ein noch einfacheres Gegenbeispiel ist
\[ f(X) := 2, \quad x := 0, \quad y := 1, \]
denn dann ist
\[ f(xy) = f(0) = 2 \neq 4 = 2 \cdot 2 = f(0) f(1). \]

\item Wir rechnen:
\begin{align*}
  (q-1) \cdot (1 + q + q^2 + \cdots + q^n) &=
    \phantom{1\mathop{-}{}} q + q^2 + q^3 + \cdots + q^n + q^{n+1} \\
  &-1 - q - q^2 - q^3 - \cdots - q^n \\
  &= q^{n+1} - 1.
\end{align*}
Nach Division durch~$q-1$ steht die zu zeigende Identität da.

\emph{Bemerkung:} Wem die Auslassungszeichen unlieb sind, kann auch einen
Induktionsbeweis führen.
\end{loesungE}
\end{aufgabe}

\begin{aufgabe}{Teiler von Polynomen}
\begin{enumerate}
\item Ist~$X+\sqrt{2}$ ein Teiler von~$X^3-2X$?
\item Besitzt~$X^7 + 11\,X^3 - 33\,X + 22$ einen Teiler der
Form~$(X-a)(X-b)$ mit~$a,b \in \QQ$?
\item Sei $f = 3\,X^4 - X^3 + X^2 - X + 1$ und $g = X^3 - 2\,X + 1$. \\
Finde Polynome $q$ und $r$ mit $f = q g + r$ und
$\deg r < \deg g$.
\item Sei~$d$ ein gemeinsamer Teiler zweier Polynome~$f$ und~$g$ und seien~$p$
und~$q$ weitere Polynome.
Zeige, dass~$d$ dann auch ein Teiler von~$pf + qg$ ist.
\item Seien~$f$, $g$ und~$h$ Polynome mit ganzzahligen Koeffizienten und~$f =
g \cdot h$. Zeige, dass für jede ganze Zahl~$n$ die ganze Zahl~$g(n)$ ein Teiler
von~$f(n)$ ist.
\end{enumerate}
\begin{loesungE}
\item \emph{Variante 1:} Ja, denn~$-\sqrt{2}$ ist eine Nullstelle von~$X^3 -
2X$ (wieso?).

\emph{Variante 2:} Ja, denn es gilt:
$X^3 - 2X = X (X^2 - 2) = X (X - \sqrt{2}) (X + \sqrt{2})$.

\emph{Variante 3:} Ja, denn eine Nebenrechnung zeigt, dass die Polynomdivision
von~$X^3 - 2X$ durch~$X+\sqrt{2}$ keinen Rest lässt.

\item Nach Blatt~0, Aufgabe~3b) und Blatt~1, Aufgabe~1 können
rationale Nullstellen des gegebenen Polynoms nur Teiler von~22 sein. Einsetzen
zeigt aber, dass keine der Zahlen
\[ \pm 1, \quad \pm 2, \quad \pm 11, \quad \pm 22 \]
Nullstellen sind. Also besitzt das Polynom keinerlei rationale Nullstellen und
daher auch keine Teiler der Form~$(X-a)(X-b)$ mit~$a,b \in \QQ$.

\item Polynomdivision liefert
\begin{align*}
  q &= 3X - 1, \\
  r &= 7X^2 - 6X + 2.
\end{align*}

\item Nach Voraussetzung gibt es Polynome~$\tilde f$ und~$\tilde g$ mit~$f = d
\tilde f$ und~$g = d \tilde g$. Damit folgt
\[ pf + qg = pf\tilde f + qd\tilde g = d \cdot (p\tilde f + q\tilde g), \]
also ist~$d$ tatsächlich ein Teiler von~$pf+qg$.

\item Für jede ganze Zahl~$n$ folgt $f(n) = g(n) \cdot h(n)$ (wieso?). Da~$h(n)$ eine
ganze Zahl ist (wieso?), zeigt das schon die Behauptung.
\end{loesungE}
\end{aufgabe}

\begin{aufgabe}{Polynomielle Ausdrücke}
\begin{enumerate}
\item Schreibe~$\frac{1}{\sqrt{2} + 5\sqrt{3}}$ als polynomiellen Ausdruck
in~$\sqrt{2}$ und~$\sqrt{3}$ mit rationalen Koeffizienten.
\item Sei~$z$ eine komplexe Zahl mit~$\QQ(z) = \QQ[z]$. Zeige, dass~$z$
algebraisch ist.
\item Inwiefern kann man ein Polynom in zwei Unbestimmten~$X$ und~$Y$ als
Polynom in einer einzigen Unbestimmten~$Y$, dessen Koeffizienten
Polynome in~$X$ sind, auffassen?
\end{enumerate}
\begin{loesungE}
\item Wir bedienen uns desselben Tricks, den man auch beim Dividieren durch
komplexe Zahlen verwendet:
\[ \frac{1}{\sqrt{2} + 5\sqrt{3}} =
  \frac{\sqrt{2} - 5\sqrt{3}}{(\sqrt{2} + 5\sqrt{3}) (\sqrt{2} - 5\sqrt{3})} =
  \frac{\sqrt{2} - 5\sqrt{3}}{2 - 25 \cdot 3} =
  \frac{-1}{73} \sqrt{2} + \frac{5}{73} \sqrt{3}. \]

\emph{Bemerkung:} Im neunjährigen Gymnasium war diese Technik unter dem Titel
\emph{Nenner rational machen} bekannt.

\emph{Bemerkung:} Mit ein wenig Galoistheorie kann man verstehen, wie man
bei komplizierteren Nennern verfahren kann: Der Nenner~$x$ ist in diesem Fall
Element einer gewissen endlichen Erweiterung von~$\QQ$. Wenn wir die Elemente
ihrer Galoisgruppe mit~$\sigma_1, \ldots, \sigma_n$ bezeichnen, wobei
wir~$\sigma_1 = \id$ setzen, so können wir den Bruch mit $\sigma_2(x) \cdots
\sigma_n(x)$ erweitern. Der erweiterte Nenner ist dann
\[ \sigma_1(x) \cdots \sigma_n(x), \]
augenscheinlich invariant unter der Wirkung der~$\sigma_i$, und somit
tatsächlich rational.

\item Wir beweisen die Behauptung zunächst für den Fall, dass~$z \neq 0$. Dann
ist nämlich $1/z$ ein Element von~$\QQ(z)$ und daher auch von~$\QQ[z]$; also
gibt es ein Polynom~$f(X)$ mit rationalen Koeffizienten und
$\frac{1}{z} = f(z)$. Dieses Polynom kann nicht das Nullpolynom sein (wieso?)
und hat daher mindestens Grad~0. Die Zahl~$z$ ist also Lösung der
Polynomgleichung
\[ f(X) \cdot X - 1 = 0 \]
mit rationalen Koeffizienten. Diese ist nichttrivial (wegen der Multiplikation
mit~$X$ ist ihr Grad mindestens~1) und enttarnt daher nach Normierung~$z$ als
algebraisch.

Nun wollen wir den allgemeinen Fall behandeln. In klassischer Logik ist das
einfach, denn da ist~$z$ null oder nicht null; im ersten Fall ist~$z$ sowieso
algebraisch, im zweiten Fall haben wir das gerade gesehen. Intuitionistisch ist
diese Fallunterscheidung nicht zulässig, trotzdem können wir den Beweis retten:
Denn auch konstruktiv gilt
\[ |z| > 0 \quad\text{oder}\quad |z| < 1. \]
Im ersten Fall folgt~$z \neq 0$ und daher die Algebraizität nach obigem
Argument. Im zweiten Fall ist~$z' := z + 1$ nicht null; wegen~$\QQ(z) =
\QQ(z')$ und~$\QQ[z] = \QQ[z']$ (wieso?) zeigt obige Argumentation, dass~$z'$
algebraisch ist. Also ist auch~$z = z' - 1$ algebraisch.

\emph{Bemerkung:} Die Inklusion~$\QQ[z] \subseteq \QQ(z)$ besteht stets. Die
Umkehrung der Behauptung der Aufgabe gilt ebenfalls und wird im Skript
unmittelbar vor Folgerung~2.3 bewiesen. Zusammengefasst ist die Beweisidee
dort folgende: Die Nenner~$x$ von Zahlen aus~$\QQ(z)$ sind algebraisch, daher sind
ihre Inversen polynomielle Ausdrücke in~$x$ und damit auch in~$z$.

\item Ein Polynom aus~$\QQ[X,Y]$ kann man einfach dadurch als Polynom
in~$(\QQ[X])[Y]$ auffassen, indem man seine Terme nach~$Y$-Potenzen
umgruppiert, zum Beispiel so:
\[ 3\,X^2Y - 5\,XY - 8\,X + 4\,Y + 5 =
  (3\,X^2 - 5\,X + 4) Y^1 + (-8\,X + 5) Y^0. \]

\emph{Bemerkung:} Etwas präziser kann man mittels dieser Idee einen
\emph{Ringisomorphismus}~$\QQ[X,Y] \to (\QQ[X])[Y]$ angeben. Für formale
Potenzreihen stimmt die analoge Aussage ebenfalls.
\end{loesungE}
\end{aufgabe}

\begin{aufgabe}{Beweis des Fundamentalsatzes}
Im Beweis des Fundamentalsatzes der Algebra tritt die Zahl~3 immer wieder auf.
Kann sie durch eine kleinere Zahl~$3-\epsilon$ ersetzt werden?
\begin{loesung}
Wir erinnern an die grobe Struktur des Beweises des Fundamentalsatzes:
Ausgehend von der (möglicherweise sehr schlechten) Näherung~$0$ als Lösung der
Polynomgleichung
\[ f(X) = X^n + b_{n-1}X^{n-1} + \cdots + b_1 X + b_0 = 0 \]
konstruieren wir eine erste bessere Näherung~$z$. Diese ist insofern besser,
als dass sie die Abschätzung
\begin{equation}\label{absch}
  |z^n + b_{n-1}z^{n-1} + \cdots + b_1 z + b_0| \leq q |b_0|
\end{equation}
erfüllt, wobei~$q := 1 - \frac{1}{2} \cdot 3^{1 - n^2}$ ein fester Faktor
kleiner als Eins ist. (Zum Vergleich: Für die Näherung~$0$ ergibt sich als
Abstand~$|f(0)| = |b_0|$.) Aus dieser besseren Näherung konstruieren wir dann
eine abermals verbesserte Näherung; sukzessive erhalten wir so eine Folge immer
besser werdender Näherungslösungen, deren Grenzwert eine tatsächliche Lösung
der Gleichung ist.

Wenn man nun die Zahl~$3$ in der Definition von~$q$ durch eine kleinere Zahl
ersetzen könnte, hätte das folgenden Vorteil: Der Faktor~$q$ wäre dann kleiner
(wieso?) und damit die Konvergenz schneller -- zumindest, wenn die
Abschätzung~\eqref{absch} nicht zu pessimistisch ist, sondern ein realistisches
Bild der Konvergenzgeschwindigkeit vermittelt. (Manche numerische Verfahren
konvergieren in der Realität viel schneller als naive Abschätzungen vermuten
lassen.)

Um nun die Frage zu klären, ob man~$3$ durch eine kleinere Zahl ersetzen kann,
müssen wir den gesamten Beweis des Fundamentalsatzes durchgehen und bei jedem
Vorkommen von~$3$ prüfen, ob der jeweilige Schritt auch bei einer kleineren
Zahl durchgeht. Bei einer solchen Analyse stellt man fest: Für den Großteil des
Beweises spielt der Wert dieser Konstanten keine Rolle (solange er nur positiv
ist), erst bei den Abschätzungen~(1.47), (1.48), (1.49) und der finalen
Abschätzung (auf Seite~41) wird es kritisch.

Wir wollen unseren kleineren Ersatz für die Zahl~$3$ mit~"`$\widetilde 3$"'
bezeichnen. Entsprechend setzen wir~$\widetilde 2 := \widetilde 3 - 1$. Die
veränderten Abschätzungen lauten dann:
\begin{align}
  m(r) &\geq \widetilde 3^{k^2 - n^2} |b_0| \tag{1.47} \\
  \sum_{i=1}^{k-1} f_i(r) &\leq \frac{1 - \widetilde 3^{1-k}}{\widetilde 2} f_k(r) \tag{1.48} \\
  \sum_{i=k+1}^n f_i(r) &\leq \frac{1 - \widetilde 3^{k-n}}{\widetilde 2} f_k(r) \tag{1.49} \\
  \sum_{i\neq0,k} f_i(r) &\leq
    \left(\frac{2}{\widetilde 2} - \frac{\widetilde 3^{1-k}}{\widetilde
    2}\right) f_k(r) \nonumber
\intertext{Interessant wird es, wenn wir uns der finalen Abschätzung zuwenden:}
  \label{abschfinal}
  |f(z)| &\leq
    |b_0| - \left(\frac{\widetilde 3^{1-k}}{\widetilde 2} + 1 -
    \frac{2}{\widetilde 2}\right) m(r)
\end{align}
Auf der rechten Seite wollen wir nun mittels Abschätzung~(1.47) weiter nach
oben abschätzen. Dazu muss aber der Vorfaktor~$\left(\cdots\right)$ positiv
sein; eine kleine Nebenrechung zeigt, dass das genau dann der Fall ist, wenn
\begin{equation}\label{absch2}
  3 - \widetilde 3 < \widetilde 3^{1-k}
\end{equation}
für alle~$k \in J \subseteq \{ 1, \ldots, n \}$. Da unsere neue
Konstante~$\widetilde 3$ schon unabhängig von~$J$ sein soll (denn diese Menge
kann sich in jedem Iterationsschritt des Näherungsverfahrens ändern), müssen
wir daher fordern, dass die Abschätzung sogar im schlimmstmöglichen Fall~$k = n$ gilt:
\begin{equation}\label{absch3}
  3 - \widetilde 3 < \widetilde 3^{1-n}.
\end{equation}
Denn für~$k = n$ ist die rechte Seite der Ungleichung~\eqref{absch2} am
kleinsten. Wenn wir unsere neue Konstante~$\widetilde 3$ also so einschränken,
dass sie Abschätzung~\eqref{absch3} erfüllt, können wir
Ungleichung~\eqref{abschfinal} fortführen:
\[
  |f(z)| \leq |b_0| \cdot \left(1 - \frac{\widetilde 3^{1-k+k^2-n^2}}{\widetilde 2}
    + \frac{\widetilde 3^{k^2-n^2}}{\widetilde 2} \cdot (3 - \widetilde 3)\right). \]
Für jedes~$k$ muss nun der Faktor~$(\cdots)$ auf der rechten Seite
echt kleiner als~$1$ sein (er wird unser neues~$q$). Eine Nebenrechnung zeigt, dass das genau dann der
Fall ist, wenn~$\widetilde 3$ die Abschätzung~\eqref{absch3} erfüllt, was wir
ja sowieso voraussetzten mussten.

Als Fazit können wir also festhalten: Für festen Polynomgrad~$n$ kann die
Konstante~$3$ in der Tat durch jede kleinere positive Zahl~$\widetilde 3$
ersetzt werden, die noch Abschätzung~\eqref{absch3} erfüllt. Konkret ergeben
sich folgende Möglichkeiten (aufgerundete Werte):
\begin{center}\begin{tabular}{r|l}
  $n$ & kleinstmögliche Konstante~$\widetilde 3$ \\ \hline
  $1$ & beliebig wenig mehr als exakt~$2$ \\
  $2$ & $2{,}619$ \\
  $3$ & $2{,}880$ \\
  $4$ & $2{,}962$ \\
  $5$ & $2{,}988$ \\
  $6$ & $2{,}996$
\end{tabular}\end{center}
Es gibt aber keine bessere Konstante, die für alle Polynomgrade~$n$ gleichmäßig
funktionieren würde: Denn im Grenzwert~$n \to \infty$ lautet
Abschätzung~\eqref{absch3}
\[ 3 - \widetilde 3 \leq 0. \]
\end{loesung}
\end{aufgabe}

\end{document}

\begin{exercise}(1 Punkt)\newline
    Sei \(z_n\) eine konvergente Folge komplexer Zahlen. Warum ist
    \(\lim\limits_{n \to \infty} |z_n|
    = |\lim\limits_{n \to \infty} z_n|\)?
\end{exercise}

\begin{exercise}(2 Punkte)\newline
    Gib eine Abschätzung des Betrages aller komplexen Lösungen von
    \(X^4 - 2 X^3 + 5 X^2 - 4 X + 5\) von oben an.
\end{exercise}

\end{document} 
