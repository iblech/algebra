\documentclass{algblatt}
\loesungenfalse

\geometry{tmargin=2cm,bmargin=2cm,lmargin=2.9cm,rmargin=2.9cm}

%\setlength{\titleskip}{0.7em}
\setlength{\aufgabenskip}{1.6em}

\begin{document}

\vspace*{-1.5cm}
\maketitle{5}{Abgabe bis 21. Mai 2013, 12:00 Uhr}

\begin{aufgabe}{Elementarsymmetrische Funktionen}
\begin{enumerate}
\item Gib~$e_2(X,Y,Z,U,V)$, also
die zweite elementarsymmetrische Funktion in den fünf Unbestimmten
$X$, $Y$, $Z$, $U$ und $V$, explizit an.
\item Schreibe~$X^2 + Y^2 + Z^2$ als Polynom in den~$e_i(X,Y,Z)$.
\item Schreibe~$X_1^2 + \cdots + X_n^2$ als Polynom in
den~$e_i(X_1,\ldots,X_n)$.
\item Zeige, dass $e_k(\underbrace{1, \dots, 1}_{\text{$n$ Argumente}}) = \binom{n}{k}$.
\end{enumerate}
\end{aufgabe}

\begin{aufgabe}{Der Vietasche Satz}
\begin{enumerate}
\item Sei $X^4 + a_3 X^3 + a_2 X^2 + a_1 X + a_0 = 0$ eine normierte Polynomgleichung
vierten Grades, deren Lösungen mit Vielfachheiten $x_1$, $x_2$, $x_3$
und $x_4$ seien. Drücke die Koeffizienten $a_0$, $a_1$, $a_2$ und
$a_3$ explizit als Polynome in den $x_i$ aus.
\item Verwende den Vietaschen Satz für~$n = 2$ um die bekannte Lösungsformel
für normierte quadratische Gleichungen herzuleiten.
\end{enumerate}
\end{aufgabe}

\begin{aufgabe}{Diskriminanten kubischer Gleichungen}
\begin{enumerate}
\item Finde eine normierte Polynomgleichung dritten Grades, welche~$1$
als zweifache Lösung, $2$ als einfache Lösung und keine weiteren Lösungen
besitzt. Was ist ihre Diskriminante?
\item Sei~$X^3 + p X + q = 0$ eine allgemeine reduzierte kubische Gleichung.
Zeige, dass ihre Diskrimante durch $-4 p^3 - 27 q^2$ gegeben ist.
\end{enumerate}
\end{aufgabe}

\begin{aufgabe}{Symmetrien eines Polynoms}
Sei $f(X, Y, Z, W) := X Y + Z W + X Y Z W$.
Wie viele vierstellige Permutationen~$\sigma$ gibt es, so dass
$\sigma \cdot f = f$?
\end{aufgabe}

\begin{aufgabe}{Formale Ableitung von Polynomen}
\begin{enumerate}
\item Seien~$g(X)$ und~$h(Y)$ Polynome. Zeige:
$
  (g h)^{(k)}(X)
  = \sum_{i + j = k} \binom k i \, g^{(i)}(X) \, h^{(j)}(X).
$

\item Sei $f(X)$ ein Polynom und~$n$ eine natürliche Zahl. Zeige:
\[ f^{(n + 1)} = 0 \quad\Longleftrightarrow\quad \deg f \leq n. \]

\item Sei $f(X)$ ein Polynom und $x$ eine komplexe Zahl. Zeige, dass die
Entwicklung von $f(X)$ nach $X - x$ durch die \emph{Taylorsche Formel} gegeben
ist (nach Brooke Taylor, 1685--1731, britischer Mathematiker):
\[
    f(X) = \sum_{k = 0}^\infty \frac{f^{(k)}(x)}{k!} \, (X - x)^k
\]
\end{enumerate}
\end{aufgabe}

\end{document} 

\begin{exercise}(2 Punkte)\newline
    Gib eine normierte Polynomgleichung dritten Grades an, welche \(1\)
    als zweifache Lösung, \(2\) als einfache Lösung und keine weiteren Lösungen
    besitzt.
\end{exercise}

\begin{exercise}(3 Punkte)\newline
    Warum sind beide Seiten von 
\begin{equation}
 (gh)'(X)=g'(X)h(X)+g(X)h'(X)
\end{equation}
 sowohl in
    \(g(X)\) als auch in \(h(X)\) linear und warum reicht es daher, die
    Gleichung (1) nur für \(g(X) = X^m\) und \(h(X) = X^n\)
    nachzurechnen?
\end{exercise}
 
\end{document} 
