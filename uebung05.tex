\documentclass{algblatt}
\loesungenfalse

\geometry{tmargin=2cm,bmargin=2cm,lmargin=2.9cm,rmargin=2.9cm}

%\setlength{\titleskip}{0.7em}
\setlength{\aufgabenskip}{1.6em}

\begin{document}

\vspace*{-1.5cm}
\maketitle{5}{Abgabe bis 22. Mai 2013, 12:00 Uhr}

\begin{aufgabe}{Elementarsymmetrische Funktionen}
\begin{enumerate}
\item Gib~$e_2(X,Y,Z,U,V)$, also
die zweite elementarsymmetrische Funktion in den fünf Unbestimmten
$X$, $Y$, $Z$, $U$ und $V$, explizit an.
\item Schreibe~$X^2 + Y^2 + Z^2$ als polynomiellen Ausdruck in den~$e_i(X,Y,Z)$.
\item Schreibe~$X_1^2 + \cdots + X_n^2$ als polynomiellen Ausdruck in
den~$e_i(X_1,\ldots,X_n)$.
\item Zeige, dass $e_k(\underbrace{1, \dots, 1}_{\text{$n$ Argumente}}) = \binom{n}{k}$.
\end{enumerate}
\begin{loesungE}
\item $e_2(X,Y,Z,U,V) = XY + XZ + XU + XV + YZ + YU + YV + ZU + ZV + UV.$
\item $X^2 + Y^2 + Z^2 = (X + Y + Z)^2 - 2\,XY - 2\,XZ - 2\,YZ =
e_1(X,Y,Z)^2 - 2\,e_2(X,Y,Z).$
\item $X_1^2 + \cdots + X_n^2 = (X_1 + \cdots + X_n)^2 -
2 \sum\limits_{1 \leq i < j \leq n} X_i X_j = e_1(X_1,\ldots,X_n)^2 -
2\,e_2(X_1,\ldots,X_n).$
\item \emph{Variante 1:} $e_k(1,\ldots,1) = \sum_{1 \leq i_1 < \cdots < i_k
\leq n} 1 = \binom{n}{k}$, denn der Summand~$1$ wird genau so oft summiert, wie
es Möglichkeiten gibt, aus den Zahlen~$\{1,\ldots,n\}$ genau~$k$ Stück als
Indizes auszuwählen.

\emph{Variante 2:} Vielleicht kennt man die Formel
\[ \prod_{i=1}^n (1 + X_i\,T) = \sum_{k=0}^n e_k(X_1,\ldots,X_n) \, T^k. \]
Setzt man in dieser Identität alle~$X_i$ auf~$1$, folgt die Behauptung sofort
mit dem binomischen Lehrsatz
\[ \prod_{i=1}^n (1 + T) = (1 + T)^n = \sum_{k=0}^n \binom{n}{k} T^k \]
und Koeffizientenvergleich.

\emph{Variante 2':} Auf ähnliche Art und Weise kann man auch die Formel
\[ \prod_{i=1}^n (T + X_i) = \sum_{k=0}^n e_{n-k}(X_1,\ldots,X_n) \, T^k \]
als Ausgangspunkt verwenden. Deren Vorteil ist, dass man ihre Korrektheit ganz
mühelos beweisen kann: Denn die linke Seite der Gleichung hat -- als Polynom
in der einzigen Variablen~$T$ mit Koeffizienten aus~$\QQ[X_1,\ldots,X_n]$
gedacht -- die Nullstellen~$-X_1,\ldots,-X_n$. Daher liefert der Vietasche Satz
sofort die Behauptung.

\emph{Variante 3':} Variante 2' kann man etwas expliziter auch wie folgt
formulieren. Das Polynom
$(1 + T)^n = \sum_{k=0}^n \binom{n}{k} T^{n-k}$
hat die Zahl~$-1$ als~$n$-fache Nullstelle. Mit dem Vietaschen Satz folgt daher
\[ \text{$k$-ter Koeffizient} = \binom{n}{n-k} =
  (-1)^{n-k} e_{n-k}(-1,\ldots,-1) = e_{n-k}(1,\ldots,1). \]

\emph{Variante 3:} Analog kann man Variante~2 explizit ausführen: Das Polynom
$(T - 1)^n = \sum_{k=0}^n \binom{n}{k} (-1)^k T^{n-k}$ hat die Zahl~$+1$
als~$n$-fache Nullstelle. Mit dem Vietaschen Satz folgt daher
\[ \text{$k$-ter Koeff.} = \textstyle (-1)^{n-k} \binom{n}{n-k} =
  (-1)^{n-k} (-1)^{n-k} e_{n-k}(1,\ldots,1) = e_{n-k}(1,\ldots,1). \]
\end{loesungE}
\end{aufgabe}

\begin{aufgabe}{Der Vietasche Satz}
\begin{enumerate}
\item Sei $X^4 + a_3 X^3 + a_2 X^2 + a_1 X + a_0 = 0$ eine normierte Polynomgleichung
vierten Grades, deren Lösungen mit Vielfachheiten $x_1$, $x_2$, $x_3$
und $x_4$ seien. Drücke die Koeffizienten $a_0$, $a_1$, $a_2$ und
$a_3$ explizit als Polynome in den $x_i$ aus.
\item Verwende den Vietaschen Satz für~$n = 2$ um die bekannte Lösungsformel
für normierte quadratische Gleichungen herzuleiten.
\end{enumerate}
\begin{loesungE}
\item Wir multiplizieren~$(X - x_1) \cdots (X - x_4)$ aus und führen einen
Koeffizientenvergleich durch. Damit folgen die gesuchten Beziehungen:
\begin{align*}
  a_0 &= \phantom{+} e_4(x_1,\ldots,x_4) = \phantom{+} x_1 x_2 x_3 x_4 \\
  a_1 &= - e_3(x_1,\ldots,x_4) = -(x_1 x_2 x_3 + x_1 x_2 x_4 + x_1 x_3 x_4 + x_2 x_3 x_4) \\
  a_2 &= \phantom{+} e_2(x_1,\ldots,x_4) = \phantom{+} x_1 x_2 + x_1 x_3 + x_1 x_4 + x_2 x_3 + x_2 x_4 + x_3 x_4 \\
  a_3 &= -e_1(x_1,\ldots,x_4) = -(x_1 + x_2 + x_3 + x_4) \\
    1 &= \phantom{+} e_0(x_1,\ldots,x_4)
\end{align*}

\emph{Bemerkung:} Wenn man den Vietaschen Satz zitiert, kann man sich das
Ausmultiplizieren sparen.

\item Sei~$X^2 + bX + c = 0$ eine allgemeine normierte quadratische Gleichung
und seien~$x_1, x_2$ ihre (nach dem Fundamentalsatz der Algebra existierenden)
komplexen Lösungen (mit Vielfachheiten).
Ausmultiplizieren von~$(X-x_1) (X-x_2)$ und Koeffizientenvergleich führt zu den
Beziehungen
\begin{align*}
  c &= x_1 x_2, \\
  b &= -(x_1 + x_2),
\end{align*}
wie vom Vietaschen Satz vorausgesagt.
Dann gibt es zwei Möglichkeiten fortzufahren:

\emph{Variante 1 (tatsächliche Herleitung):} Mit den Beziehungen können wir
den (aus der Schule bekannten) Ausdruck für die Determinante herleiten, denn
\[ \Delta = (x_1 - x_2)^2 = x_1^2 + 2x_1x_2 + x_2^2 - 4x_1x_2 = b^2 - 4c. \]
Somit gilt
\[ x_1 - x_2 = \pm\sqrt{\Delta}, \]
wobei das~"`$\pm$"'-Zeichen hier bedeuten soll, dass für \emph{eine} der beiden
komplexen Wurzeln von~$\Delta$ die Gleichung stimmt. Über einen üblichen Trick,
den wir etwa schon bei den Formeln für Real- und Imaginärteil komplexer Zahlen
gesehen haben, folgt:
\begin{align*}
  x_1 &= \frac{(x_1 + x_2) + (x_1 - x_2)}{2} = \frac{-b \pm \sqrt{\Delta}}{2} \\
  x_2 &= \frac{(x_1 + x_2) - (x_1 - x_2)}{2} = \frac{-b \mp \sqrt{\Delta}}{2}
\end{align*}

\emph{Variante 2 (lediglich Verifikation):} Alternativ können wir uns damit
begnügen, die bekannte Formel zu verfizieren:
\[ \frac{-b \pm \sqrt{b^2 - 4c}}{2} =
  \frac{x_1 + x_2 \pm \sqrt{(x_1 - x_2)^2}}{2} =
  \frac{x_1 + x_2 \pm (x_1 - x_2)}{2} =
  \text{$x_1$ bzw. $x_2$}. \]

\emph{Bemerkung:} Dem ersten Anschein nach haben wir hier die Rechenregel
"`$\sqrt{a^2} = a$"' verwendet, die ja völlig falsch ist -- für reelle~$a$ gilt
stattdessen~$\sqrt{a^2} = |a|$, und für komplexe~$a$ sollte man lieber nicht
von der Wurzel reden. In der Rechnung wird der Wurzelausdruck allerdings von
einem~"`$\pm$"'-Zeichen geschützt; dann ist das okay (wieso?).
\end{loesungE}
\end{aufgabe}

\begin{aufgabe}{Diskriminanten kubischer Gleichungen}
\begin{enumerate}
\item Finde eine normierte Polynomgleichung dritten Grades, welche~$1$
als zweifache Lösung, $2$ als einfache Lösung und keine weiteren Lösungen
besitzt. Was ist ihre Diskriminante?
\item Sei~$X^3 + p X + q = 0$ eine allgemeine reduzierte kubische Gleichung.
Zeige, dass ihre Diskriminante durch $-4 p^3 - 27 q^2$ gegeben ist.
\end{enumerate}
\begin{loesungE}
\item $(X - 1)^2 (X - 2) = 0$ tut's. Ihre Diskriminante ist~$0$, da sie ja
doppelte Lösungen besitzt.

\item \emph{Variante 1 (lange Rechnung mit Tricks):} Seien~$x,y,z$ die drei Lösungen der
Gleichung. Der Vietasche Satz liefert die Beziehungen
\begin{align*}
  0 &= x+y+z, \\
  p &= xy + xz + yz, \\
  q &= -xyz.
\end{align*}
Daher folgt
\begin{align*}
  \Delta &= (x-y)^2 \cdot (x-z)^2 \cdot (y-z)^2 \\
  &= (x^2+2xy+y^2-4xy) \cdot (x^2+2xz+z^2-4xz) \cdot (y^2+2yz+z^2-4yz) \\
  &= ((x+y)^2-4xy) \cdot ((x+z)^2-4xz) \cdot ((y+z)^2-4yz) \\
  &= (z^2-4xy) \cdot (y^2-4xz) \cdot (x^2-4yz) \\
  &= -63x^2y^2z^2 -4x^3y^3 - 4x^3z^3 - 4y^3z^3 + 16x^4yz + 16xy^4z + 16xyz^4 \\
  &= -63x^2y^2z^2 - 4(x^3y^3 + x^3z^3 + y^3z^3) + 16 xyz (x^3+y^3+z^3) \\
  &= \text{(weitere Tricks)} \\
  &= -4p^3 - 27q^2.
\end{align*}

\emph{Variante 2 (lange Rechnung ohne Tricks):} Seien wieder~$x,y,z$ die drei
Lösungen der Gleichung. Wegen der obigen Beziehungen können wir~$z = -x-y$
schreiben; dann folgt
\begin{align*}
  \Delta &= (x-y)^2 \cdot (x-z)^2 \cdot (y-z)^2 \\
  &= (x-y)^2 \cdot (2x + y)^2 \cdot (x + 2y)^2 \\
  &= 4y^6+12xy^5-3x^2y^4-26x^3y^3-3x^4y^2+12x^5y+4x^6 \\
  \\
  -4p^3 - 27q^2 &= -4 \cdot (-x^2 - xy - y^2)^3 - 27 \cdot (xy^2 + x^2y) \\
  &= 4y^6+12xy^5-3x^2y^4-26x^3y^3-3x^4y^2+12x^5y+4x^6,
\end{align*}
also stimmen die beiden Seiten der zu zeigenden Gleichung überein.

\emph{Variante 3 (durch Überlegung):} Aus der Vorlesung ist bereits bekannt,
dass die Diskriminante~$\Delta$ auf genau eine Art und Weise ein polynomieller
Ausdruck in den Koeffizienten~$p$ und~$q$ (und~$0$) der Gleichung ist. Da~$p$
vom Grad~2,~$q$ vom Grad~3 und~$\Delta$ vom Grad~6 in den Nullstellen ist, gibt
es für die Gestalt dieses Ausdrucks nur eine Möglichkeit, nämlich
\[ \Delta = \alpha p^3 + \beta q^2, \]
wobei die konstanten Vorfaktoren noch zu bestimmen sind. Da diese für jede
reduzierte kubische Gleichung gleich sind (sie sind universelle Konstanten),
können wir diese dadurch bestimmen, indem wir bestimmte einfache
Beispielgleichungen betrachten:
\begin{enumerate}
\item[1.] Die Gleichung~$X^3 - X = 0$ hat~$p = -1$, $q = 0$ und die drei
Lösungen~$-1, 0, 1$. Ihre Diskriminante ist~$4$. Also folgt aus
$4 = \Delta = \alpha \cdot (-1)^3$,
dass~$\alpha = -4$ sein muss.
\item[2.] Die Gleichung~$X^3 - 3\,X + 2 = 0$ hat~$p = -3$, $q = 2$ und die drei
Lösungen~$1, 1, -2$. Ihre Diskriminante ist~$0$. Also folgt aus~$0 = \Delta =
-4 \cdot (-3)^3 + \beta \cdot 2^2$, dass~$\beta = -27$ sein muss.
\item[2.'] \emph{(alternativ)}
Die Gleichung~$X^3 - 1 = 0$ hat~$p = 0$, $q = -1$ und die drei
Lösungen~$1, \omega, \omega^2$ mit~$\omega = \exp(2\pi\i/3)$. Ihre
Diskriminante ist~$-27$ -- wenn man die Beziehung~$\omega^2 = -\omega - 1$, die
wir schon in Aufgabe~5 von Blatt~2 gesehen haben, verwendet, ist die Rechnung
relativ schmerzlos:
\begin{align*}
  \Delta &= (1-\omega)^2 \cdot (1-\omega^2)^2 \cdot (\omega-\omega^2)^2 \\
  &= (1-\omega)^2 \cdot (2+\omega)^2 \cdot (2\omega+1)^2 \\
  &= (1-2\omega+\omega^2) \cdot (4+4\omega+\omega^2) \cdot (4\omega^2+4\omega+1) \\
  &= (-3\omega) \cdot (3+3\omega) \cdot (-3) \\
  &= 27 \cdot (\omega+\omega^2) \\
  &= -27
\end{align*}
Also folgt~$\beta = -27$.
\end{enumerate}
Damit ist die Behauptung bewiesen.
\end{loesungE}
\end{aufgabe}

\ifloesungen\newpage\fi
\begin{aufgabe}{Symmetrien eines Polynoms}
Sei $f(X, Y, Z, W) := X Y + Z W + X Y Z W$.
Wie viele vierstellige Permutationen~$\sigma$ mit~$\sigma \cdot f = f$ gibt es?
\begin{loesung}
\emph{Variante 1:} Man kann alle 24 vierstelligen Permutationen~$\sigma$ durchgehen und
jeweils~$\sigma \cdot f$ mit~$f$ vergleichen:
\begin{center}\tiny\begin{tabular}{rll}
  $xyzw$ & $xy + zw + xyzw$ \\\hline
  $xyzw$ & $xy + zw + xyzw$ & \checkmark \\
  $wxyz$ & $xw + yz + xyzw$ \\
  $xwyz$ & $xw + yz + xyzw$ \\
  $xywz$ & $xy + zw + xyzw$ & \checkmark  \\
  $xzyw$ & $xz + yw + xyzw$ \\
  $wxzy$ & $xw + yz + xyzw$ \\
  $xwzy$ & $xw + yz + xyzw$ \\
  $xzwy$ & $xz + yw + xyzw$ \\
  $yxzw$ & $xy + zw + xyzw$ & \checkmark  \\
  $wyxz$ & $xz + yw + xyzw$ \\
  $ywxz$ & $xz + yw + xyzw$ \\
  $yxwz$ & $xy + zw + xyzw$ & \checkmark  \\
  $yzxw$ & $xw + yz + xyzw$ \\
  $wyzx$ & $xz + yw + xyzw$ \\
  $ywzx$ & $xz + yw + xyzw$ \\
  $yzwx$ & $xw + yz + xyzw$ \\
  $wzxy$ & $xy + zw + xyzw$ & \checkmark  \\
  $zxyw$ & $xz + yw + xyzw$ \\
  $zwxy$ & $xy + zw + xyzw$ & \checkmark  \\
  $zxwy$ & $xz + yw + xyzw$ \\
  $wzyx$ & $xy + zw + xyzw$ & \checkmark  \\
  $zyxw$ & $xw + yz + xyzw$ \\
  $zwyx$ & $xy + zw + xyzw$ & \checkmark  \\
  $zywx$ & $xw + yz + xyzw$
\end{tabular}\end{center}
Also lassen genau acht Permutationen das Polynom invariant.

\emph{Variante 2:} Der hintere Summand bleibt unter jeder Permutation
invariant. Der vordere bleibt unter Vertauschung~$x \leftrightarrow y$ (und
sowieso~$z \leftrightarrow w$) invariant, also unter insgesamt vier
Permutationen. Analog ist es für den mittleren Summanden. Außerdem kann man
aber noch mittels~$x \leftrightarrow z, y \leftrightarrow w$ die beiden
Summanden vertauschen; insgesamt gibt es daher acht Permutationen, die das
Polynom als Ganzes invariant lassen.
\end{loesung}
\end{aufgabe}

\begin{aufgabe}{Formale Ableitung von Polynomen}
\begin{enumerate}
\item Seien~$g(X)$ und~$h(X)$ Polynome. Zeige:
\[ (g h)^{(k)}(X)
  = \sum_{i + j = k} \binom k i \, g^{(i)}(X) \, h^{(j)}(X).
\]

\item Sei $f(X)$ ein Polynom und~$n$ eine natürliche Zahl. Zeige:
$f^{(n + 1)} = 0 \Longleftrightarrow \deg f \leq n.$

\item Sei $f(X)$ ein Polynom und $a$ eine komplexe Zahl. Zeige, dass die
Entwicklung von $f(X)$ nach $X - a$ durch die \emph{Taylorsche Formel} gegeben
ist (nach Brooke Taylor, 1685--1731, britischer Mathematiker):
\[ f(X) = \sum_{k = 0}^\infty \frac{f^{(k)}(a)}{k!} \, (X - a)^k \]
\end{enumerate}

\ifloesungen\newpage\fi
\begin{loesungE}
\item Wir führen einen Induktionsbeweis über~$k \geq 0$.
Dabei ist der Induktionsanfang~$k = 0$ klar. Für den Induktionsschritt~$k \to k
+ 1$ führen wir die Rechnung
\begin{align*}
  (gh)^{(k+1)} &=
  ((gh)')^{(k)} = (g'h + gh')^{(k)} = (g'h)^{(k)} + (gh')^{(k)} \\
  &\stackrel{\text{IV}}{=}
    \sum_{i+j=k} \binom{k}{i} (g')^{(i)} h^{(j)} +
    \sum_{i+j=k} \binom{k}{i} g^{(i)} (h')^{(j)} \\
  &=
    \sum_{i+j=k} \binom{k}{i} g^{(i+1)} h^{(j)} +
    \sum_{i+j=k} \binom{k}{i} g^{(i)} h^{(j+1)} \\
  &=
    \sum_{\substack{\hat\imath+\hat\jmath=k+1\\\hat\imath\geq1}} \binom{k}{\hat\imath - 1} g^{(\hat\imath)} h^{(\hat\jmath)} +
    \sum_{\substack{\hat\imath+\hat\jmath=k+1\\\hat\jmath\geq1}} \binom{k}{\hat\imath} g^{(\hat\imath)} h^{(\hat\jmath)} \\
  &=
    \sum_{\hat\imath+\hat\jmath=k+1} \binom{k}{\hat\imath - 1} g^{(\hat\imath)} h^{(\hat\jmath)} +
    \sum_{\hat\imath+\hat\jmath=k+1} \binom{k}{\hat\imath} g^{(\hat\imath)} h^{(\hat\jmath)} \\
  &= \sum_{\hat\imath+\hat\jmath=k+1}
    \left(\binom{k}{\hat\imath - 1} + \binom{k}{\hat\imath}\right) g^{(\hat\imath)} h^{(\hat\jmath)}
  = \sum_{\hat\imath+\hat\jmath=k+1}
    \binom{k+1}{\hat\imath} g^{(\hat\imath)} h^{(\hat\jmath)}.
\end{align*}
Dabei haben wir der Übersichtlichkeit halber die Konvention~$\binom{n}{-1} = 0$
für alle~$n$ und die bekannte Rekursionsgleichung für die Binomialkoeffizienten
verwendet.

\item "`$\Longrightarrow$"': Da~$f$ ein Polynom ist, können wir es als eine gewisse
endliche Summe~$f = \sum_{i=0}^m a_i X^i$ schreiben. Eine kurze Überlegung
zeigt, dass die~$k$-te Ableitung des Monoms~$X^i$ für~$i < k$ null ist und
für~$i \geq k$ durch
\[ (X^i)^{(k)} = \frac{i!}{(i-k)!} X^{i-k} \]
gegeben ist. Daher folgt aus der Voraussetzung
\[ 0 = f^{(n+1)} = \sum_{i=n+1}^m \frac{i!}{(i-(n+1))!} a_i X^i \]
durch Koeffizientenvergleich, dass für jedes~$i = n+1,\ldots,m$ der
Koeffizient~$\frac{i!}{(i-(n+1))!} a_i$ null ist. Also sind alle~$a_i$ mit~$i
\geq n+1$ null; das Polynom~$f$ hat also in der Tat Grad~$\leq n$.

"`$\Longleftarrow$"': Alle in~$f$ auftretenden Monome verschwinden bei~$(n+1)$-maligem Ableiten.

\item Beide Seiten der Gleichung sind linear in~$f$ (wieso?); daher genügt es
(wieso?), die Identität für die Spezialfälle~$f := X^n$, $n \geq 0$, zu
bestätigen. Dies gelingt mithilfe des binomischen Lehrsatzes:
\begin{align*}
  \sum_{k=0}^\infty \frac{f^{(k)}(a)}{k!} (X-a)^k &=
  \sum_{k=0}^\infty \frac{(X^n)^{(k)}(a)}{k!} (X-a)^k \\
  &= \sum_{k=0}^n \frac{n! / (n-k)! \cdot a^{n-k}}{k!} (X-a)^k \\
  &= \sum_{k=0}^n \binom{n}{k} a^{n-k} (X-a)^k \\
  &= (a + (X-a))^n = X^n = f.
\end{align*}
\end{loesungE}
\end{aufgabe}

\end{document} 

\begin{exercise}(3 Punkte)\newline
    Warum sind beide Seiten von 
\begin{equation}
 (gh)'(X)=g'(X)h(X)+g(X)h'(X)
\end{equation}
 sowohl in
    \(g(X)\) als auch in \(h(X)\) linear und warum reicht es daher, die
    Gleichung (1) nur für \(g(X) = X^m\) und \(h(X) = X^n\)
    nachzurechnen?
\end{exercise}
 
\end{document} 
