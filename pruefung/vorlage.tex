\documentclass{../algblatt}
\geometry{tmargin=2cm,bmargin=2cm,lmargin=3.0cm,rmargin=3.0cm}

\setlength{\titleskip}{0.7em}
%\setlength{\aufgabenskip}{1.5em}

\newcounter{blattnummer}

\renewenvironment{aufgabe}[1]{
  \addtocounter{aufgabennummer}{1}
  \textbf{Blatt \theblattnummer, Aufgabe \theaufgabennummer.} \emph{#1} \par
}{\vspace{\aufgabenskip}}

\thispagestyle{empty}

\begin{document}

\vspace*{-1.2cm}

Universität Augsburg \hfill Sommersemester 2013 \\
Lehrstuhl für Algebra und Zahlentheorie \hfill Ingo Blechschmidt \\
Prof. Marc Nieper-Wißkirchen \hfill Robert Gelb \\[0em]

\begin{center}
  {\Large \textbf{Thema __THEMA__}} \\[2.6em]

  \begin{minipage}{0.93\textwidth}
    \setlength\parskip{\medskipamount}
    Es ist nicht schlimm, wenn man nicht alle Aufgaben bis ins letzte Detail
    ausarbeiten kann. Überlege dir gegebenenfalls, was die
    vorkommenden Begriffe bedeuten und welche Zutaten für die Aufgabe nützlich
    sein könnten.
  \end{minipage}
\end{center}
\vspace{1em}

__AUFGABEN__

\begin{center}Viel Erfolg!\end{center}

\end{document}
