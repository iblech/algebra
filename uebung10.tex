\documentclass{algblatt}
\usepackage{tikz}
\loesungenfalse

\geometry{tmargin=2cm,bmargin=2cm,lmargin=3.0cm,rmargin=3.0cm}

%\setlength{\titleskip}{0.7em}
\setlength{\aufgabenskip}{1.6em}

\begin{document}

\vspace*{-1.5cm}
\maketitle{10}{Abgabe bis 24. Juni 2013, 17:00 Uhr}

%\begin{aufgabe}{\ldots}
%\begin{enumerate}
%\item Sei~$x \in \QQ(\sqrt{3})$. Liegt dann immer auch eine Quadratwurzel
%von~$x$ in~$\QQ(\sqrt{3})$?
%\item Zeige, dass der \emph{goldene Schnitt} $\Phi = \frac{1+\sqrt{5}}{2}$ eine
%ganz algebraische Zahl ist, obwohl in diesem Ausdruck Nenner vorkommen, die
%sich nicht offensichtlich wegkürzen lassen.
%\end{enumerate}
%\end{aufgabe}

\begin{aufgabe}{Weitere Gradformelaufgaben}
\begin{enumerate}
\item Sei~$z$ eine algebraische Zahl und seien~$x, y \in \QQ(z)$. Zeige, dass
\[ \gra{\QQ(z)}{\QQ(x)} \cdot \gra{\QQ(x)}{\QQ} = \gra{\QQ(z)}{\QQ(y)} \cdot
\gra{\QQ(y)}{\QQ}, \]
und gib ein Diagramm zur Veranschaulichung an.
\item Sei~$a$ eine algebraische Zahl und~$y \in \QQ(a)$. Sei~$f$ ein normiertes
Polynom mit Koeffizienten aus~$\QQ(y)$, das über~$\QQ(y)$
auch irreduzibel ist. Sei der Grad von~$f$ mindestens~2 und teilerfremd
zu~$\deg_{\QQ(y)} a$. Zeige, dass keine Zahl aus~$\QQ(a)$ Nullstelle von~$f$
sein kann.

\item Beweise oder widerlege: Sei~$z$ ein primitives Element zu algebraischen
Zahlen~$x, y$. Dann ist~$\deg_\QQ z$ ein Teiler von~$\deg_\QQ x \cdot \deg_\QQ
y$.
\end{enumerate}

\begin{loesungE}
\item Aus der Voraussetzung folgt~$\QQ(x) \subseteq \QQ(z)$ und~$\QQ(y)
\subseteq \QQ(z)$. Daher kann man das Diagramm
\begin{center}\begin{tikzpicture}[node distance=1.5cm]
  \node (Q)                       {$\QQ$};
  \node (Qx)  [above left of=Q]   {$\QQ(x)$};
  \node (Qy)  [above right of=Q]  {$\QQ(y)$};
  \node (Qz)  [above right of=Qx] {$\QQ(z)$};
  \draw (Q) -- (Qx);
  \draw (Q) -- (Qy);
  \draw (Qx) -- (Qz);
  \draw (Qy) -- (Qz);
\end{tikzpicture}\end{center}
zeichnen. Die Behauptung liefert nun einfach die Gradformel, angewendet auf den
linken bzw. rechten Zweig.

\item Sei~$x \in \QQ(a)$ mit~$f(x) = 0$. Dann ist~$f$ das Minimalpolynom
von~$x$ über~$\QQ(y)$, also gilt~$\deg_{\QQ(y)} x = \gra{\QQ(y,x)}{\QQ(y)} =
\deg f$; die Situation können wir in dem Diagramm
\begin{center}\begin{tikzpicture}[node distance=1.5cm]
  \node (Q)                       {$\QQ$};
  \node (Qx)  [above left of=Q]   {$\QQ(x)$};
  \node (Qy)  [above right of=Q]  {$\QQ(y)$};
  \node (Qz)  [above right of=Qx] {$\QQ(x,y)$};
  \node (Qa)  [above of=Qz]       {$\QQ(a)$};
  \draw (Q) -- (Qx);
  \draw (Q) -- (Qy);
  \draw (Qx) -- (Qz);
  \draw (Qy) -- (Qz);
  \draw (Qz) -- (Qa);
\end{tikzpicture}\end{center}
veranschaulichen. Mit der Gradformel folgt die Beziehung
\[ \deg_{\QQ(y)} a = \gra{\QQ(a)}{\QQ(y)} = \gra{\QQ(a)}{\QQ(x,y)} \cdot \gra{\QQ(x,y)}{\QQ(y)} =
  \gra{\QQ(a)}{\QQ(x,y)} \cdot \deg f, \]
die wegen~$\deg f \geq 2$ ein Widerspruch zur Teilerfremdheitsvoraussetzung
ist.

\item Das stimmt im Allgemeinen nicht: Setze~$x = \sqrt[3]{2}$ und~$y = \omega
\cdot \sqrt[3]{2}$, wobei~$\omega = \exp(2\pi\i/3)$ ist. Dann gilt
\begin{align*}
  \deg_\QQ z &= \gra{\QQ(z)}{\QQ} = \gra{\QQ(x,y)}{\QQ(x)} \cdot \gra{\QQ(x)}{\QQ} = 2 \cdot 3 = 6, \\
  \deg_\QQ x \cdot \deg_\QQ &= 3 \cdot 3 = 9,
\end{align*}
aber~$6$ ist kein Teiler von~$9$.
Dabei war der Wert des hinteren Faktors in der ersten Zeile klar (Minimalpolynom ist~$X^3 - 2$ nach
Eisenstein), und dass der vordere Faktor gleich~$2$ ist, kann man wie folgt
begründen: Das Polynom~$X^2 + X + 1$ besitzt bekanntermaßen~$\omega$ als
Nullstelle und ist über~$\QQ(x) \subset \RR$ irreduzibel, da es vom Grad~$2$
ist und seine Nullstellen~$\omega$ und~$\omega^2$ echt komplex sind.

\emph{Bemerkung:} Obige Lösung benötigt gar keine explizite Darstellung des
primitiven Elements~$z$.
Eine ähnliche und richtige Behauptung ist~$\deg_\QQ z \leq
\deg_\QQ x \cdot \deg_\QQ y$, denn
\begin{multline*}\deg_\QQ z = \gra{\QQ(x,y)}{\QQ} = \gra{\QQ(x,y)}{\QQ(x)} \cdot
  \gra{\QQ(x)}{\QQ} \\ \leq \gra{\QQ(y)}{\QQ} \cdot \gra{\QQ(x)}{\QQ} =
  \deg_\QQ y \cdot \deg_\QQ x. {\qquad}\end{multline*}
\end{loesungE}
\end{aufgabe}

\begin{aufgabe}{Galoissche Konjugierte}
\begin{enumerate}
\item Finde zwei algebraische Zahlen, die nicht zueinander galoissch konjugiert
sind.
\item Wie viele galoissch Konjugierte hat die Zahl~$x_1 = \sqrt[3]{1 +
\sqrt{2}}$?
\item Seien~$p$ und~$q$ zwei verschiedene Primzahlen. Finde alle galoissch
Konjugierten von~$\sqrt{p} + \sqrt{q}$.
\item Sei~$t$ eine algebraische Zahl. Zeige, dass die Summe von~$t$ mit all
seinen galoisschen Konjugierten eine rationale Zahl ist. Wie steht es mit dem
Produkt?
\item Seien~$x, y, z$ algebraische Zahlen sodass~$x$ zu~$y$ und~$y$ zu~$z$
galoissch konjugiert ist. Zeige, dass dann auch~$x$ galoissch konjugiert zu~$z$
ist.
\end{enumerate}

\begin{loesungE}
\item Es gibt [abzählbar] unendlich viele Beispiele. Eines ist~$(x,y) = (0,1)$
mit den Minimalpolynomen~$X$ bzw.~$X-1$.

\item Die Zahl~$x_1$ hat insgesamt genau so viele galoissch Konjugierte, wie
ihr Grad angibt. Dieser ist~$6$, relativ schmerzlos kann man das wie folgt
erkennen: Es gilt~$x_1^3 - 1 = \sqrt{2}$, also gilt~$\QQ(\sqrt{2}) \subseteq
\QQ(x_1)$. Folglich ist die Gradformel anwendbar, sie liefert die
Beziehung\ldots

\item Wir suchen zunächst ein Polynom mit rationalen Koeffizienten, dass~$z :=
\sqrt{p} + \sqrt{q}$ als Nullstelle besitzt:
\begin{align*}
  && z &= \sqrt{p} + \sqrt{q} \\
  \Longrightarrow && z^2 &= p + 2\sqrt{p}\sqrt{q} + q^2 \\
  \Longleftrightarrow && z^2 - (p+q) &= 2\sqrt{p}\sqrt{q} \\
  \Longrightarrow && \left(z^2 - (p+q)\right)^2 &= 4pq \\
  \Longleftrightarrow && 0 &= z^4 - 2(p+q)^2z^2 + (p-q)^2
\end{align*}
Kandidat für's Minimalpolynom von~$z$ ist also~$X^4 - 2(p+q)^2\,X^2 + (p-q)^2$.
Die vier Nullstellen dieses Polynoms sind
\[
  x_1 = \sqrt{p} + \sqrt{q},\quad
  x_2 = \sqrt{p} - \sqrt{q},\quad
  x_3 = -\sqrt{p} + \sqrt{q},\quad
  x_4 = -\sqrt{p} - \sqrt{q};
\]
wenn wir seine Irreduzibilität nachgewiesen haben, erkennen wir genau diese
Zahlen als die galoissch Konjugierten von~$z$.

\emph{Irreduzibilitätsnachweis mit dem Verfahren der Vorlesung:}
\begin{itemize}
\item Keine der Nullstellen ist ganzzahlig (wieso?), also
kann kein Linearfaktor abspalten.

\item Für jede zweielementige Auswahl der Nullstellen
sind stets nicht beide elementarsymmetrischen Funktionen in den Nullstellen
ganzzahlig:
\begin{align*}
  e_1(x_1,x_2) &= 2\sqrt{p} \not\in \ZZ \\
  e_1(x_1,x_3) &= 2\sqrt{q} \not\in \ZZ \\
  e_1(x_1,x_4) &= 0 \in \ZZ, & \text{aber } e_2(x_1,x_4) &= -(p+q+2\sqrt{p}\sqrt{q}) \not\in \ZZ
\end{align*}

\item Kubische Faktoren können nicht abspalten, da die
komplementären Faktoren Linearfaktoren wären.
\end{itemize}

\emph{Irreduzibilitätsnachweis mit einem Gradformelargument:}
Wir haben die Inklusionen~$\QQ \subseteq \QQ(\sqrt{p}) \subseteq
\QQ(\sqrt{p},\sqrt{q})$. Dabei gilt~$\gra{\QQ(\sqrt{p})}{\QQ} = 2$ (klar)
und~$\gra{\QQ(\sqrt{p},\sqrt{q})}{\QQ(\sqrt{p})} = \deg_{\QQ(\sqrt{p})}
\sqrt{q} = 2$ (zeigt man wie bei Aufgabe~5 von Blatt~9). Also folgt mit der
Gradformel~$\gra{\QQ(\sqrt{p},\sqrt{q})}{\QQ} = 2 \cdot 2 = 4$.
Da~$z$ ein primitives Element für diese Erweiterung ist
(wieso?), ist also der Grad von~$z$ über~$\QQ$ gleich~$4$. Somit muss obiges
Polynom irreduzibel sein -- es kann kein Polynom niedrigeren Grads geben, das
ebenfalls normiert ist, rationale Koeffizienten hat und~$z$ als Nullstelle
besitzt.

\emph{Bemerkung:} Reduktion modulo~$p$ (oder~$q$) funktioniert nicht:
Modulo~$p$ erhält man das reduzible Polynom~$(X^2 - q)^2$. Auch kann nicht aus
der Irreduzibilität von~$g(X) = X^2 - 2(p+q)\,X + (p-q)^2$ die des eigentlich
zu untersuchenden Polynoms~$g(X^2)$ gefolgert werden. Ein einfaches
Gegenbeispiel, das die Unmöglichkeit eines solchen Schlusses zeigt, ist das
Polynom~$h(X) = X - 1$: Dieses ist irreduzibel, aber~$h(X^2) = X^2 - 1 = (X+1)
\cdot (X-1)$ ist reduzibel.

\item \emph{Variante 1 (mit Vieta):}
Sei~$m(X) = X^n + a_{n-1}X^{n-1} + \cdots + a_1 X + a_0$ das
Minimalpolynom von~$t$ und~$t_1,\ldots,t_n$ seine Nullstellen (also alle
galoissch Konjugierten von~$t$). Nach dem Vietaschen Satz gilt dann
\begin{align*}
  (-1)^n a_0 &= e_n(t_1,\ldots,t_n) = t_1 \cdots t_n, \\
  a_{n-1} &= e_1(t_1,\ldots,t_n) = t_1 + \cdots + t_n,
\end{align*}
also sind Summe und Produkt der galoissch Konjugierten bis auf Vorzeichen durch
die Koeffizienten~$a_0$ bzw.~$a_{n-1}$ des Minimalpolynoms gegeben und daher
rational.

\emph{Variante 2 (mit Wirkung der galoisschen Gruppe):}
Seien~$t_1,\ldots,t_n$ alle galoissch Konjugierten von~$t$, also die
Nullstellen des Minimalpolynoms von~$t$. Dann wollen wir zeigen, dass die
Summe der~$t_i$ invariant unter der Wirkung der galoisschen Gruppe ist und
daher rational sein muss: Sei also~$\sigma \in
\operatorname{Gal}(t_1,\ldots,t_n)$ beliebig. Dann gilt in der Tat
\[ \sigma \cdot (t_1 + \cdots + t_n) =
  t_{\sigma(1)} + \cdots + t_{\sigma(n)} =
  t_1 + \cdots + t_n. \]
Analog kann man mit dem Produkt verfahren.

\item Seien~$m_x$, $m_y$ und~$m_z$ die Minimalpolynome von~$x$, $y$ bzw.~$z$.
Dann gilt nach Voraussetzung~$m_x = m_y$ und~$m_y = m_z$, also auch~$m_x =
m_z$. Damit sind~$x$ und~$z$ zueinander galoissch konjugiert.
\end{loesungE}
\end{aufgabe}

\begin{aufgabe}{Polynome sind blind für galoissch Konjugierte}
\begin{enumerate}
\item Zeige, dass eine algebraische Zahl~$t$ genau dann zu einer weiteren
algebraischen Zahl~$t'$ galoissch konjugiert ist, wenn jedes Polynom mit
rationalen Koeffizienten, welches~$t$ als Nullstelle hat, auch~$t'$ als
Nullstelle hat.
\item Seien~$t$ und~$t'$ zueinander galoissch konjugierte algebraische Zahlen
und~$f$ ein Polynom mit rationalen Koeffizienten. Zeige, dass dann auch die
Zahlen~$x := f(t)$ und~$x' := f(t')$ zueinander galoissch konjugiert sind.
\end{enumerate}

\begin{loesungE}
\item "`$\Longleftarrow$"': Sei~$m_t$ das Minimalpolynom von~$t$. Dieses hat
sicherlich~$t$ als Nullstelle. Nach Voraussetzung ist daher auch~$t'$ eine
Nullstelle. Also haben~$t$ und~$t'$ beide~$m_t$ als Minimalpolynom und sind
daher galoissch Konjugierte.

"`$\Longrightarrow$"' (schon im Skript als Proposition~4.2): Sei~$m_t$ das
gemeinsame Minimalpolynom von~$t$ und~$t'$
und sei~$f \in \QQ[X]$ ein Polynom, das~$t$ als Nullstelle hat. Dann
haben~$f$ und~$m_t$ also die gemeinsame Nullstelle~$t$. Da~$m_t$ irreduzibel
ist, folgt mit dem abelschen Irreduzibilitätssatz (Satz~3.10), dass~$f$ ein
Vielfaches von~$m_t$ ist. Somit ist jede Nullstelle von~$m_t$,
insbesondere~$t'$, auch Nullstelle von~$f$.

\item Sei~$m_{f(t)}$ das Minimalpolynom von~$x = f(t)$. Dann gilt also
\[ m_{f(t)}(f(t)) = (m_{f(t)} \circ f)(t) = 0, \]
das Polynom~$m_{f(t)} \circ f$ besitzt also~$t$ als Nullstelle. Nach
Teilaufgabe~a) besitzt dieses Polynom dann auch~$t'$ als Nullstelle, also gilt
\[ m_{f(t)}(f(t')) = (m_{f(t)} \circ f)(t') = 0. \]
Somit ist~$f(t')$ ebenfalls Nullstelle des Minimalpolynoms von~$f(t)$ und somit
zu~$f(t)$ galoissch Konjugiert.
\end{loesungE}
\end{aufgabe}

\begin{aufgabe}{Gegenbeispiele}
Zeige an jeweils einem Beispiel, dass
\begin{enumerate}
\item Hilfssatz~4.3 auf Seite 118
\item Proposition~4.4 auf Seite 119
\end{enumerate}
falsch werden, wenn man von den dort vorkommenden Zahlen~$x_1,\ldots,x_n$ nicht
voraussetzt, dass sie die gesamten Lösungen (mit Vf.) einer Polynomgleichung
mit rationalen Koeffizienten sind, sondern stattdessen beliebige algebraische
Zahlen erlaubt.

\begin{loesungE}
\item Hilfssatz~4.3 lautet:
\begin{quote}
Seien~$x_1,\ldots,x_n$ die Lösungen (mit Vielfachheiten) einer Polynomgleichung
mit rationalen Koeffizienten. Ist dann~$V(X_1,\ldots,X_n)$ ein Polynom mit
rationalen Koeffizienten, so sind die galoissch Konjugierten von~$t =
V(x_1,\ldots,x_n)$ alle von der Form~$t' =
V(x_{\sigma(1)},\ldots,x_{\sigma(n)})$, wobei~$\sigma$ eine~$n$-stellige
Permutation ist.
\end{quote}
Es gibt zahlreiche Gegenbeispiele, wenn man die Voraussetzung, dass die~$x_i$
\emph{alle} Lösungen \emph{einer} Polynomgleichung
mit rationalen Koeffizienten sind, fallen lässt. Sei etwa~$n = 1$, $x_1 = \i$
und~$V(X_1) = X_1$. Dann stimmt es nicht, dass alle galoissch Konjugierten
von~$t = V(x_1) = \i$ von der (wegen~$n = 1$ einzig möglichen) Form~$t' =
V(x_1)$ sind. Denn~$-\i$ ist ja auch noch ein galoissch Konjugiertes von~$t$.

Ein komplizierteres Gegenbeispiel ist~$n = 2$, $x_1 = 17$, $x_2 = \i$,
$V(X_1,X_2) = X_2$.

\item Proposition~4.4 lautet:
\begin{quote}
Seien~$x_1,\ldots,x_n$ die Lösungen (mit Vielfachheiten) einer Polynomgleichung
mit rationalen Koeffizienten. Ist dann~$t$ ein primitives Element
zu~$x_1,\ldots,x_n$, so ist auch jedes galoissch Konjugierte~$t'$ von~$t$ ein
primitives Element von~$x_1,\ldots,x_n$.
\end{quote}
Auch hier gibt es zahlreiche Gegenbeispiele, wenn man die Voraussetzung fallen
lässt. Sei etwa~$n = 1$, $x_1 = \omega\sqrt[3]{2}$ und~$t = x_1$,
wobei~$\omega = \exp(2\pi\i/3)$ eine primitive dritte Einheitswurzel ist. Dann stimmt es
nicht, dass das galoissch Konjugierte~$t' = \sqrt[3]{2}$ ebenfalls ein
primitives Element von~$\QQ(x_1)$ ist: Denn~$\QQ(t') \subseteq \RR$,
aber~$\QQ(x_1) \not\subseteq \RR$.
\end{loesungE}
\end{aufgabe}

\end{document}
