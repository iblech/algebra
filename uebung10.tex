\documentclass{algblatt}
\loesungenfalse

\geometry{tmargin=2cm,bmargin=2cm,lmargin=3.0cm,rmargin=3.0cm}

%\setlength{\titleskip}{0.7em}
\setlength{\aufgabenskip}{1.6em}

\begin{document}

\vspace*{-1.5cm}
\maketitle{10}{Abgabe bis 24. Juni 2013, 17:00 Uhr}

%\begin{aufgabe}{\ldots}
%\begin{enumerate}
%\item Sei~$x \in \QQ(\sqrt{3})$. Liegt dann immer auch eine Quadratwurzel
%von~$x$ in~$\QQ(\sqrt{3})$?
%\item Zeige, dass der \emph{goldene Schnitt} $\Phi = \frac{1+\sqrt{5}}{2}$ eine
%ganz algebraische Zahl ist, obwohl in diesem Ausdruck Nenner vorkommen, die
%sich nicht offensichtlich wegkürzen lassen.
%\end{enumerate}
%\end{aufgabe}

\begin{aufgabe}{Weitere Gradformelaufgaben}
\begin{enumerate}
\item Sei~$z$ eine algebraische Zahl und seien~$x, y \in \QQ(z)$. Zeige, dass
\[ \gra{\QQ(z)}{\QQ(x)} \cdot \gra{\QQ(x)}{\QQ} = \gra{\QQ(z)}{\QQ(y)} \cdot
\gra{\QQ(y)}{\QQ}, \]
und gib ein Diagramm zur Veranschaulichung an.
\item Sei~$a$ eine algebraische Zahl und~$y \in \QQ(a)$. Sei~$f$ ein normiertes
Polynom mit Koeffizienten aus~$\QQ(y)$, das über~$\QQ(y)$
auch irreduzibel ist. Sei der Grad von~$f$ mindestens~2 und teilerfremd
zu~$\deg_{\QQ(y)} x$. Zeige, dass keine Zahl aus~$\QQ(a)$ Nullstelle von~$f$
sein kann.

\item Beweise oder widerlege: Sei~$z$ ein primitives Element zu algebraischen
Zahlen~$x, y$. Dann ist~$\deg_\QQ z$ ein Teiler von~$\deg_\QQ x \cdot \deg_\QQ
y$.
\end{enumerate}
\end{aufgabe}

\begin{aufgabe}{Galoissche Konjugierte}
\begin{enumerate}
\item Finde zwei algebraische Zahlen, die nicht zueinander galoissch konjugiert
sind.
\item Wie viele galoissch Konjugierte hat die Zahl~$x_1 = \sqrt[3]{1 +
\sqrt{2}}$?
\item Seien~$p$ und~$q$ zwei verschiedene Primzahlen. Finde alle galoissch
Konjugierten von~$\sqrt{p} + \sqrt{q}$.
\item Sei~$t$ eine algebraische Zahl. Zeige, dass die Summe von~$t$ mit all
seinen galoisschen Konjugierten eine rationale Zahl ist. Wie steht es mit dem
Produkt?
\item Seien~$x, y, z$ algebraische Zahlen sodass~$x$ zu~$y$ und~$y$ zu~$z$
galoissch konjugiert ist. Zeige, dass dann auch~$x$ galoissch konjugiert zu~$z$
ist.
\end{enumerate}

\begin{loesungE}
\item Es gibt [abzählbar] unendlich viele Beispiele. Eines ist~$(x,y) = (0,1)$
mit den Minimalpolynomen~$X$ bzw.~$X-1$.

\item Die Zahl~$x_1$ hat insgesamt genau so viele galoissch Konjugierte, wie
ihr Grad angibt. Dieser ist~$6$, relativ schmerzlos kann man das wie folgt
erkennen: Es gilt~$x_1^3 - 1 = \sqrt{2}$, also gilt~$\QQ(\sqrt{2}) \subseteq
\QQ(x_1)$. Folglich ist die Gradformel anwendbar, sie liefert die
Beziehung\ldots
\end{loesungE}
\end{aufgabe}

\begin{aufgabe}{Polynome sind blind für galoissch Konjugierte}
\begin{enumerate}
\item Zeige, dass eine algebraische Zahl~$t$ genau dann zu einer weiteren
algebraischen Zahl~$t'$ galoissch konjugiert ist, wenn jedes Polynom mit
rationalen Koeffizienten, welches~$t$ als Nullstelle hat, auch~$t'$ als
Nullstelle hat.
\item Seien~$t$ und~$t'$ zueinander galoissch konjugierte algebraische Zahlen
und~$f$ ein Polynom mit rationalen Koeffizienten. Zeige, dass dann auch die
Zahlen~$x := f(t)$ und~$x' := f(t')$ zueinander galoissch konjugiert sind.
\end{enumerate}

\begin{loesungE}
\item "`$\Longleftarrow$"': Sei~$m_t$ das Minimalpolynom von~$t$. Dieses hat
sicherlich~$t$ als Nullstelle. Nach Voraussetzung ist daher auch~$t'$ eine
Nullstelle. Also haben~$t$ und~$t'$ beide~$m_t$ als Minimalpolynom und sind
daher galoissch Konjugierte.

"`$\Longrightarrow$"' (schon im Skript als Proposition~4.2): Sei~$m_t$ das
gemeinsame Minimalpolynom von~$t$ und~$t'$
und sei~$f \in \QQ[X]$ ein Polynom, das~$t$ als Nullstelle hat. Dann
haben~$f$ und~$m_t$ also die gemeinsame Nullstelle~$t$. Da~$m_t$ irreduzibel
ist, folgt mit dem abelschen Irreduzibilitätssatz (Satz~3.10), dass~$f$ ein
Vielfaches von~$m_t$ ist. Somit ist jede Nullstelle von~$m_t$,
insbesondere~$t'$, auch Nullstelle von~$f$.

\item Sei~$m_{f(t)}$ das Minimalpolynom von~$x = f(t)$. Dann gilt also
\[ m_{f(t)}(f(t)) = (m_{f(t)} \circ f)(t) = 0, \]
das Polynom~$m_{f(t)} \circ f$ besitzt also~$t$ als Nullstelle. Nach
Teilaufgabe~a) besitzt dieses Polynom dann auch~$t'$ als Nullstelle, also gilt
\[ m_{f(t)}(f(t')) = (m_{f(t)} \circ f)(t') = 0. \]
Somit ist~$f(t')$ ebenfalls Nullstelle des Minimalpolynoms von~$f(t)$ und somit
zu~$f(t)$ galoissch Konjugiert.
\end{loesungE}
\end{aufgabe}

\begin{aufgabe}{Gegenbeispiele}
Zeige an jeweils einem Beispiel, dass
\begin{enumerate}
\item Hilfssatz~4.3 auf Seite 118
\item Proposition~4.4 auf Seite 119
\end{enumerate}
falsch werden, wenn man von den dort vorkommenden Zahlen~$x_1,\ldots,x_n$ nicht
voraussetzt, dass sie die gesamten Lösungen (mit Vf.) einer Polynomgleichung
mit rationalen Koeffizienten sind, sondern stattdessen beliebige algebraische
Zahlen erlaubt.

\begin{loesungE}
\item Hilfssatz~4.3 lautet:
\begin{quote}
Seien~$x_1,\ldots,x_n$ die Lösungen (mit Vielfachheiten) einer Polynomgleichung
mit rationalen Koeffizienten. Ist dann~$V(X_1,\ldots,X_n)$ ein Polynom mit
rationalen Koeffizienten, so sind die galoissch Konjugierten von~$t =
V(x_1,\ldots,x_n)$ alle von der Form~$t' =
V(x_{\sigma(1)},\ldots,x_{\sigma(n)})$, wobei~$\sigma$ eine~$n$-stellige
Permutation ist.
\end{quote}
Es gibt zahlreiche Gegenbeispiele, wenn man die Voraussetzung, dass die~$x_i$
\emph{alle} Lösungen \emph{einer} Polynomgleichung
mit rationalen Koeffizienten sind, fallen lässt. Sei etwa~$n = 1$, $x_1 = \i$
und~$V(X_1) = X_1$. Dann stimmt es nicht, dass alle galoissch Konjugierten
von~$t = V(x_1) = \i$ von der (wegen~$n = 1$ einzig möglichen) Form~$t' =
V(x_1)$ sind. Denn~$-\i$ ist ja auch noch ein galoissch Konjugiertes von~$t$.

Ein komplizierteres Gegenbeispiel ist~$n = 2$, $x_1 = 17$, $x_2 = \i$,
$V(X_1,X_2) = X_2$.

\item Proposition~4.4 lautet:
\begin{quote}
Seien~$x_1,\ldots,x_n$ die Lösungen (mit Vielfachheiten) einer Polynomgleichung
mit rationalen Koeffizienten. Ist dann~$t$ ein primitives Element
zu~$x_1,\ldots,x_n$, so ist auch jedes galoissch Konjugierte~$t'$ von~$t$ ein
primitives Element von~$x_1,\ldots,x_n$.
\end{quote}
Auch hier gibt es zahlreiche Gegenbeispiele, wenn man die Voraussetzung fallen
lässt. Sei etwa~$n = 1$, $x_1 = \omega\sqrt[3]{2}$ und~$t = x_1$,
wobei~$\omega = \exp(2\pi\i/3)$ eine primitive dritte Einheitswurzel ist. Dann stimmt es
nicht, dass das galoissch Konjugierte~$t' = \sqrt[3]{2}$ ebenfalls ein
primitives Element von~$\QQ(x_1)$ ist: Denn~$\QQ(t') \subseteq \RR$,
aber~$\QQ(x_1) \not\subseteq \RR$.
\end{loesungE}
\end{aufgabe}

\end{document}
