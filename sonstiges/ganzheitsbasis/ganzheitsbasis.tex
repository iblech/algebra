\documentclass[12pt,a4paper,ngerman]{scrartcl}

\usepackage{ucs}
\usepackage[utf8x]{inputenc}
\usepackage[ngerman]{babel}
\usepackage{amsmath,amssymb,amscd,amsthm,color,graphicx}
\usepackage[protrusion=true,expansion=false]{microtype}
\usepackage{lmodern}
\usepackage{hyperref}

\setlength\parskip{\medskipamount}
\setlength\parindent{0pt}

\newcommand{\Kern}{\operatorname{ker}}
\newcommand{\Span}{\operatorname{span}}
\newcommand{\GL}{\mathrm{GL}}
\newcommand{\R}{\mathbb{R}}
\newcommand{\Q}{\mathbb{Q}}
\newcommand{\QQ}{\overline{\mathbb{Q}}}
\newcommand{\Z}{\mathbb{Z}}
\newcommand{\F}{\mathbb{F}}

\theoremstyle{definition}
\newtheorem*{defn}{Definition}
\newtheorem*{bsp}{Beispiel}

\theoremstyle{plain}
\newtheorem*{prop}{Proposition}
\newtheorem*{frage}{Frage}
\newtheorem*{beh}{Behauptung}

\theoremstyle{remark}
\newtheorem*{bem}{Bemerkung}

\usepackage{geometry}
\geometry{tmargin=3cm,bmargin=3cm,lmargin=3cm,rmargin=3cm}

\usepackage{fancyhdr}
\pagestyle{fancy}
\lhead{Algebra II, SS 2011}
\chead{6. September 2011}
\rhead{Diskriminante}

\newcommand{\repalg}[1]{(Rep.~d.~Alg.,~#1)}

\newcommand{\tr}[1]{\operatorname{tr}_{K/\Q}(#1)}

\begin{document}

\section*{Definition der Diskriminante}

Sei~$K$ ein Zahlkörper, also eine endliche Körpererweiterung von~$\Q$.
Sei~$b_1, \ldots, b_n$ eine Basis von~$K$ als~$\Q$-Vektorraum.

Dann heißt die rationale Zahl
\[ D(b_1,\ldots,b_n) := \operatorname{det} \begin{pmatrix}
  \tr{b_1b_1} & \cdots & \tr{b_1b_n} \\
  \vdots & & \vdots \\
  \tr{b_nb_1} & \cdots & \tr{b_nb_n}
\end{pmatrix} \]
die \emph{Diskriminante von~$K$ über~$\Q$ zur Basis~$b_1,\ldots,b_n$}.


\section*{Eigenschaften der Diskriminante}

\subsection*{Verhalten der Diskriminante unter Basistransformation}

Seien~$b_1,\ldots,b_n$ und~$b_1',\ldots,b_n'$ zwei Basen von~$K$
als~$\Q$-Vektorraum. Dann gilt für die zugehörigen Diskriminanten
\[ D(b_1',\ldots,b_n') = (\operatorname{det} A)^2 D(b_1,\ldots,b_n), \]
wobei~$A =(a_{ij})_{ij} \in \Q^{n \times n}$ die zu den beiden Basen gehörige
Basiswechselmatrix (definiert durch die Forderungen~$b_i' = \sum_j a_{ij} b_j$
für alle~$i = 1,\ldots,n$) ist.

Dadurch motiviert definiert man noch folgenden Begriff: Die \emph{Diskriminante von~$K$
über~$\Q$} (ohne eine spezielle Basis zu nennen) ist die rationale Zahl
\[ \operatorname{disc}_{K/Q} := D(b_1,\ldots,b_n), \]
wobei~$b_1,\ldots,b_n$ eine beliebige Basis von~$K$ über~$\Q$ ist. Das Ergebnis
ist nicht wohldefiniert (hängt nämlich von der speziellen Wahl der Basis ab);
man ergänzt die Definition daher noch um den Zusatz, dass man zwei
Diskriminanten genau dann als gleich ansieht, wenn sie sich nur durch einen
quadratischen Faktor unterscheiden.
Die Rechenregel für die Basistransformation zeigt dann, dass diese Definition
sinnvoll ist.


\subsection*{Diskriminante bei ganz-algebraischen Basisvektoren}

Die Diskriminante ist stets eine rationale Zahl. Sind die~$b_i$ sogar
ganz-algebraische Zahlen (d.\,h., erfüllen die~$b_i$ sogar eine normierte
Polynomgleichung mit ganzzahligen Koeffizienten), dann ist die
Diskriminante sogar eine ganze Zahl. (Die Umkehrung gilt nicht.)

Das liegt daran, weil dann schon die Spuren~$\tr{b_ib_j}$ jeweils ganze Zahlen
sind.


\section*{Berechnung der Diskriminante}

Kennt man ein primitives Element~$z$ von~$K$, d.\,h. ein Element~$z \in K$
mit~$K = \Q(z)$, so kann man die Diskriminante~$D(b_1,\ldots,b_n)$ einer
Basis~$b_1,\ldots,b_n$ wie folgt berechnen:

Zunächst sucht man sich Polynome~$B_i \in \Q[X]$ mit~$b_i = B_i(z)$. Solche
muss es nach Voraussetzung immer geben. Dann gilt für die Diskriminante
\[ D(b_1,\ldots,b_n) = \left(\operatorname{det} \begin{pmatrix}
  B_1(z_1) & \cdots & B_1(z_n) \\
  \vdots && \vdots \\
  B_n(z_1) & \cdots & B_n(z_n)
\end{pmatrix}\right)^2, \]
wobei~$z_1,\ldots,z_n$ die galoisschen Konjugierten von~$z$ in~$\QQ$ sind.


\subsection*{Beispiel}

Sei~$K = \Q(z)$ mit~$z = \sqrt[3]{2}$. Wir wollen die Diskriminante der
Basis~$(b_1,b_2,b_3) = (1, z, z^2)$ bestimmen.

Dazu setzen wir~$B_1 = 1$, $B_2 = X$ und~$B_3 = X^2$, denn dann gilt~$B_1(z) =
b_1$, $B_2(z) = b_2$ und~$B_3(z) = b_3$.

Das Minimalpolynom von~$z$ ist~$X^3-2$, also sind die galoissch Konjugierten
von~$z$ die Zahlen~$z$, $\omega z$ und~$\omega^2 z$, wobei~$\omega = e^{2\pi
i/3}$.

Folglich ergibt sich die Determinante als
\[ D(b_1,b_2,b_3) = \left(\operatorname{det} \begin{pmatrix}
  1 & 1 & 1 \\
  z & \omega z & \omega^2 z \\
  z^2 & \omega^2 z^2 & \omega^4 z^2
\end{pmatrix}\right)^2 = -108. \]


\section*{Nutzen der Diskriminante}

Die Diskriminante kann man nutzen, um zu beweisen, dass eine Vermutung für eine
Ganzheitsbasis in der Tat korrekt ist. Denn Hilfssatz~7.129 (auf Seite~336)
garantiert folgendes:
\begin{quote}
Sei~$K \supseteq \Q$ ein Zahlkörper. Sei~$b_1,\ldots,b_n$ eine Basis von~$K$
als~$\Q$-Vektorraum aus ganz-algebraischen Zahlen, also aus Elementen
aus~$\mathcal{O}_K$. Dann gilt:
\[ \Z b_1 + \cdots + \Z b_n \subseteq \mathcal{O}_K \subseteq \Z \frac{1}{d}b_1
+ \cdots + \Z \frac{1}{d}b_n, \]
wobei~$d = D(b_1,\ldots,b_n)$.
\end{quote}

Möchte man also zeigen, dass~$b_1,\ldots,b_n$ sogar eine Ganzheitsbasis ist
(und nicht nur eine Basis über~$\Q$, die zufälligerweise aus ganz-algebraischen
Zahlen besteht), muss man also nur noch zeigen, dass alle Zahlen~$x \in
\mathcal{O}_K$ mit~$x \in \Z \frac{1}{d}b_1
+ \cdots + \Z \frac{1}{d}b_n$ schon in~$\Z b_1 + \cdots + \Z b_n$ liegen.


\subsection*{Beispiel}

Wir wollen eine Ganzheitsbasis von~$K = \Q(z)$ mit~$z = \sqrt{-3}$ bestimmen.
Eine Basis von~$K$ über~$\Q$ ist durch~$1, z$ gegeben; da~$z$ als Nullstelle
des ganzzahligen Polynoms~$X^2+3$ ganz-algebraisch ist, ist also folgendes
unsere erste Vermutung:

\emph{1. Vermutung:} Eine Ganzheitsbasis von~$\mathcal{O}_K$ ist~$1, z$.

Nun gibt es zwei Möglichkeiten zu sehen, dass diese Vermutung falsch sein muss.
Die eine besteht darin, zu beobachten, dass die Zahl~$(1+z)/2$ trotz des
gegenteiligen Anscheins auch ganz-algebraisch ist (nämlich Nullstelle des
Polynoms~$X^2-X+1$), andererseits aber sicher keine ganzzahlige Linearkombination
von~$1$ und~$z$ ist.

Eine andere besteht darin, zunächst mit dem nächsten Schritt weiterzumachen und
dann zu erkennen, dass dieser fehlschlägt. In jedem Fall werden wir zu einer neuen
Vermutung gelenkt:

\emph{2. Vermutung:} Eine Ganzheitsbasis von~$\mathcal{O}_K$ ist~$(b_1,b_2) =
(1, (1+z)/2)$.

Um diese Vermutung zu überprüfen, berechnen wir zunächst die Diskriminante
dieser Basis (Basis über~$\Q$ ist sie definitiv). Dazu wählen wir (mit der
Notation von oben)~$B_1 = 1$, $B_2 = (1+X)/2$, und die galoisschen Konjugierten
von~$z$ sind~$z$ und~$-z$. Somit gilt:
\[ D(b_1,b_2) = \left(\operatorname{det} \begin{pmatrix}
  1 & 1 \\
  (1+z)/2 & (1-z)/2
\end{pmatrix}\right)^2 = -3. \]

Sei dann ein beliebiges Element~$x \in \mathcal{O}_K$ gegeben. Wir wollen
zeigen, dass~$x \in \Z b_1 + \Z b_2$; und nach dem zitierten Hilfssatz wissen
wir schon, dass~$x \in \Z \frac{1}{3} b_1 + \Z \frac{1}{3} b_2$, dass es also
ganze Zahlen~$a,b \in \Z$ mit
\[ x = \frac{1}{3}a + \frac{1}{3}b \cdot \frac{1+\sqrt{-3}}{2} \]
gibt. Wir müssen zeigen, dass~$a$ und~$b$ Vielfache von~$3$ sind.

Dazu treffen wir eine Fallunterscheidung über den Grad von~$x$:
\begin{enumerate}
\item $[\Q(x):\Q] = 1$. Dann muss~$b$ null sein (und ist somit ein Vielfaches
von~3). Das Minimalpolynom von~$x$ ist daher~$X - \frac{1}{3}a$ und besitzt
nach Voraussetzung nur ganzzahlige Koeffizienten. Daher gilt~$a/3 \in \Z$, also
ist auch~$a$ in der Tat ein Vielfaches von~$3$.
\item $[\Q(x):\Q] = 2$. Durch Umstellen und Quadrieren erhält man die
Polynomgleichung
\[ x^2 - \frac{1}{3}(2a+b) x + \frac{1}{9}(a^2+ab+b^2) = 0 \]
für~$x$. Das zugehörige Polynom~$X^2 - \frac{1}{3}(2a+b)X +
\frac{1}{9}{a^2+ab+b^2}$ muss das Minimalpolynom von~$x$ sein, also erhalten
wir, dass
\[ 3 \mathrel{|} 2a+b \quad\text{und}\quad 9 \mathrel{|} a^2+ab+b^2. \]

Schreiben wir~$a = 3\tilde a + r$ und~$b = 3\tilde b + s$ für gewisse Reste
$r,s \in \{ 0,1,2 \}$, erhalten wir
\[ 3 \mathrel{|} 2r+s \quad\text{und}\quad 9 \mathrel{|} r^2+rs+s^2. \]
Wir können nun einfach alle Möglichkeiten für~$r$ und~$s$ ausprobieren (oder
uns klüger anstellen)\ldots
\begin{center}
\begin{tabular}{cccc}
  $r$ & $s$ & $2r+s$ & $r^2+rs+s^2$ \\
  \hline
  0 & 0 & 0 & 0 \\
  0 & 1 & 1 & 1 \\
  0 & 2 & 2 & 4 \\
  1 & 0 & 2 & 1 \\
  1 & 1 & 3 & 3 \\
  1 & 2 & 4 & 7 \\
  2 & 0 & 4 & 4 \\
  2 & 1 & 5 & 7 \\
  2 & 2 & 6 & 12
\end{tabular}
\end{center}
\ldots und sehen, dass der einzige Fall, der die Teilbarkeitsbedingung erfüllt,
$r = s = 0$ ist. Das war zu zeigen.
\end{enumerate}
Damit ist bewiesen, dass~$1, (1+z)/2$ eine Ganzheitsbasis von~$\mathcal{O}_K$
ist.

\begin{small}
Hier noch, was passiert wäre, wenn man nicht erkannt hätte, dass~$1,z$ keine
Ganzheitsbasis sein konnte. Die Diskriminante hätte sich zu~$-12$ ergeben. Für
ein beliebiges~$x \in \mathcal{O}_K$ hätte es daher ganze Zahlen~$a, b \in \Z$
mit~$x = \frac{1}{12}a + \frac{1}{12}b \sqrt{-3}$ gegeben, und man hätte
versuchen müssen, zu zeigen, dass~$a$ und~$b$ Vielfache von~12 sind.

Wie oben hätte man dann eine Fallunterscheidung über den Grad von~$x$
getroffen, der erste Fall hätte auch noch funktioniert. Im zweiten Fall wäre
man auf das Minimalpolynom
\[ X^2 - \frac{1}{6}aX + \frac{1}{144}(a^2 + 3b^2) \]
gekommen und hätte daher folgern können, dass
\[ 6 \mathrel{|} a \quad\text{und}\quad 144 \mathrel{|} a^2+3b^2. \]
Beim Versuch zu zeigen, dass aus dieser Teilbarkeitsbedingung aber schon folgt,
dass~$a$ und~$b$ Vielfache von~12 sind, wäre man aber gescheitert:
Beispielsweise erfüllt~$(a,b) = (6,6)$ auch die Bedingung. Die zugehörige
Zahl~$x$ ist
\[ x = \frac{1}{12}6 + \frac{1}{12}6 \sqrt{-3} = \frac{1 + \sqrt{-3}}{2}, \]
nimmt man diese anstelle von~$z$ als zweites Basiselement, wird man zu einer
neuen Vermutung über die Ganzheitsbasis geleitet (unserer zweiten Vermutung),
mit der man das Verfahren wiederholen kann.
\end{small}


\subsection*{Allgemeines Verfahren zur Bestimmung einer Ganzheitsbasis}

Das Beispiel zeigt, dass man folgendes Verfahren zur Bestimmung einer
Ganzheitsbasis verwenden kann:

\begin{enumerate}
\item Beginne mit einer~$\Q$-Basis~$b_1,\ldots,b_n$ von~$K$, die aus Elementen
von~$\mathcal{O}_K$ besteht.
\item Berechne ihre Diskriminante~$d := D(b_1,\ldots,b_n)$.
\item Versuche zu zeigen, dass~$\mathcal{O}_K \cap \left(\Z \frac{1}{d}b_1
+ \cdots + \Z \frac{1}{d}b_n\right) \subseteq \Z b_1 + \ldots + \Z b_n$.

Nehme dazu ein beliebiges~$x = \frac{1}{d}a_1b_1 + \ldots + \frac{1}{d}a_nb_n$
mit ganzen Zahlen~$a_1,\ldots,a_n \in \Z$ und~$x \in \mathcal{O}_K$. Treffe
eine Fallunterscheidung über den Grad von~$x$ und bestimme in jedem Fall sein
Minimalpolynom.

Versuche dann aus dem Wissen, dass die Koeffizienten des Minimalpolynoms ganze
Zahlen sind, zu zeigen, dass die~$a_i$ jeweils Vielfache von~$d$ sind.
Hilfreich ist es dabei, ohne Einschränkung der Allgemeinheit anzunehmen, dass
jedes~$a_i$ schon in der Menge~$\{ 0,1,\ldots,|d|-1 \}$ liegt. (Das haben wir
oben über die Division mit Rest erreicht.)
\item War der Versuch erfolgreich? Dann ist~$b_1,\ldots,b_n$ in der Tat eine
Ganzheitsbasis von~$\mathcal{O}_K$.

Sonst kann man aus dem Fehlschlag ein Element aus~$\mathcal{O}_K$ extrahieren,
welches keine~$\Z$-Linearkombination der Basis~$b_1,\ldots,b_n$ ist; fügt man
dieses in die Basis ein (und entfernt dafür ein anderes Basiselement), kann man
das Verfahren ab Schritt~2 für die neue Basis wiederholen.
\end{enumerate}

In der Praxis hilfreich ist es, gleich zu Beginn auszuloten, ob Zahlen wie
$(1+z)/2$ ganz-algebraisch sind und so zu einer besseren
Vermutung über die Ganzheitsbasis zu gelangen. Das spart Zeit.


\section*{Siehe auch}

\begin{itemize}
\item[--] \url{http://en.wikipedia.org/wiki/Ring%20of%20integers}
\item[--]
\url{http://planetmath.org/encyclopedia/ExamplesOfRingOfIntegersOfANumberField.html}
\item[--] \url{http://www.ucl.ac.uk/~ucahmki/courses/ant/integral.pdf}
\item[--]
\url{http://www.math.uconn.edu/~kconrad/blurbs/gradnumthy/nopowerbasis.pdf}
\end{itemize}

\end{document}
