\documentclass[12pt,a4paper,ngerman]{scrartcl}

\usepackage{ucs}
\usepackage[utf8x]{inputenc}
\usepackage[ngerman]{babel}
\usepackage{amsmath,amssymb,amscd,amsthm,color,graphicx}
\usepackage[protrusion=true,expansion=false]{microtype}
\usepackage{lmodern}
\usepackage{hyperref}

\setlength\parskip{\medskipamount}
\setlength\parindent{0pt}

\newcommand{\Bild}{\operatorname{im}}
\newcommand{\Kern}{\operatorname{ker}}
\newcommand{\Span}{\operatorname{span}}
\newcommand{\GL}{\mathrm{GL}}
\newcommand{\R}{\mathbb{R}}
\newcommand{\Q}{\mathbb{Q}}
\newcommand{\C}{\mathbb{C}}
\newcommand{\I}{\mathrm{i}}
\newcommand{\Z}{\mathbb{Z}}

\theoremstyle{definition}
\newtheorem*{defn}{Definition}
\newtheorem*{bsp}{Beispiel}

\theoremstyle{plain}
\newtheorem*{prop}{Proposition}
\newtheorem*{beh}{Behauptung}

\theoremstyle{remark}
\newtheorem*{bem}{Bemerkung}

\usepackage{geometry}
\geometry{tmargin=3cm,bmargin=3cm,lmargin=2.5cm,rmargin=2.5cm}

\renewcommand{\labelitemi}{--}

\begin{document}

\section*{Wichtige Isomorphismen von Ringen}

\begin{align}
  R/(0) &\cong R \\
  R/(1) &\cong 0 \text{ (Nullring)} \\
  R/(x,y) &\cong (R/(x))\,/\,([y]) \label{dopp} \\
  (R/\mathfrak{a})[X] &\cong R[X]\,/\,\mathfrak{a}[X] \\
  R[X]/(X-a) &\cong R \label{poly} \\
\intertext{Ist außerdem~$L \supseteq K$ eine Körpererweiterung und~$u \in L$ ein
über~$K$ algebraisches Element mit Minimalpolynom~$m \in K[X]$, so gilt:}
  K(u) = K[u] &\cong K[X]/(m) \label{minp}
\intertext{Schließlich gibt es noch den chinesischen Restsatz:
Sind~$\mathfrak{a}, \mathfrak{b} \subseteq R$ Ideale mit~$\mathfrak{a} +
\mathfrak{b} = (1)$, so gilt~$\mathfrak{a} \mathfrak{b} = \mathfrak{a} \cap
\mathfrak{b}$ und}
  R/(\mathfrak{a} \mathfrak{b}) &\cong R/\mathfrak{a} \times R/\mathfrak{b}
  \label{chin}
\end{align}


\subsection*{Anwendung: Primalitäts- und Maximalitätsuntersuchung}

Diese Rechenregeln sind in Kombination mit dem Satz
\begin{quote}
Ein Ideal~$\mathfrak{a}$ eines Rings~$R$ ist genau dann ein Primideal (bzw. ein
maximales Ideal), wenn der Faktorring~$R/\mathfrak{a}$ ein
Integritätsbereich (bzw. ein Körper) ist.
\end{quote}
nützlich, um ein gegebenes Ideal auf Primalität und Maximalität zu untersuchen.

\subsubsection*{Beispiele}

\begin{itemize}
\item Das Ideal~$(5, X-3) \subseteq \Z[X]$ ist maximal, denn
\[ \Z[X]/(5, X-3) \,\stackrel{\eqref{dopp}}{\cong}\, (\Z/(5))[X]\,/\,(X-[3])
\,\stackrel{\eqref{poly}}{\cong}\, \Z/(5) \]
ist ein Körper.
\item Das Ideal~$(X^2, X-3) \subseteq \Z[X]$ ist weder prim noch maximal, denn
\[ \Z[X]/(X^2, X-3) \,=\, \Z[X]/(3^2, X-3) \,\stackrel{\eqref{dopp}}{\cong}\,
(\Z/(9))[X]\,/\,(X-[3]) \,\stackrel{\eqref{poly}}{\cong}\, \Z/(9) \]
ist weder ein Integritätsbereich noch ein Körper.
\item Das Ideal~$(X^2+1) \subseteq \R[X]$ ist maximal, denn
\[ \R[X] / (X^2+1) \stackrel{\eqref{minp}}{\cong} \R(i) = \C \]
ist ein Körper (von rechts nach links lesen!).
\item Das Ideal~$(X^2-1) = (X+1) \cdot (X-1) \subseteq \R[X]$ ist weder
maximal noch prim, denn
\[ \R[X]/(X^2-1) \,\stackrel{\eqref{chin}}{\cong}\, \R[X]/(X+1) \times \R[X]/(X-1)
\,\stackrel{\eqref{poly}}{\cong}\, \R \times \R \]
ist weder ein Integritätsbereich noch ein Körper. Der chinesische
Restsatz~\eqref{chin} war
anwendbar, denn~$(X+1) + (X-1) = (\mathrm{ggT}(X+1,X-1)) = (1)$.
\end{itemize}

\end{document}

\newpage

\subsection*{Anwendung: Invertieren in einfachen Körpererweiterungen}

Sei~$L \supseteq K$ eine Körpererweiterung und~$u \in L$ ein über~$K$
algebraisches Element mit Minimalpolynom~$m \in K[X]$. Dann gilt also~$K(u) =
K[u] \cong K[X]/(m)$. Somit übertragen sich die Techniken, um in~$K[X]/(m)$
Inverse anzugeben, auf~$K(u)$.

Sei zur Illustration~$x \in K(u)$, dann gibt es ein Polynom~$f \in K[X]$ mit~$x = f(u)$.
Sei~$d = af + bm$ mit~$a,b \in K[X]$ eine Bézoutdarstellung des größten
gemeinsamen Teilers~$d := \mathrm{ggT}(f,m)$.

Dann tritt genau einer der folgenden Fälle ein:
\begin{itemize}
\item Der größte gemeinsame Teiler $d$ ist ein konstantes Polynom. Dann ist~$x$ in~$K(u)$
invertierbar mit Inversem~$a(u)/d \in K(u)$.
\item Der größte gemeinsame Teiler $d$ hat mindestens Grad 1. Dann ist~$x = 0$.
\end{itemize}

\subsubsection*{Beispiele}

\begin{itemize}
\item Sei $u := \sqrt{2} \in \C$. Dann hat~$u$
das Minimalpolynom~$m := X^2 - 2 \in \Q[X]$ über~$\Q$ und eine~$\Q$-Basis
von~$\Q(u)$ ist~$1, u$.

Sei~$x := 3 \sqrt{2} - 5 \in \Q(u)$. Dann können wir obige Überlegung
verwenden, um das Inverse~$x^{-1}$ als Linearkombination dieser Basis zu
schreiben:
In der obigen Notation ist~$f := 3 X - 5$. Eine Nebenrechnung zeigt, dass ein
größter gemeinsamer Teiler von~$f$ und~$m$ das konstante Einspolynom ist, mit
Bé\-zout\-dar\-stel\-lung
\[ 1 = \left(-\frac{5}{7} - \frac{3}{7} X\right) f + \frac{9}{7} \, m. \]
Also lässt sich das Inverse von~$x = f(u)$ als
\[ x^{-1} = -\frac{5}{7} - \frac{3}{7} u \]
schreiben.

\item In Blatt 10, Aufgabe 1(d) geht es um ein komplizierteres Beispiel.
\end{itemize}

\end{document}
