\documentclass{../../alg2/algblatt}

\usepackage{stmaryrd}
\usepackage{tabto}
\usepackage{tikz}
\usepackage{geometry}
\usepackage{extarrows}
\geometry{tmargin=2cm,bmargin=3.0cm,lmargin=3cm,rmargin=3cm}

\pagestyle{plain}

\begin{document}

\begin{center}\Large \sffamily\textbf{Wissenswertes zur Kronecker-Konstruktion}\end{center}

Sei~$K$ ein Körper und~$f \in K[X]$ ein normiertes Polynom vom Grad
mindestens~$1$. Aus diesen Daten können wir einen neuen Ring basteln, den Faktorring
\[ K' := K[U]/(f(U)). \]
Wenn wir diese Konstruktion mit dem Ziel ausführen, eine künstliche Nullstelle
von~$f$ zu bauen (siehe unten), so trägt sie den Namen \emph{Kronecker-Konstruktion}.


\subsubsection*{Elemente von~$K'$}

Die Elemente von~$K'$ sind Äquivalenzklassen von Polynomen, wobei zwei Klassen
genau dann als gleich angesehen werden, wenn die Differenz ihrer Repräsentanten
ein Vielfaches von~$f(U)$ ist. In~$K'$ rechnet man also \emph{modulo~$f(U)$}. Ist~$g$
ein Polynom, das bei Division durch~$f(U)$ den Rest~$r$ lässt, so gilt~$[g] = [r]$
in~$K'$. Wir können daher festhalten:
\[ K' = \{ [r] \,|\, r \in K[U],\ \deg r < \deg f \}. \]
Das Nullelement von~$K'$ ist~$[0]$, die Äquivalenzklasse des Nullpolynoms. Es
gilt~$[0] = [f(U)]$. Das Element~$\alpha := [U] \in K'$ spielt eine besondere
Rolle. Jedes Element von~$K'$ ist ein rationaler (und sogar polynomieller)
Ausdruck in~$\alpha$, daher können wir auch schreiben:
\[ K' = K[\alpha] = K(\alpha). \]


\subsubsection*{Einbettung von~$K$ in~$K'$}

Der Ring~$K'$ ist nicht im wörtlichen Sinn eine Obermenge von~$K$, da~$K'$ ja
neu kon\-stru\-ier\-te Äquivalenzklassen enthält. Vermöge des kanonischen injektiven
Ringhomomorphismus
\[ K \longrightarrow K',\ z \longmapsto [z] \]
können wir jedoch~$K$ als Unterring von~$K'$ \emph{ansehen}: Immer, wenn wir in
die Verlegenheit kommen, ein Element~$z$ von~$K$ als Element von~$K'$
interpretieren zu müssen (etwa wenn eine Formel nur dann Sinn ergibt, wenn an
einer bestimmten Stelle ein Element von~$K'$ steht), so lesen wir
einfach~"`$[z]$"' statt dem wörtlichen~"`$z$"'.


\subsubsection*{Nutzen von~$K'$}

Im Vergleich zum Ausgangskörper~$K$ enthält der Ring~$K'$ das besondere neue
Element~$\alpha = [U]$. Dieses erfüllt die Rechenregel
\[ f(\alpha) = 0 \in K', \]
denn wenn~$f = \sum_{i=0}^n a_i X^i$, so gilt
\[ f(\alpha) = \sum_{i=0}^n a_i [U]^i = \sum_{i=0}^n [a_i] [U]^i =
  \Bigl[\sum_{i=0}^n a_i U^i\Bigr] = [f(U)] = 0. \]
Beim zweiten Gleichheitszeichen haben wir im Sinne des vorherigen Absatzes das
Element~$a_i$ von~$K$ als Element~$[a_i]$ von~$K'$ aufgefasst.

\emph{In~$K'$ gibt es also ein Element~$\alpha$, dass die
Rechenregel~$f(\alpha) = 0$ erfüllt.} Das ist der Grund, wieso die
Kronecker-Konstruktion wichtig ist: Mit ihrer Hilfe können wir
nach Belieben neue Ringe bauen, in denen dann ein vorgegebenes Polynom~$f$ eine
künstliche Nullstelle~$\alpha$ besitzt. Die Kenntnis dieser künstlichen
Nullstelle gibt aber keinerlei Information über den Zahlenwert richtiger
Nullstellen von~$f$ (diese Frage könnte man sich etwa dann stellen, wenn~$K =
\QQ$).


\subsubsection*{Invertierbarkeit in~$K'$}

Sei~$[g] \in K'$ ein beliebiges Element. Um zu entscheiden, ob~$[g]$ in~$K'$
Null, invertierbar oder ein Nullteiler ist, ist es hilfreich, den normierten
größten gemeinsamen Teiler~$d$ von~$g$ und~$f$ zu betrachten. Es gibt dann
nämlich drei Fälle:
\begin{itemize}
\item Fall $d = 1$:

Dann ist~$[g] \in K'$ invertierbar. Aus einer
Bézoutdarstellung der Form~$d = pg + qf$ können wir das Inverse sofort ablesen:
Es ist~$[p] \in K'$, denn
\[ 1 = [d] = [p]\,[g] + [q]\,[f] = [p]\,[g]. \]

\item Fall $d = f$:

Dann ist~$[g] \in K'$ Null, da~$g$ ein Vielfaches von~$f$
ist.

\item Fall $0 < \deg d < \deg f$:

Dann ist~$[g] \in K'$ ein Nullteiler, der
selbst nicht Null ist. Denn da~$d$ gemeinsamer Teiler von~$f$ und~$g$ ist, gibt
es Polynome~$u$ und~$v$ mit~$f = ud$ und~$g = vd$; daher gilt in~$K'$ die
Rechnung
\[ [u]\,[g] = [uvd] = [v]\,[f] = [v]\cdot0 = 0. \]
Weder~$[u]$ noch~$[g]$ sind in~$K'$ Null, da~$u$ und~$g$ keine Vielfachen
von~$f$ sind.
\end{itemize}


\subsubsection*{Der Ring~$K'$ als Oberkörper}

Falls~$f$ irreduzibel ist, kann der dritte Fall des vorherigen Abschnitts nicht
auftreten. Dann ist also jedes Element von~$K'$ entweder Null oder
invertierbar, also ist~$K'$ in diesem Fall ein Körper.

Falls~$f$ reduzibel ist, etwa~$f = gh$ mit~$\deg g, \deg h \geq 1$, so ist~$K'$
kein Körper, da es Nullteiler gibt: Es gilt~$[g]\,[h] = [f] = 0$, obwohl~$[g]
\neq 0$ und~$[h] \neq 0$.

Wenn wir also das Problem
\begin{quote}\emph{Konstruiere einen Oberkörper von~$K$, in dem das Polynom~$f$
eine Nullstelle hat!}\end{quote}
lösen wollen, können wir \emph{nicht} einfach~$K' = K[U]/(f(U))$ betrachten, da
dieser Ring vielleicht kein Körper ist. Falls wir aber~$f$ in irreduzible
Faktoren zerlegen können, etwa~$f = g_1 \cdots g_m$, so ist~$K[U]/(g_1(U))$ ein
Oberkörper von~$K$ mit der gewünschten Eigenschaft: In~$K[U]/(g_1(U))$ ist das
Element~$[U]$ eine künstliche Nullstelle von~$g_1$ und damit auch von~$f$.

Das Problem ist nicht eindeutig lösbar, auch nicht bis auf Isomorphie: Die
Körper~$K[U]/(g_2(U))$, $\ldots$, $K[X]/(g_m(U))$ enthalten auch jeweils eine künstliche
Nullstelle von~$f$, sind aber im Allgemeinen nicht isomorph zu~$K[U]/(g_1(U))$.


\subsubsection*{Der Ring~$K'$ als ideeller Oberkörper}

Wenn wir keine Zerlegung von~$f$ in irreduzible Faktoren bestimmen wollen (oder
können), können wir trotzdem $K' = K[U]/(f(U))$ betrachten. Dieser Oberring
von~$K$ ist zwar im Allgemeinen kein Körper (nur dann, wenn~$f$ irreduzibel
ist), aber es wäre auch verkehrt, ihn als völlig nutzlosen Ring abzutun.

Addition, Subtraktion und Multiplikation in~$K'$ sind unproblematisch, diese
Rechenoperationen benötigen nur, dass~$K'$ ein Ring ist, und das ist stets der
Fall. Nur bei der Division müssen wir aufpassen: Ein Element~$[g] \in K'$ kann
ja Null, invertierbar oder ein Nullteiler sein. In den ersten beiden Fällen ist
alles in Ordnung, diese beiden Fälle erwarten wir ja von einem Körper. Der
dritte Fall darf bei einem Körper nicht auftreten. Aber nicht alle Hoffnung ist
verloren:

Tritt der dritte Fall ein, so haben wir in~$d :=
\operatorname{ggT}(g,f)$ einen nichttrivialen Faktor von~$f$ gefunden. Dann
können wir die gesamte Rechnung mit~$K[U]/(d(U))$ statt~$K[U]/(f(U))$ neustarten
(also von Beginn an neu aufrollen). Wenn wir in unserer Rechnung wieder zur
Frage kommen, ob~$[g]$ invertierbar ist, wird dann die Antwort sein:
Nein,~$[g]$ ist Null (da~$g$ ein Vielfaches von~$d$ ist). Beim zweiten
Durchgang wird der problematische dritte Fall an dieser Stelle also nicht
auftreten.

Sollte an einer späteren Stelle der Rechnung wieder der dritte Fall auftreten,
erhalten wir abermals einen nichttrivialen Faktor und können die Rechnung
abermals neustarten.

\emph{Fazit.} Obwohl der Ring~$K' = K[U]/(f(U))$ nur dann ein Körper ist, wenn~$f$
irreduzibel ist, können wir mit ein wenig Umsicht auch sonst in~$K'$ so
rechnen, \emph{als ob} er ein Körper wäre: Weil wir wissen, durch welchen
besseren Ring wir ihn zu ersetzen haben, wenn wir bei einer Rechnung auf den
problematischen dritten Fall stoßen. In diesem Sinn ist~$K'$ ein \emph{ideeller
Oberkörper} von~$K$. Bei jedem Neustart finden wir einen
nichttrivialen Faktor von~$f$.


\subsubsection*{Beispiel}

Sei~$K = \QQ$ und~$f = X^2 - 5\,X + 6$. Eine Nebenrechnung
würde zeigen, dass dieses Polynom reduzibel ist, aber auf diese Nebenrechnung
haben wir keine Lust; wir versuchen trotzdem, im Ring~$K[U]/(f(U))$ zu rechnen.

Etwa können wir uns die Frage stellen, ob~$\alpha - 1$ invertierbar ist. Dazu
müssen wir den normierten größten gemeinsamen Teiler von~$g := U - 1$ und~$f(U)$
bestimmen: Dieser ist~$d = 1$, und eine Bézoutdarstellung ist durch
\[ d = 1 = -\tfrac{1}{2}(U^2 - 5\,U + 4) \cdot g + \tfrac{1}{2} \cdot f(U) \]
gegeben. Also ist~$[-\tfrac{1}{2}(U^2-5\,U+4)] =
-\tfrac{1}{2}(\alpha^2-5\alpha+4)$ ein Inverses von~$\alpha - 1$.

Auch können wir uns die Frage stellen, ob~$\alpha - 2$ invertierbar ist. Der
normierte größte gemeinsame Teiler von~$\tilde g := U - 2$ und~$f(U)$ ist~$\tilde
d = U-2$. Also tritt der dritte Fall ein: Wir haben den nichttrivialen
Faktor~$X-2$ von~$f(X)$ gefunden und Rollen unsere Rechnung neu auf,
mit~$K[U]/(\tilde d(U))$ statt~$K[U]/(f(U))$.

Nun können wir die Frage nach der Invertierbarkeit von~$\alpha - 2$ erneut
stellen; jetzt ist der normierte größte gemeinsame Teiler von~$U-2$ und~$\tilde
d$ zu berechnen. Dieser ist~$\tilde d$ und daher ist~$\alpha - 2$ im
verbesserten Ring~$K[U]/(\tilde d(U))$ Null.


\subsubsection*{Konzept des Zerfällungskörpers}

Ein Zerfällungskörper eines Polynoms über einem Körper~$K$ ist ein Oberkörper
von~$K$, in dem das Polynom vollständig in Linearfaktoren $(X-x_1) \cdots
(X-x_n)$ zerfällt und in dem jedes Element ein polynomieller Ausdruck in den
Nullstellen~$x_i$ ist.

Ein Zerfällungskörper eines Polynoms ist also minimal unter allen
Oberkörpern von~$K$, in denen das Polynom in Linearfaktoren zerfällt.

\emph{Beispiel.} Der Körper~$\QQ$ ist kein Zerfällungskörper für das
Polynom~$X^2 + 1 \in \QQ[X]$, da dieses über~$\QQ$ noch nicht in Linearfaktoren
zerfällt. Der Rechenbereich~$\CC$ ist auch kein Zerfällungskörper, da sich nicht jedes
Element von~$\CC$ als polynomiellen Ausdruck mit rationalen Koeffizienten in
den beiden Nullstellen~$\pm \i$ schreiben lässt; $\CC$ ist viel zu groß. Ein
Zerfällungskörper ist~$\QQ[\i]$.


\subsubsection*{Konstruktion von Zerfällungskörpern}

Manchmal zerfällt in~$K' = K[U]/(f(U))$ das
Polynom~$f$ schon in Linearfaktoren: Zwar haben wir nur \emph{eine} Nullstelle
künstlich hinzugefügt, es kann aber sein, dass sich die weiteren Nullstellen
von~$f$ über diese eine ausdrücken lassen. In diesem Fall ist~$K'$ ein
(vielleicht nur ideeller) Zerfällungskörper für~$f$.

\emph{Beispiel.} Im Ring~$K[U]/(f(U))$ aus dem vorherigen Zahlenbeispiel
besitzt~$f$ neben der künstlichen Nullstelle~$\alpha = [U]$ noch die
Nullstelle~$5-\alpha$, denn es gilt
\begin{align*}
  (X-\alpha) \cdot (X-(5-\alpha)) &=
  X^2 - (\alpha + 5 - \alpha)\,X + \alpha \cdot (5-\alpha) \\
  &= X^2 - 5\,X + (5\alpha - \alpha^2) \\
  &= X^2 - 5\,X + 6 = f.
\end{align*}

Es kommt aber auch vor, dass~$f$ über~$K'$ noch nicht in Linearfaktoren
zerfällt. Wenn wir dann einen nichttrivialen Faktor~$h \in K'[X]$ von~$f$
finden (zum Beispiel das Ergebnis der Polynomdivision von~$f$
durch~$X-\alpha$), können wir die Kronecker-Konstruktion einfach wiederholen
und~$K'[V]/(h(V))$ betrachten. In diesem Ring wird~$h$ eine künstliche
Nullstelle~$\beta := [V]$ besitzen und daher reduzibel sein, sodass wir einer
Zerlegung von~$f$ in Linearfaktoren ein Stück näher gekommen sind.


\subsubsection*{Bemerkung zu Blatt 11, Aufgabe 5}

Bei Aufgabe~5 von Blatt~11 ist zu zeigen, dass ein \emph{tatsächlicher}
Zerfällungskörper existiert, nicht nur ein ideeller. Dazu kann man die
Kronecker-Konstruktion verwenden, muss aber darauf aufpassen, nur modulo
irreduzibler Polynome zu rechnen. Damit das gelingt, ist die Voraussetzung,
dass der Grundkörper~$K$ endlich ist, wesentlich.

\end{document}
