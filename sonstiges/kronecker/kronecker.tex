\documentclass{../../alg2/algblatt}

\usepackage{stmaryrd}
\usepackage{tabto}
\usepackage{tikz}
\usepackage{geometry}
\usepackage{extarrows}
%\geometry{tmargin=2cm,bmargin=2.0cm,lmargin=2.6cm,rmargin=2.6cm}

\pagestyle{plain}

\begin{document}

\begin{center}\Large \sffamily\textbf{Wissenswertes zur Kronecker-Konstruktion}\end{center}

Sei~$K$ ein Körper und~$f \in K[X]$ ein normiertes Polynom vom Grad
mindestens~$1$. Aus diesen Daten können wir einen neuen Ring basteln, den Faktorring
\[ K' := K[X]/(f). \]


\subsubsection*{Elemente von~$K'$}

Die Elemente von~$K'$ sind Äquivalenzklassen von Polynomen, wobei zwei Klassen
genau dann als gleich angesehen werden, wenn die Differenz ihrer Repräsentanten
ein Vielfaches von~$f$ ist. In~$K'$ rechnet man also \emph{modulo~$f$}. Ist~$g$
ein Polynom, das bei Division durch~$f$ den Rest~$r$ lässt, so gilt~$[g] = [r]$
in~$K'$. Wir können daher festhalten:
\[ K' = \{ [r] \,|\, r \in K[X],\ \deg r < \deg f \}. \]
Das Nullelement von~$K'$ ist~$[0]$, die Äquivalenzklasse des Nullpolynoms. Es
gilt~$[0] = [f]$.


\subsubsection*{Einbettung von~$K$ in~$K'$}

Der Ring~$K'$ ist nicht im wörtlichen Sinn eine Obermenge von~$K$, da~$K'$ ja
neu kon\-stru\-ier\-te Äquivalenzklassen enthält. Vermöge des kanonischen injektiven
Ringhomomorphismus
\[ K \longrightarrow K',\ z \longmapsto [z] \]
können wir jedoch~$K$ als Unterring von~$K'$ \emph{ansehen}: Immer, wenn wir in
die Verlegenheit kommen, ein Element~$z$ von~$K$ als Element von~$K'$
interpretieren zu müssen (etwa wenn eine Formel nur dann Sinn ergibt, wenn an
einer bestimmten Stelle ein Element von~$K'$ steht), so lesen wir
einfach~"`$[z]$"' statt dem wörtlichen~"`$z$"'.


\subsubsection*{Nutzen von~$K'$}

Im Vergleich zum Ausgangskörper~$K$ enthält der Ring~$K'$ ein besonderes neues
Element, nämlich~$\alpha := [X]$. Dieses erfüllt die Rechenregel
\[ f(\alpha) = 0 \in K', \]
denn wenn~$f = \sum_{i=0}^n a_i X^i$, so gilt
\[ f(\alpha) = \sum_{i=0}^n a_i [X]^i = \sum_{i=0}^n [a_i] [X]^i =
  \Bigl[\sum_{i=0}^n a_i X^i\Bigr] = [f] = 0. \]
Beim zweiten Gleichheitszeichen haben wir im Sinne des vorherigen Absatzes das
Element~$a_i$ von~$K$ als Element~$[a_i]$ von~$K'$ aufgefasst.

\emph{In~$K'$ gibt es also ein Element~$\alpha$, dass die
Rechenregel~$f(\alpha) = 0$ erfüllt.} Das ist der Grund, wieso die
Kronecker-Konstruktion wichtig ist: Mit ihrer Hilfe können wir
nach Belieben neue Ringe bauen, in denen dann ein vorgegebenes Polynom~$f$ eine
künstliche Nullstelle~$\alpha$ besitzt. Die Kenntnis dieser künstlichen
Nullstelle gibt aber keinerlei Information über den Zahlenwert richtiger
Nullstellen von~$f$ (diese Frage könnte man sich etwa dann stellen, wenn~$K =
\QQ$).


\subsubsection*{Invertierbarkeit in~$K'$}

Sei~$[g] \in K'$ ein beliebiges Element. Um zu entscheiden, ob~$[g]$ in~$K'$
Null, invertierbar oder ein Nullteiler ist, ist es hilfreich, den normierten
größten gemeinsamen Teiler~$d$ von~$g$ und~$f$ zu betrachten. Es gibt dann
nämlich drei Fälle:
\begin{itemize}
\item Fall $d = 1$:

Dann ist~$[g] \in K'$ invertierbar. Aus einer
Bézoutdarstellung der Form~$d = pg + qf$ können wir das Inverse sofort ablesen:
Es ist~$[p] \in K'$, denn
\[ 1 = [d] = [p]\,[g] + [q]\,[f] = [p]\,[g]. \]

\item Fall $d = f$:

Dann ist~$[g] \in K'$ Null, da~$g$ ein Vielfaches von~$f$
ist.

\item Fall $0 < \deg d < \deg f$:

Dann ist~$[g] \in K'$ ein Nullteiler, der
selbst nicht Null ist. Denn da~$d$ gemeinsamer Teiler von~$f$ und~$g$ ist, gibt
es Polynome~$u$ und~$v$ mit~$f = ud$ und~$g = vd$; daher gilt in~$K'$ die
Rechnung
\[ [u]\,[g] = [uvd] = [v]\,[f] = [v]\cdot0 = 0. \]
Weder~$[u]$ noch~$[g]$ sind in~$K'$ Null, da~$u$ und~$g$ keine Vielfachen
von~$f$ sind.
\end{itemize}


\subsubsection*{Der Ring~$K'$ als Oberkörper}

Falls~$f$ irreduzibel ist, kann der dritte Fall des vorherigen Abschnitts nicht
auftreten. Dann ist also jedes Element von~$K'$ entweder Null oder
invertierbar, also ist~$K'$ in diesem Fall ein Körper.

Falls~$f$ reduzibel ist, etwa~$f = gh$ mit~$\deg g, \deg h \geq 1$, so ist~$K'$
kein Körper, da es Nullteiler gibt: Es gilt~$[g]\,[h] = [f] = 0$, obwohl~$[g]
\neq 0$ und~$[h] \neq 0$.

Wenn wir also das Problem
\begin{quote}\emph{Konstruiere einen Oberkörper von~$K$, in dem das Polynom~$f$
eine Nullstelle hat!}\end{quote}
lösen wollen, können wir \emph{nicht} einfach~$K' = K[X]/(f)$ betrachten, da
dieser Ring vielleicht kein Körper ist. Falls wir aber~$f$ in irreduzible
Faktoren zerlegen können, etwa~$f = g_1 \cdots g_m$, so ist~$K[X]/(g_1)$ ein
Oberkörper von~$K$ mit der gewünschten Eigenschaft: In~$K[X]/(g_1)$ ist das
Element~$[X]$ eine künstliche Nullstelle von~$g_1$ und damit auch von~$f$.

\end{document}
