\documentclass{../../algblatt}

\usepackage{stmaryrd}
\usepackage{tabto}
\usepackage{tikz}
\usepackage{geometry}
\usepackage{extarrows}
\geometry{tmargin=2cm,bmargin=2.0cm,lmargin=2.6cm,rmargin=2.6cm}

\pagestyle{plain}

\newcommand{\fmini}[2]{\framebox{\begin{minipage}{#1}#2\end{minipage}}}
\makeatletter
\def\underunbrace#1{\mathop{\vtop{\m@th\ialign{##\crcr
      $\hfil\displaystyle{#1}\hfil$\crcr\noalign{\kern3\p@\nointerlineskip}
      \crcr\noalign{\kern3\p@}}}}\limits}
\def\overunbrace#1{\mathop{\vbox{\m@th\ialign{##\crcr\noalign{\kern3\p@}
      \crcr\noalign{\kern3\p@\nointerlineskip}
      $\hfil\displaystyle{#1}\hfil$\crcr}}}\limits}
\makeatother

\renewcommand{\labelitemi}{--}

\begin{document}

\begin{center}\Large \sffamily\textbf{Hauptsatz der Galoistheorie}\end{center}

\emph{Situation:} \tabto{1.85cm}
Sei~$K$ ein Koeffizientenbereich (etwa~$K = \QQ$ oder~$K = \QQ(\sqrt[3]{2})$).
\\
\tabto{1.85cm} Sei~$f(X) \in K[X]$ ein normiertes separables Polynom. \\
\tabto{1.85cm} Seien~$x_1,\ldots,x_n$ die Nullstellen von~$f(X)$.
Sei~$E := K(x_1,\ldots,x_n)$. \\
\tabto{1.85cm} Sei~$G := \Gal_K(x_1,\ldots,x_n)$ die Galoisgruppe der
Nullstellen über~$K$.

\emph{Dann gilt:} \tabto{1.85cm} Die Zuordnung
\[
  \begin{array}{@{}rcl@{}}
    \fmini{0.33\textwidth}{Menge der Untergruppen von~$G$}
    &\xleftrightarrow{\text{\qquad1:1\qquad}}&
    \fmini{0.45\textwidth}{Menge der Zwischenerweiterungen von~$E|K$} \\[1.3em]
    H &\longmapsto&
      E^H := \{ x \in E \,|\, \text{$\sigma(x) = x$ für alle~$\sigma \in H$} \} \\[0.3em]
    \mathrm{Gal}_L(x_1,\ldots,x_n) &\longmapsfrom& L
%   = \{ \sigma \in G \,|\, \text{$\sigma \cdot x = x$ für alle~$x \in L$} \}
  \end{array}
\]
ist eine inklusionsumkehrende Bijektion. Der Rechenbereich~$E^H$ wird auch als
\emph{Fixkörper} bezüglich der Untergruppe~$H$ bezeichnet.
Genauer gelten für alle Zwischenerweiterungen~$L,L'$ von~$E|K$
und Untergruppen~$H,H'$ von~$G$ folgende Aussagen.


\subsubsection*{Hin und zurück}
\vspace{-1.5em}
\begin{align*}
  E^{\mathrm{Gal}_L(x_1,\ldots,x_n)} &= L. \\
  \mathrm{Gal}_{E^H}(x_1,\ldots,x_n) &= H.
\end{align*}

Der Fixkörper zur Galoisgruppe einer Zwischenerweiterung~$L$
ist wieder~$L$. Die Galoisgruppe über dem Fixkörper einer Untergruppe~$H$ ist
wieder~$H$.
Die erste Aussage umfasst die bekannte Tatsache, dass eine
Zahl aus~$E$, welche invariant unter der Wirkung der
Galoisgruppe~$\Gal_L(x_1,\ldots,x_n)$ ist, schon in~$L$ liegen muss. Wieso?


\subsubsection*{Größer und kleiner}
\vspace{-1.5em}
\begin{align*}
  L \subseteq L' &\ \Longleftrightarrow\ 
    \mathrm{Gal}_L(x_1,\ldots,x_n) \supseteq \Gal_{L'}(x_1,\ldots,x_n). \\
  H \subseteq H' &\ \Longleftrightarrow\ 
    E^H \supseteq E^{H'}.
\end{align*}

Je größer der Koeffizientenbereich, desto kleiner ist die zugehörige
Galoisgruppe; und umgekehrt: Je größer die Untergruppe, desto kleiner ist der
zugehörige Fixkörper. Wieso sind beide Aussagen anschaulich?


\subsubsection*{Grade und Indizes}
\vspace{-1.5em}
\begin{align*}
  (|H| =)\ \gra{H}{1} &= \gra{E}{E^H}. \\
  \gra{E^H}{K} &= \gra{G}{H}\ (= |G| \mathop{/} |H|).
\end{align*}

Die \emph{Ordnung} einer Untergruppe~$H$ ist durch den
Grad~$\gra{E}{E^H}$ gegeben. Der \emph{Index} einer Untergruppe~$H$ ist durch
den Grad~$\gra{E^H}{K}$ gegeben. Wieso passt das mit dem inklusionsumkehrenden
Charakter zusammen?


\subsubsection*{Normalität}

In Algebra II werden wir eine einfache Charakterisierung dafür
kennenlernen, wann~$H$ ein Normalteiler in~$G$ ist.


\newpage
\subsubsection*{Wie kann man die relativen Galoisgruppen ausrechnen?}

Wenn~$L = K(z_1,\ldots,z_m)$, gilt
\[ \Gal_L(x_1,\ldots,x_n) = \{ \sigma \in G \,|\,
  \text{$\sigma \cdot z_i = z_i$ für~$i = 1,\ldots,m$} \}. \]


\subsubsection*{Wie kann man Erzeuger der Fixkörper bestimmen?}
Falls~$H = \{ \sigma_1,\ldots,\sigma_m \}$ und~$E = K(t)$, gilt
\[ E^H = K(
  e_1(\sigma_1 \cdot t, \ldots, \sigma_m \cdot t),
  \ldots,
  e_m(\sigma_1 \cdot t, \ldots, \sigma_m \cdot t)), \]
wobei die~$e_i$ die elementarsymmetrischen Funktionen in~$m$ Unbekannten sind.


\subsubsection*{Wozu ist der Hauptsatz gut?}

\begin{itemize}
\item Der Hauptsatz klärt die Struktur der Zwischenerweiterungen, durch
Rückführung auf die zugänglichere Struktur der Untergruppen.
\item Informationen über Grade liefern Informationen über Indizes und
umgekehrt; manchmal ist das eine leichter zu berechnen als das andere.
\item Der Hauptsatz geht wesentlich im Beweis der fundamentalen Äquivalenz
\[ \framebox{Gleichung $f(X) = 0$ auflösbar}
  \Longleftrightarrow
  \framebox{Galoisgruppe~$\Gal_K(x_1,\ldots,x_n)$ auflösbar}
\]
ein: Unter geeigneten Voraussetzungen an den Koeffizientenbereich~$K$ bilden
die Galoisgruppen zu den einzelnen Stufen eines Turms aus Radikalerweiterungen
(nach Streichen mehrfach vorkommender Untergruppen) eine Normalreihe der
vollen Galoisgruppe über~$K$.
\item Der Hauptsatz ist ein erstes Beispiel für tiefe Dualitätsresultate,
von denen es noch viele weitere gibt.
\end{itemize}

\end{document}
