\documentclass{../algblatt}
\usepackage{multicol}

\newenvironment{indentblock}{%
  \list{}{\leftmargin\leftmargin}%
  \item\relax
}{%
  \endlist
}

\begin{document}

\begin{center}\Large \textsf{\textbf{Hinweise zu den Übungsaufgaben in Algebra I}}\end{center}
\vspace{1em}


\section*{Übungsblatt 1}

\paragraph{Aufgabe 4.} Hier gibt es viele verschiedene Lösungswege. Eine
Möglichkeit besteht darin, den Winkel~$\alpha$ bei den unteren Ecken der Skizze
als Innenwinkel von drei verschiedenen Teildreiecken zu erkennen und den
Tangens von~$\alpha$ dann jeweils über Gegen- und Ankathete auszudrücken.
Zusammen mit dem Satz von Pythagoras erhält man dann drei Gleichungen für drei
Unbekannte.


\section*{Übungsblatt 2}

\paragraph{Aufgabe 5.} Die Teilaufgaben~a), c) und~d) können unabhängig von~b)
bearbeitet werden.


\section*{Übungsblatt 3}

\paragraph{Aufgabe 1.} Für Teilaufgabe~a) ist es nützlich zu wissen, dass der
Realteil einer algebraischen Zahl wieder algebraisch ist (wieso stimmt das?).
Für die Teilaufgaben~b) und~c) ist es nicht nötig, eine
explizite Darstellung der Lösung~$\alpha$ zu berechnen.

\paragraph{Aufgabe 2.} Für Teilaufgabe~a) ist es ebenfalls nicht nötig,
explizite Darstellungen der Lösungen~$x$ bzw.~$y$ zu berechnen. Auch ohne deren
Kenntnis kann man nämlich das Verfahren aus Proposition~1.3 bzw. Hilfssatz~1.4 des
Skripts einsetzen. Zur Kontrolle hier eine der insgesamt sechs Teilrechnungen,
bevor man zur Bestimmung der Determinante schreiten kann:
\[ xy \cdot c_{20} = -c_{01} + c_{11}. \]

\paragraph{Aufgabe 5.} Je nachdem, wie man Teilaufgabe~b) angeht, ist folgende
für ganze Zahlen~$a$ und~$n$ gültige Äquivalenz hilfreich:
\[ \left[\exists m \in \ZZ{:}\ a\,m \equiv 1 \mod{n}\right]
  \quad\Longleftrightarrow\quad
  \text{$a$ und~$n$ sind zueinander teilerfremd.} \]
Ausgeschrieben besagt die linke Aussage, dass es eine weitere ganze Zahl~$m$
gibt, sodass die Zahl~$a\,m$ bei Division durch~$n$ den Rest~1 lässt.


\section*{Übungsblatt 4}

\paragraph{Aufgabe 1.} Bezeichne~$f$ die zugehörige Polynomfunktion. Zeige,
dass für komplexe Zahlen~$z \in \CC$, die weiter
als die angegebene Länge vom Ursprung entfernt sind, der Betrag~$|f(z)|$ echt
größer als Null ist. Unter anderem benötigt man dazu die für alle komplexen
Zahlen~$z_1,\ldots,z_n$ gültige Dreiecksungleichung
\[ |z_1 + \cdots + z_n| \leq |z_1| + \cdots + |z_n| \]
und die für alle komplexen Zahlen~$x,y$ sog. umgekehrte Dreiecksungleichung
\[ |x + y| \geq \Bigl| |x| - |y| \Bigr| \geq |x| - |y|. \]

\paragraph{Aufgabe 2.} Vorgehen kann man wie immer
bei~$\epsilon$/$\delta$-Aufgaben: Man gibt sich zunächst~$R > 0$ und~$\epsilon > 0$
beliebig vor. Dann lässt man schon an dieser Stelle Platz für die Definition
von~$\delta$, da~$\delta$ nicht von~$z$ und~$w$ abhängen darf -- banalerweise ist die einfachste
Möglichkeit, das sicherzustellen,~$\delta$ vor~$z$ und~$w$ einzuführen.
Danach gibt
man sich beliebige~$z,w \in \CC$ mit~$|z|,|w| \leq R$ und~$|z-w| < \delta$ vor.
In diesem Kontext versucht man schließlich (hierin steckt die Hauptarbeit), den
Abstand~$|f(z)-f(w)|$ nach oben durch ein Vielfaches von~$|z-w|$ abzuschätzen;
ist das gelungen, kann man nachträglich die Definition von~$\delta$ ausfüllen.
Für die Hauptarbeit ist neben der Dreiecksungleichung vielleicht die Identität
\[ z^m - w^m = (z - w) \cdot (z^{m-1} + z^{m-2} w + z^{m-3} w^2 + \cdots + z w^{m-2}
+ w^{m-1}) \]
hilfreich (wieso gilt sie?).

\paragraph{Aufgabe 5.} Die Behauptung von Teilaufgabe~b) ist nicht mit ihrer
Umkehrung zu verwechseln (diese wird im Skript auf Seite~47 bewiesen).


\section*{Übungsblatt 5}

\paragraph{Aufgabe 1.} Zum Vergleich: Die dritte elementarsymmetrische Funktion
in den Variablen~$X,Y,Z,W$ ist
\[ e_3(X,Y,Z,W) = XYZ + XYW + XZW + YZW. \]
Die in Teilaufgabe~d) auftretende Zahl~$\binom{n}{k}$
ist die Anzahl der Möglichkeiten, aus der Menge~$\{ 1,\ldots,n \}$
eine~$k$-elementige Teilmenge auszuwählen.

\paragraph{Aufgabe 3.} Teilaufgabe~b) kann man durch eine längere, aber
einfache, Rechnung lösen, wenn man direkt die Definition der Diskriminante
benutzt und die durch den Vietaschen Satz gegebenen Relationen beachtet. Dazu
ein Tipp: Als erstes die dritte Lösung über die anderen beiden Lösungen
ausdrücken, dann~$\Delta$ und~$-4p^3 - 27q^2$ beide vollständig
ausmultiplizieren und die Ergebnisse vergleichen. Man kann aber auch die
Rechenarbeit gegen Denkarbeit tauschen, wenn man den Tipp von Seite~61 des
Skripts befolgt und ausarbeitet.

\paragraph{Aufgabe 5.} In der gesamten Aufgabe bezeichnet~"`$f^{(k)}$"'
die~$k$-te Ableitung eines Polynoms~$f$. Teilaufgabe~a) kann man etwa mit einem
Induktionsbeweis und der für alle~$k,i \geq 0$ gültigen Identität
\[ \binom{k+1}{i} = \binom{k}{i-1} + \binom{k}{i} \]
in Angriff nehmen. Die Summenschreibweise in der Angabe bedeutet, dass über
alle natürlichen Zahlen~$i,j \geq 0$, die die Beziehung~$i + j = k$ erfüllen, summiert
wird. Eine sinnvolle Konvention ist $\binom{k}{-1} := 0$.
Vor der unendlichen Summe in Teilaufgabe~c) muss man keine
Angst haben: Denn ab einem gewissen Summationsindex sind die auftretenden
Ableitungen sowieso null, sodass die unendliche Summe tatsächlich eine endliche
ist. Man hat schon viel gewonnen, wenn man die Behauptung für die
Spezialfälle~$f := X^n$, $n \geq 0$, bewiesen hat; dafür ist vielleicht der
binomische Lehrsatz
\[ (a + b)^n = \sum_{i=0}^n \binom{n}{i} a^i \, b^{n-i} \]
und die Formel $\binom{n}{k} = \frac{n!}{k! \cdot (n-k)!}$ hilfreich.


\section*{Übungsblatt 6}

\paragraph{Aufgabe 2.} Bei Teilaufgabe~a) spart man sich viel Rechenaufwand,
wenn man durch die Substitution~$Y := X + \frac{a}{3}$ die gegebene Gleichung
auf die reduzierte Form
\[ Y^3 + pY + q = 0 \quad\text{mit}\quad
  p = b - \frac{a^2}{3}, q = \frac{2a^3 - 9ab + 27c}{27} \]
bringt. Die Diskriminante dieser Gleichung ist nämlich dieselbe wie die von der
ursprünglichen Gleichung (wieso?) und dank Aufgabe~3b) von Übungsblatt~5
einfacher zu berechnen. Vielleicht findet ihr aber auch andere kreative
Lösungswege. Nur zur Kontrolle: Das Ergebnis wird
\[ \Delta = a^2b^2 - 4b^3 - 4a^3c - 27c^2 + 18abc \]
sein. Teilaufgabe~b) ist unabhängig von~a) bearbeitbar.

\paragraph{Aufgabe 5.} Die Definition von~$R$ in Teilaufgabe~a) lautet etwas
ausführlicher
\[ R = \prod_{i=1}^n \prod_{j=1}^m (x_i - y_j). \]
Für die Bearbeitung der Aufgabe ist Satz~2.12 von Seite~58 des Skripts
hilfreich. Ohne Beweis kann verwendet werden, dass dieser nicht nur für
Polynome mit ganzen, rationalen, reellen, komplexen und algebraischen
Koeffizienten funktioniert, sondern auch für Polynome, deren Koeffizienten
selbst aus einem Rechenbereich von Polynomen (oder einem Rechenbereich von
symmetrischen Polynomen) stammen. Diesen Satz wird man dann insgesamt zweimal
anwenden müssen. Teilaufgabe~b) kann \emph{unabhängig} von Teilaufgabe~a)
bearbeitet werden. Mit \emph{verschwinden} ist \emph{Null sein} gemeint. Ein
Ansatz ist (wieso?), den Ausdruck
\[ R := (x_1 - y_1) \cdot (x_1 - y_2) \cdot (x_2 - y_1) \cdot (x_2 - y_2) \]
zu verwenden, wobei~$x_1,x_2$ die Lösungen der ersten und~$y_1,y_2$ die
Lösungen der zweiten Gleichung sind. Dann muss man diesen Ausdruck so
umschreiben, dass nur noch die Gleichungskoeffizienten, aber nicht mehr die
Lösungen, vorkommen. Das ist etwa mit den Beziehungen aus dem Vietaschen Satz
oder der Mitternachtsformel möglich.

\end{document}

\section*{Übungsblatt 7}

\paragraph{Aufgabe 1.} Bei Teilaufgabe~a) sollte man unbedingt den euklidischen
Algorithmus verwenden, wenn man nicht stundenlang knobeln möchte. Für
Teilaufgabe~b) hier die Erinnerung an die relevante Definition:
\begin{quote}Ein Polynom~$d$ heißt genau dann \emph{größter gemeinsamer Teiler}
zweier Polynome~$f$ und~$g$, falls
\begin{enumerate}
\item[1.] es ein Teiler von~$f$ und von~$g$ (also ein gemeinsamer Teiler) ist und
\item[2.] für jeden gemeinsamen Teiler~$\widetilde d$ von~$f$ und~$g$ gilt,
dass $\widetilde d$ seinerseits ein Teiler
von~$d$ ist (kurz: $\widetilde d \mid d$).
\end{enumerate}
\end{quote}
Ohne eine Normiertheitsbedingung haben übrigens je zwei Polynome unendlich
viele größte gemeinsame Teiler. Wenn man von \emph{dem} größten gemeinsamer
Teiler spricht, ist von diesen unendlich vielen immer der normierte gemeint.
Den Eindeutigkeitsteil von Teilaufgabe~b) kann man mit folgender Vorlage in
Angriff nehmen:
\begin{quote}Seien~$d$ und~$\widetilde d$ beides normierte größte gemeinsame
Teiler von~$f$ und~$g$. Dann\ldots, daher folgt~$d = \widetilde d$.\end{quote}

\paragraph{Aufgabe 2.} Beispiel~3.5 auf Seite~75 des Skripts zeigt eine
Möglichkeit, Teilaufgabe~a) zu lösen. Andere Lösungswege sind aber auch
möglich. Für Teilaufgabe~c) mag es hilfreich sein, dass der relevante größte
gemeinsame Teiler im Skript schon berechnet worden ist.

\paragraph{Aufgaben 3 und 4.} Der Bequemlichkeit halber hier die nötigen
Definitionen:
\begin{quote}
\begin{enumerate}
\item[1.] Eine \emph{Zerlegung} eines normierten Polynoms~$f$ mit rationalen
Koeffizienten ist eine Darstellung von~$f$ als Produkt~$f = f_1 \cdots f_n$
aus~$n \geq 1$ normierten, nichtkonstanten Polynomen mit rationalen
Koeffizienten.
\item[2.] Ein normiertes Polynom~$f$ mit rationalen Koeffizienten heißt genau
dann \emph{irreduzibel (über den rationalen Zahlen)}, wenn es genau eine
Zerlegung zulässt, und zwar die triviale: $f = f$. Sonst heißt es
\emph{reduzibel}.
\end{enumerate}
\end{quote}
Aus der präzisen Art und Weise, wie diese Definitionen formuliert sind, folgt
insbesondere, dass das Einspolynom (das ist das konstante Polynom~$1$) nicht
als irreduzibel gilt (wieso?). Das ist auch gut so, denn sonst wäre die
Eindeutigkeit der Zerlegung in irreduzible Faktoren (Proposition~3.9 im Skript)
nicht mehr gegeben (wieso?). Abschließend sei bemerkt, dass die
\emph{Umkehrung} von Teilaufgabe~4a) Gegenstand der Vorlesung war
(Folgerung~3.11 im Skript).

\end{document}

\section*{Übungsblatt 9}

A1b): "nichttrivial", "verschwindend" erklären"
