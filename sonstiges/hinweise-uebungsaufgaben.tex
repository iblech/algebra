\documentclass{../algblatt}
\usepackage{multicol}

\newenvironment{indentblock}{%
  \list{}{\leftmargin\leftmargin}%
  \item\relax
}{%
  \endlist
}

\begin{document}

\begin{center}\Large \textsf{\textbf{Hinweise zu den Übungsaufgaben in Algebra I}}\end{center}
\vspace{1em}


\section*{Übungsblatt 1}

\paragraph{Aufgabe 4.} Hier gibt es viele verschiedene Lösungswege. Eine
Möglichkeit besteht darin, den Winkel~$\alpha$ bei den unteren Ecken der Skizze
als Innenwinkel von drei verschiedenen Teildreiecken zu erkennen und den
Tangens von~$\alpha$ dann jeweils über Gegen- und Ankathete auszudrücken.
Zusammen mit dem Satz von Pythagoras erhält man dann drei Gleichungen für drei
Unbekannte.


\section*{Übungsblatt 2}

\paragraph{Aufgabe 5.} Die Teilaufgaben~a), c) und~d) können unabhängig von~b)
bearbeitet werden.


\section*{Übungsblatt 3}

\paragraph{Aufgabe 1.} Für Teilaufgabe~a) ist es nützlich zu wissen, dass der
Realteil einer algebraischen Zahl wieder algebraisch ist (wieso stimmt das?).
Für die Teilaufgaben~b) und~c) ist es nicht nötig, eine
explizite Darstellung der Lösung~$\alpha$ zu berechnen.

\paragraph{Aufgabe 2.} Für Teilaufgabe~a) ist es ebenfalls nicht nötig,
explizite Darstellungen der Lösungen~$x$ bzw.~$y$ zu berechnen. Auch ohne deren
Kenntnis kann man nämlich das Verfahren aus Proposition~1.3 bzw. Hilfssatz~1.4 des
Skripts einsetzen. Zur Kontrolle hier eine der insgesamt sechs Teilrechnungen,
bevor man zur Bestimmung der Determinante schreiten kann:
\[ xy \cdot c_{20} = -c_{01} + c_{11}. \]

\paragraph{Aufgabe 5.} Je nachdem, wie man Teilaufgabe~b) angeht, ist folgende
für ganze Zahlen~$a$ und~$n$ gültige Äquivalenz hilfreich:
\[ \left[\exists m \in \ZZ{:}\ a\,m \equiv 1 \mod{n}\right]
  \quad\Longleftrightarrow\quad
  \text{$a$ und~$n$ sind zueinander teilerfremd.} \]
Ausgeschrieben besagt die linke Aussage, dass es eine weitere ganze Zahl~$m$
gibt, sodass die Zahl~$a\,m$ bei Division durch~$n$ den Rest~1 lässt.


\section*{Übungsblatt 4}

\paragraph{Aufgabe 1.} Bezeichne~$f$ die zugehörige Polynomfunktion. Zeige,
dass für komplexe Zahlen~$z \in \CC$, die weiter
als die angegebene Länge vom Ursprung entfernt sind, der Betrag~$|f(z)|$ echt
größer als Null ist. Unter anderem benötigt man dazu die für alle komplexen
Zahlen~$z_1,\ldots,z_n$ gültige Dreiecksungleichung
\[ |z_1 + \cdots + z_n| \leq |z_1| + \cdots + |z_n| \]
und die für alle komplexen Zahlen~$x,y$ sog. umgekehrte Dreiecksungleichung
\[ |x + y| \geq \Bigl| |x| - |y| \Bigr| \geq |x| - |y|. \]

\paragraph{Aufgabe 2.} Vorgehen kann man wie immer
bei~$\epsilon$/$\delta$-Aufgaben: Man gibt sich zunächst~$R > 0$ und~$\epsilon > 0$
beliebig vor. Dann lässt man schon an dieser Stelle Platz für die Definition
von~$\delta$, da~$\delta$ nicht von~$z$ und~$w$ abhängen darf -- banalerweise ist die einfachste
Möglichkeit, das sicherzustellen,~$\delta$ vor~$z$ und~$w$ einzuführen.
Danach gibt
man sich beliebige~$z,w \in \CC$ mit~$|z|,|w| \leq R$ und~$|z-w| < \delta$ vor.
In diesem Kontext versucht man schließlich (hierin steckt die Hauptarbeit), den
Abstand~$|f(z)-f(w)|$ nach oben durch ein Vielfaches von~$|z-w|$ abzuschätzen;
ist das gelungen, kann man nachträglich die Definition von~$\delta$ ausfüllen.
Für die Hauptarbeit ist neben der Dreiecksungleichung vielleicht die Identität
\[ z^m - w^m = (z - w) \cdot (z^{m-1} + z^{m-2} w + z^{m-3} w^2 + \cdots + z w^{m-2}
+ w^{m-1}) \]
hilfreich (wieso gilt sie?).

\paragraph{Aufgabe 5.} Die Behauptung von Teilaufgabe~b) ist nicht mit ihrer
Umkehrung zu verwechseln (diese wird im Skript auf Seite~47 bewiesen).


\section*{Übungsblatt 5}

\paragraph{Aufgabe 1.} Zum Vergleich: Die dritte elementarsymmetrische Funktion
in den Variablen~$X,Y,Z,W$ ist
\[ e_3(X,Y,Z,W) = XYZ + XYW + XZW + YZW. \]
Die in Teilaufgabe~d) auftretende Zahl~$\binom{n}{k}$
ist die Anzahl der Möglichkeiten, aus der Menge~$\{ 1,\ldots,n \}$
eine~$k$-elementige Teilmenge auszuwählen.

\paragraph{Aufgabe 3.} Teilaufgabe~b) kann man durch eine längere, aber
einfache, Rechnung lösen, wenn man direkt die Definition der Diskriminante
benutzt und die durch den Vietaschen Satz gegebenen Relationen beachtet. Dazu
ein Tipp: Als erstes die dritte Lösung über die anderen beiden Lösungen
ausdrücken, dann~$\Delta$ und~$-4p^3 - 27q^2$ beide vollständig
ausmultiplizieren und die Ergebnisse vergleichen. Man kann aber auch die
Rechenarbeit gegen Denkarbeit tauschen, wenn man den Tipp von Seite~61 des
Skripts befolgt und ausarbeitet.

\paragraph{Aufgabe 5.} In der gesamten Aufgabe bezeichnet~"`$f^{(k)}$"'
die~$k$-te Ableitung eines Polynoms~$f$. Teilaufgabe~a) kann man etwa mit einem
Induktionsbeweis und der für alle~$k,i \geq 0$ gültigen Identität
\[ \binom{k+1}{i} = \binom{k}{i-1} + \binom{k}{i} \]
in Angriff nehmen. Die Summenschreibweise in der Angabe bedeutet, dass über
alle natürlichen Zahlen~$i,j \geq 0$, die die Beziehung~$i + j = k$ erfüllen, summiert
wird. Eine sinnvolle Konvention ist $\binom{k}{-1} := 0$.
Vor der unendlichen Summe in Teilaufgabe~c) muss man keine
Angst haben: Denn ab einem gewissen Summationsindex sind die auftretenden
Ableitungen sowieso null, sodass die unendliche Summe tatsächlich eine endliche
ist. Man hat schon viel gewonnen, wenn man die Behauptung für die
Spezialfälle~$f := X^n$, $n \geq 0$, bewiesen hat; dafür ist vielleicht der
binomische Lehrsatz
\[ (a + b)^n = \sum_{i=0}^n \binom{n}{i} a^i \, b^{n-i} \]
und die Formel $\binom{n}{k} = \frac{n!}{k! \cdot (n-k)!}$ hilfreich.


\section*{Übungsblatt 6}

\paragraph{Aufgabe 2.} Bei Teilaufgabe~a) spart man sich viel Rechenaufwand,
wenn man durch die Substitution~$Y := X + \frac{a}{3}$ die gegebene Gleichung
auf die reduzierte Form
\[ Y^3 + pY + q = 0 \quad\text{mit}\quad
  p = b - \frac{a^2}{3}, q = \frac{2a^3 - 9ab + 27c}{27} \]
bringt. Die Diskriminante dieser Gleichung ist nämlich dieselbe wie die von der
ursprünglichen Gleichung (wieso?) und dank Aufgabe~3b) von Übungsblatt~5
einfacher zu berechnen. Vielleicht findet ihr aber auch andere kreative
Lösungswege. Nur zur Kontrolle: Das Ergebnis wird
\[ \Delta = a^2b^2 - 4b^3 - 4a^3c - 27c^2 + 18abc \]
sein. Teilaufgabe~b) ist unabhängig von~a) bearbeitbar.

\paragraph{Aufgabe 4.} Diese Aufgabe kann man durch eine Rechnung oder auch
allein durch eine geometrische Konstruktion lösen.

\paragraph{Aufgabe 5.} Die Definition von~$R$ in Teilaufgabe~a) lautet etwas
ausführlicher
\[ R = \prod_{i=1}^n \prod_{j=1}^m (x_i - y_j). \]
Für die Bearbeitung der Aufgabe ist Satz~2.12 von Seite~58 des Skripts
hilfreich. Ohne Beweis kann verwendet werden, dass dieser nicht nur für
Polynome mit ganzen, rationalen, reellen, komplexen und algebraischen
Koeffizienten funktioniert, sondern auch für Polynome, deren Koeffizienten
selbst aus einem Rechenbereich von Polynomen (oder einem Rechenbereich von
symmetrischen Polynomen) stammen. Diesen Satz wird man dann insgesamt zweimal
anwenden müssen. Teilaufgabe~b) kann \emph{unabhängig} von Teilaufgabe~a)
bearbeitet werden. Mit \emph{verschwinden} ist \emph{Null sein} gemeint. Ein
Ansatz ist (wieso?), den Ausdruck
\[ R := (x_1 - y_1) \cdot (x_1 - y_2) \cdot (x_2 - y_1) \cdot (x_2 - y_2) \]
zu verwenden, wobei~$x_1,x_2$ die Lösungen der ersten und~$y_1,y_2$ die
Lösungen der zweiten Gleichung sind. Dann muss man diesen Ausdruck so
umschreiben, dass nur noch die Gleichungskoeffizienten, aber nicht mehr die
Lösungen, vorkommen. Das ist etwa mit den Beziehungen aus dem Vietaschen Satz
oder der Mitternachtsformel möglich.


\section*{Übungsblatt 7}

\paragraph{Aufgabe 1.} Bei Teilaufgabe~a) sollte man unbedingt den euklidischen
Algorithmus verwenden, wenn man nicht stundenlang knobeln möchte. Für
Teilaufgabe~b) hier die Erinnerung an die relevante Definition:
\begin{quote}Ein Polynom~$d$ heißt genau dann \emph{größter gemeinsamer Teiler}
zweier Polynome~$f$ und~$g$, falls
\begin{enumerate}
\item[1.] es ein Teiler von~$f$ und von~$g$ (also ein gemeinsamer Teiler) ist und
\item[2.] für jeden gemeinsamen Teiler~$\widetilde d$ von~$f$ und~$g$ gilt,
dass $\widetilde d$ seinerseits ein Teiler
von~$d$ ist (kurz: $\widetilde d \mid d$).
\end{enumerate}
\end{quote}
Ohne eine Normiertheitsbedingung haben übrigens je zwei Polynome unendlich
viele größte gemeinsame Teiler. Wenn man von \emph{dem} größten gemeinsamer
Teiler spricht, ist von diesen unendlich vielen immer der normierte gemeint.
Den Eindeutigkeitsteil von Teilaufgabe~b) kann man mit folgender Vorlage in
Angriff nehmen:
\begin{quote}Seien~$d$ und~$\widetilde d$ beides normierte größte gemeinsame
Teiler von~$f$ und~$g$. Dann\ldots, daher folgt~$d = \widetilde d$.\end{quote}
In den Teilaufgaben~c) und~d) ist (wie immer, aber diesmal steht es nicht
explizit in der Angabe) auch die Korrektheit des von euch angegebenen
Konstruktionsverfahrens zu beweisen. Natürlich ist aber die reine Angabe eines
Verfahrens auch schon viel Wert! Die Definition in Teilaufgabe~d), die ihr
finden sollt, sollte das kleinste gemeinsame Vielfache nicht durch eine
explizite Konstruktion, sondern durch seine gewünschten Eigenschaften
charakterisieren.

\paragraph{Aufgabe 2.} Beispiel~3.5 auf Seite~75 des Skripts zeigt eine
Möglichkeit, Teilaufgabe~a) zu lösen. Andere Lösungswege sind aber auch
möglich. Für Teilaufgabe~c) mag es hilfreich sein, dass der relevante größte
gemeinsame Teiler im Skript schon berechnet worden ist.

\paragraph{Aufgaben 3 und 4.} Der Bequemlichkeit halber hier die nötigen
Definitionen:
\begin{quote}
\begin{enumerate}
\item[1.] Eine \emph{Zerlegung} eines normierten Polynoms~$f$ mit rationalen
Koeffizienten ist eine Darstellung von~$f$ als Produkt~$f = f_1 \cdots f_n$
aus~$n \geq 1$ normierten, nichtkonstanten Polynomen mit rationalen
Koeffizienten.
\item[2.] Ein normiertes Polynom~$f$ mit rationalen Koeffizienten heißt genau
dann \emph{irreduzibel (über den rationalen Zahlen)}, wenn es genau eine
Zerlegung zulässt, und zwar die triviale: $f = f$. Sonst heißt es
\emph{reduzibel}.
\end{enumerate}
\end{quote}
Aus der präzisen Art und Weise, wie diese Definitionen formuliert sind, folgt
insbesondere, dass das Einspolynom (das ist das konstante Polynom~$1$) nicht
als irreduzibel gilt (wieso?). Das ist auch gut so, denn sonst wäre die
Eindeutigkeit der Zerlegung in irreduzible Faktoren (Proposition~3.9 im Skript)
nicht mehr gegeben (wieso?). Abschließend sei bemerkt, dass die
\emph{Umkehrung} von Teilaufgabe~4a) Gegenstand der Vorlesung war
(Folgerung~3.11 im Skript).

\paragraph{Aufgabe 5.} In der Vorlesung wurde die analoge Aussage für Polynome
bewiesen (Proposition~3.1 im Skript). Man kann also versuchen, den dortigen
Beweis auf die neue Situation der Aufgabe zu übertragen. Man kann auch
versuchen, etwas expliziter ein Verfahren zu beschreiben, welches das
geforderte~$d$ berechnet, und dann die Korrektheit des Verfahrens zu beweisen.


\section*{Übungsblatt 8}

\paragraph{Aufgabe 1.} Hier kann das Verfahren aus Beispiel~3.7 oder
Beispiel~3.8 des Skripts verwendet werden. Näherungswerte für die Nullstellen
erhält man etwa bei \url{http://www.wolframalpha.com/}.

\paragraph{Aufgabe 2.} Dass~$f(X)$ \emph{nicht verschwindet} bedeutet, dass es
nicht das konstante Nullpolynom~$0$ ist. Mit~$\widetilde f$ ist wie in der Vorlesung das
Polynom~$c^{-1} \cdot f$ gemeint, wobei~$c$ der Inhalt von~$f$ ist.

\paragraph{Aufgabe 3.} Für Teilaufgabe~b) ist es hilfreich, mit einer
Bézoutdarstellung des größten gemeinsamen Teilers von~$a$ und~$n$ zu arbeiten;
eine solche existiert nach Aufgabe~5 von Übungsblatt 7. Alle Teilaufgaben
können unabhängig voneinanander bearbeitet werden.

\paragraph{Aufgabe 5.} Für Teilaufgabe~a) kann ohne Beweis folgender Satz
über die sogenannte Existenz und Eindeutigkeit der Polynominterpolation verwendet werden:
\begin{quote}
Sei~$n \geq 0$. Seien~$x_0, \ldots, x_n$ paarweise verschiedene rationale Zahlen. Seien~$y_0,
\ldots, y_n$ beliebige rationale Zahlen. Dann gibt es genau ein Polynom~$f$ mit
rationalen Koeffizienten und Grad~$\leq n$, dessen Graph durch die Punkte~$(x_i,
y_i)$ geht, also sodass
\[ f(x_i) = y_i \]
für alle~$i = 0,\ldots,n$ gilt.
\end{quote}
Außerdem hilft es für Teilaufgabe~a), sich
folgende Frage zu stellen: Wie viele Teiler kann eine ganze Zahl ungleich Null
haben?
Teilaufgabe~b) kann dann mit~a) und c) kann mit~b) gelöst werden.


\section*{Übungsblatt 9}

\paragraph{Aufgabe 1.} In Teilaufgabe~b) soll die Linearkombination den Wert~0
haben (\emph{verschwinden}) ohne, dass sie \emph{trivial} wäre, d.\,h., dass
alle Koeffizienten der Linearkombination jeweils~0 wären. Alle Teilaufgaben
können unabhängig voneinander bearbeitet werden. Nur zur Kontrolle: Der
Koeffizient vor~$x^2$ in der in Teilaufgabe~a) gesuchten Linearkombination
ist~$(-20)$. Die kleinste Möglichkeit für~$n$ in Teilaufgabe~b) ist~$6$ (aber
größere Werte sind auch möglich).

\paragraph{Aufgabe 2.} Alle Teilaufgaben 
können unabhängig voneinander bearbeitet werden. Nur zur Kontrolle: Die
Ergebnisse von Teilaufgabe~a) sind~4, 2 und~2. Eine Möglichkeit für das
gesuchte Polynom in Teilaufgabe~b) ist das Minimalpolynom der Zahl aus~a)
über~$\QQ$ (also~$X^4-4\,X^2+16$). Für Teilaufgabe~d) sind vielleicht die (zu begründenden)
Beobachtungen hilfreich, dass~$\alpha$ eine primitive zehnte und~$\zeta$ eine
primitive fünfte Einheitswurzel ist. Um dann die Basis anzugeben, kann die
Beobachtung nützlich sein, dass~$-\zeta$ ebenfalls eine primitive zehnte
Einheitswurzel ist. Alternativ hilft vielleicht die Beobachtung, dass~$\zeta$
und~$\alpha$ beide vom Grad~4 über~$\QQ$ sind (das Minimalpolynom von~$\alpha$
ist~$X^4 - X^3 + X^2 - X + 1$, später werden wir diesen Umstand tiefer
verstehen).

\paragraph{Aufgabe 3.} Allgemein ist mit~$\deg_{\QQ(a)} b$ der Grad der
Zahl~$b$ über~$\QQ(a)$ gemeint. Dessen Definition findet sich im Skript direkt
nach Proposition~3.35.

\paragraph{Aufgabe 4.} Für Teilaufgabe~a) kann das Verfahren der
Vorlesung (siehe etwa Beispiel~3.30) verwendet werden. Bei Teilaufgabe~b) kann
man ein wenig mit den Potenzen~$(\sqrt{2}+\sqrt{3})^2$, $(\sqrt{2}+\sqrt{3})^3$
knobeln. Bei Teilaufgabe~c) ist
ein Induktionsbeweis möglich (denn was ist das Signalwort?). Für Teilaufgabe~d) kann man
das Ergebnis von Teilaufgabe~c) verwenden (auch ohne einen Beweis von~c) zu
kennen).

\paragraph{Aufgabe 5.} Der Titel der Aufgabe ist irreführend, vermutlich ist
die schwierigste Aufgabe des Blatts Aufgabe~2d).


\section*{Übungsblatt 10}

\paragraph{Aufgabe 1.} In der Angabe von Teilaufgabe~b) steckte kurzzeitig ein
Tippfehler, der jetzt korrigiert ist. Teilaufgabe~b) ist sehr ähnlich, aber nicht völlig
identisch, zu Aufgabe~3c) von Übungsblatt~9. Das Vorgehen dort lässt sich auf
die Aufgabe hier übertragen. Vielleicht helfen zwei allgemeine Erinnerungen:
Für den Grad einer algebraischen Zahl~$w$ über einer anderen algebraischen
Zahl~$u$ gilt die Formel
\[ \deg_{\QQ(u)} w = \gra{\QQ(u,w)}{\QQ(u)}. \]
Falls~$u \in \QQ(w)$, gilt ferner~$\QQ(u,w) = \QQ(w)$, sodass sich die Formel
dann noch ein wenig vereinfacht.

\paragraph{Aufgabe 3.} Zur Kontrolle: Eine der Nullstellen
ist~$\exp(2\pi\i/8)$. Das gegebene Polynom ist tatsächlich irreduzibel, wie man
etwa mit dem Eisensteinverschiebungstrick sehen kann. Falls ihr euer primitives
Element gegenchecken wollt, schaut einfach kurz im Büro~2031/L1 vorbei oder
schreibt eine Mail.

\paragraph{Aufgabe 4.} Eine der beiden Richtungen in Teilaufgabe~a) wurde schon
in der Vorlesung gezeigt, die andere aber noch nicht. Für Teilaufgabe~b) mag es
hilfreich sein, dass die Komposition~$(g \circ f)(X) = g(f(X))$ zweier
Polynome~$g$ und~$f$ wieder ein Polynom ist.

\paragraph{Aufgabe 5.} Der Bequemlichkeit halber hier die beiden Aussagen des
Skripts:
\begin{quote}
\textbf{Hilfssatz 4.3.}
Seien~$x_1,\ldots,x_n$ die Lösungen (mit Vielfachheiten) einer Polynomgleichung
mit rationalen Koeffizienten. Ist dann~$V(X_1,\ldots,X_n)$ ein Polynom mit
rationalen Koeffizienten, so sind die galoissch Konjugierten von~$t =
V(x_1,\ldots,x_n)$ alle von der Form~$t' =
V(x_{\sigma(1)},\ldots,x_{\sigma(n)})$, wobei~$\sigma$ eine~$n$-stellige
Permutation ist.
\end{quote}

\begin{quote}
\textbf{Proposition 4.4.}
Seien~$x_1,\ldots,x_n$ die Lösungen (mit Vielfachheiten) einer Polynomgleichung
mit rationalen Koeffizienten. Ist dann~$t$ ein primitives Element
zu~$x_1,\ldots,x_n$, so ist auch jedes galoissch Konjugierte~$t'$ von~$t$ ein
primitives Element von~$x_1,\ldots,x_n$.
\end{quote}


\section*{Übungsblatt 11}

\paragraph{Aufgabe 1.} Für Teilaufgabe~b) muss man sich daran erinnern, wie die
Rechenoperation $\text{Symmetrie der Nullstellen} \cdot \text{Zahl
aus~$\QQ(x_1,\ldots,x_n)$}$ definiert war. Die Hinrichtung von Teilaufgabe~c)
ist einfacher als die Rückrichtung. Für die Rückrichtung lohnt es sich
vielleicht, das Polynom
\[ \prod_{\sigma \in \Gal_\QQ(x_1,\ldots,x_n)} (X - \sigma \cdot x_1) \]
zu betrachten.

\paragraph{Aufgabe 2.} Um Missverständnisse zu vermeiden: Die beiden
Nullstellen~$x_1$ und~$x_2$ des Polynoms~$f(X)$ aus Teilaufgabe~a) können
wegen der vorausgesetzten Separabilität also als verschieden angenommen werden.

\paragraph{Aufgabe 3.} Die beiden Teilaufgaben können unabhängig voneinander
bearbeitet werden.

\paragraph{Aufgabe 4.} Eine Formel auf dem Merkblatt zu
Rechenbereichserweiterungen könnte hilfreich sein.

\paragraph{Aufgabe 5.} Teilaufgabe~b) kann durch Probieren oder durch das in
Teilaufgabe~c) vorgestellte Verfahren gelöst werden. Um die Behauptung von
Teilaufgabe~c) zu beweisen, kann es hilfreich sein, für zwei verschiedene
Permutationen~$\sigma,\tau \in S_n$ die Zahl
\[ |V(x_{\sigma(1)},\ldots,x_{\sigma(n)}) -
  V(x_{\tau(1)},\ldots,x_{\tau(n)})| \]
zu betrachten und zu versuchen, sie nach unten abzuschätzen. Dafür wiederum mag
es hilfreich sein, den größten Index~$k \in \{ 1,\ldots,n \}$ mit~$\sigma(k)
\neq \tau(k)$ gesondert zu betrachten.


\section*{Übungsblatt 12}

\paragraph{Aufgabe 1.} Für Teilaufgabe~a) muss man nur einen wichtigen Satz aus
der Gruppentheorie kennen. Für Teilaufgabe~b) ist dieser Satz dagegen überhaupt
nicht zu gebrauchen, besser ist es da, sich an den Definitionen zu orientieren.
Für Teilaufgabe~c) sind angesichts der Definition
\[ \sigma^i := \begin{cases}
  \sigma \circ \cdots \circ \sigma \text{ ($i$ Faktoren)}, &
  \text{falls~$i \geq 1$,} \\
  \id, &
  \text{falls~$i = 0$,} \\
  \sigma^{-1} \circ \cdots \circ \sigma^{-1} \text{ ($-i$ Faktoren)}, &
  \text{falls~$i \leq -1$,}
\end{cases} \]
vielleicht Fallunterscheidungen sinnvoll.

\paragraph{Aufgabe 2.} Wenn man Teilaufgabe~a) rechnerisch löst, ist die
Angelegenheit ein bisschen fiddelig. Einfacher ist es, wenn man die Teilaufgabe
durch eine saubere Skizze und eine präzise Begründung (als Text) löst. Die
Definition der für Teilaufgabe~b) benötigten zyklischen Gruppe steht im Skript
auf Seite 123 (oben):
\[ \CCC_n := \{ \sigma_0, \ldots, \sigma_{n-1} \} \subseteq \SSS_n. \]
Die~$\sigma_k$ sind dabei auf der Seite zuvor (ganz unten) definiert.
In Kombination mit Teil\-auf\-ga\-be~a) kann man schnell die gesuchten Ordnungen
angeben, wenn man nur erkennt, dass sich alle Elemente von~$\CCC_n$ als
Potenzen eines bestimmten Grundelements ausdrücken lassen (welchem, und
wieso?). Für Teilaufgabe~c) mag die für alle endlichen Gruppen~$G$ gültige
Äquivalenz
\[ \text{$x \in G$ ist ein Erzeuger von~$G$}
  \Longleftrightarrow
  \text{die Ordnung von~$x$ ist~$|G|$} \]
hilfreich sein (wieso gilt sie?).

\paragraph{Aufgabe 3.} Im Skript und auf der englischen Wikipedia ist ein
Verfahren beschrieben, um die in Teilaufgabe~a) verlangten
Kreisteilungspolynome zu berechnen. Teilaufgabe~b) kann man kurz und knapp mit
Kreisteilungspolynomen oder auch etwas langsamer über das übliche Verfahren
lösen.

\paragraph{Aufgabe 4.} Welche Nullstellen hat das Polynom in Teilaufgabe~a)
modulo~$p$?
Für Teilaufgabe~b) ist vielleicht die Formel
\[ \binom{n}{k} = \frac{n!}{k! \cdot (n-k)!} =
  \frac{n \cdot (n-1) \cdots (n-k+2) \cdot (n-k+1)}{k \cdot (k-1) \cdots 2
  \cdot 1} \]
hilfreich. Die beiden Teilaufgaben haben nichts miteinander zu tun.

\paragraph{Aufgabe 5.} Wenn man die Definition kennt, kann man Teilaufgabe~a)
einfach durch Probieren lösen. Teilaufgabe~b) hat mit der ersten nichts zu tun;
hier ein paar Stichworte:
Eine \emph{Bijektion} ist eine bijektive (d.\,h. injektive und surjektive)
Abbildung. Hier allerdings folgt aus Injektivität schon Surjektivität und
umgekehrt (wieso gelten diese Ausnahmeregeln?). Wenn~$\zeta_0$ eine
primitive~$n$-te Einheitswurzel ist, ist~$(\zeta_0)^d$ genau dann ebenfalls eine
primitive~$n$-te Einheitswurzel, wenn~$d$ zu~$n$ teilerfremd ist (wieso?).

\end{document}
