\documentclass{../algblatt}
\usepackage{multicol}

\begin{document}

\begin{center}\Large \textsf{\textbf{Hinweise zu den Übungsaufgaben in Algebra I}}\end{center}
\vspace{1em}


\section*{Übungsblatt 1}

\paragraph{Aufgabe 4.} Hier gibt es viele verschiedene Lösungswege. Eine
Möglichkeit besteht darin, den Winkel~$\alpha$ bei den unteren Ecken der Skizze
als Innenwinkel von drei verschiedenen Teildreiecken zu erkennen und den
Tangens von~$\alpha$ dann jeweils über Gegen- und Ankathete auszudrücken.
Zusammen mit dem Satz von Pythagoras erhält man dann drei Gleichungen für drei
Unbekannte.


\section*{Übungsblatt 2}

\paragraph{Aufgabe 5.} Die Teilaufgaben~a), c) und~d) können unabhängig von~b)
bearbeitet werden.


\section*{Übungsblatt 3}

\paragraph{Aufgabe 1.} Für die Teilaufgaben~b) und~c) ist es nicht nötig, eine
explizite Darstellung der Lösung~$\alpha$ zu berechnen.

\paragraph{Aufgabe 2.} Für Teilaufgabe~a) ist es ebenfalls nicht nötig,
explizite Darstellungen der Lösungen~$x$ bzw.~$y$ zu berechnen. Auch ohne deren
Kenntnis kann man nämlich das Verfahren aus Proposition~1.3 bzw. Hilfssatz~1.4 des
Skripts einsetzen. Zur Kontrolle hier eine der insgesamt sechs Teilrechnungen,
bevor man zur Bestimmung der Determinante schreiten kann:
\[ xy \cdot c_{20} = -c_{01} + c_{11}. \]

\paragraph{Aufgabe 5.} Je nachdem, wie man Teilaufgabe~b) angeht, ist folgende
für ganze Zahlen~$a$ und~$n$ gültige Äquivalenz hilfreich:
\[ \left[\exists m \in \ZZ{:}\ a\,m \equiv 1 \mod{n}\right]
  \quad\Longleftrightarrow\quad
  \text{$a$ und~$n$ sind zueinander teilerfremd.} \]
Ausgeschrieben besagt die linke Aussage, dass es eine weitere ganze Zahl~$m$
gibt, sodass die Zahl~$a\,m$ bei Division durch~$n$ den Rest~1 lässt.

\end{document}
