\documentclass{../../algblatt}

\usepackage{geometry}
\geometry{tmargin=3cm,bmargin=2cm,lmargin=2.3cm,rmargin=2.3cm}

\usepackage{fancyhdr}
\pagestyle{fancy}
\lhead{Rechenbereiche}
\rhead{20. Juni 2013}
\fancyfoot{}

\renewcommand{\labelitemi}{--}

\begin{document}

\begin{center}\huge \sffamily\textbf{Erweiterungen der rationalen Zahlen}\end{center}


\section*{Vereinfachung der Beschreibung durch Erzeuger}

Folgende Regeln kann man verwenden, um Darstellungen von Erweiterungen der
Form
\[ \QQ(x_1,\ldots,x_n) \] zu vereinfachen (wieso gelten die Regeln?):
\begin{enumerate}
\item Die Reihenfolge der Erzeuger (damit sind die~$x_i$ gemeint) spielt keine Rolle.
\item Erzeuger, die in~$\QQ$ liegen, kann man weglassen.
\item Man kann beliebige Elemente aus~$\QQ$ zu Erzeugern addieren und
subtrahieren, sowie (falls nicht null) multiplizieren und dividieren.
\item Man kann beliebige~$\QQ$-Vielfache eines Erzeugers auf einen anderen
addieren und subtrahieren, sowieso (falls nicht null) multiplizieren und
dividieren.
\end{enumerate}


\subsection*{Beispiele}

\begin{enumerate}
\item[1.] $\QQ(\sqrt{2}, {-}\sqrt{2}) = \QQ(\sqrt{2})$
\item[2.] $\QQ(\frac{1 + \sqrt{5}}{2}, \frac{1 - \sqrt{5}}{2}) = \QQ(1+\sqrt{5},
1-\sqrt{5}) = \QQ(\sqrt{5},-\sqrt{5}) = \QQ(\sqrt{5})$
\item[3.] $\QQ(\zeta^0,\zeta^1,\ldots,\zeta^5) = \QQ(\zeta) = \QQ(1
+ \sqrt{3}\,\i) = \QQ(\sqrt{3}\,\i),$ \\[0.3em]
${\qquad}$ für $\zeta := e^{2 \pi \i/6} = \cos 60^\circ + \i \sin 60^\circ =
\frac{1}{2}(1 + \sqrt{3}\,\i)$.
\item[4.] $\QQ(\sqrt{3})(\zeta^0,\ldots,\zeta^5) = \QQ(\sqrt{3})(\sqrt{3}\,\i) =
\QQ(\sqrt{3},\sqrt{3}\,\i) = \QQ(\sqrt{3},\i),$ \\[0.3em]
${\qquad}$ für~$\zeta$ wie in Beispiel 3.
\item[5.]
$\QQ(\sqrt[8]{2}\,\zeta^0, \ldots, \sqrt[8]{2}\,\zeta^7) = \QQ(\sqrt[8]{2},
\zeta, \zeta^2, \ldots, \zeta^7) = \QQ(\sqrt[8]{2}, \zeta) = \QQ(\sqrt[8]{2}, 1 + \i) = \QQ(\sqrt[8]{2},
\i),$ \\[0.3em]
${\qquad}$ für $\zeta := e^{2 \pi \i/8} = \frac{1}{\sqrt{2}}(1 + \i)$.
\item[6.] $\QQ(\text{alle sechs Nullstellen von~$(X^4 - 2) (X^2 + 1)$}) =
\QQ(\sqrt[4]{2}, \i)$
\end{enumerate}


\subsection*{Anwendungen}

Die Darstellung zu vereinfachen ist hilfreich, wenn man\ldots
\begin{itemize}
\item \ldots Erweiterungen von~$\QQ$ in knapper Form angeben möchte.

\item \ldots den Grad einer Erweiterung bestimmen
möchte.

\emph{Beispiel:} Die Erweiterung von Beispiel~3 hat über~$\QQ$ den Grad~2, denn das
Minimalpolynom von~$\sqrt{3}\,\i$ über~$\QQ$ ist~$X^2 + 3$ (wieso?). In der
Ausgangsformulierung~$\QQ(\zeta^0,\zeta^1,\ldots,\zeta^5)$ erkennt man den Grad
dagegen nicht so schnell.

\item \ldots primitive Elemente bestimmen möchte.

\emph{Beispiel:} Ein primitives Element für die Zahlen~$\zeta^0,\ldots,\zeta^5$
aus Beispiel~3 ist~$\sqrt{3}\,\i$. Dank der Vereinfachungsregeln sieht man das ganz
mühelos, ohne langwierige wiederholte Anwendung des Verfahrens aus der
Vorlesung.

\item \ldots Galoisgruppen bestimmen möchte (denn dazu benötigt man ja diese
Dinge).
\end{itemize}


\section*{Nachweis von Rechenbereichsinklusionen}

Seien $F$ und~$\widetilde F$ beliebige weitere Erweiterungen von~$\QQ$. Dann
gilt (wieso?):

\begin{enumerate}
\item $\QQ(x_1,\ldots,x_n)(y_1,\ldots,y_m) = \QQ(x_1,\ldots,x_n,y_1,\ldots,y_m)$.
\item $\QQ[x] = \QQ(x)$ genau dann, wenn~$x$ algebraisch ist.
\item $\QQ(x_1,\ldots,x_n) \subseteq F$ genau dann, wenn
$x_1,\ldots,x_n \in F$.
\item $\QQ(x_1,x_2) = \QQ(x_1)$ genau dann, wenn~$x_2 \in \QQ(x_1)$ (wie folgt
das aus~c)?).
\item Gelte~$F \subseteq \widetilde F$. Dann gilt genau dann~$F =
\widetilde F$, wenn $[F:\QQ] = [\widetilde F:\QQ]$.

Ohne die Zusatzvoraussetzung~$F \subseteq \widetilde F$ ist das Quatsch!
\end{enumerate}

Im Allgemeinen gilt nicht, dass~$\QQ(x,y) = \QQ(x+y)$.


\section*{Gradbestimmung}

\begin{enumerate}
\item Seien~$w$ und~$u$ algebraische Zahlen. Dann gilt
\begin{align*}
  \deg_{\QQ(u)} w &= \text{Grad von~$w$ über~$\QQ(u)$} \\
  &= \text{Grad des Minimalpolynoms von~$w$ über~$\QQ(u)$} \\
  &= \gra{\QQ(u,w)}{\QQ(u)}. \\
\intertext{Falls außerdem~$u \in \QQ(w)$ gelten sollte, gilt ferner~$\QQ(u,w) =
\QQ(w)$, sodass man in diesem Fall die Formel noch weiter vereinfachen kann:}
  &= \gra{\QQ(w)}{\QQ(u)}.
\end{align*}

Für den Grad über~$\QQ$ folgt daraus (mit~$u := 1$):
\[ \deg_\QQ w =
  \text{(Grad des Minimalpolynoms von~$w$ über~$\QQ$)} =
  \gra{\QQ(w)}{\QQ}. \]

\item Gelte~$F \subseteq F' \subseteq F''$. Dann gilt die \emph{Gradformel}:
\[ \gra{F''}{F} = \gra{F''}{F'} \cdot \gra{F'}{F}. \]

\item Verbindung zur Galoistheorie: Sind~$x_1,\ldots,x_n$ die Nullstellen
eines normierten separablen Polynoms mit rationalen Koeffizienten und ist~$t$
ein primitives Element für diese Nullstellen, so
gilt~$[\QQ(x_1,\ldots,x_n):\QQ] = [\QQ(t):\QQ] =
|\mathrm{Gal}_\QQ(x_1,\ldots,x_n)|$. Dies folgt aus der fundamentalen
1:1--Korrespondenz zwischen den Elementen der Galoisgruppe und den galoissch
Konjugierten von~$t$ (Proposition~4.8).
\end{enumerate}


\section*{Basen}

Eine mögliche~$\QQ$-Basis der Erweiterung~$\QQ(\alpha)$ ist durch
\[ 1,\quad \alpha,\quad \alpha^2,\quad \ldots,\quad \alpha^{n-1} \]
gegeben, wobei~$n$ der Grad von~$\alpha$ sei. Ausbuchstabiert bedeutet das:
\emph{Jede} Zahl aus~$\QQ(\alpha)$ lässt sich auf \emph{eindeutige} Art und Weise als
rationale Linearkombination in den Zahlen~$\alpha^0,\ldots,\alpha^{n-1}$ schreiben.

\emph{Beispiel:} Der Grad von~$\sqrt[3]{2}$ über~$\QQ$ ist~$3$ (wieso?). Daher
gibt es für jede Zahl~$x$ aus~$\QQ(\sqrt[3]{2})$ genau einen Satz von
rationalen Koeffizienten~$a, b, c$ mit
\[ x = a \cdot 1 + b \cdot \sqrt[3]{2} + c \cdot \sqrt[3]{2}^2. \]

\end{document}
