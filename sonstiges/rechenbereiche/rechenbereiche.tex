\documentclass[12pt,a4paper,ngerman]{scrartcl}

\usepackage{ucs}
\usepackage[utf8x]{inputenc}
\usepackage[ngerman]{babel}
\usepackage{amsmath,amssymb,amscd,amsthm,color,graphicx}
\usepackage[protrusion=true,expansion=false]{microtype}
\usepackage{lmodern}
\usepackage{hyperref}

\setlength\parskip{\medskipamount}
\setlength\parindent{0pt}

\newcommand{\Bild}{\operatorname{im}}
\newcommand{\Kern}{\operatorname{ker}}
\newcommand{\Span}{\operatorname{span}}
\newcommand{\GL}{\mathrm{GL}}
\newcommand{\R}{\mathbb{R}}
\newcommand{\C}{\mathbb{C}}
\newcommand{\QQ}{\mathbb{Q}}
\renewcommand{\i}{\mathrm{i}}

\theoremstyle{definition}
\newtheorem*{defn}{Definition}
\newtheorem*{bsp}{Beispiel}

\theoremstyle{plain}
\newtheorem*{prop}{Proposition}
\newtheorem*{beh}{Behauptung}

\theoremstyle{remark}
\newtheorem*{bem}{Bemerkung}

\usepackage{geometry}
\geometry{tmargin=3cm,bmargin=3cm,lmargin=2.7cm,rmargin=2.7cm}

\usepackage{fancyhdr}
\pagestyle{fancy}
\lhead{Rechenbereiche}
\rhead{19. Juni 2013}
\fancyfoot{}

\renewcommand{\labelitemi}{--}

\begin{document}

\section*{Rechenregeln für Erweiterungen der rationalen Zahlen}

Folgende Regeln kann man verwenden, um Darstellungen von Erweiterungen der
Form
\[ \QQ(x_1,\ldots,x_n) \] zu vereinfachen:
\begin{enumerate}
\item[a)] Die Reihenfolge der Erzeuger (damit sind die~$x_i$ gemeint) spielt keine Rolle.
\item[b)] Erzeuger, die in~$\QQ$ liegen, kann man weglassen.
\item[c)] Man kann beliebige Elemente aus~$\QQ$ zu Erzeugern addieren und
subtrahieren, sowie (falls nicht null) multiplizieren und dividieren.
\item[d)] Man kann beliebige~$\QQ$-Vielfache eines Erzeugers auf einen anderen
addieren und subtrahieren, sowieso (falls nicht null) multiplizieren und
dividieren.
\end{enumerate}

Außerdem helfen oft folgende Tatsachen (seien $F$ und~$\widetilde F$ beliebige
weitere Erweiterungen von~$\QQ$):
\begin{enumerate}
\item[e)] $\QQ(x_1,\ldots,x_n)(y_1,\ldots,y_m) = \QQ(x_1,\ldots,x_n,y_1,\ldots,y_m)$
\item[f)] $\QQ(x_1,\ldots,x_n) \subseteq F$ genau dann, wenn
$x_1,\ldots,x_n \in \QQ$.
\item[g)] $\QQ(x_1,x_2) = \QQ(x_1)$ genau dann, wenn~$x_2 \in \QQ(x_1)$.
\item[h)] Gelte~$F \subseteq \widetilde F$. Dann gilt genau dann~$F =
\widetilde F$, wenn $[F:\QQ] = [\widetilde F:\QQ]$.

Ohne die Zusatzvoraussetzung~$F \subseteq \widetilde F$ ist das Quatsch!
\item[i)] Verbindung zur Galoistheorie: Sind~$x_1,\ldots,x_n$ die Nullstellen
eines normierten separablen Polynoms mit rationalen Koeffizienten und ist~$t$
ein primitives Element für diese Nullstellen, so
gilt~$[\QQ(x_1,\ldots,x_n):\QQ] = [\QQ(t):\QQ] =
|\mathrm{Gal}_\QQ(x_1,\ldots,x_n)|.$
\end{enumerate}
Im Allgemeinen gilt nicht, dass~$\QQ(x,y) = \QQ(x+y)$.

\subsection*{Beispiele}

\begin{enumerate}
\item $\QQ(\sqrt{2}, {-}\sqrt{2}) = \QQ(\sqrt{2})$
\item $\QQ(\frac{1 + \sqrt{5}}{2}, \frac{1 - \sqrt{5}}{2}) = \QQ(1+\sqrt{5},
1-\sqrt{5}) = \QQ(\sqrt{5},-\sqrt{5}) = \QQ(\sqrt{5})$
\item $\QQ(\zeta^0,\zeta^1,\ldots,\zeta^5) = \QQ(\zeta) = \QQ(1
+ \sqrt{3}\,\i) = \QQ(\sqrt{3}\,\i),$ \\[0.3em]
${\qquad}$ für $\zeta := e^{2 \pi \i/6} = \cos 60^\circ + \i \sin 60^\circ =
\frac{1}{2}(1 + \sqrt{3}\,\i)$.
\item $\QQ(\sqrt{3})(\zeta^0,\ldots,\zeta^5) = \QQ(\sqrt{3})(\sqrt{3}\,\i) =
\QQ(\sqrt{3},\sqrt{3}\,\i) = \QQ(\sqrt{3},\i),$ \\[0.3em]
${\qquad}$ für~$\zeta$ wie in Beispiel 3.
\item
$\QQ(\sqrt[8]{2}\,\zeta^0, \ldots, \sqrt[8]{2}\,\zeta^7) = \QQ(\sqrt[8]{2},
\zeta, \zeta^2, \ldots, \zeta^7) = \QQ(\sqrt[8]{2}, \zeta) = \QQ(\sqrt[8]{2}, 1 + \i) = \QQ(\sqrt[8]{2},
\i),$ \\[0.3em]
${\qquad}$ für $\zeta := e^{2 \pi \i/8} = \frac{1}{\sqrt{2}}(1 + \i)$.
\item $\QQ(\text{alle sechs Nullstellen von~$(X^4 - 2) (X^2 + 1)$}) =
\QQ(\sqrt[4]{2}, \i)$
\end{enumerate}

\newpage

\subsection*{Anwendungen}

Die Darstellung zu vereinfachen ist hilfreich, wenn man\ldots
\begin{itemize}
\item \ldots den Grad einer Körpererweiterung bestimmen
möchte.

\emph{Beispiel:} Die Erweiterung von Beispiel~3 hat über~$\QQ$ den Grad~2, denn das
Minimalpolynom von~$\sqrt{3}\,\i$ über~$\QQ$ ist~$X^2 + 3$ (wieso?). In der
Ausgangsformulierung~$\QQ(\zeta^0,\zeta^1,\ldots,\zeta^5)$ erkennt man den Grad
dagegen nicht so schnell.

\item \ldots Erweiterungen von~$\QQ$ in knapper Form angeben möchte.

\item \ldots primitive Elemente bestimmen möchte.

\emph{Beispiel:} Ein primitives Element für die Zahlen~$\zeta^0,\ldots,\zeta^5$
aus Beispiel~3 ist~$\sqrt{3}\,\i$. Dank der Vereinfachungsregeln sieht man das ganz
mühelos, ohne langwierige wiederholte Anwendung des Verfahrens aus der
Vorlesung.

\item \ldots Galoisgruppen bestimmen möchte (denn dazu benötigt man ja diese
Dinge).
\end{itemize}

\end{document}
