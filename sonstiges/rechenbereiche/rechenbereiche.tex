\documentclass[12pt,a4paper,ngerman]{scrartcl}

\usepackage{ucs}
\usepackage[utf8x]{inputenc}
\usepackage[ngerman]{babel}
\usepackage{amsmath,amssymb,amscd,amsthm,color,graphicx}
\usepackage[protrusion=true,expansion=false]{microtype}
\usepackage{lmodern}
\usepackage{hyperref}

\setlength\parskip{\medskipamount}
\setlength\parindent{0pt}

\newcommand{\Bild}{\operatorname{im}}
\newcommand{\Kern}{\operatorname{ker}}
\newcommand{\Span}{\operatorname{span}}
\newcommand{\GL}{\mathrm{GL}}
\newcommand{\R}{\mathbb{R}}
\newcommand{\C}{\mathbb{C}}
\newcommand{\Q}{\mathbb{Q}}
\newcommand{\I}{\mathrm{i}}

\theoremstyle{definition}
\newtheorem*{defn}{Definition}
\newtheorem*{bsp}{Beispiel}

\theoremstyle{plain}
\newtheorem*{prop}{Proposition}
\newtheorem*{beh}{Behauptung}

\theoremstyle{remark}
\newtheorem*{bem}{Bemerkung}

\usepackage{geometry}
\geometry{tmargin=3cm,bmargin=3cm,lmargin=2.5cm,rmargin=2.5cm}

\usepackage{fancyhdr}
\pagestyle{fancy}
\lhead{Algebra I, SS 2012}
\chead{1. Juli 2012}
\rhead{Körpererweiterungen}

\renewcommand{\labelitemi}{--}

\begin{document}

\section*{Rechenregeln für Körpererweiterungen}

Folgende Regeln kann man benutzen, um Darstellungen von Körpererweiterungen der
Form~$L = K(x_1,\ldots,x_n)$ über~$K$ zu vereinfachen:
\begin{enumerate}
\item[a)] Die Reihenfolge der Erzeuger spielt keine Rolle.
\item[b)] Erzeuger, die in~$K$ liegen, kann man weglassen.
\item[c)] Man kann beliebige Elemente aus~$K$ zu Erzeugern addieren und
subtrahieren, sowie (falls nicht null) multiplizieren und dividieren.
\item[d)] Man kann beliebige~$K$-Vielfache eines Erzeugers auf einen anderen
addieren und subtrahieren, sowieso (falls nicht null) multiplizieren und
dividieren.
\end{enumerate}

Außerdem helfen folgende Tatsachen ($F$ beliebiger Körper):
\begin{enumerate}
\item[e)] $K(x_1,\ldots,x_n)(y_1,\ldots,y_m) = K(x_1,\ldots,x_n,y_1,\ldots,y_m)$
\item[f)] $K(x_1,\ldots,x_n) \subseteq F$ genau dann, wenn $K \subseteq F$ und
$x_1,\ldots,x_n \in F$.
\end{enumerate}
Im Allgemeinen gilt nicht, dass~$K(x,y) = K(x+y)$.

\subsection*{Beispiele}

\begin{enumerate}
\item $\Q(\sqrt{2}, {-}\sqrt{2}) = \Q(\sqrt{2})$
\item $\Q(\frac{1 + \sqrt{5}}{2}, \frac{1 - \sqrt{5}}{2}) = \Q(1+\sqrt{5},
1-\sqrt{5}) = \Q(\sqrt{5},-\sqrt{5}) = \Q(\sqrt{5})$
\item $\Q(\zeta^0,\zeta^1,\ldots,\zeta^5) = \Q(\zeta) = \Q(1
+ \sqrt{3}\,\I) = \Q(\sqrt{3}\,\I),$ \\
${\qquad}$ für $\zeta := e^{2 \pi \I/6} = \cos 60^\circ + \I \sin 60^\circ =
\frac{1}{2}(1 + \sqrt{3}\,\I)$.
\item $\Q(\sqrt{3})(\zeta^0,\ldots,\zeta^5) = \Q(\sqrt{3})(\sqrt{3}\,\I) =
\Q(\sqrt{3},\sqrt{3}\,\I) = \Q(\sqrt{3},\I),$ \\
${\qquad}$ für~$\zeta$ wie in Beispiel 3.
\item
$\Q(\sqrt[8]{2}\,\zeta^0, \ldots, \sqrt[8]{2}\,\zeta^7) = \Q(\sqrt[8]{2},
\zeta, \zeta^2, \ldots, \zeta^7) = \Q(\sqrt[8]{2}, \zeta) = \Q(\sqrt[8]{2}, 1 + \I) = \Q(\sqrt[8]{2},
\I),$ \\
${\qquad}$ für $\zeta := e^{2 \pi \I/8} = \frac{1}{\sqrt{2}}(1 + \I)$.
\item $\Q(\text{alle sechs Nullstellen von~$(X^4 - 2) (X^2 + 1)$}) =
\Q(\sqrt[4]{2}, \I)$
\item $\Q(\sqrt{2}, \I) = \Q(\sqrt{2} + \I)$ (ein Ausnahmefall; knifflig)
\end{enumerate}

\subsection*{Anwendungen}

Die Darstellung zu vereinfachen ist hilfreich, wenn man\ldots
\begin{itemize}
\item \ldots den Grad einer Körpererweiterung bestimmen
möchte.

\emph{Bsp.:} Die Erweiterung von Beispiel~3 hat über~$\Q$ den Grad~2, denn das
Minimalpolynom von~$\sqrt{3}\,\I$ über~$\Q$ ist~$X^2 + 3$ (wieso?). In der
Ausgangsformulierung~$\Q(\zeta^0,\zeta^1,\ldots,\zeta^5)$ erkennt man den Grad
dagegen nicht so schnell.
\item \ldots Zerfällungskörper in knapper Form angeben möchte.
\item \ldots Galoisgruppen bestimmen möchte.
\end{itemize}

\end{document}
