\documentclass[10pt,pdftex,a4paper,headsepline]{scrartcl}


\usepackage{ngerman}
\usepackage[latin1]{inputenc}
\usepackage[T1]{fontenc}
\usepackage{array}
\usepackage{amsmath, amsthm, amssymb}
\usepackage{mathtools}
\usepackage[right]{eurosym}
\usepackage[automark]{scrpage2}
\usepackage[bookmarks, colorlinks=true, linkcolor=blue, pagebackref=true]{hyperref}
\usepackage{color}
\usepackage[left=2cm,right=2cm,top=3cm,bottom=3.5cm]{geometry}


\pagestyle{scrheadings}


\begin{document}

\section{Wissenswertes �ber $S_n$}
\begin{itemize}
\item $S_n$ ist eine Gruppe mittels der Verkn�pfung von Abbildungen $\circ$ und dem neutralen Element $Id$
\item $S_n$ hat $n!$ viele Elemente
\item $A_n$ ist eine Untergruppe der $S_n$ und enth�lt alle geraden Permutationen ($sgn(\sigma)=1$)
\item $A_n$ hat genau $\frac{n!}{2}$ Elemente
\end{itemize}

\section{Rechnen mit $S_n$}
\begin{itemize}
\item Ein Element in $S_n$ ist eine Bijektion von einer $n$-elementigen Menge in sich selbst. Diese schreibt man am besten so:
\begin{align*}
\sigma = \begin{pmatrix}
1 & 2 & \cdots & n \\
\sigma (1) & \sigma (2) & \cdots & \sigma (n)
\end{pmatrix}
\end{align*}
\item Diese Schreibweise eignet sich hervorragend zum Rechnen:
\begin{align*}
\sigma \circ \tau = \begin{pmatrix}
1 & 2 & 3 & 4 & 5 & 6 \\
3 & 6 & 1 & 4 & 2 & 5
\end{pmatrix} \circ
\begin{pmatrix}
1 & 2 & 3 & 4 & 5 & 6 \\
5 & 4 & 3 & 2 & 1 & 6
\end{pmatrix} =
\begin{pmatrix}
1 & 2 & 3 & 4 & 5 & 6 \\
2 & 4 & 1 & 6 & 3 & 5
\end{pmatrix}
\end{align*}
\item Auch zur Inversenbestimmung muss man hierf�r nicht viel denken. Einfach umdrehen und sortieren.
\begin{align*}
\tau = \begin{pmatrix}
1 & 2 & 3 & 4 & 5 & 6 \\
6 & 5 & 4 & 1 & 2 & 3
\end{pmatrix}
\end{align*}
Umdrehen:
\begin{align*}
\begin{pmatrix}
6 & 5 & 4 & 1 & 2 & 3 \\
1 & 2 & 3 & 4 & 5 & 6
\end{pmatrix}
\end{align*}
Sortieren und fertig:
\begin{align*}
\tau^{-1}=
\begin{pmatrix}
1 & 2 & 3 & 4 & 5 & 6 \\
4 & 5 & 6 & 3 & 2 & 1
\end{pmatrix}
\end{align*}
\item die Zykelschreibweise eignet sich nicht zum Rechnen, da es dazu sehr viel Erfahrung ben�tigt. Also zuerst umrechnen, rechnen und wieder umrechnen.
\begin{align*}
\begin{pmatrix}
3 & 1
\end{pmatrix}
\circ
\begin{pmatrix}
1 & 2
\end{pmatrix}
=
\begin{pmatrix}
1 & 2 & 3 & 4 \\
3 & 2 & 1 & 4 
\end{pmatrix}
\circ
\begin{pmatrix}
1 & 2 & 3 & 4 \\
2 & 1 & 3 & 4 
\end{pmatrix}
=
\begin{pmatrix}
1 & 2 & 3 & 4 \\
2 & 3 & 1 & 4 
\end{pmatrix}
=
\begin{pmatrix}
1 & 2 & 3
\end{pmatrix}
\end{align*}
\end{itemize}
\section{Was n�tzt uns dann die Zykelschreibweise?}
\begin{itemize}
\item Bestimmung des Signums einer Permutation. Denn es gilt:
\begin{description}
\item $sgn(\tau)=(-1)^{k+1}$, wenn $\tau$ ein Zusammenh�ngender k-Zykel ist
\item $sgn(\tau \circ \sigma)= sgn(\tau) \cdot sgn(\sigma)$, falls $\tau$ und $\sigma$ disjunkte k-Zykel sind.
\end{description}
Bsp: ges: $sgn(\tau)$ mit
\begin{align*}
\tau=
\begin{pmatrix}
1 & 2 & 3 & 4 & 5 & 6 & 7 & 8 & 9 & 10 \\
6 & 7 & 5 & 9 & 8 & 4 & 10 & 3 & 1 & 2
\end{pmatrix}
=
\begin{pmatrix}
1 & 6 & 4 & 9
\end{pmatrix}
\begin{pmatrix}
2 & 7 & 10
\end{pmatrix}
\begin{pmatrix}
3 & 5 & 8
\end{pmatrix}
\end{align*}
\begin{align*}
\Longrightarrow \ sgn(\tau)= (-1)\cdot 1 \cdot 1= -1
\end{align*}
\item Bestimmung der Ordnung k f�r ein $\sigma \in S_n$\\
Die Ordung ist def. als kleinste nat�rliche Zahl: $\sigma^k=\underbrace{\sigma \circ \sigma \circ \cdots}_{k}= id$\\
Haben:
\begin{description}
\item $ord(\tau)=$k, falls $\tau$ ein zusammenh�ngender k-Zykel ist.
\item $ord(\tau \circ \sigma)=kgV(ord(\tau), ord(\sigma))$, wenn $\sigma$ und $\tau$ zwei disjunkte Zykel sind.
\end{description}
Bsp: Gesucht ist die Ordnung von $\sigma$ mit
\begin{align*}
\sigma=
\begin{pmatrix}
1 & 2 & 3 & 4 & 5 & 6\\
3 & 6 & 1 & 4 & 2 & 5
\end{pmatrix}
=
\begin{pmatrix}
1 & 3 
\end{pmatrix}
\begin{pmatrix}
2 & 6 & 5
\end{pmatrix}
\end{align*}
$\Longrightarrow ord(\sigma)= kgV(2,3)=6$
\end{itemize}
\end{document}
