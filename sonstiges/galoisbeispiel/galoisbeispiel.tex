\documentclass{../../algblatt}

\usepackage{geometry}
\geometry{tmargin=2cm,bmargin=2cm,lmargin=2.3cm,rmargin=2.3cm}

\pagestyle{empty}

\renewcommand{\labelitemi}{--}

\begin{document}

\begin{center}\huge \sffamily\textbf{Beispielberechnung einer Galoisgruppe}\end{center}

Wir wollen die Galoisgruppe der Nullstellen des Polynoms
\[ f(X) = X^4 - 1 \in \QQ[X] \]
bestimmen.

\begin{enumerate}
\item[\textbf{1.}] Die vier Nullstellen sind
\[ x_1 = 1,\quad
  x_2 = \i,\quad
  x_3 = -1,\quad
  x_4 = -\i. \]

\item[\textbf{2.}] Es gilt
\[ \QQ(x_1,x_2,x_3,x_4) = \QQ(1,\i,-1,-\i) = \QQ(\i,-\i) = \QQ(\i), \]
also ist~$t := \i$ ein primitives Element.

\item[\textbf{3.}] Wir müssen die Nullstellen als polynomielle Ausdrücke in~$t$ mit
rationalen Koeffizienten schreiben: Für die vier Nullstellen gilt jeweils~$x_i
= h_i(t)$, wobei
\begin{align*}
  h_1(X) &= 1, \\
  h_2(X) &= X, \\
  h_3(X) &= -1, \\
  h_4(X) &= -X.
\end{align*}
Nicht zulässig wäre beispielsweise die Setzung~$h_2(X) = \i$ gewesen (wieso?).
Eine korrekte, aber unnötig komplizierte, Alternative ist~$h_2(X) = X + X^2 +
1$ (wieso?).

\item[\textbf{4.}] Das Minimalpolynom von~$t$ ist bekanntermaßen~$X^2 + 1$: Denn
dieses Polynom ist normiert, hat nur rationale Koeffizienten, besitzt~$t$ als
Nullstelle und ist irreduzibel (es ist vom Grad~$2$ und seine Nullstellen sind
nicht rational).

\item[\textbf{5.}] Die zwei galoissch Konjugierten von~$t$ (die Nullstellen
seines Minimalpolynoms) sind
\[ t_1 = \i,\quad
  t_2 = -\i. \]

\item[\textbf{6.}] Damit können wir die Elemente der Galoisgruppe auflisten:
\begin{center}
  \begin{tabular}{@{}r|cccc|l@{}}
    $t_i$ & $h_1(t_i)$ & $h_2(t_i)$ & $h_3(t_i)$ & $h_4(t_i)$ & $\sigma_i$ \\\hline
    $t_1$ & $x_1$ & $x_2$ & $x_3$ & $x_4$ &
      $\begin{pmatrix}1&2&3&4\\1&2&3&4\end{pmatrix} = \id$ \\[1.5em]
    $t_2$ & $x_1$ & $x_4$ & $x_3$ & $x_2$ &
      $\begin{pmatrix}1&2&3&4\\1&4&3&2\end{pmatrix} = (2,4)$
  \end{tabular}
\end{center}
\end{enumerate}

Es gilt also
\[ \Gal_\QQ(x_1,x_2,x_3,x_4) = \left\{
  \begin{pmatrix}1&2&3&4\\1&2&3&4\end{pmatrix}\!,
  \begin{pmatrix}1&2&3&4\\1&4&3&2\end{pmatrix}
\right\} = \left\{ \id,\ (2, 4) \right\}\!. \]

\end{document}
