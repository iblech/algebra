\documentclass[12pt,a4paper,ngerman]{scrartcl}

\usepackage{ucs}
\usepackage[utf8x]{inputenc}
\usepackage[ngerman]{babel}
\usepackage{amsmath,amssymb,amscd,amsthm,color,graphicx}
\usepackage[protrusion=true,expansion=false]{microtype}
\usepackage{lmodern}
\usepackage{hyperref}

\setlength\parskip{\medskipamount}
\setlength\parindent{0pt}

\newcommand{\Bild}{\operatorname{im}}
\newcommand{\Kern}{\operatorname{ker}}
\newcommand{\Span}{\operatorname{span}}
\newcommand{\GL}{\mathrm{GL}}
\newcommand{\R}{\mathbb{R}}
\newcommand{\C}{\mathbb{C}}
\newcommand{\I}{\mathrm{i}}

\theoremstyle{definition}
\newtheorem*{defn}{Definition}
\newtheorem*{bsp}{Beispiel}

\theoremstyle{plain}
\newtheorem*{prop}{Proposition}
\newtheorem*{beh}{Behauptung}

\theoremstyle{remark}
\newtheorem*{bem}{Bemerkung}

\usepackage{geometry}
\geometry{tmargin=3cm,bmargin=3cm,lmargin=3cm,rmargin=3cm}

\begin{document}

\section*{Rechenregeln für Ideale}

\begin{enumerate}
\item $(x_1,\ldots,x_n) + (y_1,\ldots,y_m) = (x_1,\ldots,x_n,y_1,\ldots,y_m)$
\item $(x_1,\ldots,x_n) \cdot (y_1,\ldots,y_m) = (x_1 y_1,\ldots,x_1 y_m,x_2
y_1,\ldots,x_2 y_m,\ldots\ldots,x_n y_1,\ldots,x_n y_m)$
\item Die Reihenfolge der Erzeuger spielt keine Rolle.
\item Ein Ideal ändert sich nicht, wenn man zu einem Erzeuger ein beliebiges
Vielfaches eines anderen Erzeugers addiert.
\item Ist ein Erzeuger ein Vielfaches eines anderen, so kann man ihn
weglassen.
\item Ein Ideal ändert sich nicht, wenn man einen Erzeuger mit einer beliebigen
Einheit multipliziert.

Ist einer der Erzeuger eine Einheit, so ist das Ideal schon das
Einsideal.
\item Speziell in Bézoutschen Ringen:
\begin{align*}
  (x_1,\ldots,x_n) &= (\mathrm{ggT}(x_1,\ldots,x_n)) \\
  (x) \cap (y) &= (\mathrm{kgV}(x,y))
\end{align*}
\item Speziell in Polynomringen:
\[ (f_1(X),\ldots,f_n(X),X-a) =
(f_1(a),\ldots,f_n(a),X-a) \]
\end{enumerate}

\subsection*{Beispiele}

\begin{itemize}
\item in~$\mathbb{Z}$: $(8,6,4) = (\mathrm{ggT}(8,6,4)) = (2)$
\item in~$\mathbb{Z}$: $(2,4) \cdot (3,6) = (2\cdot3,2\cdot6,4\cdot3,4\cdot6) =
(6,12,12,24) = (6)$
\item in~$\mathbb{Z}[X]$: $(X^2-25, X-3) = (3^2-25, X-3) = (-16, X-3) =
(16,X-3)$
\item in~$\mathbb{Q}$: $(8,6,4) = (1) = (2) = (3173)$
\item in~$\mathbb{R}[X]$: $(X^2-25, X-3) = (3^2-25, X-3) = (-16, X-3) = (1)$
\end{itemize}

\end{document}
