\documentclass{algblatt}
\loesungenfalse

\geometry{tmargin=2cm,bmargin=2cm,lmargin=3.0cm,rmargin=3.0cm}

%\setlength{\titleskip}{0.7em}
\setlength{\aufgabenskip}{1.6em}

\begin{document}

\vspace*{-1.5cm}
\maketitle{3}{Abgabe bis 6. Mai 2013, 17:00 Uhr}

\begin{aufgabe}{Beispiele für algebraische Zahlen}
\begin{enumerate}
\item Ist die Zahl~$\cos 10^\circ$ algebraisch?
\item Zeige, dass die Polynomgleichung $X^3 - 2X + 5 = 0$ genau eine reelle
Lösung~$\alpha$ besitzt.
\item Zeige, dass diese Lösung~$\alpha$ invertierbar ist, und finde eine normierte
Polynomgleichung mit rationalen Koeffizienten, die~$\alpha^{-1}$ als Lösung besitzt.
\end{enumerate}
\end{aufgabe}

\begin{aufgabe}{Produkt algebraischer Zahlen}
\begin{enumerate}
\item Seien~$x$ und~$y$ Zahlen mit~$x^5-x+1=0$ und~$y^2-2=0$.
Finde eine normierte Polynomgleichung mit rationalen
Koeffizienten, die die Zahl~$x \cdot y$ als Lösung besitzt.
\item Der \emph{Grad} einer algebraischen Zahl~$z$ ist der kleinstmögliche Grad
einer normierten Polynomgleichung mit rationalen Koeffizienten, die~$z$ als Lösung
besitzt. Finde eine Abschätzung für den Grad des Produkts zweier algebraischer
Zahlen in Abhängigkeit der Grade der Faktoren.
\end{enumerate}
\end{aufgabe}

\begin{aufgabe}{Eigenschaften algebraischer Zahlen}
\begin{enumerate}
\item Zeige, dass der Betrag einer jeden algebraischen Zahl algebraisch ist.
\item Zeige, dass rationale ganz-algebraische Zahlen schon ganzzahlig sind.
\item Sei~$f$ ein normiertes Polynom mit rationalen Koeffizienten und~$z$ eine
transzendente Zahl. Zeige, dass dann auch~$f(z)$ eine transzendente Zahl ist.
\end{enumerate}
\end{aufgabe}

\begin{aufgabe}{Spielen mit Einheitswurzeln}
\begin{enumerate}
\item Finde alle komplexen Lösungen der Gleichung~$X^6 + 1 = 0$.
\item Finde eine Polynomgleichung, deren Lösungen genau die Ecken
desjenigen re\-gel\-mä\-ßi\-gen Siebenecks in der komplexen Zahlenebene sind, dessen Zentrum
der Ursprung der Ebene ist und das deine Lieblingszahl als eine Ecke besitzt.
\item Zeige, dass die Gleichung~$X^{n-1} + X^{n-2} + \cdots + X + 1 = 0$
genau~$n-1$ Lösungen besitzt, und zwar alle~$n$-ten Einheitswurzeln bis auf
die~$1$.
\item Sei~$\zeta$ eine $n$-te und~$\vartheta$ eine~$m$-te Einheitswurzel.
Zeige, dass~$\zeta \cdot \vartheta$ eine~$k$-te Einheitswurzel ist, wobei~$k$
das kleinste gemeinsame Vielfache von~$n$ und~$m$ ist.
\end{enumerate}
\end{aufgabe}

\begin{aufgabe}{Primitive Einheitswurzeln}
Eine~$n$-te Einheitswurzel~$\zeta$ heißt genau dann \emph{primitiv}, wenn
\emph{jede}~$n$-te Einheitswurzel eine ganzzahlige Potenz von~$\zeta$ ist.
Sei~$\Phi(n)$ die Anzahl der zu~$n$ teilerfremden Zahlen
in~$\{1,\ldots,n\}$.
\begin{enumerate}
\item Kläre ohne Verwendung von~b): Wie viele primitive vierte Einheitswurzeln gibt es?
\item Zeige, dass es genau~$\Phi(n)$ primitive~$n$-te
Einheitswurzeln gibt.
\end{enumerate}
\end{aufgabe}

\end{document}

\begin{exercise}(3 Punkte)\newline
    Folgere die Additionstheoreme für die Sinus- und die Kosinusfunktion aus der
    Identität
    \(\exp({x \mathrm i}) \cdot \exp({y \mathrm i}) = \exp({(x + y) \mathrm i})\).
\end{exercise}
