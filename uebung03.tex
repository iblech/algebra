\documentclass{algblatt}
\usepackage{xstring}
\IfSubStr{\jobname}{\detokenize{loesung}}{\loesungentrue}{\loesungenfalse}

\geometry{tmargin=2cm,bmargin=2cm,lmargin=3.0cm,rmargin=3.0cm}

%\setlength{\titleskip}{0.7em}
\setlength{\aufgabenskip}{1.6em}

\begin{document}

\vspace*{-1.5cm}
\maketitle{3}{Abgabe bis 6. Mai 2013, 17:00 Uhr}

\begin{aufgabe}{Beispiele für algebraische Zahlen}
\begin{enumerate}
\item Ist die Zahl~$\cos 10^\circ$ algebraisch?
\item Zeige, dass die Polynomgleichung $X^3 - 2X + 5 = 0$ genau eine reelle
Lösung~$\alpha$ besitzt.
\item Zeige, dass diese Lösung~$\alpha$ invertierbar ist, und finde eine normierte
Polynomgleichung mit rationalen Koeffizienten, die~$\alpha^{-1}$ als Lösung besitzt.
\end{enumerate}
\begin{loesungE}
\item Ja, denn die Zahl~$\cos 10^\circ$ ist der Realteil der komplexen
Zahl~$e^{\pi\i/18}$, und diese ist algebraisch, da sie die Gleichung
\[ X^{18} + 1 = 0 \]
erfüllt (wieso?). Da Realteile algebraischer Zahlen selbst ebenfalls
algebraisch sind, begründet das die Algebraizität von~$\cos 10^\circ$.
\item Wir setzen~$f := X^3-2X+5$. Da~$f(-3) =
-16 < 0 < 1 = f(-2)$, besitzt die Gleichung~$f(X) = 0$ nach Blatt~1, Aufgabe~2
mindestens eine reelle Lösung~$\alpha$ im Intervall~$(-3, -2)$. Mit einer Polynomdivision
durch~$(X-\alpha)$ kann man~$f$ faktorisieren:
\[ f = (X - \alpha) (X^2 + \alpha X + \alpha^2-2). \]
Das verbleibende Polynom hat nun keine weiteren reellen Nullstellen, denn seine
Diskriminante ist negativ:
\[ D = \alpha^2 - 4(\alpha^2-2) = 8 - 3 \alpha^2 \leq 8 - 3 \cdot 2^2 = -4 < 0. \]
\item Die Zahl~$\alpha$ kann nicht Null sein, da Null keine Lösung der
Gleichung~$f(X) = 0$ ist:
\[ f(0) = 0^3 - 2 \cdot 0 + 5 = 5 \neq 0. \]
Also ist~$\alpha$ invertierbar. Für die Zahl~$\alpha^{-1}$ gilt
\[ (\alpha^{-1})^{-3} - 2 (\alpha^{-1})^{-1} + 5 = 0; \]
das ist zwar eine Gleichung, aber keine Polynomgleichung für~$\alpha^{-1}$.
Wenn wir mit~$(\alpha^{-1})^3$ durchmultiplizieren, erhalten wir die
äquivalente Gleichung
\[ 1 - 2 (\alpha^{-1})^2 + 5 (\alpha^{-1})^3 = 0. \]
Also ist~$\alpha^{-1}$ Lösung der normierten Polynomgleichung mit rationalen
Koeffizienten
\[ X^3 - \frac{2}{5} X^2 + \frac{1}{5} = 0. \]
\end{loesungE}
\end{aufgabe}

\begin{aufgabe}{Produkt algebraischer Zahlen}
\begin{enumerate}
\item Seien~$x$ und~$y$ Zahlen mit~$x^5-x+1=0$ und~$y^2-2=0$.
Finde eine normierte Polynomgleichung mit rationalen
Koeffizienten, die die Zahl~$x \cdot y$ als Lösung besitzt.
\item Der \emph{Grad} einer algebraischen Zahl~$z$ ist der kleinstmögliche Grad
einer normierten Polynomgleichung mit rationalen Koeffizienten, die~$z$ als Lösung
besitzt. Finde eine Abschätzung für den Grad des Produkts zweier algebraischer
Zahlen in Abhängigkeit der Grade der Faktoren.
\end{enumerate}
\end{aufgabe}

\begin{aufgabe}{Eigenschaften algebraischer Zahlen}
\begin{enumerate}
\item Zeige, dass der Betrag einer jeden algebraischen Zahl algebraisch ist.
\item Zeige, dass rationale ganz-algebraische Zahlen schon ganzzahlig sind.
\item Sei~$f$ ein normiertes Polynom mit rationalen Koeffizienten und~$z$ eine
transzendente Zahl. Zeige, dass dann auch~$f(z)$ eine transzendente Zahl ist.
\end{enumerate}
\begin{loesungE}
\item Sei~$z$ eine algebraische Zahl. Dann gilt
\[ |z|^2 = z \overline{z}. \]
Da mit~$z$ auch~$\overline{z}$ algebraisch ist und das Produkt algebraischer
Zahlen algebraisch ist, ist die rechte Seite dieser Identität algebraisch. Der
Betrag von~$z$ ist also als eine der Lösungen der Gleichung mit algebraischen
Koeffizienten
\[ X^2 - z \overline{z} = 0 \]
ebenfalls algebraisch.
\item Sei~$z$ eine rationale ganz-algebraische Zahl. Dann erfüllt~$z$ also eine
normierte Polynomgleichung mit ganzzahligen Koeffizienten. Nach Blatt~0,
Aufgabe~3b) ist~$z$ daher schon ganzzahlig.
\item Angenommen,~$y := f(z)$ wäre algebraisch. Dann gibt es ein normiertes
Polynom~$g$ mit rationalen Koeffizienten, sodass~$y$ die Gleichung
\[ g(Y) = 0 \]
erfüllt, sodass also~$g(f(z)) = 0$ ist. Setzt man~$h := g \circ f$ -- das ist
wieder ein normiertes Polynom mit rationalen Koeffizienten (wieso?) -- sieht
man, dass~$z$ Lösung der Gleichung~$h(X) = 0$ ist. Das ist ein Widerspruch zur
Transzendenz von~$z$.
\end{loesungE}
\end{aufgabe}

\begin{aufgabe}{Spielen mit Einheitswurzeln}
\begin{enumerate}
\item Finde alle komplexen Lösungen der Gleichung~$X^6 + 1 = 0$.
\item Finde eine Polynomgleichung, deren Lösungen genau die Ecken
desjenigen re\-gel\-mä\-ßi\-gen Siebenecks in der komplexen Zahlenebene sind, dessen Zentrum
der Ursprung der Ebene ist und das die Zahl~$1 + \i$ als eine Ecke besitzt.
\item Zeige, dass die Gleichung~$X^{n-1} + X^{n-2} + \cdots + X + 1 = 0$
genau~$n-1$ Lösungen besitzt, und zwar alle~$n$-ten Einheitswurzeln bis auf
die~$1$.
\item Sei~$\zeta$ eine $n$-te und~$\vartheta$ eine~$m$-te Einheitswurzel.
Zeige, dass~$\zeta \cdot \vartheta$ eine~$k$-te Einheitswurzel ist, wobei~$k$
das kleinste gemeinsame Vielfache von~$n$ und~$m$ ist.
\end{enumerate}
\begin{loesungE}
\item Bezeichne~$\xi$ eine primitive sechste Einheitswurzel, etwa~$\xi =
e^{2\pi\i / 6}$. Eine Lösung der Gleichung ist~$\i$. Daher sind die
insgesamt sechs Lösungen der Gleichung durch
\[ \i,\quad \xi \i,\quad \xi^2 \i,\quad \xi^3 \i,\quad \xi^4 \i,\quad \xi^5 \i \]
gegeben (wieso?).
\item Die Gleichung~$X^7 - (1+\i)^7 = 0$ tut's (wieso?).

\emph{Bemerkung:} Wenn man möchte, kann man die Gleichung auch ausfaktorisiert
hinschreiben. Sei dazu~$\xi$ eine primitive siebte Einheitswurzel, etwa~$\xi =
e^{2\pi\i / 7}$. Dann ist obige Gleichung äquivalent zu
\[ \prod_{k=0}^6 (X - \xi^k (1+\i)) = 0. \]

\emph{Bemerkung:} Es kann keine Polynomgleichung mit \emph{rationalen}
Koeffizienten geben, die genau die sieben Ecken als Lösungen besitzt. Denn jede
solche Gleichung würde mit~$1+\i$ auch das komplex Konjugierte~$1-\i$ als
Lösung besitzen, das ist aber keine der Ecken.

\item Sei~$x$ eine beliebige komplexe Zahl. Dann gilt:
\begin{align*}
  && x^{n-1} + x^{n-2} + \cdots + x + 1 &= 0 \\
  \stackrel{?}{\Longleftrightarrow} &&
    x^{n-1} + x^{n-2} + \cdots + x + 1 &= 0 \ \wedge\  x \neq 1 \\
  \Longleftrightarrow &&
    (x - 1) \cdot (x^{n-1} + x^{n-2} + \cdots + x + 1) &= 0 \ \wedge\  x \neq 1 \\
  \Longleftrightarrow &&
    x^n - 1 &= 0 \ \wedge\  x \neq 1 \\
  \Longleftrightarrow &&
    \omit\rlap{\text{$x$ ist eine der~$n$-ten Einheitswurzeln, aber nicht
    die~$1$.}}
\end{align*}
Da wir durchgängig Äquivalenzumformungen verwendet haben, zeigt diese
Überlegung tatsächlich die Behauptung.

\emph{Bemerkung:} Bei einem~"`$\Rightarrow$"'-Schritt können Phantomlösungen
entstehen, bei einem~"`$\Leftarrow$"'-Schritt können Lösungen verloren gehen.

\item Da~$k$ ein Vielfaches von~$n$ ist, gilt~$\zeta^k = 1$. Analog
gilt~$\vartheta^k = 1$. Daher folgt:
\[ (\zeta \cdot \vartheta)^k = \zeta^k \cdot \vartheta^k = 1 \cdot 1 = 1. \]
\end{loesungE}
\end{aufgabe}

\begin{aufgabe}{Primitive Einheitswurzeln}
Eine~$n$-te Einheitswurzel~$\zeta$ heißt genau dann \emph{primitiv}, wenn
\emph{jede}~$n$-te Einheitswurzel eine ganzzahlige Potenz von~$\zeta$ ist.
Sei~$\Phi(n)$ die Anzahl der zu~$n$ teilerfremden Zahlen
in~$\{1,\ldots,n\}$.
\begin{enumerate}
\item Kläre ohne Verwendung von~b): Wie viele primitive vierte Einheitswurzeln gibt es?
\item Zeige, dass es genau~$\Phi(n)$ primitive~$n$-te
Einheitswurzeln gibt.
\end{enumerate}
\begin{loesungE}
\item Insgesamt gibt es vier vierte Einheitswurzeln:
\[ 1,\quad \i,\quad -1,\quad -\i. \]
Von diesen sind~$\i$ und~$-\i$ primitiv: Denn die Potenzen von~$\i$ geben
gerade diese vier Zahlen, und für~$-\i$ stimmt es auch. Die anderen beiden
Wurzeln sind aber nicht primitiv: Denn die Potenzen von~$1$ sind nur~$1$
selbst, und die von~$-1$ sind nur~$\pm 1$.
\end{loesungE}
\end{aufgabe}

\end{document}

\begin{exercise}(3 Punkte)\newline
    Folgere die Additionstheoreme für die Sinus- und die Kosinusfunktion aus der
    Identität
    \(\exp({x \mathrm i}) \cdot \exp({y \mathrm i}) = \exp({(x + y) \mathrm i})\).
\end{exercise}
