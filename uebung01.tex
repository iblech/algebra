\documentclass{algblatt}

\begin{document}

\maketitle{1}{Abgabe am ??.4.2013}

\begin{aufgabe}{Lösungskriterium}
Sei eine normierte Polynomgleichung mit ganzzahligen Koeffizienten
der Form
\[ X^n + a_{n-1} X^{n-1} + \cdots + a_1 X + a_0 = 0 \]
gegeben. Zeige, dass
jede ganzzahlige Lösung ein Teiler von~$a_0$ sein muss.
\end{aufgabe}

\begin{aufgabe}{Polynomgleichungen ungeraden Grads}
Zeige, dass jede normierte Polynomgleichung ungeraden Grads mit rationalen
Koeffizienten in den reellen Zahlen eine Lösung besitzt.
\end{aufgabe}

\begin{aufgabe}{Beispiele für Polynomgleichungen}
Finde eine normierte Polynomgleichung\ldots
\begin{enumerate}
\item vierten Grads mit rationalen
Koeffizienten, welche in den reellen Zahlen keine Lösung besitzt.
\item fünften Grads mit rationalen
Koeffizienten, welche als einzige Lösung die Zahl~$1$ besitzt.
\item mit ganzzahligen Koeffizienten, die
$\sqrt[7]{3 + \sqrt[3]{4}}$ als eine Lösung besitzt.
\item mit ganzzahligen Koeffizienten, die
$\cos 15^\circ$ als eine Lösung besitzt.
\end{enumerate}
\end{aufgabe}

\begin{aufgabe}{Calabis Dreieck}
Neben dem gleichseitigen Dreieck gibt es nur ein Dreieck, das folgende erstaunliche
Eigenschaft hat: Das größte 
einbeschreibbare Quadrat lässt sich auf drei verschiedene Arten einbeschrieben.
Dieses zweite Dreieck hat Eugenio Calabi (1923--, italienisch-amerikanischer
Mathematiker) gefunden und ist gleichschenklig.

Zeige, dass das
Längenverhältnis der längsten zu
einer der kürzeren Seiten die Gleichung
\[ 2 X^3 - 2 X^2 - 3 X + 2 = 0 \]
erfüllt. \\[1em]

\begin{center}
  \scalebox{0.5}{\input{calabi.pspdftex}}
\end{center}
\end{aufgabe}

\end{document}
