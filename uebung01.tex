\documentclass{algblatt}
\loesungenfalse


\begin{document}

\maketitle{1}{Abgabe bis 23. April 2013, 12:00 Uhr}

\begin{aufgabe}{Lösungskriterium}
Sei eine normierte Polynomgleichung mit ganzzahligen Koeffizienten
der Form
\[ X^n + a_{n-1} X^{n-1} + \cdots + a_1 X + a_0 = 0 \]
gegeben. Zeige, dass
jede ganzzahlige Lösung ein Teiler von~$a_0$ sein muss.
\begin{loesung}Sei~$x$ eine ganzzahlige Lösung der Gleichung. Dann ist~$n \geq
1$ und es folgt
\[ x \cdot (x^{n-1} + a_{n-1} x^{n-2} + \cdots + a_1) = -a_0. \]
Da der Ausdruck in Klammern eine ganze Zahl ist (wieso?), ist daher~$x$ ein
Teiler von~$a_0$ (wieso?).

\emph{Bemerkung:} Zusammen mit Aufgabe~3b) von Blatt~0 hat man damit ein
nützliches Kriterium gefunden: Jede rationale Lösung einer solchen
Polynomgleichung muss sogar schon ganzzahlig und ein Teiler von~$a_0$ sein.
Weitere irrationale Lösungen werden damit aber nicht ausgeschlossen.
\end{loesung}
\end{aufgabe}

\begin{aufgabe}{Polynomgleichungen ungeraden Grads}
Zeige, dass jede normierte Polynomgleichung ungeraden Grads mit rationalen
Koeffizienten eine Lösung in den reellen Zahlen besitzt.
\begin{loesung}Sei~$f$ ein Polynom ungeraden Grads mit rationalen
Koeffizienten. Dann konstruieren wir in drei Schritten eine Lösung der
Gleichung~$f(X) = 0$.
\begin{enumerate}
\item[1.] Zunächst finden wir zwei Schranken für die zu findende Lösung, genauer
zwei Zahlen~$x_0, y_0 \in \QQ$ mit~$f(x_0) < 0 < f(y_0)$.

Das beweist man entweder über eine Grenzwertüberlegung oder direkt über
Abschätzungen: Sei~$R := |a_{n-1}| + \cdots + |a_0|$. Dann gilt für
alle~$x \in \RR$ mit~$|x| \geq R + 1$ folgende Hilfsüberlegung (wieso?):
\begin{align*}
  \left|a_{n-1} \frac{1}{x} + \cdots + a_1 \frac{1}{x^{n-1}} + a_0
  \frac{1}{x^n}\right| &\stackrel{\triangle}{\leq}
  |a_{n-1}| \frac{1}{|x|} + \cdots + |a_1| \frac{1}{|x|^{n-1}} + |a_0|
  \frac{1}{|x|^n} \\
  &\leq R \cdot \frac{1}{|x|}
  \leq \underbrace{\frac{R}{R+1}}_{=:\,q} < 1
\end{align*}
Somit gilt (wieso?) für alle~$x \geq R + 1$
\[ f(x) = x^n \left(1 + a_{n-1} \frac{1}{x} + \cdots + a_0 \frac{1}{x^n}\right)
  \geq x^n \left(1 - q\right) > 0 \]
und für alle~$x \leq -R - 1$
\[ f(x) = x^n \left(1 + a_{n-1} \frac{1}{x} + \cdots + a_0 \frac{1}{x^n}\right)
  \leq x^n \left(1 - q\right) < 0. \]
Die Zahlen~$x_0 := -R - 1$ und~$y_0 := R + 1$ leisten also das Gewünschte.

\item[2.] Nun geben wir eine Konstruktion an, um sukzessive immer bessere
Schranken
\[ x_n, y_n \in \QQ \text{ mit } x_n \leq y_n \text{ und } f(x_n) \leq 0 \leq
f(y_n) \]
zu finden. Der erste Schritt wird durch obige Argumentation erledigt. Ist
nun~$(x_n,y_n)$ schon konstruiert, schreiben wir~$m := \frac{x_n + y_n}{2}$
und konstruieren die nächstbesseren Schranken (Skizze!):
\[ \begin{cases}
  \text{Falls~$f(m) < 0$:} &
    \text{Setze $x_{n+1} := m$, $y_{n+1} := y_n$.} \\
  \text{Falls~$f(m) = 0$:} &
    \text{Setze $x_{n+1} := m$, $y_{n+1} := m$.} \\
  \text{Falls~$f(m) > 0$:} &
    \text{Setze $x_{n+1} := x_n$, $y_{n+1} := m$.}
\end{cases} \]
Diese Fallunterscheidung kann man tatsächlich (in der Praxis, intuitionistisch)
durchführen, da~$f(m)$ eine rationale Zahl ist (wieso?).
\item[3.] Die so konstruierten Schranken~$x_n$ und~$y_n$ legen nach dem
Intervallschachtelungsprinzip eine bestimmte Zahl~$z \in \RR$ fest (wieso?):
\[ x_n \xra{n \to \infty} z, \quad y_n \xra{n \to \infty} z. \]
Daher konvergieren auch die Folgen~$(f(x_n))_n$ und~$(f(y_n))_n$:
\[ f(x_n) \xra{n \to \infty} f(z), \quad f(y_n) \xra{n \to \infty} f(z). \]
Denn~$f$ setzt sich ja nur aus Konstanten, Addition und Multiplikation
zusammen, und dafür gibt es in der Analysis entsprechende Grenzwertregeln
(etwas abstrakter: Die Polynomfunktion~$f$ ist stetig!).

Die Behauptung~$f(z) = 0$ folgt nun wegen der Tatsache, dass schwache
Ungleichungen im Grenzwert erhalten bleiben, aus folgender Beobachtung:
\begin{align*}
  \text{$f(x_n) \leq 0$ für alle~$n \geq 0$} &\quad\Longrightarrow\quad f(z) \leq 0 \\
  \text{$f(y_n) \geq 0$ für alle~$n \geq 0$} &\quad\Longrightarrow\quad f(z) \geq 0
\end{align*}
\end{enumerate}

\emph{Bemerkung:} Im Wesentlichen wiederholt obige Argumentation einfach den
Beweis des Zwischenwertsatzes für stetige Funktionen aus der Analysis. In
seiner Allgemeinheit steht dieser uns nicht zur Verfügung, da wir die im
zweiten Schritt benötigten Fallunterscheidungen bei beliebigen stetigen
Funktionen intuitionistisch nicht treffen können.

\emph{Bemerkung:} Wenn man den Fundamentalsatz der Algebra schon kennt, kann
man kürzer so argumentieren: Echt komplexe Lösungen treten stets in
komplex-konjugierten Paaren auf. Da die Gleichung~$f(X) = 0$ aber insgesamt
eine ungerade Anzahl von Lösungen hat, muss mindestens eine der Lösungen rein
reell sein.
\end{loesung}
\end{aufgabe}

\ifloesungen\newpage\fi
\begin{aufgabe}{Beispiele für Polynomgleichungen}
Finde eine normierte Polynomgleichung mit rationalen Koeffizienten\ldots
\begin{enumerate}
\item vierten Grads, welche in den reellen Zahlen keine Lösung besitzt.
\item fünften Grads, welche als einzige komplexe Lösung die Zahl~$1$ besitzt.
\item die $\sqrt[7]{3 + \sqrt[3]{4}}$ als eine Lösung besitzt.
\item die $\cos 15^\circ$ als eine Lösung besitzt.
\end{enumerate}
\begin{loesung}\begin{enumerate}
\item $X^4 + 1 = 0$ oder $(X^2 + 1) (X^2 + 1) = 0$ oder viele andere (wieso?).
\item $(X - 1)^5 = 0$ (wieso?), d.\,h. $X^5-5\,X^4+10\,X^3-10\,X^2+5\,X-1 = 0$.

\emph{Bemerkung:} Die Gleichung~$X^5 - 1 = 0$ hat als einzige reelle Lösung die
Zahl~$1$, besitzt in den komplexen Zahlen aber noch vier weitere Lösungen,
nämlich die vier restlichen fünften Einheitswurzeln.
\item Wir schreiben~$x := \sqrt[7]{3 + \sqrt[3]{4}}$ und formen diese Identität
so lange um, bis keine Wurzeln mehr übrig bleiben: Konkret rechnen wir hoch
sieben, minus drei, hoch drei und minus vier. Mit diesen Schritten sieht man,
dass
\[ (x^7 - 3)^3 - 4 = 0 \]
folgt. Also ist $(X^7-3)^3-4=0$ die gesuchte Gleichung, ausmultipliziert
$X^{21}-9\,X^{14}+27\,X^7-31 = 0$.

\emph{Bemerkung:} Eine Probe ist nicht nötig, da man beim Herleiten der
Gleichung "`$\Rightarrow$"'-Zeichen verwendet hat. In der Gleichung am Ende sollte die
Polynomvariable groß geschrieben werden.
\item Nach einem Additionstheorem gilt
\[ \cos 30^\circ = 2 \, (\cos 15^\circ)^2 - 1. \]
Stellt man diese Beziehung unter Beachtung von~$\cos 30^\circ = \sqrt{3}/2$ um, erkennt man~$\cos 15^\circ$ als Lösung der
Gleichung
\[ X^4 - X^2 + \frac{1}{16} = 0. \]

\emph{Bemerkung:} Durch Verwendung eines anderen Additionstheorems kann man den
Wert von~$\cos 15^\circ$ auch direkt bestimmen,
\[ \cos 15^\circ = \cos (45^\circ - 30^\circ) =
  \cos 45^\circ \cos 30^\circ + \sin 45^\circ \cos 30^\circ = \cdots =
  \frac{1 + \sqrt{3}}{2 \sqrt{2}}, \]
und dann die Technik von Teilaufgabe~c) verwenden.

\emph{Bemerkung:} Eine abstraktere Argumentation ist folgende.
Die Zahl~$e^{\i \pi / 12}$ ist algebraisch (da sie die Gleichung~$X^{24} - 1 =
0$ erfüllt). Nach einem späteren Satz der Vorlesung muss daher auch ihr Realteil
algebraisch sein. Dieser ist gerade~$\cos 15^\circ$. Da der Beweis dieses Satzes
konstruktiv ist, kann man aus ihm sicher auch eine entsprechende Gleichung für~$\cos
15^\circ$ ablesen, allerdings wird diese recht großen Grad haben.
\end{enumerate}
\end{loesung}
\end{aufgabe}

\begin{aufgabe}{Calabis Dreieck}
Neben dem gleichseitigen Dreieck gibt es nur ein Dreieck, das folgende erstaunliche
Eigenschaft hat: Das größte 
einbeschreibbare Quadrat lässt sich auf drei verschiedene Arten einbeschreiben.
Dieses zweite Dreieck hat Eugenio Calabi (1923--, italienisch-amerikanischer
Mathematiker) gefunden und ist gleichschenklig.

Zeige, dass das Längenverhältnis der längsten zu einer der kürzeren Seiten die
Gleichung
\[ 2 X^3 - 2 X^2 - 3 X + 2 = 0 \]
erfüllt. \ifloesungen\else\\[1em]\fi

\begin{center}
  \scalebox{0.5}{\input{calabi.pspdftex}}
\end{center}

\begin{loesung}\addtolength{\jot}{0.3em}%
Da es nur um das Verhältnis der Längen geht, können wir die Längen der
beiden Katheten jeweils als~$1$ festlegen. Die Länge~$x$ der Hypotenuse ist
dann gesucht. Außerdem ist es hilfreich, drei weitere Größen zu betrachten
(Skizze!):
\begin{itemize}
\item die Länge~$h$ der Höhe des Dreiecks,
\item die Seitenlänge~$s$ der Quadrate und
\item den spitzen Winkel~$\alpha$ (in der Skizze unten links).
\end{itemize}
\emph{Variante 1:} Man kann folgende drei Gleichungen aufstellen:
\begin{align}
  \frac{s}{1 - s} &= \frac{h}{x/2} \\
  \frac{s}{1 - s} &= \frac{s}{x/2 - s/2} \\
  (x/2)^2 + h^2 &= 1
\end{align}
Gleichung~(1) drückt den Tangens von~$\alpha$ auf zwei verschiedene Arten und
Weisen aus: über das "`schiefe"' Dreieck unten links und über das (nicht
eingezeichnete) Teildreieck, dessen eine Seite die Höhe ist. Auch die zweite
Gleichung drückt den Tangens aus, über das schiefe Dreieck und das, dessen eine
Seite die linke Kante des blauen Quadrats ist. Gleichung~(3) ist eine Instanz
des Satzes von Pythagoras.

Dieses Gleichungssystem kann man dann sukzessive lösen. Etwa kann man zunächst
die erste Gleichung nach~$s$ und die dritte nach~$h$ auflösen:
\begin{align*}
  s &= \frac{2h}{x + 2h} \\
  h &= \sqrt{1 - x^2/4}
\end{align*}
(Wieso weiß man, dass~$h$ die positive Lösung der quadratischen Gleichung sein
muss?) Setzt man diese Darstellungen in Gleichung~(2) ein, erhält man
\[ 2x^4 - 6x^3 + x^2 + 8x - 4 = 0; \]
das ist schon fast die gesuchte Gleichung. Mittels einer Polynomdivision kann
man den Faktor~$(x-2)$ ausklammern:
\[ (2x^3 - 2x^2 - 3x + 2) (x-2) = 0. \]
Da dieser Faktor nicht null ist (sonst wäre das Dreieck entartet), kann man ihn
kürzen.

\emph{Variante 2:} Ein anderes zielführendes Gleichungssystem ist folgendes
(nach Tim~Baumann):
\begin{align*}
  h &= s + \tan\alpha \cdot s/2 \\
  h &= \cos\alpha \cdot s + \sin\alpha \cdot s \\
  x/2 &= \cos\alpha
\end{align*}
Die erste Gleichung kommt durch Betrachtung des kleinen Teildreiecks oben
zustande, die zweite durch Betrachtung des Winkels~$\alpha$ an einer zu
Hypotenuse parallelen Hilfslinie durch die obere Spitze des "`schiefen
Dreiecks"' (wie genau?).

Setzt man die ersten beiden Gleichungen gleich und formt unter Beachtung
von~$0^\circ < \alpha < 90^\circ$ um, erhält man nach
einigem Rechnen die Beziehung
\[ y^4 + 3y^3 - \frac{1}{4}y^2 - y + \frac{1}{4} = 0 \]
für~$y := \cos\alpha$. Da~$y \neq 1$ (wieso?), kann man durch den
Faktor~$(y-1)$ dividieren und erhält
\[ 2 (2y)^3 - 2(2y)^2 - 3 (2y) + 2 = 0. \]
Mit der dritten Gleichung folgt die Behauptung.
\end{loesung}
\end{aufgabe}

\end{document}
